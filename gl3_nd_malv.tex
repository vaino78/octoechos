%<%[Въ суббw'ту ве'чера,  на ма'лjьй вече'рни%],%>

%<на Г%>д\си воззва'хъ, %<стiхи^ры воскр\сны г~, 
повторя'юще пе'рвую. Гла'съ г~.%>

%<Стi'хъ: W\т%> стра'жи о_у='треннiя до но'щи, w\т 
стра'жи о_у='треннiя, да о_у=пова'етъ i=и~ль на гд\са.

%<Т%>вои'мъ кр\сто'мъ хр\сте` сп~се, сме'рти держа'ва 
разруши'ся, и= дiа'воля пре'лесть о_у=праздни'ся: ро'дъ 
же человjь'ческiй вjь'рою сп~са'емый, пjь'снь тебjь` 
всегда` прино'ситъ.

%<П%>росвjьти'шася вся'ч_еская воскр\снiемъ твои'мъ 
гд\си, и= ра'й па'ки w\тве'рзеся: вся' же тва'рь 
восхваля'ющи тя`, пjь'снь тебjь` всегда` прино'ситъ.

%<С%>ла'влю _о=ц~а` и= сн~а си'лу, и= ст~а'гw дх~а 
пою` вла'сть, нераздjь'льное несозда'нное бж~ество`, 
тр\оцу _е=диносу'щную, ца'рствующую въ вjь'къ вjь'ка.

%<Сла'ва, и= ны'нjь, бг~оро'диченъ, догма'тiкъ. Гла'съ 
г~:%>

%<П%>реве'лiе чу'до: дв~а ро'ждшая, и= рожде'нное бг~ъ 
пре'жде вjь^къ, пред\ъявле'нное рожде'нiе, и= 
соверше'нное па'че _е=стества`. _w та'инства стра'шнагw! 
_е='же и= мы'слимое неизрече'нно пребыва'етъ, и= зри'мое 
не пое'млется. блаже'нна ты` _е=си` пречи'стая 
_о=трокови'це, а=да'ма земна'гw дщи`, и= бг~а вы'шнягw 
jа='вльшаяся мт~и: того` моли`, сп~сти'ся душа'мъ 
на'шымъ.

%<Та'же, С%>вjь'те ти'хiй: П%<осе'мъ прокi'менъ: 
Г%>д\сь воцр~и'ся: %<три'жды. Стi'хъ: W=%>блече'ся гд\сь 
въ си'лу, и= препоя'сася. Та'же, (c. 346) %<С%>подо'би 
гд\си въ ве'черъ се'й: %<I=ере'й же _е=ктенiи` не 
глаго'летъ, но пое'мъ на стiхо'внjь стiхи'ру воскр\сну, 
гла'съ г~:%>

%<С%>тр\стiю твое'ю хр\сте`, w=мрачи'вый со'лнце, и= 
свjь'томъ твоегw` воскр\снiя просвjьти'вый вся'ч_еская, 
прiими` на'шу вече'рнюю пjь'снь чл~вjьколю'бче.

%<Та'же г~ бг~оро'дичны подо'бны. Гла'съ г~.%>

%<Стi'хъ: П%>омяну` и='мя твое` во вся'комъ ро'дjь и= 
ро'дjь.

%<К%>овче'гъ тя` о_у='мный ч\стая вjь'мы, нося'щiй 
скрижа^ли бг~опи^санныя, на'мъ же законоположи'теля и= 
созда'теля: _е=му'же моли'ся сп~сти'ся душа'мъ на'шымъ.

%<Стi'хъ: С%>лы'ши дщи` и= ви'ждь, и= приклони` 
о_у='хо твое`.

%<Н%>а земли' тя нб~о ч\стая показа`, и='же во 
о_у=тро'бjь твое'й всели'выйся бг~ъ сло'во, пло'ть 
прiи'мъ, и='же _е=стество'мъ незри'мый: и= дожди'ши 
всjь^мъ безсме'ртную ро'су неискусобра'чная.

%<Стi'хъ: Л%>ицу` твоему` помо'лятся бога'тiи 
лю'дстiи.

%<С%>п~се'нiе мл~твами твои'ми и='мамы, и=`же тjь'ми 
дв~о w\т бjь'дъ при'снw и= и=скуше'нiй бу'ри 
и=збjьга'емъ: и= сегw` ра'ди про'симъ непреста'ннw w= 
на'съ _е='же моли'ти, сп~сти'ся душа'мъ на'шымъ.

%<Сла'ва, и= ны'нjь, бг~оро'диченъ.%> %<Гла'съ г~:%>

%<П%>а'ки назда'нiе и= живо'тъ тобо'ю вторы'й вjь'мы 
преч\стая мр~i'е чл~вjь'ческагw _е=стества`, во чре'вjь 
твое'мъ смjьси'вшагося созда'теля тва'ри, и= 
воскреси'вшаго на'съ w\т а='да и= сме'рти, и= жи'знь 
вjь'чную пода'вшаго на'мъ, _е='же вопи'ти тебjь` 
приснодв~о: ра'дуйся, совокупи'вшая дw'льная нб\снымъ. 
ра'дуйся, наде'ждо всjь'хъ конце'й земли`, и= 
предста'тельство и= заступле'нiе. ра'дуйся, jа='же 
воскр\снiемъ сн~а твоегw` просвjьти'вши вся'ч_еская, и= 
подаю'щи мi'рови ве'лiю мл\сть.

%<Та'же, Н%>ы'нjь w\тпуща'еши: Т%<рист~о'е. И= по 
_О='%>ч~е на'шъ: тропа'рь воскр\снъ. %<Сла'ва, и= ны'нjь, 
бг~оро'диченъ _е=гw`, и= w\тпу'стъ. %>(с. 347)%<%>
