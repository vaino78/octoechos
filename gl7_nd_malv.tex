%[Въ суббw'ту ве'чера,  на ма'лjьй вече'рни%],

на Гд\си воззва'хъ, поста'вимъ стiхw'въ д~: и= пое'мъ 
г~ воскр\сны, повторя'юще а~-й. Стiхи^ры воскр\сны, 
гла'съ з~:

Стi'хъ: W\т стра'жи о_у='треннiя до но'щи, w\т стра'жи 
о_у='треннiя да о_у=пова'етъ i=и~ль на гд\са.

Прiиди'те, возра'дуемся гд\севи, сокруши'вшему сме'рти 
держа'ву, и= просвjьти'вшему человjь'ческiй ро'дъ, со 
безпло'тными зову'ще: содjь'телю и= сп~се на'шъ, сла'ва 
тебjь`.

Кр\стъ претерпjь'лъ _е=си` сп~се, и= погребе'нiе на'съ 
ра'ди, сме'ртiю же, jа='кw бг~ъ сме'рть о_у=мертви'лъ 
_е=си`. тjь'мже покланя'емся тридне'вному воскр\снiю 
твоему`: гд\си, сла'ва тебjь`.

А=п\сли ви'дjьвше воскр\снiе содjь'теля, чудя'хуся, 
пою'ще хвалу` а='гг~льскую: сiя` сла'ва _е='сть 
цр~ко'вная, сiе` бога'тство цр\ствiя, пострада'вый на'съ 
ра'ди, гд\си, сла'ва тебjь`.

Сла'ва, и= ны'нjь, бг~оро'диченъ, догма'тiкъ, гла'съ 
з~:

Стра'шно и= неизрече'нно вои'стинну, _е='же w= тебjь` 
содjь'янное та'инство нескве'рная: сло'во бо всjь'хъ 
вино'вное, па'че вины` и= сло'ва, ст~ы'мъ дх~омъ 
вопло'щшееся родила` _е=си`, и=з\ъ себе` пло'ть прiи'мша, 
своегw` _е=стества` непрело'жнw пребы'вша. соше'дшымся бо 
_о=бои'мъ самобы'тнjь, по v=поста'си _е=ди'нственной, 
сугу'бъ _е=стество'мъ происхо'дитъ: ве'сь бг~ъ, и= ве'сь 
чл~вjь'къ, во _о=бои'хъ соверше'нствiихъ, 
дjьйстви'тельными сво'йствы и=з\ъявля'я: пострада'въ бо 
на кр\стjь` пло'тiю, безстра'стный то'йже пребы'сть 
бж~е'ственнjь: jа='кw человjь'къ о_у=мры'й, па'ки w=живе` 
jа='кw бг~ъ тридне'венъ, держа'ву сме'ртную низложи'въ, 
и=з\ъ и=стлjь'нiя и=зба'вль человjь'чество. того` jа='кw 
и=зба'вителя и= сп~са ро'да на'шегw бг~омт~и моли`, 
низпосла'ти на'мъ щедро'тъ _е=гw` ве'лiю мл\сть.

Та'же, Свjь'те ти'хiй: Посе'мъ прокi'менъ: Гд\сь 
воцр~и'ся три'жды. Стi'хъ: W=блече'ся гд\сь въ си'лу, и= 
препоя'сася.

Та'же, Сподо'би гд\си, въ ве'черъ се'й: I=ере'й же 
_е=ктенiи` не глаго'летъ, но пое'мъ на стiхо'внjь 
стiхи'ру воскр\сну, гла'съ з~:

Воскр\снъ _е=си` и=з\ъ гро'ба сп~се мi'ра, и= 
совоздви'глъ _е=си` человjь'ки съ пло'тiю твое'ю: гд\си, 
сла'ва тебjь`.

И='ны стiхи^ры прест~jь'й бц\дjь, гла'съ з~:

Подо'бенъ: Неради'вше w= вре'менныхъ:

Стi'хъ: Помяну` и='мя твое` во вся'комъ ро'дjь и= 
ро'дjь.

Восто'къ сл~нца мы'сленнагw дв~о была` _е=си`, на 
за'падjьхъ _е='же по на'мъ бы'вшагw _е=стества`: но 
jа='кw и=му'щи дерзнове'нiе, того` моли` бц\де 
всепjь'тая, w\т безмjь'рныхъ прегрjьше'нiй свободи'ти 
ду'шы на'шя.

Стi'хъ: Слы'ши дщи` и= ви'ждь, и= приклони` о_у='хо 
твое`.

Же'злъ и=з\ъ ко'рене i=ессе'ева дв~о, jа='вjь 
прозябла` _е=си`, насажд_е'нiя пре'лести и=з\ъ ко'рене 
потре'бльшаго: но jа='кw и=му'щи дерзнове'нiе, 
непреста'ннw моли`, и=скорени'ти всепjь'тая, се'рдца 
моегw` стра^сти, и= тогw` всади'ти стра'хъ, и= сп~сти' 
мя.

Стi'хъ: Лицу` твоему` помо'лятся бога'тiи лю'дстiи.

Врата` бж~iя, прест~а'я, w\т вра'тъ а='довыхъ и=зба'ви 
мя`, и= пу'ть покая'нiя покажи' ми, и='мже w=бря'щу 
врата` къ жи'зни ввwдя'щая: заблу'ждшихъ наста'внице, 
ро'дъ вjь'рныхъ человjь^къ соблюда'й, и= сп~са'й ду'шы 
на'шя.

Сла'ва, и= ны'нjь, бг~оро'диченъ: И=з\ъ тебе` 
всест~а'я бц\де дв~о, неизрече'ннw роди'ся хр\сто'съ бг~ъ 
на'шъ, вои'стинну сы'й бг~ъ превjь'чный, и= чл~вjь'къ 
но'въ. _о='во о_у='бw сы'й, присносу'щенъ: _о='во же 
на'съ ра'ди бы'въ: сп~са'етъ бо собо'ю, коегw'ждо 
_е=стества` сво'йство, _о='вjьмъ о_у='бw сiя'я чудесы`, 
_о='вjьмъ же о_у=вjьря'я страстьми`, тjь'мже _е=ди'нъ и= 
то'йже, и= о_у=мира'етъ jа='кw человjь'къ, и= jа='кw бг~ъ 
востае'тъ. _е=го'же моли` ч\стая, неискусобра'чная, 
сп~сти'ся душа'мъ на'шымъ.

Та'же, Ны'нjь w\тпуща'еши: Трист~о'е, и= по _О='ч~е 
на'шъ: Тропа'рь воскр\снъ. Сла'ва, и= ны'нjь, 
бг~оро'диченъ _е=гw`, и= w\тпу'стъ.
