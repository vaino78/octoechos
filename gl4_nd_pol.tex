%<%[Въ недjь'лю о_у='тра на полу'нощницjь%],%>

%<канw'нъ тр\оченъ, [_е=гw'же краестро'чiе: 
Ч%>етве'ртое пjь'нiе бг~у. %<Творе'нiе митрофа'ново.] 
Гла'съ д~.%>

%<Пjь'снь а~%>

%<I=рмо'съ: %>Мо'ря чермну'ю пучи'ну, невла'жными 
стопа'ми, дре'внiй пjьшеше'ствовавъ i=и~ль, 
кр\стоwбра'зныма мwv"се'овыма рука'ма, а=мали'кову си'лу 
въ пусты'ни побjьди'лъ _е='сть.

%<Т%>р\оцу бг~онача'льную да просла'вимъ v=поста'сьми, 
_е=ди'нственное же _е=стество` трiе'хъ, соприсносу'щную, 
сопр\сто'льную, ю='же моля'ще, глаго'лемъ: сп~си` и=`же 
вjь'рою тебе` сла'вящихъ.

%<П%>ома'зася w\т _о=ц~а` дх~омъ ра'дованiя, 
бг~одjь'тельнымъ _е=ле'емъ, сн~ъ, и= чл~вjь'къ бы'сть, и= 
_е=ди'нагw бж~ества` трiv"поста'сное научи'лъ _е='сть.

%<Сла'ва: Д%>обро'ту непристу'пныя сла'вы твоея` 
_е=ди'нице трисо'лнечная, серафi'ми не терпя'ще зрjь'ти, 
спокрыва'ются кри'лы: и= трист~ы'ми пjь'сньми 
непреста'ннw тебе` сла'вятъ.

%<Бг~оро'диченъ: Н%>еизрече'ннw творца` родила` _е=си` 
всjь'хъ преч\стая, и=збавля'ющаго дре'внiя кля'твы 
человjь'ки, и= сме'ртныя тли`, и= тобо'ю позна'хомъ 
_е=ди'наго бг~а трiv"поста'снаго.

%<%[Пjь'снь г~%]%>

%<I=рмо'съ: %>Не му'дростiю и= си'лою и= бога'тствомъ 
хва'лимся, но тобо'ю _о='ч~ею v=поста'сною му'дростiю 
хр\сте`: нjь'сть бо ст~ъ, па'че тебе` чл~вjьколю'бче.

%<С%>и'лу свы'ше ст~ы^мъ твои^мъ пре'жде а=п\слwмъ, 
jа='кw посла'лъ _е=си` хр\сте`, w\т _о=ц~а` 
о_у=тjь'шителя, _е=ди'но jа=ви'лъ _е=си` _е=стество` 
трисо'лнечное. (с. 509)

%<П%>атрiа'рху а=враа'му _е=гда` jа=ви'лася _е=си` во 
w='бразjь му'жестjь, тр\очня _е=ди'нице, непремjь'нное 
показа'ла _е=си` твоея` бл~гости и= гд\сьства.

%<Сла'ва: И='%>же во w='бразjьхъ трiе'хъ, _е=ди'нъ 
бг~ъ вjь'руемый: неwпи'санный jа='вjь, недомы'слимый 
всjь'ми, и=зба'ви ду'шы на'шя w\т вся'кiя ско'рби.

%<Бг~оро'диченъ: Н%>аста'вльшеся сн~а твоегw` 
прему'дрыми приведе'ньми, _е=ди'нственное и= трисвjь'тлое 
бг~онача'лiе сла'вимъ, и= тебе` бл~жи'мъ приснодв~у.

%<Та'же сjьда'ленъ, гла'съ д~. Подо'бенъ: С%>ко'рw 
предвари`:

%<Т%>рисо'лнечная, несозда'нная и= _е=диносу'щная 
_е=ди'нице, трiv"поста'сная и= непостижи'мая, рабы^ твоя^ 
о_у=ще'дри: сп~си` w\т бjь'дъ, jа='кw бг~ъ мл\стивъ, тя' 
бо гд\си, _е=ди'наго и=зба'вителя и= вл\дку и='мамы, 
вопiю'ще: бу'ди на'мъ мл\стивъ.

%<Сла'ва, и= ны'нjь, бг~оро'диченъ:%> Мно'гими 
w=бстоя'ньми и= напа'стьми лю'тыхъ дв~о, w=кружа'еми, и= 
ко w\тча'янiю при'снw впа'дающе, _е=ди'ну тя` сп~се'нiе 
и= наде'жду, и= стjь'ну и='мамы бц\де, и= тебе` по до'лгу 
вjь'рою и= ны'нjь мо'лимъ: сп~си` рабы^ твоя^.

%<%[Пjь'снь д~%]%>

%<I=рмо'съ: %>Сjьдя'й въ сла'вjь на пр\сто'лjь 
бж~ества`, во _о='блацjь ле'гцjь прiи'де i=и~съ 
пребж~е'ственный, нетлjь'нною дла'нiю, и= сп~се` 
зову'щыя: сла'ва хр\сте` си'лjь твое'й.

%<П%>ресу'щную тр\оцу, во _е=ди'ницjь бж~ества` и= 
гд\сонача'лiе, съ серафi'мы тебе` сла'вимъ, jа='кw 
нераздjь'льно _е=стество`, jа='кw непристу'пное, jа='кw 
равноста'тно сла'вою, бж~е непостижи'мый.

%<Р%>аздjьле'нну су'щу неизрече'ннw ли'цы бж~ества`, 
и= соединя'ему держа'вою вку'пjь _е=ди'нjьмъ гд\сьствомъ, 
безпредjь'льну _е=ди'ну, неwпи'санну, воспjьва'емъ тя` 
творца` всея` тва'ри. (с. 510)

%<Сла'ва: О_у='%>мъ безнача'льный, сло'во 
неизглаго'ланнjь роди`, и= бж~е'ственнаго дх~а 
равномо'щна и=спусти`: и= сегw` ра'ди тр\оцу 
_е=диносу'щную, вл\дку всjь'хъ бг~а проповjь'даемъ.

%<Бг~оро'диченъ: В%>и'димь быва'я дре'вними 
w=бра'знjь, предвозвjьсти'ло _е='сть, _е='же w\т тебе` 
воплоще'нiе, сло'во: но по'слjьжде jа='влься 
человjь'кwмъ, пои'стиннjь трiv"поста'сное _е=динонача'лiе 
jа=ви`.

%<%[Пjь'снь _е~%]%>

%<I=рмо'съ: О_у=%>жасо'шася вся'ч_еская w= 
бж~е'ственнjьй сла'вjь твое'й: ты' бо неискусобра'чная 
дв~о, и=мjь'ла _е=си` во о_у=тро'бjь над\ъ всjь'ми бг~а, 
и= родила` _е=си` безлjь'тнаго сн~а, всjь^мъ 
воспjьва'ющымъ тя` ми'ръ подава'ющая.

%<Р%>азумjь'вше w\т вjь'ры вседjь'тельнагw бж~ества`, 
_е=ди'но о_у='бw непристу'пно существо`, три' же 
v=поста'си живонача^льны, сра'слены чте'мъ: _о=ц~а`, и= 
сн~а, и= дх~а ст~а'гw соприсносу'щное бытiе`.

%<С%>вjь'те трисо'лнечне, су'щественнагw свjь'та 
твоегw` возсiя'й ми` _е=ди'нственное бж~ество`, 
несозда'нное _е=стество`, и= свjьтодjь'йственный 
и=сто'чниче вся'кiя свjьтода'тельныя зари`: да созерца'ю 
твою` добро'ту неизрече'нную.

%<Сла'ва: Jа='%>кw _е=ди'ному су'щу содjь'телю 
вся'ческихъ, и= содержи'телю, и= ко'рмчiю всепрему'дру, 
вои'стинну и= жи'зни пода'телю, сегw` ра'ди и= вопiе'мъ 
ти` вjь'рнw: вл\дко трисо'лнечне, пою'щыя тя` соблюди`.

%<Бг~оро'диченъ: W=%>божи'ти хотя` дре'вле 
и=стлjь'вшаго человjь'ка, за бл~гость дв~о созда'вый, и= 
показа'вый w='браза бж~е'ственный зра'къ, чл~вjь'къ 
бы'сть и=з\ъ тебе`, _е=ди'но тричи'сленное бг~онача'лiе 
проповjь'да.

%<%[Пjь'снь s~%]%>

%<I=рмо'съ: %>Возопи`, проwбразу'я погребе'нiе 
тридне'вное, пр\оро'къ i=w'на въ ки'тjь моля'ся: w\т тли` 
и=зба'ви мя`, i=и~се цр~ю` си'лъ. (с.  511)

%<Jа=%>ви` _о=ц~ъ и=зглаго'луя сн~овство`, и= дх~ъ, 
хр\сту` кре'щшуся, ви'димь бы'въ: сегw` ра'ди _е=ди'но и= 
тр\оческое бг~онача'лiе сла'вимъ.

%<Jа='%>кw ви'дjь тя` трист~ы'ми гла'сы воспjьва'емаго 
и=са'iа, на высо'цjь пр\сто'лjь сjьдя'ща, тр\оческую 
позна` _е=ди'нагw бг~онача'лiя v=поста'сь.

%<Сла'ва: В%>озвыше'но се'рдце покажи` и= на'съ ра^бъ 
твои'хъ, высо'кiй цр~ю` трiv"поста'сне: да твоея` сла'вы 
зри'мъ jа='снw свjь'тлость.

%<Бг~оро'диченъ: В%>осхотjь` воwбрази'тися jа='вjь въ 
на'ше, w\т дв~ы сн~ъ бж~iй jа='кw чл~вjьколю'бецъ, и= 
бж~е'ственныя сла'вы _о='бщники человjь'ки сотвори`.

%<Сjьда'ленъ, гла'съ д~. Подо'бенъ: С%>ко'рw 
предвари`:

%<_О=%>ц~а` нерожде'нна, сн~а же рожде'нна, и= дх~а 
ст~а'го и=схо'дна w\т _о=ц~а` му'дрствующе, 
проповjь'даемъ безнача'льное цр\ство и= бж~ество` 
_е=ди'но, _е='же славосло'вяще _е=диному'дреннw вопiе'мъ: 
тр\оце _е=диносу'щная, сп~си` на'съ бж~е.

%<Сла'ва, и= ны'нjь, бг~оро'диченъ:%> Лjь'тъ 
превы'шше, и= пре'жде вjь^къ бг~а, въ лjь'то родила` 
_е=си` преесте'ственнjь пло'тiю, бг~а чл~вjь'ка 
преч\стая. тjь'мже тя` бц\ду, и='стиннw и= гд\сьственнjь 
вси` и=сповjь'дающе, прилjь'жнw ти` вопiе'мъ: сла'вы 
вjь'чныя вся^ сподо'би.

%<%[Пjь'снь з~%]%>

%<I=рмо'съ: %>Въ пещи` а=враа'мстiи _о='троцы 
персi'дстjьй, любо'вiю бл~гоче'стiя па'че, не'жели 
пла'менемъ w=паля'еми, взыва'ху: бл~гослове'нъ _е=си` въ 
хра'мjь сла'вы твоея` гд\си.

%<О_у=%>чин_е'ная нб\сная _е=ст_ества`, и= о_у='мныя 
чи'ны правосла'внw вси` земноро'днiи подража'юще, 
сла'вимъ _е=ди'но бж~ество` въ трiе'хъ равнодjь'тельныхъ 
v=поста'сехъ. %<[Два'жды.]%>

%<Сла'ва: Р%>jь^чи ст~ы'хъ пр\орw'къ, тя` дре'вле 
w=бра'знw _е=ди'наго вjькw'въ всjь'хъ содjь'теля 
прояви'ша, неизрече'ннаго бг~а и= гд\са, бг~онача'льными 
треми` v=поста'сьми. (с. 512)

%<Бг~оро'диченъ: И='%>же по существу` неви'димое 
сло'во и= вседjь'тельное, jа=ви'лся _е=си` человjь'кwмъ, 
чл~вjь'къ w\т ч\стыя бг~омт~ре, человjь'ка призыва'я ко 
прича'стiю твоегw` бж~ества`.

%<%[Пjь'снь и~%]%>

%<I=рмо'съ: %>Ру'цjь распросте'ръ данiи'лъ, львw'въ 
зiя^нiя въ ро'вjь затче`: _о='гненную же си'лу 
о_у=гаси'ша, добродjь'телiю препоя'савшеся, бл~гоче'стiя 
рачи'тели _о='троцы, взыва'юще: бл~гослови'те вся^ дjьла` 
гд\сня гд\са.

%<С%>вjь'те _е=динонача'льный и= трисiя'нный, 
существо` безнача'льное, добро'то недовjь'домая, въ 
се'рдцы мое'мъ всели'ся, и= хра'мъ твоегw` бж~ества`, 
свjьтови'денъ и= чи'стъ покажи' мя, зову'ща: 
бл~гослови'те вся^ дjьла` гд\сня гд\са, по'йте и= 
превозноси'те _е=го` во вjь'ки.

%<Сла'ва: W\т%> разли'чныхъ мя` страсте'й тр\оце 
нераздjь'льная, _е=ди'нице несли'тная, и= w=мраче'нiя 
прегрjьше'нiй и=зба'ви, и= w=зари` луча'ми твои'ми 
бж~е'ственными: да воwбразу'ю твою` сла'ву, и= воспjьва'ю 
тя` сла'вы гд\са.

%<Бг~оро'диченъ: О_у='%>мъ о_у='бw нерожде'нный 
_о=ц~ъ, и= сло'во соwбра'зно, и= дх~ъ сопресто'ленъ, 
существо`, си'ла, бытiе`: пресу'щная, неизрече'нная, 
великодjь'йственная тр\оце, _е=ди'нице, соблюда'й ста'до 
твое` мл~твами бц\ды, jа='кw _е=стество'мъ 
чл~вjьколю'бецъ.

%<%[Пjь'снь f~%]%>

%<I=рмо'съ: %>Вся'къ земноро'дный да взыгра'ется 
дх~омъ просвjьща'емь, да торжеству'етъ же безпло'тныхъ 
о_у=мw'въ _е=стество`, почита'ющее сщ~е'нное торжество` 
бг~ома'тере, и= да вопiе'тъ: ра'дуйся всебл~же'нная 
бц\де, ч\стая приснодв~о.

%<В%>се` къ тебjь` ны'нjь дви'жу се'рдце мое` и= 
мы'сль, и= предлож_е'нiя же всея` души` и= тjь'ла, 
содjь'телю и= и=зба'вителю моему`, _е=динонача'льне, и= 
трисвjь'тле, и= вопiю' ти: сп~си' мя раба` твоего` w\т 
вся'кихъ и=скуше'нiй и= скорбе'й. %<[Два'жды.] %>(с. 513)

%<Сла'ва: В%>озвы'си на'шъ о_у='мъ, и= мы'сль къ 
тебjь` вы'шнему, просвjьти` твои'ми сiя'ньми преч\стыми, 
_о='ч~е, сло'ве, о_у=тjь'шителю, во свjь'тjь живы'й 
непристу'пнjьмъ, сла'вы сл~нце, свjьтоде'ржче, всегда` 
сла'вити тя` _е=динонача'льнаго бг~а трiv"поста'снаго.

%<Бг~оро'диченъ: С%>п~си` и=`же въ тя` вjь'рующыя 
гд\си, и= проповjь'дающыя безнача'льное присносу'щное 
существо` _е=ди'но, три' же ли'ца бг~онача^льна и= 
соwбра'зна, твоегw` гд\сьствiя, и= бж~е'ственныя сла'вы 
твоея` сподо'би, мольба'ми ч\стыя бг~ома'тере.

%<Посе'мъ припjь'вы григо'рiа сiнаи'та: Д%>осто'йно 
_е='сть: %<И= про'чее полу'нощницы.%>