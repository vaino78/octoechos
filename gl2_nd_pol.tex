%<%[Въ недjь'лю на полу'нощницjь%],%>

%<канw'нъ тр\очный, творе'нiе митрофа'ново, _е=гw'же 
краестро'чiе: Т%>ригу'бый пою` бг~онача'лiя свjь'тъ. 
%<Гла'съ в~.%>

%<%[Пjь'снь а~%]%>

%<I=рмо'съ: В%>о глубинjь` постла` и=ногда`, 
фараwни'тское всево'инство преwруже'нная си'ла, 
вопло'щшееся же сло'во всеsло'бный грjь'хъ потреби'ло 
_е='сть, препросла'вленный гд\сь, сла'внw бо просла'вися.

%<Т%>р\очное и= _е=динонача'льное _е=стество` 
бж~ества`, пjь'сненнw воспои'мъ глаго'люще: мл\сти 
пучи'ну неизчерпа'емую, суще'ственную jа='кw и=му'щи, 
тебjь` кла'няющихся соблюди` и= сп~си`, jа='кw 
чл~вjьколю'бецъ.

%<И='%>же и=сто'чникъ, и= ко'рень _о=ц~ъ сы'й, jа='кw 
вино'венъ, и='же въ сн~jь, и= ст~jь'мъ твое'мъ ду'сjь, 
сра'сленнагw бж~ества` трисо'лнечный се'рдцу моему` 
и=сточи` свjь'тъ, и= прича'стiемъ w=сiя'й, 
бг~одjь'тельнагw сiя'нiя.

%<Сла'ва: Т%>рисвjь'тлая _е=ди'нице бг~онача'льная, 
ве'сь разори` грjьхw'въ и= страсте'й мои'хъ мра'къ, 
свjь'тлыхъ луче'й твои'хъ сладча'йшими причаще'ньми, и= 
сотвори' мя твоея` непристу'пныя сла'вы хра'мъ и= сjь'нь 
преч\стую.

%<Бг~оро'диченъ: Т%>о'къ дре'внiй _е=стества` на'шегw, 
страда'вшiй безмjь'стнw, и= къ тли` попо'лзшiйся 
преч\стая, вопло'щься во о_у=тро'бjь твое'й бг~ъ сло'во, 
человjьколю'бнjь w=сiя`, и= бг~онача'лiю трисвjь'тлому 
на'съ та'йнw научи`.

%<%[Пjь'снь г~%]%>

%<I=рмо'съ: Н%>а ка'мени мя` вjь'ры о_у=тверди'въ, 
разшири'лъ _е=си` о_у=ста` моя^ на враги` моя^, 
возвесели' бо ся ду'хъ мо'й, внегда` пjь'ти: нjь'сть 
ст~ъ, jа='коже бг~ъ на'шъ, и= нjь'сть пра'веденъ, па'че 
тебе` гд\си.

%<Р%>а'венствомъ _е=стества` бг~онача'лiе 
_е=диноче'стное сла'влю тя` ли^цы: живо'тъ бо w\т живота` 
ты` произше'дъ нетлjь'ннw, _е=ди'нъ сы'й бг~ъ на'шъ, и= 
нjь'сть ст~ъ, па'че тебе` гд\си. (с. 206)

%<Т%>ы` чи'ны невеще'ств_енныя и= нб\сныя соста'вилъ 
_е=си`, jа='кw зерца^ла твоея` добро'ты: тр\оце 
нераздjь'льное _е=динонача'лiе, пjь'ти непреста'ннw 
тебjь`: но и= ны'нjь на'ше w\т бре'нныхъ о_у='стъ прiими` 
хвале'нiе.

%<Сла'ва: О_у=%>тверди` на ка'мени вjь'ры, и= разшири` 
любве` твоея` пучи'ною сердца` и= мы'сль твои'хъ ра^бъ, 
_е=ди'нице трисо'лнечная: ты' бо бг~ъ на'шъ, на него'же 
о_у=пова'юще, да не посрами'мся.

%<Бг~оро'диченъ: И='%>же вся'къ пре'жде соста'въ 
w=существова'вый тва'ри, во о_у=тро'бjь твое'й 
w=существова'ся, неизче'тною бл~гостiю бц\де, и= свjь'тъ 
трисо'лнечный всjь^мъ возсiя` _е=ди'нагw бж~ества` и= 
гд\сьства.

%<Сjьда'ленъ, гла'съ в~.%>

%<Подо'бенъ: Б%>л~гоутро'бiя:

%<_Е=%>гда` въ нача'лjь а=да'ма созда'лъ _е=си` гд\си, 
тогда` сло'ву твоему` v=поста'сному возопи'лъ _е=си` 
бл~гоутро'бне: сотвори'мъ по на'шему подо'бiю, дх~ъ же 
ст~ы'й сопрису'тствоваше содjь'тель. тjь'мже вопiе'мъ 
ти`: тво'рче бж~е на'шъ, сла'ва тебjь`.

%<Сла'ва, и= ны'нjь, бг~оро'диченъ: _Е=%>гда` къ на'мъ 
бг~ъ прiити` и=зво'ли, тогда` въ твою` преч\стая, 
ч\стjь'йшую о_у=тро'бу всели'ся, и= сп~се` тобо'ю 
человjь'ческое смjьше'нiе, дарова'вый всjь^мъ цр\ство 
нб\сное. тjь'мже вопiе'мъ ти`, бц\де ч\стая, ра'дуйся 
вл\дчце.

%<%[Пjь'снь д~%]%>

%<I=рмо'съ: П%>ою' тя, слу'хомъ бо гд\си о_у=слы'шахъ 
и= о_у=жасо'хся, до мене' бо и='деши, мене` и=щя` 
заблу'ждшаго. тjь'мъ мно'гое твое` снизхожде'нiе, _е='же 
на мя`, прославля'ю многомл\стиве.

%<Р%>азумjь'ти тя` ниже` чи'нове мо'гутъ 
невеще'ственнiи а='гг~льстiи, _е=ди'нице, тр\оце 
безнача'льная: но о_у='бw мы` бре'ннымъ я=зы'комъ твою` 
су'щественную бл~гость воспjьва'емъ, и= вjь'рою сла'вимъ. 
(с. 207)

%<С%>ы'й созда'тель _е=стества` человjь'ческагw, 
вседержи'телю, все` мое` ви'диши ны'нjь, jа='кw 
всеви'децъ неможе'нiе: тjь'мже о_у=ще'дри раба` твоего`, 
и= къ жи'зни лу'чшей па'ки возведи`.

%<Сла'ва: _Е=%>ди'ницы нача'льныя несмjь'сна три` 
ли'ца воспjьва'емъ, jа='кw сво'йственнjь и=му'щая, и= 
раздjь'льнjь v=поста^си: но о_у='бw соедине'на и= 
нераздjь'льна, въ совjь'тjь же, и= сла'вjь, и= 
бж~ествjь`.

%<Бг~оро'диченъ: Х%>ра'мъ тя` чи'стъ и= 
пренепоро'ченъ, приснодв~о бц\де, вседjь'тель w=брjь'те 
_е=ди'ну jа='вjь w\т вjь'ка, въ ню'же все'лься: воwбрази` 
человjь'ческое _е=стество`, jа='кw чл~вjьколю'бецъ.

%<%[Пjь'снь _е~%]%>

%<I=рмо'съ: П%>росвjьще'нiе во тьмjь` лежа'щихъ, 
сп~се'нiе w\тча'янныхъ хр\сте` сп~се мо'й, къ тебjь` 
о_у='тренюю цр~ю` ми'ра, просвjьти' мя сiя'нiемъ твои'мъ: 
и=но'гw бо ра'звjь тебе` бг~а не зна'ю.

%<Jа='%>кw вся'чески на вся^ су'щыя твоегw` про'мысла, 
мирода^рныя простира'яй лучы`, и= сп~си'т_ельныя, цр~ю` 
ми'ра, соблюди' мя въ ми'рjь твое'мъ: ты' бо живо'тъ и= 
ми'ръ вся'ческому.

%<М%>wv"се'ю въ купинjь` jа='кw jа=ви'лся _е=си` въ 
видjь'нiи _о='гненнjь, а='гг~лъ наре'клся _е=си` _о='ч~ее 
сло'во, _е='же къ на'мъ твое` пред\ъявля'я прише'ствiе: 
и='мже всjь^мъ jа='вjь возвjьсти'лъ _е=си` держа'ву 
бг~онача'лiя _е=ди'нагw, трiv"поста'сную.

%<Сла'ва: Jа='%>же _е=сте'ственную, соприсносу'щную 
сла'ву предло'жиши, _е=динонача'льнjьйшая тр\оце ст~а'я, 
воспjьва'ющихъ тя` правосла'вною вjь'рою, твоея` сла'вы 
ви'дjьти сподо'би, безнача'льную, и= _е=ди'ну зарю` 
трисл~нечную.

%<Бг~оро'диченъ: С%>одержи'тельный по существу` сы'й 
бг~ъ сло'во, всjь^мъ вjькw'мъ, дв~о мт~и, во чре'вjь 
твое'мъ о_у=держа'ся неизрече'ннw, человjь'ки призыва'я 
ко _е=ди'нству _е=ди'нагw гд\сьства. (с. 208)

%<%[Пjь'снь s~%]%>

%<I=рмо'съ: В%>ъ бе'зднjь грjьхо'внjьй валя'яся, 
неизслjь'дную милосе'рдiя твоегw` призыва'ю бе'здну: w\т 
тли` бж~е мя` возведи`.

%<В%>оли'телю мл\сти, поми'луй въ тя` вjь'рующихъ, 
бж~е трисо'лнечне, и= прегрjьше'нiй и=зба'ви, и= 
страсте'й, и= бjь'дъ, рабы^ твоя^. %<[Два'жды.]%>

%<Сла'ва: Н%>еизрече'нною пучи'ною бл~гости, 
неwбмы'слимую твоегw` сiя'нiя, и= трисiя'ннагw бж~ества` 
свjьтода'тельную зарю` пода'ждь ми`.

%<Бг~оро'диченъ: Н%>еизрече'ннjь дв~о, вы'шнiй 
чл~вjь'къ бы'сть и=з\ъ тебе`, въ чл~вjь'ка по всему` 
w=бо'лкся, и= свjь'томъ мя` трисл~нчнымъ w=зари`.

%<Сjьда'ленъ, гла'съ в~. Подо'бенъ: Б%>л~гоутро'бiя:

%<Б%>л~гоутро'бiя пучи'ну на'мъ простры'й, прiими` 
на'съ мл\стиве, при'зри на лю'ди тя` сла'вящыя, прiими` 
пjь'снь прося'щихъ тя`, тр\оце _е=ди'нице безнача'льная: 
на тя' бо о_у=пова'емъ всjь'хъ бг~а, прегрjьше'нiй да'ти 
проще'нiе.

%<Сла'ва, и= ны'нjь, бг~оро'диченъ: Б%>л~гоутро'бiя 
ро'ждши и=сто'чникъ, мл\стива ты` _е=си` бл~га'я бц\де: 
ты' бо вjь'рныхъ _е=ди'но заступле'нiе, ты` скорбя'щихъ 
о_у=тjьше'нiе. тjь'мже тебjь` ны'нjь вси` вjь'рою 
припа'даемъ, w=брjьсти` разрjьше'нiе лю'тыхъ, 
w=богатjь'ющеся _е=ди'ною тебе` по'мощiю.

%<%[Пjь'снь з~%]%>

%<I=рмо'съ: W='%>бразу злато'му на по'лjь деи'рjь 
служи'му, трiе` твои` _о='троцы небрего'ша безбо'жнагw 
велjь'нiя: посредjь' же _о=гня` вве'ржени, w=роша'еми 
поя'ху: бл~гослове'нъ _е=си` бж~е _о=т_е'цъ на'шихъ.

%<О_у=%>ставля'еши при'снw а='гг~льская вw'инства къ 
непреложе'нiю, _е=ди'не сы'й неизмjь'нне и= 
трiv"поста'сне гд\си: покажи` о_у='бw и= мое` се'рдце 
непрело'жно всегда`, во _е='же сла'вити тя` те'плjь, и= 
воспjьва'ти бл~гоче'стнw. %<[Два'жды.]%>

%<Сла'ва: Л%>и'цы о_у='мнiи невеще'ственныхъ 
суще'ствъ, твои'ми луча'ми бж~е _е=динонача'льне, и= 
трисо'лнечне, w=заря'еми (с. 209) быва'ютъ, положе'нiемъ 
вторi'и свjь'тове: и='хже и= мене` сiя'ньми, и= 
прича'стiемъ покажи` свjь'тъ, jа='кw свjьтодjь'тель 
трисiя'ненъ.

%<Бг~оро'диченъ: Н%>аправля'ти на'съ и= возвыша'ти къ 
нб~с_е'мъ не w=скудjь'й, тебе` лю'бящихъ, и='же за 
неизрече'нное чл~вjьколю'бiе, бы'въ чл~вjь'къ во 
о_у=тро'бjь дв~ы, и= w=божи'въ человjь'ка, и= на 
пр\сто'лjь сла'вы со _о=ц~е'мъ сjьдя'й.

%<%[Пjь'снь и~%]%>

%<I=рмо'съ: W=%> подо'бiи зла'тjь небре'гше 
требл~же'ннiи ю='нwши, неизмjь'нный и= живы'й бж~iй 
w='бразъ ви'дjьвше, среди` _о=гня` воспjьва'ху: 
w=существова'нная да пое'тъ гд\са вся` тва'рь, и= 
превозно'ситъ во вся^ вjь'ки.

%<Н%>епристу'пная тр\оце соприсносу'щная, 
собезнача'льная, бг~онача'льная, неизмjь'нная во всjь'хъ, 
кромjь` свjьтоно'сныхъ сво'йствъ, вся'кiй лука'вый 
о_у=праздни` сопротиволежа'щихъ совjь'тъ, и= стуже'нiя 
де'мwнwвъ, невре'дна соблюда'яй мя` при'снw, гд\си 
всjь'хъ. %<[Два'жды.]%>

%<Сла'ва: П%>рему'дрjь и= всемо'щнjь, неwпи'санное, 
трисл~нчное _е=динонача'лiе соста'вльшее мi'ръ, и= 
соблюда'ющее въ невреди'момъ чи'нjь всесоверше'нномъ, 
всели'ся въ мое` се'рдце, пjь'ти и= сла'вити тя` 
немо'лчнw съ ли'ки а='гг~льскими во вся^ вjь'ки.

%<И= ны'нjь: П%>рему'дросте _о='ч~ая, непостижи'ме, 
неизрече'нне, бж~iй сло'ве, непрело'жное твое` 
_е=стество` не и=змjьни'въ, _е=стество` человjь'ческое 
мл\стивнjь воспрiя'лъ _е=си`: и= _е=ди'нственную тр\оцу 
чести` всjь'хъ научи'лъ _е=си`, jа='кw гд\сонача'лiе 
всjь'хъ вjькw'въ.

%<%[Пjь'снь f~%]%>

%<I=рмо'съ: Jа='%>же пре'жде сл~нца свjьти'льника бг~а 
возсiя'вшаго, пло'тски къ на'мъ прише'дшаго, и=з\ъ боку` 
дв~и'чу неизрече'ннw воплоти'вшая, бл~гослове'нная 
всеч\стая, тя` бц\де велича'емъ. (с. 210)

%<W\т%> свjь'та безнача'льна, собезнача'ленъ сн~ъ 
свjь'тъ возсiя`, и= соесте'ственный свjь'тъ дх~ъ и=зы'де, 
неизрече'ннw бг~олjь'пнw, нетлjь'нну рж\ству` 
о_у=вjьря'еми, вку'пjь же и= неизрече'нному и=схожде'нiю.

%<В%>озсiя'й въ сердца'хъ трисл~нчное бж~ество`, 
воспjьва'ющихъ тя`, трисiя'ннымъ свjь'томъ твои'мъ: и= 
да'ждь ра'зумъ _е='же во всjь'хъ разумjь'ти, и= дjь'яти 
твое` хотjь'нiе бл~го'е и= соверше'нное, и= велича'ти и= 
сла'вити тя`.

%<Сла'ва: Н%>еизче'тенъ _е=стество'мъ сы'й jа='кw 
бг~ъ, неизче'тную пучи'ну щедро'тъ jа='кw и=мjь'я, 
о_у=ще'дрила _е=си` тр\оце пре'жде: та'кw и= ны'нjь 
о_у=ще'дри рабы^ твоя^, и= w\т прегрjьше'нiй и=зба'ви, и= 
напа'стей и= w=бстоя'нiй.

%<Бг~оро'диченъ: С%>п~си' мя бж~е мо'й, w\т вся'кагw 
и=скуше'нiя и= w=sлобле'нiя, и='же въ трiе'хъ ли'цjьхъ 
воспjьва'емый несказа'ннw, _е=ди'нственнjь бг~ъ и= 
всеси'льный, и= твое` ста'до сохрани`, бц\ды мл~твами.

%<Та'же тр\очны григо'рiа сiнаи'та. Д%>осто'йно 
_е='сть: %<И= про'чее полу'нощницы, пи'сано въ концjь` 
кни'ги сея`.%>
