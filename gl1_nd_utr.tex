%<%[На о_у='трени по шесто_пса'лмiи%]%>

%<Б%>г~ъ гд\сь, %<на гла'съ а~ и= глаго'лемъ тропа'рь 
воскр\сный, два'жды, и= бг~оро'диченъ _е=ди'ножды, 
пи'саны на вели'цjьй вече'рни. Та'же _о=бы'чное 
стiхосло'вiе _псалти'ря.%>

%<По а~-мъ стiхосло'вiи сjьда'льны воскр\сны, гла'съ 
а~:%>

%<Г%>ро'бъ тво'й сп~се, во'ини стрегу'щiи, ме'ртвiи w\т 
w=блиста'нiя jа='вльшагwся а='гг~ла бы'ша, проповjь'дающа 
жена'мъ воскр~се'нiе. тебе` сла'вимъ тли` потреби'теля, 
тебjь` припа'даемъ воскр\сшему и=з\ъ гро'ба, и= 
_е=ди'ному бг~у на'шему.

%<Стi'хъ: В%>оскр\сни` гд\си бж~е мо'й, да вознесе'тся 
рука` твоя`, не забу'ди о_у=бо'гихъ твои'хъ до конца`.

%<К%>о кр\сту` пригво'ждься во'лею ще'дре, во гро'бjь 
положе'нъ бы'въ jа='кw ме'ртвъ животода'вче, держа'ву 
сте'рлъ _е=си` си'льне сме'ртiю твое'ю: тебе' бо 
вострепета'ша вра'тницы а='дwвы, ты` совоздви'глъ _е=си` 
w\т вjь'ка о_у=ме'ршыя, jа='кw _е=ди'нъ чл~вjьколю'бецъ.

%<Сла'ва, и= ны'нjь, бг~оро'диченъ: М%>т~рь тя` бж~iю 
свjь'мы вси` дв~у вои'стинну, и= по рж\ствjь` 
jа='вльшуюся, и=`же любо'вiю прибjьга'ющiи къ твое'й 
бл~гости: тебе' бо и='мамы грjь'шнiи предста'тельство, 
тебе` стяжа'хомъ въ напа'стехъ сп~се'нiе, _е=ди'ну 
всенепоро'чную.

%<По в~-мъ стiхосло'вiи, сjьда'ленъ, гла'съ а~.%>

%<Подо'бенъ: К%>а'мени запеча'тану:

%<Ж%>_ены` ко гро'бу прiидо'ша о_у=ра'ншя, и= 
а='гг~льское jа=вле'нiе ви'дjьвшя трепета'ху: гро'бъ 
w=блиста` жи'знь, (с. 40) чу'до о_у=дивля'ше я=`: сегw` 
ра'ди ше'дшя о_у=ч~нкw'мъ проповjь'даху воста'нiе: а='дъ 
плjьни` хр\сто'съ, jа='кw _е=ди'нъ крjь'покъ и= си'ленъ, 
и= и=стлjь'вшыя вся^ совоздви'же, w=сужде'нiя стра'хъ 
разруши'въ кр\сто'мъ.

%<Стi'хъ: И=%>сповjь'мся тебе` гд\си всjь'мъ се'рдцемъ 
мои'мъ, повjь'мъ вся^ чудеса` твоя^.

%<Н%>а кр\стjь` пригвозди'лся _е=си` животе` всjь'хъ, и= 
въ ме'ртвыхъ вмjьни'лся _е=си` безсме'ртный гд\си, 
воскр~слъ _е=си` тридне'венъ сп~се, совоздви'гъ а=да'ма 
w\т тлjь'нiя. сегw` ра'ди си^лы нб\сныя вопiя'ху тебjь`, 
жизнода'вче хр\сте`: сла'ва воскр\снiю твоему`, сла'ва 
снизхожде'нiю твоему`, _е=ди'не чл~вjьколю'бче.

%<Сла'ва, и= ны'нjь, бг~оро'диченъ: М%>р~i'е, ч\стно'е 
вл\дки прiя'телище, воскр~си' ны па'дшыя въ про'пасть 
лю'тагw w\тча'янiя, и= прегрjьше'нiй и= скорбе'й, ты' бо 
_е=си` грjь^шнымъ сп~се'нiе и= по'мощь, и= крjь'пкое 
предста'тельство, и= сп~са'еши рабы^ твоя^.

%<Та'же, Б%>л~же'ни непоро'чнiи: %<Посе'мъ тропари`: 
А='%>гг~льскiй собо'ръ: %<пи'саны въ концjь` кни'ги сея`. 
Та'же _е=ктенiа` ма'лая, и=%>

%<V=пакои`, гла'съ а~:%>

%<Р%>азбо'йничо покая'нiе ра'й w=кра'де, пла'чь же 
мv"роно'сицъ ра'дость возвjьсти`, jа='кw воскр\слъ _е=си` 
хр\сте` бж~е, подая'й мi'рови ве'лiю мл\сть.

%<Степ_е'нны: %[а=нтiфw'нъ%] а~, гла'съ а~, стiхи` 
повторя'юще:%>

%<В%>негда` скорбjь'ти ми`, о_у=слы'ши моя^ болjь^зни, 
гд\си тебjь` зову`.

%<П%>усты^ннымъ непреста'нное бж~е'ственное жела'нiе 
быва'етъ, @мi'ра су'щымъ су'етнагw кромjь`@{@су'щымъ 
внjь` су'етнагw мi'ра@}.

%<Сла'ва: С%>т~о'му дх~у че'сть и= сла'ва, jа='коже 
_о=ц~у` подоба'етъ, ку'пнw же и= сн~у, сегw` ра'ди да 
пое'мъ тр\оцjь _е=динодержа'вiе.

%<И= ны'нjь, то'йже. %>(л. 41)

%<%[А=нтiфw'нъ%] в~:%>

%<Н%>а го'ры твои'хъ возне'слъ _е=си` мя` зако'нwвъ, 
добродjь'тельми просвjьти` бж~е, да пою' тя.

%<Д%>есно'ю твое'ю руко'ю прiи'мъ ты` сло'ве, сохрани' 
мя, соблюди`, да не _о='гнь мене` w=пали'тъ грjьхо'вный.

%<Сла'ва: С%>т~ы'мъ дх~омъ вся'кая тва'рь w=бновля'ется, 
па'ки теку'щи на пе'рвое: равномо'щенъ бо _е='сть _о=ц~у` 
и= сло'ву.

%<И= ны'нjь, то'йже.%>

%<%[А=нтiфw'нъ%] г~:%>

%<W=%> ре'кшихъ мнjь`, вни'демъ во дворы` гд\сни, 
возвесели'ся мо'й ду'хъ, сра'дуется се'рдце.

%<В%>ъ дому` дв~довjь стра'хъ вели'къ: та'мw во 
пр\сто'лwмъ поста'вл_еннымъ, су'дятся вся^ племена` 
земна^я, и= jа=зы'цы.

%<Сла'ва: С%>т~о'му дх~у, че'сть, поклоне'нiе, сла'ву и= 
держа'ву, jа='коже _о=ц~у` досто'итъ, и= сн~ови 
подоба'етъ приноси'ти: _е=ди'ница бо _е='сть тр\оца 
_е=стество'мъ, но не ли'цы.

%<И= ны'нjь, то'йже.%>

%<Прокi'менъ, гла'съ а~:%>

%<Н%>ы'нjь воскр\сну`, глаго'летъ гд\сь, положу'ся во 
сп~се'нiе, не w=биню'ся w= не'мъ.

%<Стi'хъ: С%>ловеса` гд\сня, словеса` чи'ста.

%<В%>ся'кое дыха'нiе: %<_Е=v\глiе о_у='треннее 
рядово'е.%>

%<В%>оскресе'нiе хр\сто'во ви'дjьвше, поклони'мся ст~о'му 
гд\су i=и~су, _е=ди'ному безгрjь'шному. кр\сту` твоему` 
покланя'емся хр\сте`, и= ст~о'е воскресе'нiе твое` пое'мъ 
и= сла'вимъ: ты' бо _е=си` бг~ъ на'шъ, ра'звjь тебе` 
и=но'гw не зна'емъ, и='мя твое` и=мену'емъ. прiиди'те 
вси` вjь'рнiи, поклони'мся ст~о'му хр\сто'ву 
воскресе'нiю: се' бо прiи'де кр\сто'мъ ра'дость всему` 
мi'ру. всегда` бл~гословя'ще гд\са, пое'мъ воскресе'нiе 
_е=гw`: распя'тiе бо претерпjь'въ, сме'ртiю сме'рть 
разруши`.

%<_Псало'мъ н~: П%>оми'луй мя` бж~е: 

%<Сла'ва: М%>л~твами а=п\слwвъ, мл\стиве, w=чи'сти 
мно'ж_ества согрjьше'нiй на'шихъ. (л. 42)

%<И= ны'нjь: М%>л~твами бц\ды, мл\стиве, w=чи'сти 
мно'ж_ества согрjьше'нiй на'шихъ.

%<Та'же, гла'съ s~: П%>оми'луй мя` бж~е по вели'цjьй 
мл\сти твое'й, и= по мно'жеству щедро'тъ твои'хъ w=чи'сти 
беззако'нiе мое`.

%<Посе'мъ стiхи'ра: В%>оскре'съ i=и~съ w\т гро'ба, 
jа='коже прорече`, даде` на'мъ живо'тъ вjь'чный и= ве'лiю 
ми'лость.

%<С%>п~си` бж~е лю'ди твоя^:

%<И= возгла'съ: М%>л\стiю и= щедро'тами и= 
чл~вjьколю'бiемъ:

%<Канw'ны: воскр\сный на д~: кр\стовоскр\сный на г~: и= 
бг~оро'диченъ на г~: мине'и на д~. А='ще же пра'зднуется 
ст~ы'й, на s~: кр\стовоскр\сный на в~, и= бц\ды на в~.%>

%<Канw'нъ воскр\сный, гла'съ а~.%>

%<%[Пjь'снь а~%]%>

%<I=рмо'съ: Т%>воя` побjьди'тельная десни'ца бг~олjь'пнw 
въ крjь'пости просла'вися: та' бо безсме'ртне, jа='кw 
всемогу'щая, проти^вныя сотре`, i=и~льтянwмъ пу'ть 
глубины` новосодjь'лавшая.

%<Припjь'въ: С%>ла'ва гд\си ст~о'му воскр\снiю твоему`.

%<Тропа'рь: И='%>же рука'ма пречи'стыма w\т пе'рсти 
бг~одjь'тельнjь и=спе'рва созда'въ мя`; ру'цjь 
распросте'рлъ _е=си` на кр\стjь`, w\т земли` взыва'я 
тлjь'нное мое` тjь'ло, _е='же w\т дв~ы прiя'лъ _е=си`.

%<О_у=%>мерщвле'нiе под\ъя'лъ _е=си` мене` ра'ди, и= 
ду'шу сме'рти пре'далъ _е=си`, и='же вдохнове'нiемъ 
бж~е'ственнымъ ду'шу ми` вложи'вый, и= w\трjьши'въ 
вjь'чныхъ о_у='зъ, и= совоскр~си'въ нетлjь'нiемъ 
просла'вилъ _е=си`.

%<Бг~оро'диченъ: Р%>а'дуйся бл~года'ти и=сто'чниче. 
ра'дуйся лjь'ствице, и= две'ре нб\сная, ра'дуйся 
свjь'щниче, и= ру'чко злата'я, и= горо` несjько'мая, 
jа='же жизнода'вца хр\ста` мi'рови ро'ждшая.

%<И='нъ канw'нъ, кр\стовоскр\сный.%>

%<Пjь'снь а~, гла'съ то'йже. I=рмо'съ: Х%>р\сто'съ 
ражда'ется:

%<Х%>р\сто'съ w=божа'етъ мя` воплоща'яся, хр\сто'съ мя` 
возно'ситъ смиря'яся, хр\сто'съ безстра'стна мя` 
содjь'ловаетъ, (с. 43) стражда` жизнода'вецъ 
_е=стество'мъ пло'ти. тjь'мже воспjьва'ю 
бл~года'рственную пjь'снь: jа='кw просла'вися.

%<Х%>р\сто'съ возно'ситъ мя` распина'емь, хр\сто'съ 
совоскреша'етъ мя` о_у=мерщвля'емь, хр\сто'съ жи'знь 
мнjь` да'руетъ. тjь'мже съ весе'лiемъ рука'ма плеща'я, 
пою` сп~си'телю побjь'дную пjь'снь: jа='кw просла'вися.

%<Бг~оро'диченъ: Б%>г~а дв~о зачала` _е=си`, хр\ста' же 
въ дв~ствjь родила` _е=си` и=з\ъ тебе` вопло'щшася 
преч\стая, _е=ди'наго v=поста'сiю _е=диноро'днаго, во 
двою' же сущ_еству` познава'емаго сн~а: jа='кw 
просла'вися.

%<И='нъ канw'нъ, прест~jь'й бц\дjь.%>

%<Пjь'снь а~, гла'съ то'йже. I=рмо'съ: Т%>воя` 
побjьди'тельная:

%<К%>у'ю ти` досто'йную пjь'снь на'ше принесе'тъ 
неможе'нiе; то'чiю w=бра'довательную, _е='йже на'съ 
гаврiи'лъ та'йнw научи'лъ _е='сть: ра'дуйся бц\де дв~о, 
мт~и неневjь'стная.

%<П%>риснодв~jь и= мт~ри цр~я` вы'шнихъ си'лъ, w\т 
чистjь'йша се'рдца вjь'рнiи духо'внjь возопiи'мъ: 
ра'дуйся бц\де дв~о, мт~и неневjь'стная.

%<Б%>езмjь'рная бе'здна твоегw` непостижи'магw рж\ства` 
всеч\стая, вjь'рою несумнjь'нною о_у='бw чи'стjь 
прино'симъ ти` глаго'люще: ра'дуйся бц\де дв~о, мт~и 
неневjь'стная.

%<Та'же мине'и. Катава'сiа: W\т%>ве'рзу о_у=ста` моя^:

%<%[Пjь'снь г~%]%>

%<I=рмо'съ: _Е=%>ди'не вjь'дый человjь'ческагw существа` 
не'мощь, и= ми'лостивнw въ не` воwбра'жся, препоя'ши мя` 
съ высоты` си'лою, _е='же вопи'ти тебjь` ст~ы'й: 
w=душевле'нный хра'ме неизрече'нныя сла'вы твоея` 
чл~вjьколю'бче.

%<Б%>г~ъ сы'й мнjь` бл~же, па'дшаго о_у=ще'дрилъ _е=си`, 
и= сни'ти ко мнjь` бл~говоли'въ, возне'слъ мя` _е=си` 
распя'тiемъ, _е='же вопи'ти тебjь` @ст~ы'й: хра'ме 
w=душевле'нный неизрече'нныя твоея` сла'вы 
чл~вjьколю'бче@{@ст~ъ и='же сла'вы гд\сь неизмjь'нный 
бл~гостiю@}. (с. 44)

%<Ж%>иво'тъ v=поста'сный хр\сте` сы'й, въ и=стлjь'вша 
мя`, jа='кw милосе'рдый бг~ъ w=бо'лкся, въ пе'рсть 
сме'ртную соше'дъ вл\дко, сме'ртную держа'ву разруши'лъ 
_е=си`, и= ме'ртвъ тридне'венъ воскр~съ, въ нетлjь'нiе 
мя` w=бле'клъ _е=си`.

%<Бг~оро'диченъ: Б%>г~а заче'нши во чре'вjь дв~о, дх~омъ 
прест~ы'мъ, пребыла` _е=си` неwпали'ма, поне'же тя` 
купина` законополо'жнику мwv"се'ю, пали'мую нежего'мw, 
jа='вjь предвозвjьсти`, _о='гнь нестерпи'мый прiе'мшую.

%<%[И='нъ%] I=рмо'съ: П%>ре'жде вjь^къ w\т _о=ц~а`:

%<И='%>же на свое` ра'мо заблужда'емое _о=вча` взе'мшему, 
и= низложи'вшему дре'вомъ _е=гw` грjь'хъ, хр\сту` бг~у 
возопiи'мъ: воздви'гнувый ро'гъ на'шъ, ст~ъ _е=си` гд\си.

%<В%>озве'дшему па'стыря вели'каго и=з\ъ а='да хр\ста`, 
и= сщ~еннонача'лiемъ _е=гw` а=п\слы jа='вjь jа=зы'ки 
о_у=па'сшему, и='стиною и= бж~е'ственнымъ вjь'рнiи дх~омъ 
да послу'жимъ.

%<И='%>же w\т дв~ы воплоти'вшемуся без\ъ сjь'мене во'лею 
сн~у, и= ро'ждшую по рж\ствjь`, бж~е'ственною си'лою 
ч\стую дв~у сохра'няему, и='же над\ъ всjь'ми бг~у 
возопiи'мъ: ст~ъ _е=си` гд\си.

%<%[И='нъ%] I=рмо'съ: _Е=%>ди'не вjь'дый:

%<_О='%>блакъ тя` ле'гкiй нело'жнw дв~о и=мену'емъ, 
проро'ч_ескимъ возслjь'дующе рече'нi_емъ: прiи'де бо на 
тебjь` гд\сь низложи'ти _е=гv'петскiя пре'лести 
рукотвор_е'нiя, и= просвjьти'ти си^мъ служа'щыя.

%<Т%>я` запеча'танный вои'стинну ли'къ пр\оро'ческiй 
и=сто'чникъ, и= заключе'нную две'рь и=менова`, дjь'вства 
твоегw` всепjь'тая, jа='вственнjь зна'м_енiя на'мъ 
пи'шуще: _е='же сохрани'ла _е=си` и= по рж\ствjь`.

%<О_у=%>ма` пресу'щественна ви'дjьти, jа='коже мо'щно, 
сподо'блься гаврiи'лъ, дв~о всенепоро'чная, ра'достный 
тебjь` гла'съ принесе`, сло'ва зача'тiе jа='вственнjь 
возвjьща'ющiй, и= неизрече'нное рж\ство` проповjь'дающiй. 
(с. 45)

%<%[Пjь'снь д~%]%>

%<I=рмо'съ: %>Го'ру тя` бл~года'тiю бж~iею 
прiwсjьне'нную, прозорли'выма а=вваку'мъ о_у=смотри'въ 
_о=чи'ма, и=з\ъ тебе` и=зы'ти i=и~леву провозглаша'ше 
ст~о'му, во сп~се'нiе на'ше и= w=бновле'нiе.

%<К%>то` се'й сп~съ и='же и=з\ъ _е=дw'ма и=сходя`, 
вjьне'цъ нося` терно'венъ, w=червле'нну ри'зу и=мы'й, на 
дре'вjь ви'ся; i=и~левъ _е='сть се'й ст~ы'й, во сп~се'нiе 
на'ше и= w=бновле'нiе.

%<В%>и'дите лю'дiе непокори'вiи, и= о_у=стыди'теся: 
_е=го'же бо jа='кw sлодjь'я вы` вознести` на кр\стъ о_у= 
пiла'та и=спроси'сте о_у=мовре'днjь, сме'рти разруши'въ 
си'лу, бг~олjь'пнw воскр~съ и=з\ъ гро'ба.

%<Бг~оро'диченъ: Д%>ре'во тя` дв~о жи'зни вjь'мы: не бо` 
снjь'ди пло'дъ смертоно'сный человjь'кwмъ и=з\ъ тебе` 
прозябе`, но живота` присносу'щнагw наслажде'нiе, во 
сп~се'нiе на'съ пою'щихъ тя`.

%<%[И='нъ%] I=рмо'съ: Ж%>е'злъ и=з\ъ ко'рене i=ессе'ова:

%<К%>то` се'й красе'нъ и=з\ъ _е=дw'ма, и= сегw` 
w=червле'нiе ри'зное, w\т вiногра'да восо'рска, красе'нъ 
jа='кw бг~ъ, jа='кw человjь'къ же, кро'вiю пло'ти ри'зу 
w=червле'ну нося`; _е=му'же пое'мъ вjь'рнiи: сла'ва 
си'лjь твое'й гд\си.

%<Х%>р\сто'съ бу'дущихъ бла^гъ jа='влься а=рхiере'й, 
грjь'хъ на'шъ разори'лъ _е='сть: и= показа'въ стра'ненъ 
пу'ть свое'ю кро'вiю, въ лу'чшую и= соверше'ннjьйшую 
вни'де ски'нiю, пр\дте'ча на'шъ во ст~а^я.

%<Бг~оро'диченъ: _Е='%>vинъ дре'внiй до'лгъ и=спроси'ла 
_е=си` всепjь'тая о_у= и='же на'съ ра'ди jа='вльшагwся 
но'вагw а=да'ма. соедини'въ бо себjь` чи'стымъ зача'тiемъ 
пло'ть о_у='мную, w=душевле'нную, и=з\ъ тебе` произы'де 
хр\сто'съ, _е=ди'нъ во _о=бою` гд\сь.

%<%[И='нъ%] I=рмо'съ: Г%>о'ру тя` бл~года'тiю:

%<С%>лы'ши чуде'съ нб~о, и= внуша'й земле`, jа='кw дщи` 
пе'рстнагw о_у='бw па'дшагw а=да'ма, бг~у нарече'на 
бы'сть, и= (с. 46) своему` содjь'телю роди'тельница, на 
сп~се'нiе на'ше и= w=бновле'нiе.

%<П%>ое'мъ вели'кое и= стра'шное твое` та'инство, 
преме'рныхъ бо о_у=таи'вся чинонача'лiй, на тя` и='же 
сы'й сни'де jа='кw до'ждь на руно`, всепjь'тая, на 
сп~се'нiе на'съ пою'щихъ тя`.

%<С%>т~ы'хъ ст~а'я бц\де всепjь'тая, ча'янiе jа=зы'кwвъ, 
и= сп~се'нiе вjь'рныхъ, и=з\ъ тебе` возсiя` и=зба'витель 
и= жизнода'вецъ, и= гд\сь: _е=го'же моли`, сп~сти'ся 
рабw'мъ твои^мъ.

%<%[Пjь'снь _е~%]%>

%<I=рмо'съ: %>Просвjьти'вый сiя'нiемъ прише'ствiя твоегw` 
хр\сте`, и= w=свjьти'вый кр\сто'мъ твои'мъ мi'ра концы`, 
сердца` просвjьти` свjь'томъ твоегw` бг~оразу'мiя, 
правосла'внw пою'щихъ тя`.

%<П%>а'стыря _о=вца'мъ вели'каго и= гд\са, i=уд_е'и 
дре'вомъ кр\стнымъ о_у=мертви'ша: но то'й jа='кw _о='вцы, 
м_е'ртвыя во а='дjь погреб_е'нныя, держа'вы сме'ртныя 
и=зба'ви.

%<К%>р\сто'мъ твои'мъ ми'ръ бл~говjьсти'въ, и= 
проповjь'давъ плjь^ннымъ сп~се мо'й w=ставле'нiе, 
держа'ву и=му'щаго посрами'лъ _е=си` хр\сте` на'га, 
w=бнища'вша показа'вый бж~е'ственнымъ воста'нiемъ 
твои'мъ.

%<Бг~оро'диченъ: П%>роше'нiя вjь'рнw прося'щихъ, 
всепjь'тая, не пре'зри: но прiими`, и= сiя^ доноша'й сн~у 
твоему` преч\стая, бг~у _е=ди'ному бл~годjь'телю, тебе' 
бо предста'тельницу стяжа'хомъ.

%<%[И='нъ%] I=рмо'съ: Б%>г~ъ сы'й ми'ра:

%<_W%> бога'тство, и= глубино` прему'дрости бж~iя! 
Прему^дрыя w=б\ъе'мляй гд\сь, w\т си'хъ кова'рства 
и=зба'вилъ _е='сть на'съ: пострада'въ бо во'лею не'мощiю 
плотско'ю, свое'ю крjь'постiю, животворя'й м_е'ртвыя 
воскр~си'лъ _е='сть.

%<Б%>г~ъ сы'й соединя'ется пло'ти на'съ ра'ди: и= 
распина'ется, и= о_у=мира'етъ: погреба'ется, и= па'ки 
воскр~са'етъ, и= восхо'дитъ свjь'тлw съ пло'тiю свое'ю 
хр\сто'съ ко _о=ц~у`: съ не'юже прiи'детъ, и= сп~се'тъ 
бл~гоче'стнw тому` служа'щыя. (с. 47)

%<Бг~оро'диченъ: С%>т~ы'хъ ст~а'я дв~о ч\стая, ст~ы'хъ 
ст~а'го родила` _е=си`, всjь'хъ w=свяща'ющаго хр\ста` 
и=зба'вителя. тjь'мже тя` цр~и'цу, и= вл\дчцу всjь'хъ 
jа='кw мт~рь зижди'теля тва'рей проповjь'дуемъ.

%<%[И='нъ%] I=рмо'съ: П%>росвjьти'вый сiя'нiемъ:

%<В%>еселя'тся нб\сныя си^лы зря'ще тя`: ра'дуются съ 
ни'ми человjь'кwвъ собра^нiя: рж\ство'мъ бо твои'мъ 
совокупи'шася, дв~о бц\де, _е='же досто'йнw сла'вимъ.

%<Д%>а дви'жатся вси` я=зы'цы человjь'честiи и= мы^сли, 
къ похвалjь` человjь'ческагw вои'стинну о_у=добре'нiя, 
дв~а предстои'тъ jа='вjь сла'вящи, вjь'рою тоя` пою'щихъ 
чудеса`.

%<С%>ла'вится пjь'нiе всепрему'дрыхъ и= похвала`, дв~jь 
и= мт~ри бж~iи приноси'мая: сла'вы бо бы'сть сiя` хра'мъ 
пребж~е'ственныя, ю='же досто'йнw сла'вимъ.

%<%[Пjь'снь s~%]%>

%<I=рмо'съ: W=%>бы'де на'съ послjь'дняя бе'здна, нjь'сть 
и=збавля'яй, вмjьни'хомся jа='кw _о='вцы заколе'нiя, 
сп~си` лю'ди твоя^, бж~е на'шъ: ты' бо крjь'пость 
немощству'ющихъ и= и=справле'нiе.

%<С%>огрjьше'нiемъ первозда'ннагw гд\си, лю'тjь 
о_у=язви'хомся, ра'ною же и=сцjьли'хомся твое'ю, _е='юже 
за ны` о_у=язви'лся _е=си` хр\сте`: ты' бо крjь'пость 
немощству'ющихъ и= и=справле'нiе.

%<В%>озве'лъ ны` _е=си` и=з\ъ а='да гд\си, ки'та 
о_у=би'въ всея'дца, всеси'льне, твое'ю держа'вою 
низложи'въ тогw` си'лу: ты' бо живо'тъ, и= свjь'тъ 
_е=си`, и= воскр\снiе.

%<Бг~оро'диченъ: В%>еселя'тся w= тебjь` дв~о преч\стая, 
ро'да на'шегw пра'_отцы, _е=де'мъ воспрiе'мше тобо'ю, 
_е=го'же преступле'нiемъ погуби'ша: ты' бо ч\стая, и= 
пре'жде рж\ства`, и= по рж\ствjь` _е=си`.

%<%[И='нъ%] I=рмо'съ: И=%>з\ъ о_у=тро'бы i=w'ну:

%<О_у='%>мъ сы'й безстра'стенъ и= невеще'ственъ, 
примjьша'ется хр\сто'съ бг~ъ человjь'ческому о_у=му`, 
хода'тайствующему бж~е'ственнымъ _е=стество'мъ, пло'ти же 
дебельство'мъ, и= всему` (с. 48) мнjь` непрело'женъ 
весьма` соедини'ся: да сп~се'нiе всему` мнjь` па'дшему 
пода'стъ распина'емь.

%<П%>а'даетъ прельсти'вся а=да'мъ, и= @запя'выйся 
сокруша'ется@{@запя'вся а=да'мъ и= сокруша'ется@}, 
наде'ждою w=бо'лганъ сы'й дре'вле w=боже'нiя: но 
востае'тъ соедине'нiемъ сло'ва w=божа'емь, и= стра'стiю 
безстра'стiе прiе'млетъ, на пр\сто'лjь jа='кw сн~ъ 
сла'вится, сjьдя'й со _о=ц~е'мъ же и= дх~омъ.

%<Бг~оро'диченъ: Н%>jь'дръ не w\тсту'пль безнача'льна 
роди'теля, въ нjь'дрjьхъ ч\стыя _о=трокови'цы 
водворя'ется, и= быва'етъ, и='же пре'жде безма'теренъ, 
без\ъ _о=ц~а` воплоща'емый, и='же пра'вдою цр\ствуяй 
бг~ъ: сегw` неродосло'венъ стра'шный ро'дъ и= 
неизрече'ненъ.

%<%[И='нъ%] I=рмо'съ: W=%>бы'де на'съ:

%<П%>редстоя'тъ раболjь'пнjь рж\ству` твоему` чи'ни 
нб\снiи, дивя'щеся досто'йнw твоему` безсjь'менному 
рождеству` приснодв~о: ты' бо ч\стая, и= пре'жде 
рж\ства`, и= по рж\ствjь` _е=си`.

%<В%>оплоти'ся пре'жде сы'й безпло'тенъ, сло'во и=з\ъ 
тебе` преч\стая, вся'ч_еская во'лею творя'й, 
безтjьле'сныхъ вw'инства приведы'й w\т небытiя` jа='кw 
всеси'ленъ.

%<О_у=%>мерщвле'нъ бы'сть вра'гъ живоно'снымъ твои'мъ 
плодо'мъ бг~облагода'тная: и= попра'нъ бы'сть а='дъ 
проявле'ннjь, и= и=`же во о_у='захъ свободи'хомся. 
тjь'мже вопiю`: стра^сти разруши` се'рдца моегw`.

%<Конда'къ, гла'съ а~: Подо'бенъ: _Е=%>гда` прiи'деши:

%<В%>оскр\слъ _е=си` jа='кw бг~ъ и=з\ъ гро'ба во сла'вjь, 
и= мi'ръ совоскр~си'лъ _е=си`, и= _е=стество` 
человjь'ческое jа='кw бг~а воспjьва'етъ тя`, и= сме'рть 
и=счезе`: а=да'мъ же лику'етъ вл\дко, _е='vа ны'нjь w\т 
о_у='зъ и=збавля'ема ра'дуется зову'щи: ты` _е=си`, и='же 
всjь^мъ подая` хр\сте` воскр\снiе.

%<I='косъ: В%>оскр\сшаго тридне'внw воспои'мъ jа='кw бг~а 
всеси'льна, и= врата` а='дwва сте'ршаго, и= jа='же w\т 
вjь'ка и=з\ъ гро'ба воздви'гшаго, мv"роно'сицамъ 
jа='вльшагося, jа='коже благоизво'лилъ _е='сть, пре'жде 
си^мъ _е='же ра'дуйтеся, рекi'й: и= (с. 49)  а=п\слwмъ 
ра'дость возвjьща'я, jа='кw _е=ди'нъ жизнода'вецъ. 
тjь'мже вjь'рою ж_ены` о_у=ч~нкw'мъ зна'менiя побjь'ды 
бл~говjьству'ютъ, и= а='дъ стене'тъ, и= сме'рть рыда'етъ: 
мi'ръ же весели'тся. и= вси` съ ни'мъ ра'дуются. ты' бо 
по'далъ _е=си` хр\сте` всjь^мъ воскр\снiе.

%<%[Пjь'снь з~%]%>

%<I=рмо'съ: Т%>ебе` о_у='мную бц\де, пе'щь разсмотря'емъ 
вjь'рнiи: jа='коже бо _о='троки сп~се` три` 
превозноси'мый, мi'ръ w=бнови` во чре'вjь твое'мъ 
всецjь'лъ, хва'льный _о=тц_е'въ бг~ъ, и= препросла'вленъ.

%<О_у=%>боя'ся земля`, сокры'ся со'лнце, и= поме'рче 
свjь'тъ, раздра'ся цр~ко'вная бж~е'ственная завjь'са, 
ка'менiе же разсjь'деся: на кр\стjь' бо ви'ситъ прв\дный, 
хва'льный _о=тц_е'въ бг~ъ, и= препросла'вленъ.

%<Т%>ы` бы'въ а='ки безпомо'щенъ, и= о_у=я'звенъ въ 
ме'ртвыхъ во'лею на'съ ра'ди превозноси'мый, вся^ 
свободи'лъ _е=си`, и= держа'вною руко'ю совоскр~си'лъ 
_е=си`, хва'льный _о=тц_е'въ бг~ъ, и= препросла'вленъ.

%<Бг~оро'диченъ: Р%>а'дуйся, и=сто'чниче присноживы'я 
воды`. ра'дуйся, раю` пи'щный. ра'дуйся, стjьно` 
вjь'рныхъ. ра'дуйся неискусобра'чная. ра'дуйся всемi'рная 
ра'досте, _е='юже на'мъ возсiя` хва'льный _о=тц_е'въ бг~ъ 
и= препросла'вленъ.

%<%[И='нъ%] I=рмо'съ: _О='%>троцы бл~гоче'стiю:

%<Д%>ре'вле о_у='бw проклята` бы'сть земля` а='велевою 
w=червлени'вшися кро'вiю, братоубi'йственною руко'ю: 
бг~ото'чною же твое'ю кро'вiю бл~гослови'ся w=кропле'на, 
и= взыгра'ющи вопiе'тъ: _о=тц_е'въ бж~е, бл~гослове'нъ 
_е=си`.

%<Д%>а рыда'ютъ i=уде'йстiи бг~опроти'внiи лю'дiе, 
де'рзости о_у=бiе'нiя хр\сто'ва: jа=зы'цы же да 
веселя'тся, и= рука'ми да воспле'щутъ, и= вопiю'тъ: 
_о=тц_е'въ бж~е бл~гослове'нъ _е=си`.

%<С%>е` мv"роно'сицамъ w=блиста'яй, вопiя'ше а='гг~лъ: 
воскр\снiя хр\сто'ва прiиди'те и= ви'дите зна'м_енiя, 
плащани'цу и= гро'бъ, и= возопi'йте: _о=тц_е'въ бж~е 
бл~гослове'нъ _е=си`. (с. 50)

%<%[И='нъ%] I=рмо'съ: Т%>ебе` о_у='мную:

%<Т%>я` бц\де лjь'ствицу i=а'кwвъ пр\оро'чески 
разумjьва'етъ: тобо'ю бо превозноси'мый на земли` 
jа=ви'ся, и= съ человjь'ки поживе`, jа='кw бл~говоли`, 
хва'льный _о=тц_е'въ бг~ъ и= препросла'вленъ.

%<Р%>а'дуйся ч\стая, и=з\ъ тебе` про'йде па'стырь, и='же 
во а=да'мову ко'жу w=бо'лкся вои'стинну, превозноси'мый, 
во всего' мя человjь'ка, за бл~гоутро'бiе непости'жное: 
хва'льный _о=тц_е'въ бг~ъ и= препросла'вленъ.

%<Н%>о'вый а=да'мъ w\т чи'стыхъ крове'й твои'хъ 
превjь'чный бг~ъ бы'сть вои'стинну, _е=го'же ны'нjь 
моли`, w=бетша'вшаго мя` w=бнови'ти зову'ща: хва'льный 
_о=тц_е'въ бг~ъ и= препросла'вленъ.

%<%[Пjь'снь и~%]%>

%<I=рмо'съ: %>Въ пещи` _о='троцы i=и~левы, jа='коже въ 
горни'лjь добро'тою бл~гоче'стiя, чистjь'е зла'та 
блеща'хуся, глаго'люще: бл~гослови'те вся^ дjьла` гд\сня 
гд\са, по'йте и= превозноси'те во вся^ вjь'ки.

%<И='%>же во'лею вся^ творя'й, и= претворя'яй, w=браща'яй 
сjь'нь сме'ртную въ вjь'чную жи'знь, стр\стiю твое'ю 
сло'ве бж~iй, тебе` непреста'ннw вся^ дjьла` гд\сня гд\са 
пои'мъ, и= превозно'симъ во вся^ вjь'ки.

%<Т%>ы` разори'лъ _е=си` сокруше'нiе хр\сте`, и= 
_о=кая'нство, во вратjь'хъ и= тверды'няхъ а='довыхъ, 
воскр~съ и=з\ъ гро'ба тридне'венъ. тебе` непреста'ннw 
вся^ дjьла` jа='кw гд\са пою'тъ, и= превозно'сятъ во вся^ 
вjь'ки.

%<Бг~оро'диченъ: Jа='%>же без\ъ сjь'мене и= 
преесте'ственнjь w\т w=блиста'нiя бж~е'ственнагw ро'ждшую 
би'сера многоцjь'ннаго хр\ста`, воспои'мъ глаго'люще: 
бл~гослови'те вся^ дjьла` гд\сня гд\са, по'йте и= 
превозноси'те _е=го` во вся^ вjь'ки.

%<%[И='нъ%] I=рмо'съ: Ч%>у'да преесте'ственнагw:

%<П%>рiиди'те лю'дiе, поклони'мся мjь'сту, на не'мже 
стоя'стjь преч\стjьи но'зjь, и= на дре'вjь бж~е'ственнjьи 
хр\сто'вjь (с. 51) дла^ни животворя'щiи простро'стjься, 
на сп~се'нiе всjь'хъ человjь'кwвъ, и= гро'бъ живо'тный 
w=бстоя'ще, пои'мъ: да бл~гослови'тъ тва'рь вся'кая 
гд\са, и= превозно'ситъ во вся^ вjь'ки.

%<W=%>бличи'ся бг~оубi'йцъ i=уде'wвъ пребеззако'нное 
w=клевета'нiе: _е=го'же бо лестца` нареко'ша, воста` 
jа='кw си'ленъ, наруга'вся безу'мнымъ печа'темъ. тjь'мже 
ра'дующеся воспои'мъ: да бл~гослови'тъ тва'рь вся'кая 
гд\са, и= превозно'ситъ во вся^ вjь'ки.

%<Тр\оченъ: В%>ъ трiе'хъ сщ~е'нiихъ бг~осло'вяще и= 
_е=ди'номъ гд\сьствjь сла'ву серафi'ми пречи'стiи, со 
стра'хомъ раболjь'пнw трiv"поста'сное сла'вятъ бж~ество`. 
съ ни'миже и= мы` бл~гоче'ствующе воспои'мъ: да 
бл~гослови'тъ тва'рь вся'кая гд\са, и= превозно'ситъ во 
вся^ вjь'ки.

%<%[И='нъ%] I=рмо'съ: В%>ъ пещи` _о='троцы i=и~левы:

%<Ч%>ерто'гъ свjьтови'дный, и=з\ъ негw'же всjь'хъ вл\дка, 
jа='кw жени'хъ произы'де хр\сто'съ, воспои'мъ вси` 
вопiю'ще: вся^ дjьла` гд\сня гд\са по'йте, и= 
превозноси'те во вся^ вjь'ки.

%<Р%>а'дуйся пр\сто'ле сла'вный бж~iй, ра'дуйся вjь'рныхъ 
стjьно`, _е='юже су'щымъ во тьмjь` возсiя` свjь'тъ 
хр\сто'съ, тебе` бл~жа'щымъ, и= вопiю'щымъ: вся^ дjьла` 
гд\сня гд\са по'йте, и= превозноси'те во вся^ вjь'ки.

%<С%>п~се'нiю вино'внаго на'мъ гд\са ро'ждши, моли` w= 
всjь'хъ вопiю'щихъ прилjь'жнw, дв~о всепjь'тая: 
бл~гослови'те вся^ дjьла` гд\сня гд\са, по'йте и= 
превозноси'те во вся^ вjь'ки.

%<Та'же пое'мъ пjь'снь бц\ды: В%>ели'читъ душа` моя` 
гд\са: %<Съ припjь'вомъ: Ч\с%>тнjь'йшую херувi^мъ:

%<%[Пjь'снь f~%]%>

%<I=рмо'съ: W='%>бразъ чи'стагw рж\ства` твоегw`, 
_о=гнепали'мая купина` показа` неwпа'льная: и= ны'нjь на 
на'съ напа'стей свирjь'пjьющую о_у=гаси'ти мо'лимся 
пе'щь, да тя` бц\де непреста'ннw велича'емъ. (с. 52)

%<_W%> ка'кw, лю'дiе беззако'ннiи и= непокори'вiи, 
лука^вая совjьща'вше, го'рдаго и= нечести'ваго 
w=правди'ша: прв\днаго же на дре'вjь w=суди'ша гд\са 
сла'вы, _е=го'же досто'йнw велича'емъ!

%<С%>п~се а='гнче непоро'чне, и='же мi'ра грjьхи` 
взе'мый, тебе` сла'вимъ воскр\сшаго тридне'внw, со 
_о=тце'мъ и= бж~е'ственнымъ твои'мъ дх~омъ, и= гд\са 
сла'вы: _е=го'же бг~осло'вяще, велича'емъ.

%<Бг~оро'диченъ: С%>п~си` лю'ди твоя^, гд\си, и=`хже 
стяжа'лъ _е=си` честно'ю твое'ю кро'вiю, цр~квамъ твои'мъ 
подая` ми'ръ, чл~вjьколю'бче, бц\ды мл~твами.

%<%[И='нъ%] I=рмо'съ: Т%>а'инство стра'нное:

%<П%>росла'вися неизрече'нною си'лою твое'ю кр\стъ тво'й, 
гд\си, немощно'е бо твое` па'че си'лы всjь'мъ jа=ви'ся: 
и='мже си'льнiи о_у='бw низложе'ни бы'ша на зе'млю, и= 
ни'щiи къ нб~си` возводи'ми быва'ютъ.

%<О_у=%>мертви'ся ме'рзкая на'ша сме'рть, @и=з\ъ 
ме'ртвыхъ воскр\снiемъ: ты' бо jа=ви'вся су'щымъ во 
а='дjь хр\сте`, живо'тъ дарова'лъ _е=си`@{@и=з\ъ 
ме'ртвыхъ бо воскр\снiе ты`, jа=ви'вся су'щымъ во а='дjь, 
хр\сте`, дарова'лъ _е=си`@}. тjь'мже тя` jа='кw жи'знь и= 
воскр\снiе и= свjь'тъ v=поста'сный пою'ще велича'емъ.

%<Тр\оченъ: Б%>езнача'льное _е=стество` и= 
непредjь'льное, въ трiе'хъ познава'ется _е=ди'нствjьхъ, 
бг~онача'льныхъ v=поста'сехъ _е=ди'но бж~ество`, во 
_о=ц~jь`, и= сн~jь, и= дх~jь: на не'же бг~ому'дрiи лю'дiе 
о_у=пова'юще, спаса'емся.

%<%[И='нъ%] I=рмо'съ: W='%>бразъ чи'стагw:

%<И=%>з\ъ ко'рене дв~дова прозябла` _е=си` пр\оро'ческагw 
дв~о, и= бг~о_оч~ескагw: но и= дв~да jа='кw вои'стинну 
ты` просла'вила _е=си`, jа='кw ро'ждши 
пр\оро'чествованнаго гд\са сла'вы: _е=го'же досто'йнw 
велича'емъ.

%<В%>ся'къ похва'льный, преч\стая, зако'нъ побjьжда'ется 
вели'чествомъ сла'вы твоея`. но _w вл\дчце, w\т ра^бъ (с. 
53) твои'хъ недосто'йныхъ, w\т любве` тебjь` приноси'мое 
прiими`, бц\де, со о_у=се'рдiемъ пjь'нiе похва'льное.

%<_W%> па'че о_у=ма` чуде'съ твои'хъ! ты' бо дв~о 
_е=ди'на па'че сл~нца, всjь^мъ дала` _е=си` разумjь'ти 
новjь'йшее чу'до, всеч\стая, твоегw` рж\ства` 
непостижи'магw. тjь'мже тя` вси` велича'емъ.

%<По катава'сiи _е=ктенiа` ма'лая. Та'же, С%>т~ъ гд\сь 
бг~ъ на'шъ: %<три'жды. _е=_ксапостiла'рiй о_у='треннiй.%>

%<На хвали'техъ стiхи^ры воскр\сны, гла'съ а~:%>

%<Стi'хъ: С%>отвори'ти въ ни'хъ су'дъ напи'санъ: сла'ва 
сiя` бу'детъ всjь^мъ прп\дбнымъ _е=гw`.

%<П%>ое'мъ твою` хр\сте`, сп~си'тельную стр\сть, и= 
сла'вимъ твое` воскр\снiе.

%<Стi'хъ: Х%>вали'те бг~а во ст~ы'хъ _е=гw`, хвали'те 
_е=го` во о_у=тверже'нiи си'лы _е=гw`.

%<К%>р\стъ претерпjь'вый, и= сме'рть о_у=праздни'вый, и= 
воскр~сы'й и=з\ъ ме'ртвыхъ, о_у=мири` на'шу жи'знь гд\си, 
jа='кw _е=ди'нъ всеси'ленъ.

%<Стi'хъ: Х%>вали'те _е=го` на си'лахъ _е=гw`, хвали'те 
_е=го` по мно'жеству вели'чествiя _е=гw`.

%<А='%>да плjьни'вый, и= человjь'ка воскр~си'вый, 
воскр\снiемъ твои'мъ хр\сте`, сподо'би на'съ чи'стымъ 
се'рдцемъ, тебе` пjь'ти и= сла'вити.

%<Стi'хъ: Х%>вали'те _е=го` во гла'сjь тру'бнjьмъ: 
хвали'те _е=го` во _псалти'ри и= гу'слехъ.

%<Б%>г~олjь'пное твое` снисхожде'нiе сла'вяще, пое'мъ тя` 
хр\сте`. роди'лся _е=си` w\т дв~ы, и= не разлуче'нъ бы'лъ 
_е=си` w\т _о=ц~а`, пострада'лъ _е=си` jа='кw чл~вjь'къ, 
и= во'лею претерпjь'лъ _е=си` кр\стъ, воскр\слъ _е=си` 
w\т гро'ба, jа='кw w\т черто'га произше'дъ да сп~се'ши 
мi'ръ, гд\си сла'ва тебjь`.

%<И='ны стiхи^ры, а=нато'лiевы, гла'съ то'йже:%>

%<Стi'хъ: Х%>вали'те _е=го` въ тv"мпа'нjь и= ли'цjь, 
хвали'те _е=го` во стру'нахъ и= _о=рга'нjь. (с. 54)

%<_Е=%>гда` пригвозди'лся _е=си` на дре'вjь кр\стнjьмъ, 
тогда` о_у=мертви'ся держа'ва вра'жiя: тва'рь поколеба'ся 
стра'хомъ твои'мъ: и= а='дъ плjьне'нъ бы'сть держа'вою 
твое'ю: м_е'ртвыя w\т грw'бъ воскр~си'лъ _е=си`, и= 
разбо'йнику ра'й w\тве'рзлъ _е=си`: хр\сте` бж~е на'шъ 
сла'ва тебjь`.

%<Стi'хъ: Х%>вали'те _е=го` въ кv"мва'лjьхъ 
доброгла'сныхъ, хвали'те _е=го` въ кv"мва'лjьхъ 
восклица'нiя: вся'кое дыха'нiе да хва'литъ гд\са.

%<Р%>ыдаю'щя со тща'нiемъ гро'ба твоегw` доше'дшя 
честны^я ж_ены`, w=брjь'тшя же гро'бъ w\тве'рстъ, и= 
о_у=вjь'дjьвшя w\т а='гг~ла но'вое и= пресла'вное чу'до, 
возвjьсти'ша а=п\слwмъ: jа='кw воскр~се гд\сь, да'руяй 
мi'рови ве'лiю мл\сть.

%<Стi'хъ: В%>оскр\сни` гд\си бж~е мо'й, да вознесе'тся 
рука` твоя`, не забу'ди о_у=бо'гихъ твои'хъ до конца`.

%<С%>тр\сте'й твои'хъ бж~е'ств_еннымъ jа='звамъ 
покланя'емся хр\сте` бж~е, и= _е='же въ сiw'нjь вл\дчнему 
сщ~еннодjь'йствiю, на коне'цъ вjькw'въ бг~оявле'ннjь 
бы'вшему: и='бо во тьмjь` спя'щыя, сл~нце просвjьти` 
пра'вды, къ невече'рнему наставля'я сiя'нiю: гд\си сла'ва 
тебjь`.

%<Стi'хъ: И=%>сповjь'мся тебjь` гд\си всjь'мъ се'рдцемъ 
мои'мъ, повjь'мъ вся^ чудеса` твоя^.

%<Л%>юбомяте'жный ро'де _е=вре'йскiй внуши'те, гдjь` 
су'ть, и=`же къ пiла'ту прише'дшiи: да реку'тъ стрегу'щiи 
во'ини: гдjь` су'ть печа^ти грw'бныя; гдjь` преложе'нъ 
бы'сть погребе'нный; гдjь` про'данъ бы'сть непрода'нный; 
ка'кw о_у=кра'дено бы'сть сокро'вище; что` w=клевету'ете 
сп~сово воста'нiе пребеззако'нiи i=уд_е'и; воскр~се и='же 
въ ме'ртвыхъ свобо'дь, и= подае'тъ мi'рови ве'лiю мл\сть.

%<Сла'ва, стiхи'ра _е=v\гльская о_у='тренняя%>

%<И= ны'нjь: П%>ребл~гослове'нна _е=си` бц\де дв~о, 
вопло'щшимъ бо ся и=з\ъ тебе` а='дъ плjьни'ся, а=да'мъ 
воззва'ся, кля'тва потреби'ся, _е='vа свободи'ся, сме'рть 
о_у=мертви'ся, и= мы` w=жи'хомъ, тjь'мъ воспjьва'юще 
вопiе'мъ: бл~гослове'нъ хр\сто'съ бг~ъ бл~говоли'вый 
та'кw, сла'ва тебjь`. (с. 55)

%<Славосло'вiе вели'кое. Та'же, тропа'рь воскр\снъ:%>

%<Д%>не'сь сп~се'нiе мi'ру бы'сть, пое'мъ воскр\сшему 
и=з\ъ гро'ба, и= нача'льнику жи'зни на'шея: разруши'въ бо 
сме'ртiю сме'рть, побjь'ду даде` на'мъ, и= ве'лiю мл\сть.

%<И= w\тпу'стъ%>
