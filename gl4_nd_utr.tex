%<%[Въ недjь'лю о_у='тра по шесто_пса'лмiи%],%>

%<Б%>г~ъ гд\сь, и= jа=ви'ся на'мъ: %<на гла'съ д~. 
Та'же тропа'рь: С%>вjь'тлую воскр\снiя про'повjьдь: 
%<два'жды. Сла'ва, и= ны'нjь, бг~оро'диченъ: _Е='%>же w\т 
вjь'ка о_у=тае'ное: %<Та'же _о=бы^чная стiхослw'вiя 
_псалти'ра.%>

%<По а~-мъ стiхосло'вiи сjьда'льны воскр\сны, гла'съ 
д~:%>

%<В%>оззрjь'вшя на гро'бный вхо'дъ, и= пла'мене 
а='гг~льскагw не терпя'щя мv"ронw'сицы, съ тре'петомъ 
дивля'хуся, глаго'лющя: _е=гда` о_у=кра'деся w\тве'рзый 
разбо'йнику ра'й; _е=да' ли воста`, и='же и= пре'жде 
стра'сти проповjь'давый воста'нiе; вои'стинну воскр~се 
хр\сто'съ, су'щымъ во а='дjь подая` живо'тъ, и= 
воскр\снiе.

%<Стi'хъ: В%>оскр\сни` гд\си бж~е мо'й, да вознесе'тся 
рука` твоя`, не забу'ди о_у=бо'гихъ твои'хъ до конца`.

%<В%>о'льнымъ твои'мъ совjь'томъ кр\стъ претерпjь'лъ 
_е=си` сп~се: и= во гро'бjь но'вjь человjь'цы положи'ша 
тя` сме'ртнiи, сло'вомъ концы` соста'вльшаго. тjь'мже 
свя'занъ бы'сть чу'ждiй: сме'рть лю'тw плjьня'шеся, и= 
су'щiи во (с. 514) а='дjь вси` взыва'ху живоно'сному 
воста'нiю твоему`: хр\сто'съ воскр~се, жизнода'вецъ 
пребыва'яй во вjь'ки.

%<Сла'ва, и= ны'нjь, бг~оро'диченъ:%> О_у=диви'ся 
i=w'сифъ, _е='же па'че _е=стества` зря`, и= внима'ше 
мы'слiю и='же на руно` до'ждь, въ безсjь'меннjьмъ 
зача'тiи твое'мъ бц\де, купину` _о=гне'мъ неwпали'мую, 
же'злъ а=арw'новъ прозя'бшiй, и= свидjь'тельствуя 
_о=бру'чникъ тво'й и= храни'тель, сщ~е'нникwмъ взыва'ше: 
дв~а ражда'етъ, и= по рж\ствjь` па'ки дв~а пребыва'етъ.

%<По в~-мъ стiхосло'вiи сjьда'льны воскр\сны, гла'съ 
д~:%>

%<Подо'бенъ: С%>ко'рw предвари`:

%<В%>оскр~слъ _е=си` jа='кw безсме'ртный w\т гро'ба 
сп~се, совоздви'глъ _е=си` мi'ръ тво'й си'лою твое'ю 
хр\сте` бж~е на'шъ, сокруши'лъ _е=си` въ крjь'пости 
сме'рти держа'ву, показа'лъ _е=си` мл\стиве, воскр\снiе 
всjь^мъ: тjь'мже тя` и= сла'вимъ _е=ди'не чл~вjьколю'бче.

%<Стi'хъ: И=%>сповjь'мся тебjь`, гд\си, всjь'мъ 
се'рдцемъ мои'мъ, повjь'мъ вся^ чудеса` твоя^.

%<С%>ъ го'рнихъ высо'тъ соше'дъ гаврiи'лъ, и= къ 
ка'меню присту'пль, и=дjь'же ка'мень жи'зни, бjьлонося'й 
взыва'ше ко пла'чущымъ: преста'ните вы` w\т рыда'нiя 
во'пля, и=мjь'ющыя и= ны'нjь мл\стивное: _е=го'же бо 
и='щете пла'чущя, дерза'йте, jа='кw вои'стинну воста'лъ 
_е='сть. тjь'мже возопi'йте а=п\слwмъ: jа='кw воскр~се 
гд\сь, воста'вшему поклони'теся ра'дость прiе'мшя. 
дерза'йте о_у='бw, да дерза'етъ о_у='бw и= _е='vа.

%<Сла'ва, и= ны'нjь, бг~оро'диченъ:%> О_у=диви'шася 
ч\стая, вси` а='гг~лwвъ ли'цы та'инству твоегw` рожде'нiя 
стра'шному: ка'кw и='же вся^ содержа'й ма'нiемъ 
_е=ди'нjьмъ, w=б\ъя'тiи твои'ми jа='кw человjь'къ 
содержава'ется, и= прiе'млетъ нача'ло превjь'чный и= 
млеко'мъ пита'ется, и='же вся'кое дыха'нiе пита'яй 
неизрече'нною бл~гостiю; и= тя` jа='кw вои'стинну бж~iю 
мт~рь хва'ляще сла'вятъ. (с. 515)

%<По непоро'чныхъ v=пакои`, гла'съ д~:%>

%<Jа=`%>же твоегw` пресла'внагw воста'нiя, предте'кшя 
мv"ронw'сицы, а=п\слwмъ проповjь'даху хр\сте`, jа='кw 
воскр~слъ _е=си` jа='кw бг~ъ, подая` мi'рови ве'лiю 
мл\сть.

%<Степ_е'нна, гла'съ д~. %[А=нтiфw'нъ%] а~, и='хже 
стiхи` повторя'юще пое'мъ:%>

%<W\т%> ю='ности моея` мно'зи бо'рютъ мя` стра^сти, но 
са'мъ мя` заступи`, и= сп~си` сп~се мо'й.

%<Н%>енави'дящiи сiw'на, посрами'теся w\т гд\са, 
jа='кw трава' бо _о=гне'мъ бу'дете и=зсо'хше.

%<Сла'ва: С%>т~ы'мъ дх~омъ вся'ка душа` живи'тся, и= 
чистото'ю возвыша'ется, свjьтлjь'ется тр\оческимъ 
_е=ди'нствомъ сщ~еннота'йнjь.

%<И= ны'нjь, то'йже.%>

%<%[А=нтiфw'нъ%] в~:%>

%<В%>оззва'хъ тебjь` гд\си те'плjь, и=з\ъ глубины` 
души` моея`, и= мнjь` да бу'дутъ на послуша'нiе 
бж~е'ств_енная твоя^ о_у=шеса`.

%<Н%>а гд\са наде'жду вся'къ кто` стяжа'въ, вы'шшiй 
_е='сть всjь'хъ скорбя'щихъ.

%<Сла'ва: С%>т~ы'мъ дх~омъ то'чатся бл~года^тныя 
струи^, напая'ющя вся'ку тва'рь ко w=живле'нiю.

%<И= ны'нjь, то'йже.%>

%<%[А=нтiфw'нъ%] г~:%>

%<С%>е'рдце мое` къ тебjь` сло'ве да возвы'сится, и= 
да ничто'же о_у=слади'тъ мя` w\т мiрски'хъ красо'тъ на 
сла'бость.

%<К%>ъ ма'тери свое'й jа='коже и='мать кто` любо'вь, 
ко гд\су те'пльше любле'нiемъ до'лжни _е=смы`.

%<Сла'ва: С%>т~ы'мъ дх~омъ бг~овjь'дjьнiя бога'тство, 
зрjь'нiя, и= прему'дрости: вся^ бо въ се'мъ _о=те'ч_еская 
велjь^нiя сло'во w\ткрыва'етъ.

%<И= ны'нjь, то'йже.%>

%<Прокi'менъ, гла'съ д~: В%>оскр\сни` гд\си, помози` 
на'мъ, и= и=зба'ви на'съ и='мене твоегw` ра'ди. (с. 516) 
%<Стi'хъ: Б%>ж~е о_у=ши'ма на'шима о_у=слы'шахомъ: 
%<В%>ся'кое дыха'нiе: %<_Е=v\глiе воскр\сно. В%>оскр\снiе 
хр\сто'во: %<_Псало'мъ н~. И= прw'чая по _о=бы'чаю.%>

%<Канw'нъ воскр\снъ, творе'нiе i=wа'нна дамаскина`.%>

%<%[Пjь'снь а~%]%>

%<I=рмо'съ: %>Мо'ря чермну'ю пучи'ну невла'жными 
стопа'ми, дре'внiй пjьшеше'ствовавъ i=и~ль, 
кр\стоwбра'зныма мwv"се'овыма рука'ма а=мали'кову си'лу 
въ пусты'ни побjьди'лъ _е='сть.

%<Припjь'въ: С%>ла'ва гд\си ст~о'му воскр\снiю 
твоему`.

%<В%>озне'слся _е=си` на преч\стjьмъ дре'вjь 
кр\стнjьмъ, на'ше w\тпаде'нiе и=справля'я, _е='же на 
дре'вjь и=сцjьля'я всегуби'тельство вл\дко, jа='кw бл~гъ 
и= всеси'ленъ.

%<В%>о гро'бjь пло'тски, во а='дjь же съ душе'ю, 
jа='кw бг~ъ: въ раи' же съ разбо'йникомъ, и= на 
пр\сто'лjь бы'лъ _е=си` хр\сте`, со _о=ц~е'мъ и= дх~омъ, 
вся^ и=сполня'я неwпи'санный.

%<Бг~оро'диченъ: Б%>ез\ъ сjь'мене _о='ч~ею во'лею w\т 
бж~е'ственнагw дх~а бж~iя зачала` _е=си` сн~а, и= пло'тiю 
родила` _е=си`: и='же и=з\ъ _о=ц~а` без\ъ ма'тере, на'съ 
же ра'ди, и=з\ъ тебе` без\ъ _о=ц~а`.

%<Другi'й канw'нъ кр\стовоскре'сенъ. Гла'съ д~.%>

%<%[Пjь'снь а~%]%>

%<I=рмо'съ: W\т%>ве'рзу о_у=ста` моя^, и= напо'лнятся 
дх~а:

%<И=%>сцjьли'лъ _е=си` сокруше'нiе человjь'чества 
гд\си, бж~е'ственною твое'ю кро'вiю w=бнови'вый то`: и= 
сокруши'лъ _е=си` си'льнаго въ крjь'пости, и='же дре'вле 
сокруши'вшаго твое` созда'нiе.

%<М%>е'ртвыхъ воста'нiе, о_у=мерщвле'нiемъ бы'лъ 
_е=си`: крjь'пость бо w\тя'тся о_у=мерщвле'нiя, бра'вшися 
съ жи'знiю вjь'чною, и='же всjь'ми влады'чествующу 
воплоще'нному бг~у.

%<Бг~оро'диченъ: К%>расе'нъ превы'шши нб\сныхъ си'лъ, 
бж~е'ственный тво'й бы'сть хра'мъ w=душевле'нный, @во 
о_у=тро'бjь ты` носи'вши дв~о, горо` ст~а'я@{@jа='же во 
о_у=тро'бjь тя` носи'вшая дв~о, гора` ст~а'я твоя`@}, 
бг~а на'шего. (с. 517)

%<И='нъ канw'нъ прест~jь'й бц\дjь, [_е=гw'же 
краестро'чiе: Ч%>етве'ртая пjь'снь всесла'внjьй 
_о=трокови'цjь.%<] Гла'съ д~.%>

%<%[Пjь'снь а~%]%>

%<I=рмо'съ: %>Трiста'ты крjь^пкiя, рожде'йся w\т дв~ы:

%<С%>отрясо'шася лю'дiе, смято'шася jа=зы'цы, цр\ствiя 
же держа^вная о_у=клони'шася ч\стая, w\т стра'ха рж\ства` 
твоегw`: прiи'де бо цр~ь мо'й, и= низложи` мучи'теля, и= 
мi'ръ w\т тли` и=зба'ви.

%<Ж%>или'ще свое` живы'й въ вы'шнихъ, къ человjь'кwмъ 
соше'дъ, w=святи` хр\сте`, и= непоколеби'мо jа=ви`: 
_е=ди'на бо по рж\ствjь` дв~ства сокро'вище, зижди'теля 
ро'ждши пребыла` _е=си`.

%<%[Пjь'снь г~%]%>

%<I=рмо'съ: %>Весели'тся w= тебjь` цр~ковь твоя` 
хр\сте`, зову'щи: ты` моя` крjь'пость гд\си, и= 
прибjь'жище, и= о_у=твержде'нiе.

%<Д%>ре'во живо'тное, мы'сленный и='стинный 
вiногра'дъ, на кр\стjь` ви'ситъ, всjь^мъ и=сточа'я 
нетлjь'нiе.

%<Jа='%>кw вели'къ, jа='кw стра'шенъ, jа='кw а='дово 
низло'жь шата'нiе, и= jа='кw бг~ъ нетлjь'ненъ, ны'нjь 
пло'тски воскр~се.

%<Бг~оро'диченъ: Т%>ы` _е=ди'на су'щымъ на земли`, 
jа='же па'че _е=стества` бл~ги'хъ хода'таица, мт~и бж~iя 
была` _е=си`: тjь'мже ти`, ра'дуйся, прино'симъ.

%<%[И='нъ%]%>

%<I=рмо'съ: %>Твоя^ пjьсносло'вцы бц\де:

@%<Jа='%>домъ и=спо'лненный мнjь`@{@jа='домъ 
и=спо'лнены въ мя`@} sмi'й зу'бы вонзе`, сп~се, jа=`же 
вседержи'телю вл\дко, гвоздьми` ру'къ твои'хъ сокруши'лъ 
_е=си`: jа='кw нjь'сть ст~ъ во ст~ы'хъ, па'че тебе` 
чл~вjьколю'бче.

%<В%>и'дjьнъ бы'лъ _е=си` чл~вjьколю'бче во'лею во 
гро'бjь ме'ртвъ, животво'рче, и= врата` разве'рглъ _е=си` 
а='дwва, jа=`же w\т вjькw'въ душа'мъ: jа='кw нjь'сть ст~ъ 
во ст~ы'хъ ра'звjь тебе` чл~вjьколю'бче.

%<Бг~оро'диченъ: Н%>еwра'на бразда` jа=ви'лася _е=си`, 
кла'съ живо'тный ро'ждши, всjь^мъ причаща'ющымся 
безсме'ртiю хода'тая, во ст~ы'хъ ст~а'го свя'тw 
почива'ющаго. (с. 518)

%<И='нъ. I=рмо'съ: %>Съ высоты` снизше'лъ _е=си` 
во'лею на зе'млю:

%<W=%>чища'ется человjь'кwвъ существо`, тобо'ю 
присовоку'пльшееся нестерпи'мому бж~е'ственному _о=гню`: 
jа='кw сокрове'нный, преч\стая дв~о, въ тебjь` хлjь'бъ 
и=спе'кшееся, и='же и= тебе` неврежде'нну сохра'ншему.

%<К%>а'я сiя` jа='же вои'стинну бли'зъ бг~а; jа='кw 
превозше'дши вся^ а='гг~льскiя чи'ны, _е=ди'на добро'тою 
дв~ства, jа='кw мт~и сiя'ющи вседержи'теля.

%<%[Пjь'снь д~%]%>

%<I=рмо'съ: %>Вознесе'на тя` ви'дjьвши цр~ковь на 
кр\стjь`, со'лнце пра'ведное, ста` въ чи'нjь свое'мъ, 
досто'йнw взыва'ющи: сла'ва си'лjь твое'й гд\си.

%<В%>озше'лъ _е=си`, стра^сти моя^ и=сцjьля'я, на 
кр\стъ стр\стiю преч\стыя пло'ти твоея`, въ ню'же во'лею 
w=бле'клся _е=си`. тjь'мже ти` взыва'емъ: сла'ва си'лjь 
твое'й гд\си.

%<Б%>езгрjь'шнагw сме'рть вкуси'вши, животворя'щагw 
тjь'ла твоегw`, досто'йнw вл\дко о_у=мертви'ся: мы' же 
вопiе'мъ ти`, сла'ва си'лjь твое'й гд\си.

%<Бг~оро'диченъ: Н%>еискусобра'чнw родила` _е=си` 
дв~о, и= по рж\ствjь` jа=ви'лася _е=си` дв~ствующи па'ки: 
тjь'мже немо'лчными гла'сы, _е='же ра'дуйся тебjь` 
вл\дчце, вjь'рою несумнjь'нною взыва'емъ.

%<И='нъ. I=рмо'съ: %>Неизслjь'дный бж~iй совjь'тъ:

%<В%>зако'ненъ сы'й i=и~ль, тебе` хр\сте` 
взако'нившагw бг~а не позна`: но jа='кw беззако'нника, 
законопреступа'я, на кр\стjь` пригвозди`, и='же 
законоположе'нiю недосто'йный.

%<W=%>боже'на твоя^ сп~се дш~а`, а='дwва сокрw'вища 
плjьни'вши, jа='же w\т вjь'ка совоскр~си` ду'шы: 
живоно'сное же тjь'ло всjь^мъ нетлjь'нiе и=сточи`.

%<Бг~оро'диченъ: Т%>ебе` приснодв~у и= и='стинную 
бц\ду вси` сла'вимъ, ю='же проwбразова'ше бг~ови'дцу 
мwv"се'ю неwпа'льнw преч\стая, купина`, _о=гню` 
примjьси'вшися. (с. 519)

%<И='нъ. I=рмо'съ: %>Сjьдя'й въ сла'вjь:

%<П%>оживе` съ человjь'ки, ви'димь бы'въ неви'димый, 
во зра'цjь сы'й непостижи'магw бж~ества`, и= воwбра'жься 
и=з\ъ тебе` _о=трокови'це въ чужде'е, вjь'дущихъ тя` 
ч\стую бг~ома'терь сп~са'етъ.

%<П%>рiя'тъ въ веще'ственнjь дв~а невеще'ственнаго, въ 
прича'стiи вещества`, мл\днца w\т нея` бы'вша. тjь'мже во 
дву` сущ_еству`, _е=ди'нъ познава'ется плотоно'сецъ бг~ъ, 
и= человjь'къ пресу'щественный.

%<П%>о рж\ствjь' тя дв~у, и='же въ дв~у тя` всели'ся, 
и= ро'ждься без\ъ сjь'мене, сло'во и= бг~ъ пребы'сть, и= 
въ рж\ствjь` дв~у сохрани`, jа='кw вл\дка и= зижди'тель 
всея` тва'ри.

%<%[Пjь'снь _е~%]%>

%<I=рмо'съ: Т%>ы` гд\си мо'й, свjь'тъ въ мi'ръ 
прише'лъ _е=си`, свjь'тъ ст~ы'й, w=браща'яй и=з\ъ мра'чна 
невjь'дjьнiя вjь'рою воспjьва'ющыя тя`.

%<Т%>ы` гд\си, къ земли` мл\стивнw соше'лъ _е=си`: ты` 
возне'слъ _е=си` па'дшее человjь'ческое существо`, на 
дре'вjь воздвиза'емь.

%<Т%>ы` взя'лъ ми` _е=си` хр\сте`, прегрjьше'нiй 
w=сужде'нiе: ты` разруши'лъ _е=си` бwлjь'зни см_е'ртныя 
ще'дре, бж~е'ственнымъ воскр~се'нiемъ твои'мъ.

%<Бг~оро'диченъ: Т%>я` _о=ру'жiе непобjьди'мое на 
враги` предлага'емъ, тя` о_у=твержде'нiе, и= наде'жду 
на'шегw сп~се'нiя бг~оневjь'сто стяжа'хомъ.

%<И='нъ. I=рмо'съ: О_у=%>жасо'шася вся'ч_еская w= 
бж~е'ственнjьй:

%<П%>рiя'тъ тя` всего` о_у=сты` а='дъ безу'мный: на 
кр\стjь' бо пригвожде'на тя` ви'дjьвъ, копiе'мъ 
прободе'на, бездыха'нна, жива'го бг~а, про'ста вмjьня'ше 
человjь'ка. разумjь' же и=скуси'вый крjь'пость твоегw` 
бж~ества`.

%<Р%>азруше'ный, чл~вjьколю'бче, хра'мъ твоегw` 
тjьлесе`, гро'бъ раздjьли'вый и= а='дъ нево'лею @_о='ба 
суда` и=стязу'еми су'ть@{@_о='ба и=стязу'етася@}: _о='въ 
о_у='бw ст~ы'хъ твои'хъ ду'шы, тjьлеса' же другi'й 
соw\тсыла'юще, безсме'ртне. (с. 520)

%<Бг~оро'диченъ: С%>е` ны'нjь и=спо'лнися 
пр\оро'ческое прорече'нiе: ты' бо неискусобра'чная дв~о, 
и=мjь'ла _е=си` во о_у=тро'бjь над\ъ всjь'ми бг~а и= 
родила` _е=си` безлjь'тнаго сн~а, всjь^мъ воспjьва'ющымъ 
тя`, ми'ръ подава'юща.

%<И='нъ. I=рмо'съ: Н%>ы'нjь воста'ну:

%<Д%>о'мъ тя` сла'вы, го'ру бж~iю ст~у'ю ч\стая, 
невjь'сту, черто'гъ, хра'мъ w=сщ~е'нiя, сн~ъ бж~iй, въ 
тя` всели'вся, и= ра'й сла'дости присносу'щныя на'мъ 
содjь'ла.

%<П%>ло'ть w\т кро'ве дв~ственныя прiя'лъ _е=си` 
хр\сте`, безсjь'менну, преч\сту, v=поста'сну, и= 
слове'сну и= о_у='мну, w=душевле'нну, дjь'йственну, 
хотjь'тельну, самовлады'чну и= самовла'стну.

%<М%>учи'телей ра'зумъ дв~ственное посрами` чре'во: 
мл\днцъ бо jа='зву а='спiдную душегу'бную и=спыта` 
руко'ю, и= w\тсту'пника го'рдаго низло'жъ, вjь'рныхъ 
под\ъ но'зjь покори`.

%<%[Пjь'снь s~%]%>

%<I=рмо'съ: %>Пожру' ти со гла'сомъ хвале'нiя гд\си, 
цр~ковь вопiе'тъ ти`, w\т бjьсо'вскiя кро'ве w=чи'щшися, 
ра'ди мл\сти w\т ре'бръ твои'хъ и=сте'кшею кро'вiю.

%<В%>озше'лъ _е=си` на кр\стъ, си'лою препоя'сався, и= 
сопле'тся съ мучи'телемъ jа='кw бг~ъ, съ высоты` све'рглъ 
_е=си`, а=да'ма же непобjьди'мою си'лою воскр~си'лъ 
_е=си`.

%<В%>оскр\слъ _е=си` блиста'яйся кра'сный и=з\ъ гро'ба 
хр\сте`, и= разгна'лъ _е=си` вся^ враги` бж~е'ственною 
си'лою твое'ю, и= вся^ jа='кw бг~ъ, весе'лiя и=спо'лнилъ 
_е=си`.

%<Бг~оро'диченъ: _W%> чу'до всjь'хъ чуде'съ 
новjь'йшее! jа='кw дв~а во о_у=тро'бjь, вся'ч_еская 
w=бдержа'щаго неискусому'жнw заче'нши, не тjьсновмjьсти`.

%<И='нъ. I=рмо'съ: %>Прiидо'хъ во глубины^ мwрскi'я:

%<W\т%>ве'рзе горта'нь сво'й а='дъ, и= пожре' мя, и= 
ду'шу разшири` безу'мный: но хр\сто'съ соше'дъ, возведе` 
жи'знь мою`, jа='кw чл~вjьколю'бецъ. (с. 521)

%<П%>оги'бе сме'ртiю сме'рть, о_у=ме'рый бо воскр~се, 
нетлjь'нiе мнjь` да'руя: jа='влься же жена'мъ провjьща` 
ра'дость безсме'ртный.

%<Бг~оро'диченъ: Н%>естерпи'магw бж~ества` 
вмjьсти'лище чи'стое о_у=тро'ба твоя` jа=ви'ся, _w бц\де! 
_е='же без\ъ стра'ха нб\снiи чи'нове воззрjь'ти не 
возмого'ша.

%<И='нъ. I=рмо'съ то'йже.%>

%<Д%>ре'вле о_у='бw прельсти' мя sмi'й, и= о_у=мори' 
мя, прама'терiю мое'ю _е='vою: ны'нjь же ч\стая, тобо'ю 
созда'вый мя` и=з\ъ и=стлjь'нiя воззва`.

%<Б%>е'здна тя` бе'здну неизрече'ннw бл~гоутро'бiя 
_о=трокови'це, и=збра'нную показа` чуде'съ: и='бо и=з\ъ 
тебе` мо'лнiею бж~ества`, би'серъ хр\сто'съ возсiя`.

%<Конда'къ, гла'съ д~. Подо'бенъ: Jа=%>ви'лся _е=си` 
дне'сь:

%<С%>п~съ и= и=зба'витель мо'й, и=з\ъ гро'ба jа='кw 
бг~ъ воскр~си` w\т о_у='зъ земнорw'дныя, и= врата` 
а='дwва сокруши`, и= jа='кw вл\дка воскр~се тридне'венъ.

%<I='косъ: В%>оскр~сшаго и=з\ъ ме'ртвыхъ, хр\ста` 
жизнода'вца тридне'вна и=з\ъ гро'ба, и= врата` см_е'ртная 
дне'сь сокру'шшаго си'лою свое'ю, и= а='да 
о_у=мертви'вшаго, и= жа'ло сме'ртное сте'ршаго, и= 
а=да'ма со _е='vою свободи'вшаго, воспои'мъ вси` 
земноро'днiи, вопiю'ще хвалу` прилjь'жнw: то'й бо jа='кw 
_е=ди'нъ крjь'пкiй, бг~ъ и= вл\дка, воскр~се тридне'венъ.

%<%[Пjь'снь з~%]%>

%<I=рмо'съ: %>Въ пещи` а=враа'мстiи _о='троцы 
персi'дстjьй любо'вiю бл~гоче'стiя па'че, не'жели 
пла'менемъ w=паля'еми взыва'ху: бл~гослове'нъ _е=си` въ 
хра'мjь сла'вы твоея` гд\си.

%<К%>ъ нетлjь'нiю человjь'чество призва'ся, 
бж~е'ственною и=змове'но кро'вiю хр\сто'вою, бл~года'рнw 
воспjьва'ющее: бл~гослове'нъ _е=си` въ хра'мjь сла'вы 
твоея` гд\си.

%<Jа='%>кw живоно'сецъ, jа='кw рая` краснjь'йшiй 
вои'стинну, и= черто'га вся'кагw ца'рскагw показа'ся 
свjьтлjь'йшiй хр\сте`, гро'бъ тво'й, и=сто'чникъ на'шегw 
воскр\снiя. (с. 522)

%<Бг~оро'диченъ: В%>ы'шнягw w=свяще'нное бж~е'ственное 
селе'нiе ра'дуйся, тобо'ю бо даде'ся ра'дость бц\де, 
зову'щымъ: бл~гослове'на ты` въ жена'хъ _е=си` 
всенепоро'чная вл\дчце.

%<И='нъ. I=рмо'съ: %>Не послужи'ша тва'ри бг~ому'дрiи 
па'че созда'вшагw:

%<С%>мири'лъ _е=си` на дре'во воздвиза'емь, _о='ко 
высо'кое, и= превознесе'нную бро'вь на зе'млю низложи'лъ 
_е=си`, сп~сы'й человjь'ка: препjь'тый _о=тц_е'въ гд\сь 
и= бг~ъ бл~гослове'нъ _е=си`.

%<С%>и'лою твое'ю ро'гъ на'шъ возвы'си служа'щихъ ти`, 
воскр~сы'й и=з\ъ ме'ртвыхъ, и= а='дово и=стощи'вый 
пре'жде многочеловjь'чное бога'тство, вл\дко: препjь'тый 
_о=тц_е'въ гд\сь и= бг~ъ бл~гослове'нъ _е=си`.

%<Тр\оченъ: Р%>ече'нi_емъ бж~е'ств_еннымъ 
послjь'дующе, сла'вимъ _е=ди'но бж~ество`, jа='кw въ 
трiе'хъ свjь'тjьхъ неслiя'нно, непресjько'мо, пла'мень 
вjь'чный просвjьща'ющiй всю` тва'рь, зову'щую: бж~е 
бл~гослове'нъ _е=си`.

%<И='нъ. I=рмо'съ: Ю='%>ношы три` въ вавv"лw'нjь:

%<П%>ривлачи'тъ мя` къ пjь'нiю @любо'вь дjь'вственная, 
_о='гнь@{@любве` дjь'вственныя _о='гнь@}, и='же въ 
се'рдцы, вопи'ти мт~ри и= дв~jь: бл~гослове'нная, гд\сь 
си'ламъ съ тобо'ю.

%<П%>ревы'шши тва'ри jа=ви'лася _е=си`, jа='кw творца` 
ро'ждши и= гд\са. тjь'мже ти` вопiю` бц\де: 
бл~гослове'нная, гд\сь си'ламъ съ тобо'ю.

%<Тр\оченъ: Г%>д\сьство тя` _е=ди'но чты'й, въ трiе'хъ 
сщ~е'нiихъ нераздjь'льно, воспjьва'ю трiv"поста'сное 
_е=стество`, бл~гослове'нная, взыва'я тебjь`, jа='же вся^ 
о_у=правля'ющая.

%<%[Пjь'снь и~%]%>

%<I=рмо'съ: %>Ру'цjь распросте'ръ данiи'лъ, львw'въ 
зiя^нiя въ ро'вjь затче`: _о='гненную же си'лу 
о_у=гаси'ша, добродjь'телiю препоя'савшеся, бл~гоче'стiя 
рачи'тели _о='троцы, взыва'юще: бл~гослови'те вся^ дjьла` 
гд\сня гд\са.

%<Р%>у'цjь распросте'ръ на кр\стjь`, jа=зы'ки вся^ 
собра'лъ _е=си`, и= _е=ди'ну jа=ви'лъ _е=си` вл\дко 
цр~ковь воспjьва'ющую тя`, (с. 523) @земну'ю и= 
нб\сную@{@з_емны'мъ и= неб_е'снымъ@}, согла'снw пою'щымъ: 
бл~гослови'те вся^ дjьла` гд\сня гд\са, по'йте и= 
превозноси'те _е=го` во вjь'ки.

%<Б%>jьлоwбра'зенъ jа=ви'ся жена'мъ, непристу'пнымъ 
свjь'томъ воскр\снiя блиста'яйся а='гг~лъ, что` жива'го 
во гро'бjь, вопiя`, и='щете jа='кw ме'ртва; вои'стинну 
воста` хр\сто'съ, _е=му'же вопiе'мъ: вся^ дjьла` по'йте 
гд\са, и= превозноси'те во вся^ вjь'ки.

%<Бг~оро'диченъ: Т%>ы` _е=ди'на во всjь'хъ ро'дjьхъ 
дв~о преч\стая, мт~и jа=ви'лася _е=си` бж~iя: ты` 
бж~ества` была` _е=си` жили'ще всенепоро'чная, не 
w=па'льшися _о=гне'мъ непристу'пнагw свjь'та. тjь'мже 
вси' тя бл~гослови'мъ, мр~i'е бг~оневjь'сто.

%<И='нъ. I=рмо'съ: _О='%>троки бл~гочести^выя въ 
пещи`:

%<Н%>епра'ведное ви'дящи заколе'нiе твое` тва'рь, 
w=мрача'ющися рыда'ше: земли' бо смуща'ющейся, во мра'къ 
jа='кw въ ри'зу че'рну со'лнце w=блече'ся: мы' же тя` 
непреста'ннw пое'мъ, и= превозно'симъ хр\сте`, во вjь'ки.

%<С%>ше'дый ко мнjь` да'же до а='да, и= всjь^мъ 
путесотвори'вый воскр\снiе, па'ки возше'лъ _е=си`, взе'мъ 
мя` на ра'му тво_е'ю, и= _о=ц~у` приве'лъ _е=си`. тjь'мже 
зову' ти: гд\са по'йте дjьла`, и= превозноси'те во вся^ 
вjь'ки.

%<Тр\оченъ: О_у=%>ма` пе'рваго и= вино'внаго всjь'хъ, 
_о=ц~а` _е=ди'наго безвино'внаго сла'вимъ, сло'ва же 
безнача'льнаго, и= дх~а о_у=тjь'шителя, _е=ди'наго бг~а 
всjь'хъ, тр\оцjь сра'сленнjьй покланя'ющеся, и= 
превознося'ще во вся^ вjь'ки.

%<И='нъ. I=рмо'съ: И=%>зба'вителю всjь'хъ всеси'льне:

%<W\т%> ребра` а=да'мова созда'вый тя`, w\т твоегw` 
дв~ства воплоти'ся, и='же всjь'хъ гд\сь, _е=го'же пою'ще, 
вопiе'мъ: вся^ дjьла` бл~гослови'те, по'йте гд\са, и= 
превозноси'те _е=го` во вjь'ки.

%<В%>ъ сjь'ни а=враа'мъ о_у=зрjь`, _е='же въ тебjь` 
бц\дjь, та'инство, сн~а бо твоего` безпло'тнаго прiя'тъ, 
поя`: вся^ дjьла` бл~гослови'те, по'йте гд\са, и= 
превозноси'те _е=го` во вjь'ки. (с. 524)

%<Р%>авночи'сл_енныя тр\оцы спасло` _е='сть твоегw` 
дв~ства проwбраже'нiе: въ дjь'вственныхъ бо тjьлесjь'хъ 
попира'ху пла'мень _о=трокови'це, вопiю'ще: 
бл~гослови'те, по'йте гд\са, и= превозноси'те _е=го` во 
вjь'ки.

%<%[Пjь'снь f~%]%>

%<I=рмо'съ: %>Ка'мень нерукосjь'чный, w\т несjько'мыя 
горы` тебе` дв~о краеуго'льный w\тсjьче'ся, хр\сто'съ, 
совокупи'вый разстоя^щаяся _е=ст_ества`. тjь'мъ 
веселя'щеся тя` бц\де велича'емъ.

%<В%>сего' мя воспрiя'лъ _е=си` ве'сь въ совокупле'нiи 
несли'тнw, всему' ми дая` бж~е мо'й, сп~се'нiе стр\стiю 
твое'ю, ю='же на кр\стjь` претерпjь'лъ _е=си` тjьле'снjь, 
бл~гоутро'бiя ра'ди мно'гагw.

%<Т%>вои` о_у=ч~нцы` зря'ще w\тве'рзенъ гро'бъ тво'й, 
и= бг~онw'сныя плащани^цы и=спражне'ны воскр\снiемъ 
твои'мъ, со а='гг~ломъ глаго'лаху: вои'стинну воста` 
гд\сь.

%<Тр\оченъ: _Е=%>ди'ницjь о_у='бw бж~е'ственнагw 
существа`, но v=поста'сьми тр\оцjь, вси` вjь'рнiи 
покланя'ющеся, въ неслiя'нныхъ v=поста'сехъ равноси'льную 
_е=диноче'стную ны'нjь благочту'ще велича'емъ.

%<И='нъ. I=рмо'съ: %>Вся'къ земноро'дный:

%<Л%>ьсти'внw попо'лзъ sмi'й, и=з\ъ _е=де'ма поя'тъ 
мя` плjьне'на: на кра'нiевjьмъ же тве'рдjьмъ ка'мени 
разби` сего` вседержи'тель гд\сь, jа='коже младе'нца: и= 
сла'дости па'ки мнjь` вхо'дъ дре'вомъ кре'стнымъ 
w\тве'рзе.

%<П%>оложи'лъ _е=си` крjь^пкiя вра^жiя тверды^ни 
ны'нjь въ запустjь'нiе: всеси'льнjьйшею же руко'ю 
бога'тство _е=гw` расхи'тилъ _е=си` и=з\ъ и=стоща'нiй 
а='довыхъ совоскр~си'вый мя` хр\сте`, и= дре'вле 
безмjь'рнw хва'лящагося jа='кw смjь'хъ руга'ема jа=ви'лъ 
_е=си`.

%<П%>рiиди`, ни'щихъ люде'й твои'хъ w=sлобле'нiе 
посjьща'я, мл\стивною же и= держа'вною твое'ю руко'ю 
о_у=крjьпи` кр\стонw'сныя лю'ди, твое` и=зря'дное 
достоя'нiе, хр\сте`, jа='кw человjьколю'бецъ. (с. 525)

%<И='нъ. I=рмо'съ: С%>окрове'нное бж~iе неизрече'нное 
въ тебjь`:

%<З%>ри'мъ jа='кw крi'нъ тя` ри'зою w=багре'ною 
о_у=кра'шену, преч\стая, бж~е'ственнагw дх~а, посредjь` 
те'рнiя сiя'ющу, и= бл~гоуха'нiя и=сполня'ющу, и=`же 
тебе` и='скреннw велича'ющихъ.

%<Т%>лjь'нное прiи'мъ человjь'ческое _е=стество` и=з\ъ 
твоегw`, всенепоро'чная, чре'ва нетлjь'нный. въ себjь` 
показа` нетлjь'нно, за бл~гоутро'бiе: тjь'мже тя` jа='кw 
бц\ду велича'емъ.

%<Jа='%>же всjь'ми вл\дчествующи тва'рьми, лю'демъ 
твои^мъ да'руй побjь'дное w=долjь'нiе, врага` полага'ющи 
примири'тельна цр~кви: да тя` jа='кw бц\ду велича'емъ.

%<По катава'сiи _е=ктенiа`, и= С%>т~ъ гд\сь бг~ъ 
на'шъ: %<посе'мъ свjьти'ленъ. Сла'ва, и= ны'нjь, 
бг~оро'диченъ.%>

%<На хвали'техъ стiхи^ры воскр\сны, гла'съ д~:%>

%<Стi'хъ: С%>отвори'ти въ ни'хъ су'дъ напи'санъ: 
сла'ва сiя` бу'детъ всjь^мъ прп\дбнымъ _е=гw`.

%<К%>р\стъ претерпjь'вый и= сме'рть, и= воскр~сы'й 
и=з\ъ ме'ртвыхъ всеси'льне гд\си, сла'вимъ твое` 
воскр\снiе.

%<Стi'хъ: Х%>вали'те бг~а во ст~ы'хъ _е=гw`, хвали'те 
_е=го` во о_у=тверже'нiи си'лы _е=гw`.

%<К%>р\сто'мъ твои'мъ хр\сте`, w\т дре'внiя кля'твы 
свободи'лъ _е=си` на'съ, и= сме'ртiю твое'ю _е=стество` 
на'ше му'чащаго дiа'вола о_у=праздни'лъ _е=си`: 
воста'нiемъ же твои'мъ ра'дости вся^ и=спо'лнилъ _е=си`. 
тjь'мже вопiе'мъ ти`: воскр~сы'й и=з\ъ ме'ртвыхъ гд\си, 
сла'ва тебjь`.

%<Стi'хъ: Х%>вали'те _е=гw` на си'лахъ _е=гw`, 
хвали'те _е=го` по мно'жеству вели'чествiя _е=гw`.

%<Т%>вои'мъ кр\сто'мъ хр\сте` сп~се`, наста'ви на'съ 
на и='стину твою`, и= и=зба'ви на'съ w\т сjь'тей 
вра'жiихъ, воскр~сы'й и=з\ъ ме'ртвыхъ, возста'ви ны` 
па'дшыяся грjьхо'мъ, просте'ръ ру'ку твою` чл~вjьколю'бче 
гд\си, мл~твами ст~ы'хъ твои'хъ.

%<Стi'хъ: Х%>вали'те _е=го` во гла'сjь тру'бнjьмъ, 
хвали'те _е=го` во _псалти'ри и= гу'слехъ. (с. 526)

%<_О=%>ч~ескихъ твои'хъ нjь'дръ не разлучи'вся, 
_е=диноро'дный сло'ве бж~iй, прише'лъ _е=си` на зе'млю за 
чл~вjьколю'бiе, чл~вjь'къ бы'въ непрело'женъ, и= кр\стъ 
и= сме'рть претерпjь'лъ _е=си` пло'тiю, безстра'стный 
бж~ество'мъ: воскр~съ же и=з\ъ ме'ртвыхъ, безсме'ртiе 
по'далъ _е=си` ро'ду человjь'ческому, jа='кw _е=ди'нъ 
всеси'ленъ.

%<И='ны стiхи^ры а=натw'лiевы, гла'съ то'йже.%>

%<Стi'хъ: Х%>вали'те _е=го` въ тv"мпа'нjь и= ли'цjь, 
хвали'те _е=го` во стру'нахъ и= _о=рга'нjь.

%<С%>ме'рть прiя'лъ _е=си` пло'тiю, на'мъ безсме'ртiе 
хода'тайствуя сп~се, и= во гро'бъ всели'лся _е=си`, да 
на'съ w\т а='да свободи'ши, воскр~си'въ съ собо'ю: 
пострада` о_у='бw jа='кw человjь'къ, но воскр~съ jа='кw 
бг~ъ. сегw` ра'ди вопiе'мъ: сла'ва тебjь` жизнода'вче 
гд\си, _е=ди'не чл~вjьколю'бче.

%<Стi'хъ: Х%>вали'те _е=го` въ кv"мва'лjьхъ 
доброгла'сныхъ, хвали'те _е=го` въ кv"мва'лjьхъ 
восклица'нiя: вся'кое дыха'нiе да хва'литъ гд\са.

%<К%>а'менiе распада'шеся сп~се, _е=гда` на ло'бнjьмъ 
кр\стъ тво'й водрузи'ся, о_у=страши'шася а='дwвы 
вра'тницы, _е=гда` во гро'бjь jа='кw ме'ртвъ положе'нъ 
бы'лъ _е=си`: и='бо сме'ртную о_у=праздни'вый крjь'пость, 
о_у=ме'ршымъ всjь^мъ нетлjь'нiе по'далъ _е=си` 
воскр\снiемъ твои'мъ сп~се, жизнода'вче гд\си, сла'ва 
тебjь`.

%<Стi'хъ: В%>оскр\сни` гд\си бж~е мо'й, да вознесе'тся 
рука` твоя`, не забу'ди о_у=бо'гихъ твои'хъ до конца`.

%<В%>озжелjь'ша ж_ены` ви'дjьти твое` воскр\снiе, 
хр\сте` бж~е, прiи'де предва'рши марi'а магдали'на, 
w=брjь'те ка'мень w\твале'нъ w\т гро'ба, и= а='гг~ла 
сjьдя'ща, и= глаго'люща: что` и='щете жива'го съ 
ме'ртвыми: воскр~се jа='кw бг~ъ, да сп~се'тъ вся'ч_еская.

%<Стi'хъ: И=%>сповjь'мся тебjь` гд\си, всjь'мъ 
се'рдцемъ мои'мъ, повjь'мъ вся^ чудеса` твоя^.

%<Г%>дjь` _е='сть i=и~съ, _е=го'же вмjьни'сте стрещи`, 
рцы'те i=уд_е'и; гдjь` _е='сть, _е=го'же положи'сте во 
гро'бjь, ка'мень (с. 527) запечатлjь'вше; дади'те 
ме'ртва, и=`же живота` w\тве'ргшiися: дади'те 
погребе'ннаго, и=ли` вjь'руйте воскр\сшему. а='ще и= вы` 
о_у=молчите` гд\сне воста'нiе, ка'менiе возопiе'тъ, па'че 
же w\твале'нный w\т гро'ба. вели'кая твоя` мл\сть, ве'лiе 
та'инство смотре'нiя твоегw` сп~се на'шъ, сла'ва тебjь`.

%<Сла'ва, стiхи'ра о_у='тренняя _е=v\гльская. И= 
ны'нjь, бг~оро'диченъ: П%>ребл~гослове'нна _е=си`: 
%<Славосло'вiе вели'кое. По славосло'вiи тропа'рь 
воскр\снъ:%>

%<В%>оскр~съ и=з\ъ гро'ба, и= о_у='зы растерза'лъ 
_е=си` а='да, разруши'лъ _е=си` w=сужде'нiе сме'рти 
гд\си, вся^ w\т сjьте'й врага` и=зба'вивый: jа=ви'вый же 
себе` а=п\слwмъ твои^мъ, посла'лъ _е=си` я=` на 
про'повjьдь, и= тjь'ми ми'ръ тво'й по'далъ _е=си` 
вселе'ннjьй, _е=ди'не многомл\стиве.

%<Та'же _е=кт_енiи`, и= w\тпу'стъ. Посе'мъ ча'съ 
пе'рвый, и= про'чее _о=бы'чное, и= коне'чный w\тпу'стъ.%>
