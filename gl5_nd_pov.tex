%[Въ суббw'ту на повече'рiи%],

глаго'летъ i=ере'й: Бл~гослове'нъ бг~ъ: И= мы`: Сла'ва 
тебjь` бж~е на'шъ, сла'ва тебjь`. Цр~ю` нб\сный: 
Трист~о'е. и= по _О='ч~е на'шъ: Гд\си поми'луй, в~_i. 
Сла'ва, и= ны'нjь: Прiиди'те поклони'мся: три'жды. 
_псало'мъ н~, и= про'чее _о=бы'чно.

Посе'мъ канw'нъ прест~jь'й бц\дjь, гла'съ _е~:

%[Пjь'снь а~%]

I=рмо'съ: Коня` и= вса'дника въ мо'ре чермно'е, 
сокруша'яй бра^ни мы'шцею высо'кою, хр\сто'съ и=стрясе`: 
i=и~ля же сп~се`, побjь'дную пjь'снь пою'ща. 

Припjь'въ: Прест~а'я бц\де сп~си` на'съ.

Пjь'снь ти` по достоя'нiю приноси'ти, вл\дчце, 
недоумjь'емъ вси`: сла'ва бо па'че всjь'хъ твоя`. но 
_о=ба'че бг~оневjь'сто, не w=мерзи` моле'нiе приноси'мое 
со стра'хомъ и= любо'вiю тебjь`.

Вси` къ водjь` твоегw` неwску'днагw и=сто'чника, бц\де 
дв~о, притека'емъ зову'ще: _е=ди'на ра'досте ро'ду 
на'шему, преч\стая, и=спроси` ми'ръ цр~квамъ твои^мъ.

Сла'ва: Положи' тя приста'нище су'щымъ въ бjьда'хъ, 
ч\стая, бг~ъ, и='же w\т тебе` бл~говоли'вый воспрiя'ти 
пло'ть. тjь'мже ти` припа'дающе взыва'емъ: да'руй твои^мъ 
рабw'мъ твою` по'мощь.

И= ны'нjь: W=блегче'нiе твоя` мл~тва, дв~о преч\стая, 
твои^мъ рабw'мъ да бу'детъ, страсте'й w\тгна'нiе, 
грjьхw'въ разруше'нiе, и= вся'ческихъ болjь'зней, 
бг~ороди'тельнице, w=чище'нiе.

%[Пjь'снь г~%]

I=рмо'съ: Водрузи'вый на ничесо'мже зе'млю 
повелjь'нiемъ твои'мъ, и= повjь'сивый неwдержи'мw 
тяготjь'ющую, на недви'жимjьмъ хр\сте`, ка'мени 
за'повjьдей твои'хъ, цр~ковь твою` о_у=тверди`, _е=ди'не 
бл~же и= чл~вjьколю'бче.

Ты` _е=си` земноро'дныхъ о_у=пова'нiе, и= по'мощь и= 
ра'дость, покро'въ и= прибjь'жище, вл\дчце мт~и живота`. 
тjь'мже тя` мо'лимъ: твою` по'мощь низпосли` всjь^мъ 
пою'щымъ тя`, преч\стая.

Неду'гующiи, и= напа'стьми лю'тыми w=держи'мiи, 
вседjь'телю ще'дрый, и='же вся'ческихъ бж~е, преч\стую 
сjь'нь твою` мт~рь сп~се, на мольбу` ти` приво'димъ: 
разрjьши` пл_ени'цы прегрjьше'нiй на'шихъ.

Сла'ва: Всjь'хъ содержи'теля и= зижди'теля и= гд\са 
ро'ждшая _е=ди'на, и= дв~ствующи, и=`же вои'стинну тя` 
бж~iю мт~рь сла'вятъ, сп~се'нiе бг~оневjь'сто, пода'ждь 
свы'ше рабw'мъ твои^мъ.

И= ны'нjь: Во'дъ животво'рныхъ мя` и=спо'лни вл\дчце, 
бж~е'ственную во'ду мi'рови и=сточи'вшая: и= беззако'нiй 
мои'хъ лю^тыя пото'ки и= се'рдца моегw` во'лны 
бж~е'ственною твое'ю тишино'ю о_у=кроти`.

%[Пjь'снь д~%]

I=рмо'съ: Бж~е'ственное твое` разумjь'въ и=стоща'нiе, 
прозорли'вw а=вваку'мъ, хр\сте`, со тре'петомъ вопiя'ше 
тебjь`: во сп~се'нiе люде'й твои'хъ сп~сти` пома^занныя 
твоя^ прише'лъ _е=си`.

Jа='кw всjь'хъ вы'шшу су'щую тва'рей, тя` бг~ъ су'щымъ 
на земли` дарова`, _е='же къ нему` хода'тайствы вино'вну, 
бц\де препjь'тая.

Цр~ковь тя` бж~iю вjь'дуще дв~о мт~и, прилjь'жнw 
мо'лимся тебе` чту'щiи: не затвори` рабw'мъ твои^мъ 
бц\де, твоея` мл\сти двере'й.

Сла'ва: Тя` вси` бж~iю вjь'дуще преукра'шену 
вои'стинну _о=де'жду, мт~и браконеиску'сная: и=`же тя` 
чту'щiи про'симъ, во w=ставле'нiя _о=де'жду w=блецы` 
на'съ. 

И= ны'нjь: Мi'ръ въ рж\ствjь` твое'мъ преч\стая, ве'сь 
и=спо'лнися ра'дости: w\тне'лjьже, _е='же ра'дуйся, 
вели'кiй тебjь` дв~о мт~и мр~i'е, свы'ше гаврiи'лъ 
возгласи`.

%[Пjь'снь _е~%]

I=рмо'съ: W=дjья'йся свjь'томъ jа='кw ри'зою, къ 
тебjь` о_у='тренюю, и= тебjь` зову`: ду'шу мою` 
просвjьти` w=мраче'нную хр\сте`, jа='кw _е=ди'нъ 
бл~гоутро'бенъ.

Вои'стинну наказу'емся, не по ра'венству же до'лга 
согрjьше'нiй на'шихъ: но, _w дв~о мт~и преч\стая, 
w\тврати` ве'сь гнjь'въ сн~а твоегw` w\т на'съ.

_Е=ди'наго и=зве'дшаго свjь'тъ w\т тьмы` ч\стая, бг~а 
ро'ждши неискусобра'чнw, прилjь'жнw моли` _е=го`, 
низпосла'ти бж~е'ственный свjь'тъ рабw'мъ твои^мъ.

Сла'ва: W\т каже'нiя мл~твы твоея` ч\стая, и= 
бж~е'ственная невjь'сто w\т лiва'на, ю='же соломw'нъ 
прорече`, рабы^ твоя^ мт~и зижди'теля, бл~гоуха'й.

И= ны'нjь: Ты` пра'вду же и= и=збавле'нiе на'мъ 
ро'ждши, хр\ста` без\ъ сjь'мене, свобо'дно содjь'яла 
_е=си` бц\де, w\т кля'твы _е=стество` пра'дjьднее.

%[Пjь'снь s~%]

I=рмо'съ: Неи'стовствующееся бу'рею душетлjь'нною, 
вл\дко хр\сте`, страсте'й мо'ре о_у=кроти` и= w\т тли` 
возведи` мя`, jа='кw бл~гоутро'бенъ.

Бц\де вл\дчце, ро'ждшая содjь'теля, твои^мъ рабw'мъ 
и=спроси` w=ставле'нiе: и= держа'вны на'съ воздви'гни, во 
_е='же пjь'ти тя`.

Бу'ди на'мъ по'мощь твои^мъ рабw'мъ, вл\дчце ч\стая 
вjь'рнw моля'щымъ тя`, jа='кw мл\стива: и= держа'вны 
на'съ воздви'гни, во _е='же пjь'ти тя`.

Сла'ва: По достоя'нiю и=му'щи, _е='же мощи` вл\дчце 
ч\стая, твои'мъ мл\стивнымъ _о='комъ при'зри, и= w\т тли` 
на'съ твоя^ рабы^ возведи`.

И= ны'нjь: Непреста'ннw точа'щи щедро'тъ струи^ 
бл~га'я прося'щымъ, w=дожди` и= мнjь` свjь'тъ за'повjьдей 
твоегw` сн~а, пренепоро'чная.

Та'же, Гд\си поми'луй, три'жды. 

Сла'ва, и= ны'нjь, сjьда'ленъ, гла'съ _е~:

Всест~а'я дв~о, поми'луй на'съ прибjьга'ющихъ вjь'рою 
къ тебjь` мл\срдой, и= прося'щихъ те'плагw твоегw` 
заступле'нiя: мо'жеши бо всjь'хъ сп~сти`, jа='кw бл~га'я 
су'щи мт~и бг~а вы'шнягw, мт~рними твои'ми мл~твами 
при'снw w=б\ъе'мши дв~о бг~ора'дованная.

%[Пjь'снь з~%]

I=рмо'съ: Превозноси'мый _о=тц_е'въ гд\сь пла'мень 
о_у=гаси`, _о='троки w=роси` согла'снw пою'щыя: бж~е 
бл~гослове'нъ _е=си`.

Неизслjь'дная бж~iя му'дросте хр\сте`, рабы^ твоя^ 
о_у=ще'дри, ро'ждшiя тя` ра'ди, непреста'ннw пою'щыя: 
бж~е, бл~гослове'нъ _е=си`.

Твою` бл~гость мо'лимъ гд\си, jа='зву и=сцjьли` 
ро'ждшiя тя` ра'ди со стра'хомъ пою'щихъ: бж~е 
бл~гослове'нъ _е=си`.

Сла'ва: _О='комъ мл\стивнымъ твои'мъ бг~ома'ти при'зри 
и= и=зба'ви рабы^ твоя^ вся'кагw w=бстоя'нiя, вjь'рою 
пою'щыя: бж~е, бл~гослове'нъ _е=си`.

И= ны'нjь: Sло` дjь'лающе, вл\дчце, w\т тебе` 
w\тпадо'хомъ: но w=брjьто'хомъ преч\стая, а='бiе по'мощь 
твою`, внегда` зва'ти: бж~е, бл~гослове'нъ _е=си`.

%[Пjь'снь и~%]

I=рмо'съ: И=з\ъ _о=ц~а` пре'жде вjь^къ рожде'ннаго 
сн~а и= бг~а, и= въ послjь^дняя лjь^та воплоще'ннаго w\т 
дв~ы мт~ре, сщ~е'нницы по'йте, лю'дiе превозноси'те во 
вся^ вjь'ки.

Вл\дчце на'ша, бл~ги'хъ пода'тельнице, рабw'мъ твои^мъ 
да'руй страсте'й и=сцjьле'нiе: jа='кw да непреста'ннw 
пое'мъ тя` дв~о, и= превозно'симъ во вjь'ки.

Неизглаго'ланнw ч\стая, и=зба'вителя ро'ждши, 
несказа'ннw дои'ла _е=си` дв~а пребы'вши. _е=го'же 
о_у='бw моли` w= пою'щихъ тя`, и= славосло'вящихъ во вся^ 
вjь'ки.

Сла'ва: Тебjь` свjь'тлому и=зба'вителя свjь'щнику, 
прекра'сный спле'тше ли'къ пое'мъ: вся^ дjьла` гд\сня 
по'йте непреста'ннw дв~у мр~i'ю, и= превозноси'те ю=` во 
вjь'ки.

И= ны'нjь: Ч\стая а='гнице, дв~о мт~и _о=трокови'це, 
чи'ста мя` сотвори` w\т страсте'й тjьле'сныхъ: jа='кw да 
льсти'вагw и=зба'влюся сjь'тей, пjьсносло'вя тя` 
бг~ора'дованная.

%[Пjь'снь f~%]

I=рмо'съ: И=са'iе лику'й, дв~а и=мjь` во чре'вjь, и= 
роди` сн~а _е=мману'ила, бг~а же и= человjь'ка, восто'къ 
и='мя _е=му`: _е=го'же велича'юще, дв~у о_у=бл~жа'емъ.

Соверша'ется пjь'нiе о_у='бw, наде'жда же вл\дко 
хр\сте` непрехо'дна, jа='же къ тебjь` содjь'телю, 
jа='коже и= бл~года'ть твоя`: но w\т_обою'ду си'лу 
крjь'пку дава'й рабw'мъ твои^мъ, мл~твами ро'ждшiя тя`.

Неду'гующымъ си'ла, и= боля'щымъ бли'зъ ты` ч\стая 
_е=си`, jа='кw и='стинная мт~и жи'зни: тjь'мже къ тебjь` 
прибjьга'юще, премjьне'нiе всjь'хъ ско'рбныхъ 
w=брjьто'хомъ, вл\дчце, и= кро'вомъ твои'мъ сп~со'хомся.

Сла'ва: Тво'й бж~е'ственный ви'дяще w=бра'знw вл\дчце 
зра'къ, зри'мъ тя` въ не'мъ jа='коже jа='вjь, вся'кw 
ненави'дяще _е=ретi^къ безу'мiе на земли`. _е=му'же 
припа'дающе и=сцjьле'нiе прiе'млемъ.

И= ны'нjь: И=сцjьле'нi_емъ бе'здну, и= бл~года'т_емъ 
пучи'ну, ч\стая позна'хомъ тя` мы` грjь'шнiи. тjь'мже ти` 
мо'лимся: w\т ну'ждныхъ всjь'хъ и=зми` _е=ди'на 
преч\стая, притека'ющихъ къ покро'ву твоему`.

Та'же, Досто'йно _е='сть: Трист~о'е. По _О='ч~е на'шъ: 
Конда'къ, и= про'чее _о=бы'чно: и= w\тпу'стъ.
