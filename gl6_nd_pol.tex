Въ недjь'лю о_у='тра, на полу'нощницjь,

канw'нъ прест~jь'й и= живонача'льнjьй тр\оцjь, 
[_е=гw'же краестро'чiе: Шесто'е пjь'нiе приноша'ю тебjь` 
бж~ество`.] Творе'нiе митрофа'ново. Гла'съ s~.

%[Пjь'снь а~%]

I=рмо'съ: Jа='кw по су'ху пjьшеше'ствовавъ i=и~ль по 
бе'зднjь стопа'ми, гони'теля фараw'на ви'дя потопля'ема, 
бг~у побjь'дную пjь'снь пои'мъ, вопiя'ше.

Припjь'въ: Прест~а'я тр\оце бж~е на'шъ, сла'ва тебjь`. 

Три` v=поста'си пое'мъ бг~онача'льныя, 
_е=ди'нственнагw _е=стества` неизмjь'нный зра'къ, 
бл~га'гw чл~вjьколю'бца бг~а, прегрjьше'нiй w=чище'нiе 
на'мъ да'рующа.

Пресу'щный _е=ди'не, и= трисiя'нный начерта'ньми 
гд\си, въ бж~ествjь` _е=ди'номъ сы'й, вразуми` на'съ, и= 
сподо'би твоегw` бж~е'ственнагw сiя'нiя.

Сла'ва: Невjьстоукраси'въ па'vелъ, ю='же w\т jа=зы^къ 
цр~ковь, _е=ди'ному тебjь` трiv"поста'сному бг~у 
покланя'тися научи`, w\т негw'же, и= и='мже, и= въ не'мже 
вся^ бы'ша.

И= ны'нjь, бг~оро'диченъ: И=з\ъ чре'ва твоегw` про'йде 
о_у='мное бц\де сл~нце, и= w=сiя` на'съ трисвjь'тлагw 
бж~ества` заря'ми: _е=го'же пою'ще, бл~гоче'стнw тя` 
о_у=бл~жа'емъ.

%[Пjь'снь г~%]

I=рмо'съ: Нjь'сть ст~ъ, jа='коже ты` гд\си бж~е мо'й, 
вознесы'й ро'гъ вjь'рныхъ твои'хъ бл~же, и= 
о_у=тверди'вый на'съ на ка'мени и=сповjь'данiя твоегw`.

О_у=краси'въ трисвjь'тлый бж~е чи'ны нб\сныя, 
о_у=стро'илъ _е=си` пjь'ти тя` трист~ы'ми гла'сы: съ 
ни'миже прiими` и= на'съ воспjьва'ющихъ твою` бл~гость. 

_Е=ди'но непрело'жное тр\очное, соwбра'зное 
_е=ди'нственное бг~онача'лiе пою'ще, мо'лимъ тя` те'плjь, 
грjьхw'въ мно'гихъ низпосла'ти ны'нjь на'мъ проще'нiе.

Сла'ва: О_у='ме безнача'льный _о='ч~е, соwбра'зный 
бж~iй сло'ве, и= дш~е бж~е'ственный, бл~гi'й и= пра'вый, 
воспjьва'ющыя вjь'рнw твою` держа'ву, соблюди` jа='кw 
бл~гоутро'бенъ.

И= ны'нjь, бг~оро'диченъ: Па'жить потреби` тли`, 
человjь'къ по существу` бы'въ бг~ъ мо'й, во о_у=тро'бjь 
твое'й ч\стая: и= родонача'льники пре'жднягw w=сужде'нiя 
_е=ди'нъ свободи`. 

Та'же, Гд\си поми'луй, три'жды. 

Сjьда'ленъ, гла'съ s~. Подо'бенъ: _Е='же w= на'съ:

Вл\дко бж~е, при'зри съ нб~се`, и= ви'ждь на'ше 
смире'нiе jа='кw ще'дръ, и= о_у=мл\срдися чл~вjьколю'бче 
пребл~гi'й: ни w\тку'ду бо надjь'емся проще'нiе получи'ти 
sлы'хъ, и='миже согрjьши'хомъ. тjь'мже бу'ди съ на'ми, и= 
никто'же на ны`.

Сла'ва, и= ны'нjь, бг~оро'диченъ: Вл\дчце ч\стая, 
при'зри бц\де, ви'ждь на'шихъ jа='звъ болjь^зни, и= 
о_у=мл\срдися преч\стая, и= и=сцjьли` со'вjьстное 
жже'нiе, твое'ю мл\стiю w=роша'ющи, и= вопiю'щи рабw'мъ 
твои^мъ: а='зъ _е='смь съ ва'ми, и= никто'же на вы`.

%[Пjь'снь д~%]

I=рмо'съ: Хр\сто'съ моя` си'ла, бг~ъ и= гд\са, 
ч\стна'я цр~ковь, бг~олjь'пнw пое'тъ взыва'ющи, w\т 
смы'сла чи'ста w= гд\сjь пра'зднующи.

Возвыша'яй мы'сль, _е=ди'нице трисвjь'тлая, и= ду'шу 
и= се'рдце твои'хъ пjьв_е'цъ ско'рw возведи`, и= сiя'нiя 
твоегw` и= свjь'тлости сподо'би.

Претвори` и= преwбрази` w\т sло'бы мя` вся'кiя къ 
добродjь'тели, _е=ди'на неиз\ъwбраже'нная, и= 
неизмjь'нная тр\оце, и= твои'ми заря'ми просвjьти`.

Сла'ва: Помы'сливъ пре'жде, му'дрjь соста'вилъ _е=си` 
а='гг~лwвъ чи'ны, служи'тельныя твоея` бл~гости, 
трiv"поста'сная _е=ди'нице, съ ни'миже прiими` мою` 
хвалу`.

И= ны'нjь, бг~оро'диченъ: И='же _е=стество'мъ 
несозда'нный бг~ъ присносу'щный, созда'нное воспрiе'мъ 
чл~вjь'ческое _е=стество`, воз\ъwбрази` во ст~jь'й твое'й 
о_у=тро'бjь, бц\де приснодв~о.

%[Пjь'снь _е~%]

I=рмо'съ: Бж~iимъ свjь'томъ твои'мъ бл~же, 
о_у='тренюющихъ ти` ду'шы любо'вiю w=зари`, молю'ся: тя` 
вjь'дjьти сло'ве бж~iй, и='стиннаго бг~а, w\т мра'ка 
грjьхо'внагw взыва'юща.

Помышля'юще _е=стество` бг~онача'льное, 
промысли'тельное и= сп~си'тельное всjь'хъ су'щее вл\дко, 
трисвjь'тлое же _е=ди'но, къ тебjь` о_у='тренюемъ, 
проще'нiя прося'ще грjьхопаде'нiй.

_О='ч~е безнача'льный бж~е, и= соприсносу'щный сн~е, 
и= дш~е ст~ы'й, о_у=тверди` _е=динонача'льная тр\оце 
твоя^ пjьвцы`, и= w\т вся'кiя напа'сти и=зба'ви и= 
ско'рби.

Сла'ва: О_у=правля'яй сiя'ньми бг~одjь'тельными, и= къ 
бл~гоугожде'нiю твоегw` трiv"поста'снагw бж~ества`, 
со'лнце сла'вы, наставля'я мя` при'снw, и= бж~е'ственнагw 
цр\ствiя сотвори` прича'стника.

И= ны'нjь, бг~оро'диченъ: И='же вся^ нося'й и= 
соблюда'яй всеси'льною твое'ю руко'ю сло'ве бж~iй 
неизмjь'нне, сохрани` и= соблюди' тя сла'вящыя, мл~твами 
ро'ждшiя тя` бг~ома'тере.

%[Пjь'снь s~%]

I=рмо'съ: Жите'йское мо'ре воздвиза'емое зря` 
напа'стей бу'рею, къ ти'хому приста'нищу твоему` прите'къ 
вопiю' ти: возведи` w\т тли` живо'тъ мо'й, многомл\стиве.

Прему'дрость и= ра'зумъ, бг~онача'лiе трисвjь'тлое 
пjьвц_е'мъ твои^мъ да'руй, и= добро'ты луча'ми 
свjьтодjь'тельныя твоея` бл~гости, w=сiява'тися всjь'хъ 
сподо'би. [Два'жды.]

Сла'ва: Свjь'те нераздjь'льный по существу`, 
трисiя'нне, вседержи'тельне, непристу'пне, сердца` 
w=зари` вjь'рнw хва'лящихъ держа'ву твою`, и= къ 
бж~е'ственнjьй любви` впери`. 

И= ны'нjь, бг~оро'диченъ: Въ тя` всели'ся приснодв~о, 
jа='вjь вседержи'тель и= гд\сь всjь'хъ, и= _е=ди'ному 
трисiя'нному зра'ку бж~ества` человjь'ки покланя'тися 
научи`. 

Гд\си поми'луй, три'жды. 

Сjьда'ленъ, гла'съ s~: Подо'бенъ: _Е='же w= на'съ:

_О='ч~е и= сн~е со дх~омъ ст~ы'мъ, при'зри на ны` 
вjь'рою тебjь` покланя'ющыяся, и= сла'вящыя держа'ву 
твою` бл~гоутро'бне, со _о='гненными бре'ннiи, и=но'гw бо 
ра'звjь тебе` не вjь'мы, и= возопi'й пою'щымъ тя`: а='зъ 
_е='смь съ ва'ми, и= никто'же на вы`.

Сла'ва, и= ны'нjь, бг~оро'диченъ: При'зри на ны` 
всепjь'тая бц\де, возсiя'й просвjьще'нiе сердца'мъ 
w=мрач_е'ннымъ, и= w=зари` ста'до твое` преч\стая. 
_е=ли'кw бо хо'щеши и= мо'жеши, jа='кw мт~и су'щи 
зижди'теля твоегw`, и= возопi'й моля'щымъ тя`: а='зъ 
_е='смь съ ва'ми, и= никто'же на вы`.

%[Пjь'снь з~%]

I=рмо'съ: Росода'тельну о_у='бw пе'щь содjь'ла 
а='гг~лъ прп\дбнымъ _о=трокw'мъ, халд_е'и же w=паля'ющее 
велjь'нiе бж~iе мучи'теля о_у=вjьща` вопи'ти: 
бл~гослове'нъ _е=си` бж~е _о=т_е'цъ на'шихъ.

Крjь'пкую мнjь` мы'сль о_у=стро'й трисвjь'тлая 
начерта'ньми _е=ди'нице, _е='же храни'ти и= соблюда'ти 
бж~е'ств_енныя за'пwвjьди твоя^, и= всегда` пjь'ти тебjь` 
вjь'рнw: бл~гослове'нъ _е=си` бж~е _о=т_е'цъ на'шихъ.

Jа='кw то'ждествомъ _е=стества` пое'мый, неизрече'ннw 
_е=ди'нственный бж~е, ли'цы же тр\оцы нося` число`, 
соблюди` всjь'хъ на'съ w\т разли'чныхъ и=скуше'нiй и= 
w=бстоя'нiй.

Сла'ва: Соесте'ственна и= соприсносу'щна сла'вимъ, 
_е=ди'наго тя` по существу` бг~а, сво'йствы неслiя'ннjь 
v=поста'сными тр\оце, разли'чное про'стw предлага'юще, во 
зра'цjь непремjь'ннjьмъ jа='вjь.

И= ны'нjь, бг~оро'диченъ: Бг~ъ пресу'щный прiя'тъ, 
преч\стая, w\т чре'ва твоегw` ч\стагw на'ше 
чл~вjьколюбе'знjь смjьше'нiе jа='вjь, и= всjь'хъ научи` 
вопи'ти: бл~гослове'нъ _е=си` бж~е _о=т_е'цъ на'шихъ.

%[Пjь'снь и~%]

I=рмо'съ: И=з\ъ пла'мене прп\дбнымъ ро'су и=сточи'лъ 
_е=си`, и= пра'веднагw же'ртву водо'ю попали'лъ _е=си`: 
вся^ бо твори'ши хр\сте`, то'кмw _е='же хотjь'ти: тя` 
превозно'симъ во вся^ вjь'ки.

W=чище'нiе прегрjьше'нiй ско'рw пода'ждь ми`, и= 
страсте'й многоwбра'зныхъ и=збавле'нiе, соwбра'зная 
тр\оце, _е=ди'нице трiv"поста'сная: да тя` сла'влю во 
вся^ вjь'ки.

Воли'тель мл\сти и=звjь'ствованъ, jа='кw бг~ъ 
мл\стивъ, всjь'хъ поми'луй трисвjь'тлая _е=ди'нице и= 
тр\оце пребл~га'я, славосло'вящихъ твое` вели'чествiе.

Сла'ва: W\т свjь'та присносу'щнагw _о=ц~а`, свjь'та 
соприсносу'щна ро'ждшася сло'ва, со дх~омъ и=схо'днымъ 
свjь'томъ, вjь'рою сла'вимъ, и= превозно'симъ во вся^ 
вjь'ки.

И= ны'нjь, бг~оро'диченъ: Врача` человjь'кwмъ 
преч\стая, родила` _е=си` всеси'льнаго сло'ва, хр\ста` 
гд\са, прароди'т_ельныя jа='звы всjь'хъ и=сцjьля'юща, 
превознося'щихъ _е=го` во вjь'ки.

%[Пjь'снь f~%]

I=рмо'съ: Бг~а человjь'кwмъ не возмо'жно ви'дjьти, на 
него'же не смjь'ютъ чи'ни а='гг~льстiи взира'ти: тобо'ю 
же всеч\стая, jа=ви'ся человjь'кwмъ сло'во воплоще'нно. 
_е=го'же велича'юще, съ нб\сными во'и тя` о_у=бл~жа'емъ.

Зрjь'ти чи'нове херувi'мстiи вл\дко, добро'ты твоея` 
сла'вы не могу'ще, кри'лы покрыва'ющеся, непреста'ннw 
тр\очную пjь'снь вопiю'тъ, трiv"поста'сную твоегw` 
бг~онача'лiя _е=ди'нственнагw держа'ву сла'вяще.

Твоя^ сiя^нiя сл~нце незаходи'мое, твои'хъ рабw'въ 
пода'ждь сердца'мъ, и= просвjьти` ду'шы, и= и=зба'ви w\т 
мно'гихъ прегрjьше'нiй _е=ди'не всемл\стиве, и= 
трiv"поста'сне: и= нетлjь'нныя твоея` жи'зни на'съ 
сподо'би. 

Сла'ва: _Е='же свjь'тъ _е=диноче'стный, и= 
трисо'лнечный, и= свjьтодjь'тельный, бж~ество` су'щее 
w=зари` вjь'рою тя` пою'щихъ, и= w\т мра'чнагw и=зба'ви 
sлодjь'йства, и= сподо'би свjьтлjь'йшихъ твои'хъ 
селе'нiй, jа='кw пребл~гi'й.

И= ны'нjь, бг~оро'диченъ: Прему'дрjь человjь'ка 
пре'жде созда` сн~ъ тво'й дв~о, и= и=стлjь'вшаго w=бнови` 
тобо'ю всепjь'тая: и= бж~е'ственнагw свjь'та зари` 
невече'рнiя всjь'хъ и=спо'лни, тя` бц\ду и='стинну 
вjь'рою сла'вящихъ.

Посе'мъ припjь'въ григо'рiя сiнаи'та: Досто'йно 
_е='сть: И= про'чее полу'нощницы, пи'сано въ концjь` 
кни'ги сея`.
