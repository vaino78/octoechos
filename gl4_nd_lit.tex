%<На лiтургi'и бл~ж_е'нна, гла'съ д~:%>

%<Д%>ре'вомъ а=да'мъ рая` бы'сть и=зселе'нъ: дре'вомъ 
же кре'стнымъ разбо'йникъ въ ра'й всели'ся. _о='въ 
о_у='бw вку'шъ, за'повjьдь w\тве'рже сотво'ршагw: _о='въ 
же сраспина'емь, бг~а и=сповjь'да тая'щагося, помяни' мя, 
вопiя`, во цр\ствiи твое'мъ.

%<Стi'хъ: Б%>л~же'ни чи'стiи се'рдцемъ, jа='кw тi'и 
бг~а о_у='зрятъ.

%<В%>ознесы'йся на кр\стъ, сме'ртную разруши'вый 
си'лу, и= загла'дивый jа='кw бг~ъ _е='же на ны` 
рукописа'нiе гд\си, разбо'йниче покая'нiе и= на'мъ 
пода'ждь _е=ди'не чл~вjьколю'бче, вjь'рою служа'щымъ, 
хр\сте` бж~е на'шъ, и= вопiю'щымъ ти`: помяни` и= на'съ 
во цр\ствiи твое'мъ.

%<Стi'хъ: Б%>л~же'ни миротво'рцы, jа='кw тi'и сы'нове 
бж~iи нареку'тся.

%<Р%>укописа'нiе на'ше на кр\стjь` копiе'мъ раздра'лъ 
_е=си`, и= вмjьни'вся въ ме'ртвыхъ, та'мошняго мучи'теля 
связа'лъ _е=си`, и=зба'вивый всjь'хъ w\т о_у='зъ 
а='довыхъ воскр\снiемъ твои'мъ, и='мже просвjьти'хомся 
чл~вjьколю'бче гд\си, и= вопiе'мъ ти`: помяни` и= на'съ 
во цр\ствiи твое'мъ. (с. 528)

%<Стi'хъ: Б%>л~же'ни и='згнани пра'вды ра'ди, jа='кw 
тjь'хъ _е='сть цр\ство нб\сное.

%<Р%>аспны'йся и= воскр~сы'й jа='кw си'ленъ и=з\ъ 
гро'ба тридне'венъ, и= первозда'ннаго а=да'ма 
воскр~си'вый _е=ди'не безсме'ртне: и= мене` на покая'нiе 
w=брати'тися гд\си, сподо'би w\т всегw` се'рдца моегw`, 
и= те'плою вjь'рою при'снw взыва'ти ти`: помяни' мя сп~се 
во цр\ствiи твое'мъ.

%<Стi'хъ: Б%>л~же'ни _е=сте`, _е=гда` поно'сятъ ва'мъ, 
и= и=зжену'тъ и= реку'тъ вся'къ sо'лъ глаго'лъ, на вы` 
лжу'ще мене` ра'ди.

%<Н%>а'съ ра'ди и='же безстра'стенъ стра'стный бы'сть 
человjь'къ, и= во'лею на кр\стjь` пригвожде'йся, на'съ 
совоскр~си`, тjь'мже и= сла'вимъ со кр\сто'мъ стр\сть и= 
воскр\снiе, и='миже возсозда'хомся, и='миже и= 
сп~са'емся, взыва'юще: помяни` и= на'съ во цр\ствiи 
твое'мъ.

%<Стi'хъ: Р%>а'дуйтеся и= весели'теся, jа='кw мзда` 
ва'ша мно'га на нб~сjь'хъ.

%<В%>оскр\сшаго и=з\ъ ме'ртвыхъ, и= а='дову держа'ву 
плjьни'вшаго, и= ви'дима жена'ми мv"роно'сицами, 
ра'дуйтеся, глаго'лющаго, вjь'рнiи о_у=мо'лимъ, w\т 
и=стлjь'нiя и=зба'вити ду'шы на'шя, зову'ще всегда` 
разбо'йника бл~горазу'мнагw гла'сомъ къ нему`: помяни` и= 
на'съ во цр\ствiи твое'мъ.

%<Сла'ва, тр\оченъ: _О=%>ц~а`, и= сн~а, и= ст~а'го 
дх~а, вси` _е=диному'дреннw вjь'рнiи славосло'вити 
досто'йнw помо'лимся: _е=ди'нство бж~ества`, въ трiе'хъ 
су'щее v=поста'сехъ, неслiя'нно пребыва'ющее, про'сто, 
нераздjь'льно и= непристу'пно, и='мже и=збавля'емся 
_о='гненнагw муче'нiя.

%<И= ны'нjь, бг~оро'диченъ: М%>т~рь твою` хр\сте`, 
пло'тiю без\ъ сjь'мене ро'ждшую тя`, и= дв~у вои'стинну, 
и= по рж\ствjь` пребы'вшу нетлjь'нну, сiю` ти` приво'димъ 
въ моли'тву, вл\дко многомл\стиве, прегрjьше'нiй 
проще'нiе да'руй, всегда` вопiю'щымъ ти`: помяни` и= 
на'съ во цр\ствiи твое'мъ.

%<Прокi'менъ, гла'съ д~: Jа='%>кw возвели'чишася 
дjьла` твоя^ гд\си, вся^ прему'дростiю сотвори'лъ _е=си`. 
(с. 529) %<Стi'хъ: Б%>л~гослови` душе` моя` гд\са, гд\си 
бж~е мо'й, возвели'чился _е=си` sjьлw`. %<А=ллилу'iа: 
Н%>аляцы`, и= о_у=спjьва'й, и= цр\ствуй и='стины ра'ди, 
и= кро'тости, и= пра'вды. %<Стi'хъ: В%>озлюби'лъ _е=си` 
пра'вду, и= возненави'дjьлъ _е=си` беззако'нiе.

