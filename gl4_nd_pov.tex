%<%[Въ суббw'ту на повече'рiи%],%>

%<канw'нъ прест~jь'й бц\дjь. Гла'съ д~.%>

%<%[Пjь'снь а~%]%>

%<I=рмо'съ: %>Мо'ря чермну'ю пучи'ну невла'жными 
стопа'ми дре'внiй пjьшеше'ствовавъ i=и~ль, 
кр\стоwбра'зныма мwv"се'овыма рука'ма, а=мали'кову си'лу 
въ пусты'ни побjьди'лъ _е='сть.

%<Припjь'въ: П%>рест~а'я бц\де, сп~си` на'съ.

%<Jа='%>же _е=ди'на въ напа'стехъ и= ско'рбехъ 
защища'ющая, под\ъ кро'въ тво'й преч\стая, прибjьга'ющихъ 
те'плjь, прiими` jа='кw пребл~га'я, jа=`же w\т се'рдца 
мол_е'нiя.

%<П%>риста'нище невла'емое w=брjь'тъ тя` неразу'мный, 
напа^стныя же и= ну^ждныя прило'ги w\трjьва'я, 
бл~года'рное пою' ти воспjьва'нiе, бг~ому'жная 
роди'тельнице.

%<Сла'ва: М%>л\стивнымъ и= кро'ткимъ твои'мъ _о='комъ, 
бг~ороди'тельнице, зря'щи мя` во w=бстоя'нiи и= ско'рби 
w=держи'ма, вско'рjь свободи`: тя' бо призыва'ю на 
по'мощь. (с. 504)

%<И= ны'нjь: П%>реклоне'на мя` вл\дчце, скорбьми` 
ну'ждными лю'тjь, jа='кw мл\стива _е=ди'на бл~га'я 
предста'тельница рабw'въ твои'хъ, ру'ку мольбы` простри`, 
и= лю'тыхъ бjь'дъ и=зба'ви мя`.

%<%[Пjь'снь г~%]%>

%<I=рмо'съ: %>Лу'къ си'льныхъ и=знемо'же, и= 
немощству'ющiи препоя'сашася си'лою: сегw` ра'ди 
о_у=тверди'ся въ гд\сjь се'рдце мое`.

%<_О=%>ру'жiе крjь'пко, и= стjь'ну тя` стяжа'въ а='зъ, 
побjьжда'ю сопроти'вныхъ полки`, и= пою` вели^чiя твоя^, 
бц\де неискусобра'чная.

%<П%>е'щь разоря'еши печа'лей, и= погаша'еши 
w\тча'янiя зно'й: кто' бо та'кw, jа='коже ты` дв~о бц\де, 
о_у=пова'нiе на'ше;

%<Сла'ва: В%>нуши` гла'съ раба` твоегw`, твоея` 
тре'бующагw по'мощи, бг~омт~и: о_у=пова'нiе мое`, 
о_у=слы'ши мя`, и= напа'стей и=схити`.

%<И= ны'нjь: П%>рiи'де w\т прегрjьше'нiй мно'жества 
мучи'тельство на'мъ, нося` сме'рть па'губную: но сп~си` 
твоя^ рабы^ бц\де, ты' бо мо'жеши.

%<%[Пjь'снь д~%]%>

%<I=рмо'съ: %>Вознесе'на тя` ви'дjьвши цр~ковь на 
кр\стjь`, сл~нце пра'ведное, ста` въ чи'нjь свое'мъ, 
досто'йнw взыва'ющи: сла'ва си'лjь твое'й гд\си.

%<П%>обjь'ждши вражду'ющихъ мнjь` всу'е, jа='кw ду'шу 
мою` тща'щихся лю'тjь прiя'ти, сохрани' мя вл\дчце, 
поми'луй, и= сп~си`: къ тебjь' бо прибjьга'ю ра'бъ тво'й.

%<И=%>збавля'ющи мя` w\т jа=зы'ка льстивоглаго'лива, 
бл~га'я засту'пнице моя`, безпа'костна покажи`, и= 
жите'йскихъ дjья'нiй: мно'гw бо мо'жеши, jа='кw 
зижди'теля мт~и су'щи.

%<Сла'ва: Б%>езболjь'зненну вjь'дый тя` цjьле'бницу 
немощны'й, дх~омъ и= о_у=сты` зову`: и=сцjьли' мя 
вл\дчце, поми'луй, и= сп~си`: къ тебjь' бо прибjьга'ю 
ра'бъ тво'й. (с. 505)

%<И= ны'нjь: Н%>е w=ста'ви мене` напа'стемъ пре'дану 
бы'ти, мт~и бг~а на'шегw, но w\т вся'кiя ско'рби и= 
sло'бы человjь'чи сохрани` неврежде'на: ты' бо _е=си` 
помо'щница всjь^мъ на'мъ.

%<%[Пjь'снь _е~%]%>

%<I=рмо'съ: Т%>ы` гд\си мо'й, свjь'тъ въ мi'ръ 
прише'лъ _е=си`, свjь'тъ ст~ы'й, w=браща'яй и=з\ъ мра'чна 
невjь'дjьнiя, вjь'рою воспjьва'ющыя тя`.

%<И=%>спра'ви чи'стая, моли'тву раба` твоегw` ко гд\су 
сн~у твоему`: да w=бря'щу разрjьше'нiе мно'гихъ мои'хъ 
прегрjьше'нiй.

%<И=%>зба'ви мя` страсте'й и= бjь'дъ, бг~оневjь'сто: 
тя' бо положи` бг~ъ w=чище'нiе вои'стинну моему` 
смире'нiю.

%<Сла'ва: П%>окро'въ мо'й ты` _е=си`, и= при'сное 
хвале'нiе, _w вл\дчце бц\де! ника'коже бо презира'еши къ 
тебjь` прибjьга'ющихъ.

%<И= ны'нjь: П%>оми'луй ч\стая, чту'щихъ рж\ство` 
твое`, и= и=зба'ви мучи'тельства, и= го'рести 
человjь'ческiя: и='бо и='маши _е='же мощи`.

%<%[Пjь'снь s~%]%>

%<I=рмо'съ: %>Пожру' ти со гла'сомъ хвале'нiя гд\си, 
цр~ковь вопiе'тъ ти`, w\т бjьсо'вскiя кро'ве w=чи'щшися, 
ра'ди мл\сти w\т ре'бръ твои'хъ и=сте'кшею кро'вiю.

%<К%>рjь'пость ми` сама` _е=си` преч\стая вл\дчце, во 
w=брjь'тшихъ мя` sjьлw` ненача'янныхъ ско'рбехъ, и= 
вопiю' ти: jа='кw ве'лiя _е=си` покрови'тельница рабу` 
твоему`.

%<И=%>сцjьли` душ_е'вныя моя^ jа='звы преч\стая, 
заступи' мя дв~о, и= и=зба'ви раба` твоего` w\т 
w=клевета'нiя, навjь'та же и= развраще'нiя непра'ведна.

%<Сла'ва: С%>окруши` на мя`, при'снw прибjьга'ющаго къ 
тебjь`, навjь'тники непра'в_едныя, и= не w=ста'ви мя` 
поги'бнути: jа='кw вся^ тебjь` возмw'жна ч\стая, jа='кw 
бг~о_отрокови'цjь. (с. 506)

%<И= ны'нjь: П%>обjьди` души` моея` свирjь'пую во'лну, 
jа='кw мно'жество прегрjьше'нiй, напа'стей и= скорбе'й, 
вл\дчце, воста'ша на мя`, но сама' мя сп~си`.

%<Г%>д\си поми'луй, %<три'жды, Сла'ва, и= ны'нjь:%>

%<Сjьда'ленъ, гла'съ д~:%>

%<М%>но'гими прегрjьше'нiй а='зъ блу'дный о_у='мъ 
помрачи'въ, вопiю` твоему` крjь'пкому заступле'нiю бц\де: 
просвjьти` души` моея` зjь^ницы, возсiя'й ми` покая'нiя 
свjь'тлую зарю`, и= w=блецы' мя во _о=ру'жiе свjь'та, 
бг~ороди'тельнице ч\стая.

%<%[Пjь'снь з~%]%>

%<I=рмо'съ: %>Сп~сы'й во _о=гни` а=враа^мскiя твоя^ 
_о='троки, и= халд_е'и о_у=би'въ, jа=`же пра'вда 
пра'веднw о_у=ловля'ше, препjь'тый гд\си бж~е _о=т_е'цъ 
на'шихъ, бл~гослове'нъ _е=си`.

%<И='%>же w\т а=га'рянъ наси'лованiе ско'рw 
потре'бльши мече'мъ мл~твъ твои'хъ мр~i'е, лю'ди и= 
ста'до твое` сохрани`, сн~у твоему` зову'щыя: бж~е 
_о=т_е'цъ на'шихъ, бл~гослове'нъ _е=си`.

%<Р%>авнолjь'пная ски'нiя, прiими' мя къ тебjь` 
прибjьга'ющаго, да не прiи'метъ мя` вра'гъ погуби'ти 
хотя`, зову'щаго: превозноси'мый _о=т_е'цъ на'шихъ бж~е, 
бл~гослове'нъ _е=си`.

%<Сла'ва: Б%>г~ороди'тельнице мр~i'е, предвари` раба` 
твоего` вско'рjь, въ треволне'нiихъ напа'стей 
потопля'емаго, не и=му'щаго по'мощи, къ тебjь' же 
зову'ща: о_у=пова'нiе конц_е'въ, поми'луй мя`.

%<И= ны'нjь: Ч%>еловjь'ч_ескiя по'мыслы jа='кw 
грjьхw'мъ вино'вны, ны'нjь бц\де бл~га'я разори`, 
бж~е'ственными твои'ми мл~твами, и= рабы^ твоя^ и=зба'ви 
болjь'зненныя напа'сти и= вся'кагw вре'да.

%<%[Пjь'снь и~%]%>

%<I=рмо'съ: И=%>зба'вителю всjь'хъ всеси'льне, 
посредjь` пла'мене бл~гоче'ствовавшыя, снизше'дъ 
w=роси'лъ _е=си`, и= научи'лъ _е=си` пjь'ти: вся^ дjьла` 
бл~гослови'те, по'йте гд\са. (с. 507)

%<Н%>а'йде на ны` jа=зы'къ беззако'ненъ, хваля'ся 
погуби'ти служи'тели твоя^: _е=го'же потре'бльши 
преч\стая, покры'й взыва'ющыя: вся^ дjьла` бл~гослови'те 
гд\сня гд\са.

%<М%>нw'гiя твоя^ щедрw'ты мл\стивнw на'съ сп~са'ютъ, 
_е=ди'на бг~омт~и, w\т грjьхо'внагw суда`, и= разли'чныхъ 
напа'стей: ты' бо ро'ждши бг~а, ми'луеши мi'ръ _е=гw`.

%<Сла'ва: Jа='%>кw ты` _е=си` крjь'пость и= по'мощь, 
не бою'ся врагw'въ негодова'нiя, но пою' тя вл\дчце, и= 
вопiю` сн~у твоему`: бл~гослови'те вся^ дjьла` гд\сня 
гд\са.

%<И= ны'нjь: Н%>а мольбу` мою` ны'нjь о_у=мл\срдися, 
и= ра'дость въ печа'ли мjь'сто да'руй ми`: да пою' тя 
вл\дчце, и= вопiю` сн~у твоему`: бл~гослови'те вся^ 
дjьла` гд\сня гд\са.

%<%[Пjь'снь f~%]%>

%<I=рмо'съ: _Е='%>vа о_у='бw неду'гомъ преслуша'нiя 
кля'тву всели'ла _е='сть: ты' же дв~о бц\де, 
прозябе'нiемъ чревоноше'нiя мi'рови бл~гослове'нiе 
процвjьла` _е=си`, тjь'мъ тя` вси` велича'емъ.

%<_О=%>ру'жiе о_у='бw на ны` w=бостри'въ, 
совjьщава'етъ льсти'вый а=ра'влянинъ беззако'нный: ты' же 
дв~о бц\де, си'лою кр\ста` и= мл~твъ твои'хъ, воwружа'еши 
на него` рабы^ твоя^. тjь'мже проповjь'даемъ сла'ву 
твою`.

%<К%>рjь'пость тебjь` на враги` даде'ся вл\дчце, и= 
и=збавле'нiе ми` w\т бjь'дъ: что' же а='зъ тебjь` 
принесу`, не довjь'мъ. _о=ба'че, _е='же и='мамъ, 
бл~годаре'нiе приношу' ти: прiими` сiе` ны'нjь, и= сп~си' 
мя.

%<Сла'ва: _W%> мт~и вся'ческихъ творца` всесвjь'тлая, 
печа'льныхъ о_у=тjь'ха, потопля'емыхъ предста'тельница, 
и= пренемога'ющихся засту'пница, до живота` моегw` ты' мя 
сохрани`.

%<И= ны'нjь: О_у=%>тjьсня'ема мя` грjьхи` мно'гими, и= 
бjьда'ми, не пре'зри мене` всепjь'тая ны'нjь, тебjь` 
хвале'нiя же'ртву приношу`, прилjь'жнw взыва'я ти`: ста'я 
бц\де, помози' ми, тя' бо сла'вя пjь'снь скончава'ю.

%<Та'же, Д%>осто'йно _е='сть: %<Трист~о'е, и= про'чее 
_о=бы'чно, и= w\тпу'стъ.%> (с. 508)
