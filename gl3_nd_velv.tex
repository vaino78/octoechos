%<%[Въ суббw'ту на вели'цjьй вече'рни%],%>

%<на Г%>д\си воззва'хъ %<поста'вимъ стiхw'въ _i~, и= 
пое'мъ стiхи^ры воскр\сны, гла'съ г~.%>

%<Стi'хъ: И=%>зведи` и=з\ъ темни'цы ду'шу мою`, 
и=сповjь'датися и='мени твоему`.

%<Т%>вои'мъ кр\сто'мъ хр\сте` сп~се, сме'рти держа'ва 
разруши'ся, и= дiа'воля пре'лесть о_у=праздни'ся: ро'дъ 
же человjь'ческiй вjь'рою сп~са'емый: пjь'снь тебjь` 
всегда` прино'ситъ.

%<Стi'хъ: М%>ене` жду'тъ пра'в_едницы, до'ндеже 
возда'си мнjь`.

%<П%>росвjьти'шася вся'ч_еская воскр\снiемъ твои'мъ 
гд\си, и= ра'й па'ки w\тве'рзеся: вся' же тва'рь 
восхваля'ющи тя`, пjь'снь тебjь` всегда` прино'ситъ.

%<Стi'хъ: И=%>з\ъ глубины` воззва'хъ къ тебjь` гд\си, 
гд\си, о_у=слы'ши гла'съ мо'й.

%<С%>ла'влю _о=ц~а` и= сн~а си'лу, и= ст~а'гw дх~а 
пою` вла'сть, нераздjь'льное, несозда'нное бж~ество`, 
тр\оцу _е=диносу'щную, ца'рствующую въ вjь'къ вjь'ка.

%<И='ны стiхи^ры а=нато'лiевы, гла'съ то'йже.%>

%<Стi'хъ: Д%>а бу'дутъ о_у='ши твои` вне'млющjь гла'су 
моле'нiя моегw`.

%<К%>р\сту` твоему` честно'му покланя'емся хр\сте`, и= 
воскр\снiе твое` пое'мъ и= сла'вимъ: ра'ною бо твое'ю мы` 
вси` и=сцjьлjь'хомъ.

%<Стi'хъ: А='%>ще беззакw'нiя на'зриши гд\си, гд\си, 
кто` постои'тъ; jа='кw о_у= тебе` w=чище'нiе _е='сть.

%<П%>ое'мъ сп~са w\т дв~ы вопло'щшагося: на'съ бо 
ра'ди распя'тся, и= въ тре'тiй де'нь воскр~се, да'руя 
на'мъ ве'лiю мл\сть.

%<Стi'хъ: И='%>мене ра'ди твоегw` потерпjь'хъ тя` 
гд\си, потерпjь` душа` моя` въ сло'во твое`, о_у=пова` 
душа` моя` на гд\са.

%<С%>у'щымъ во а='дjь соше'дъ хр\сто'съ благовjьсти`: 
дерза'йте, глаго'ля, ны'нjь побjьди'хъ, а='зъ _е='смь 
воскр\снiе, а='зъ вы` возведу`, разруши'въ см_е'ртная 
врата`. (с. 348) 

%<Стi'хъ: W\т%> стра'жи о_у='треннiя до но'щи, w\т 
стра'жи о_у='треннiя, да о_у=пова'етъ i=и~ль на гд\са.

%<Н%>едосто'йнw стоя'ще въ пречи'стомъ дому` твое'мъ, 
вече'рнюю пjь'снь возсыла'емъ, и=з\ъ глубины` взыва'юще 
хр\сте` бж~е: просвjьти'вый мi'ръ тридне'внымъ 
воскр\снiемъ твои'мъ, и=зми` лю'ди твоя^ w\т руки` 
врагw'въ твои'хъ чл~вjьколю'бче.

%<И='ны стiхи^ры прест~jь'й бц\дjь. Творе'нiе па'vла 
а=морре'йскагw, пое'мъ и=`хъ, и=дjь'же нjь'сть мине'и. 
Гла'съ з~.%>

%<Подо'бенъ: Д%>не'сь бди'тъ i=у'да:

%<Стi'хъ: Jа='%>кw о_у= гд\са мл\сть, и= мно'гое о_у= 
негw` и=збавле'нiе: и= то'й и=зба'витъ i=и~ля w\т всjь'хъ 
беззако'нiй _е=гw`.

%<М%>и'лостива, благопребы'тна мнjь` дв~о, и= 
благопослу'шна jа=ви'ся призыва'ющу бж~е'ственную твою` 
бл~года'ть во всjь'хъ вои'стинну случа'ющихъ ми ся: всю' 
бо наде'жду души` моея` возложи'хъ къ тебjь`, и= во 
всjь'хъ о_у=пова'ю на бж~е'ственный про'мыслъ тво'й, ты` 
и= бу'дущiя мя` сла'вы, и= живота` бж~е'ственнагw 
сподо'би.

%<Стi'хъ: Х%>вали'те гд\са вси` jа=зы'цы, похвали'те 
_е=го` вси` лю'дiе.

%<О_у='%>глiе во мнjь` бц\де страсте'й мои'хъ 
возгорjь'шася, w\т гнjь'ва же и= jа='рости, w\т пiя'нства 
и= блуда`, w\т сребролю'бiя и= же'стости се'рдца, 
о_у=мерщвле'нiя лю'тагw, w\т о_у=ны'нiя и= смуще'нiя, w\т 
тщесла'вiя же и= попра'нiя со'вjьстнагw, w\т ни'хже 
и=зба'ви ду'шу мою`, молю'ся, вл\дчце, и= спаси' мя.

%<Стi'хъ: Jа='%>кw о_у=тверди'ся мл\сть _е=гw` на 
на'съ, и= и='стина гд\сня пребыва'етъ во вjь'къ.

%<В%>си` чи'стою со'вjьстiю бц\дjь припаде'мъ, 
вопiю'ще непреста'ннw w\т среды` серде'чныя: вл\дчце 
ст~а'я, сп~си` всjь'хъ w\т гнjь'ва и= w=sлобле'нiя, 
бjь'дъ и= паде'жей: jа='кw тебе` стяжа'хомъ стjь'ну, и= 
о_у=твержде'нiе, тобо'ю сп~са'еми под\ъ кро'въ тво'й 
притека'юще.

%<Сла'ва, и= ны'нjь, бг~оро'диченъ: К%>а'кw не 
диви'мся бг~ому'жному рж\ству` твоему`, преч\стна'я; 
и=скуше'нiя бо му'жескагw не прiе'мши всенепоро'чная, 
родила` _е=си` без\ъ _о=ц~а` сн~а (с. 349) пло'тiю 
пре'жде вjь^къ w\т _о=ц~а` рожде'ннаго без\ъ мт~ре, 
ника'коже претерпjь'вшаго и=змjьне'нiя, и=ли` смjьше'нiя, 
и=ли` раздjьле'нiя, но _о=бою` сущ_еству` сво'йство 
цjь'ло сохра'ншаго. тjь'мже мт~и дв~о вл\дчце, того` 
моли` сп~сти'ся душа'мъ, правосла'внw бц\ду 
и=сповjь'дающихъ тя`.

%<Та'же вхо'дъ съ кади'ломъ. С%>вjь'те ти'хiй: 
П%<рокi'менъ и= _е=кт_енiи`.%>

%<На стiхо'внjь стiхи^ры воскр\сны, гла'съ г~:%>

%<С%>тр\стiю твое'ю хр\сте`, w=мрачи'вый со'лнце и= 
свjь'томъ твоегw` воскр\снiя, просвjьти'вый вся'ч_еская, 
прiими` на'шу вече'рнюю пjь'снь чл~вjьколю'бче.

%<По а=лфави'ту:%>

%<Стi'хъ: %>Гд\сь воцари'ся, въ лjь'поту w=блече'ся.

%<Ж%>ивопрiе'мное твое` воста'нiе гд\си вселе'нную 
всю` просвjьти`, и= твое` созда'нiе и=стлjь'вшее призва`. 
тjь'мже кля'твы а=да'мовы и=змjь'ншеся, вопiе'мъ: 
всеси'льне гд\си, сла'ва тебjь`.

%<Стi'хъ: И='%>бо о_у=тверди` вселе'нную, jа='же не 
подви'жится.

%<Б%>г~ъ сы'й неизмjь'ненъ, пло'тiю стражда` 
и=змjьни'лся _е=си`, _е=го'же тва'рь не терпя'щи ви'сяща 
зрjь'ти, стра'хомъ прекланя'шеся, и= стеня'щи пое'тъ 
твое` долготерпjь'нiе: соше'дъ же во а='дъ, тридне'венъ 
воскре'слъ _е=си`, жи'знь да'руя мi'рови, и= ве'лiю 
мл\сть.

%<Стi'хъ: Д%>о'му твоему` подоба'етъ ст~ы'ня гд\си, въ 
долготу` днi'й.

%<Д%>а ро'дъ на'шъ w\т сме'рти хр\сте` и=зба'виши, 
сме'рть претерпjь'лъ _е=си`: и= тридне'венъ и=з\ъ 
ме'ртвыхъ воскре'съ, съ собо'ю воскреси'лъ _е=си`, и=`же 
тя` бг~а позна'вшихъ: и= мi'ръ просвjьти'лъ _е=си`. гд\си 
сла'ва тебjь`.

%<Сла'ва, и= ны'нjь, бг~оро'диченъ: Б%>ез\ъ сjь'мене 
w\т бж~е'ственнагw дх~а, во'лею же _о='ч~ею зачала` 
_е=си` сн~а бж~iя, w\т _о=ц~а` без\ъ мт~ре пре'жде вjь^къ 
су'ща: на'съ же ра'ди, и=з\ъ тебе` без\ъ _о=ц~а` бы'вша, 
пло'тiю родила` _е=си`, и= мл\днца млеко'мъ пита'ла 
_е=си`. тjь'мже не преста'й, моли'ти, и=зба'витися w\т 
бjь'дъ душа'мъ на'шымъ.

%<Та'же, Н%>ы'нjь w\тпуща'еши: Т%<рист~о'е. По 
_О='%>ч~е на'шъ: (с. 350)

%<Тропа'рь воскр\снъ, гла'съ г~:%>

%<Д%>а веселя'тся нб\сная, да ра'дуются земна^я: 
jа='кw сотвори` держа'ву мы'шцею свое'ю гд\сь, попра` 
сме'ртiю сме'рть, пе'рвенецъ ме'ртвыхъ бы'сть, и=з\ъ 
чре'ва а='дова и=зба'ви на'съ, и= подаде` мi'рови ве'лiю 
мл\сть.

%<Сла'ва, и= ны'нjь, бг~оро'диченъ: Т%>я` 
хода'тайствовавшую сп~се'нiе ро'да на'шегw, воспjьва'емъ 
бц\де дв~о: пло'тiю бо w\т тебе` воспрiя'тою сн~ъ тво'й, 
и= бг~ъ на'шъ, кр\сто'мъ воспрiи'мъ стр\сть, и=зба'ви 
на'съ w\т тли` jа='кw чл~вjьколю'бецъ.

