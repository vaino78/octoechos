%[Въ суббw'ту на вели'цjьй вече'рни%],

на Гд\си воззва'хъ, поста'вимъ стiхw'въ _i~ и= пое'мъ 
стiхи^ры воскре'сны, гла'съ _е~:

Стi'хъ: И=зведи` и=з\ъ темни'цы ду'шу мою`, 
и=сповjь'датися и='мени твоему`.

Ч\стны'мъ твои'мъ кр\сто'мъ хр\сте`, дiа'вола 
посрами'лъ _е=си`, и= воскр\снiемъ твои'мъ жа'ло 
грjьхо'вное притупи'лъ _е=си`, и= сп~слъ _е=си` ны` w\т 
вра'тъ сме'ртныхъ: сла'вимъ тя` _е=диноро'дне. 

Стi'хъ: Мене` жду'тъ пра'в_едницы, до'ндеже возда'си 
мнjь`. 

Воскр\снiе дая'й ро'ду человjь'ческому, jа='кw _о=вча` 
на заколе'нiе веде'ся: о_у=страши'шася сегw` кня'зи 
а='дстiи, и= взя'шася врата` плач_е'вная. вни'де бо цр~ь 
сла'вы хр\сто'съ, глаго'ля су'щымъ во о_у='захъ, 
и=зыди'те: и= су'щымъ во тьмjь`, w\ткры'йтеся.

Стi'хъ: И=з\ъ глубины` воззва'хъ къ тебjь` гд\си, 
гд\си, о_у=слы'ши гла'съ мо'й.

Ве'лiе чу'до, неви'димыхъ содjь'тель, за 
чл~вjьколю'бiе пло'тiю пострада'въ, воскр~се 
безсме'ртный, прiиди'те _о=те'ч_ествiя jа=зы^къ, тому` 
поклони'мся: бл~гоутро'бiемъ бо _е=гw` w\т пре'лести 
и=зба'вльшеся, въ трiе'хъ v=поста'сjьхъ _е=ди'наго бг~а 
пjь'ти навыко'хомъ. 

И='ны стiхи^ры а=натw'лiевы, гла'съ то'йже.

Стi'хъ: Да бу'дутъ о_у='ши твои`, вне'млющjь гла'су 
моле'нiя моегw`. 

Вече'рнее поклоне'нiе прино'симъ тебjь` невече'рнему 
свjь'ту, на коне'цъ вjькw'въ, jа='кw въ зерца'лjь пло'тiю 
возсiя'вшему мi'рови, и= да'же до а='да низше'дшему, и= 
та'мw су'щую тьму` разруши'вшему, и= свjь'тъ воскр\снiя 
jа=зы'кwмъ показа'вшему: свjьтода'вче гд\си сла'ва 
тебjь`.

Стi'хъ: А='ще беззакw'нiя на'зриши гд\си, гд\си, кто` 
постои'тъ; jа='кw о_у= тебе` w=чище'нiе _е='сть.

Нача'льника сп~се'нiя на'шегw, хр\ста` славосло'вимъ: 
тому' бо и=з\ъ ме'ртвыхъ воскр\сшу, мi'ръ w= пре'лести 
сп~се'нъ бы'сть. ра'дуется ли'къ а='гг~льскiй, бjь'гаетъ 
де'монwвъ пре'лесть, а=да'мъ пады'й воста`, дiа'волъ 
о_у=праздни'ся.

Стi'хъ: И='мене ра'ди твоегw`, потерпjь'хъ тя` гд\си, 
потерпjь` душа` моя` въ сло'во твое`, о_у=пова` душа` 
моя` на гд\са.

И=`же w\т кустwдi'и науче'ни быва'ху w\т 
беззакw'нникъ, покры'йте хр\сто'во воста'нiе, и= 
прiими'те сре'бреники, и= рцы'те jа='кw на'мъ спя'щымъ, 
и=з\ъ гро'ба о_у=кра'денъ бы'сть ме'ртвый. кто` ви'дjь, 
кто` слы'ша, мертвеца` о_у=кра'дена когда`, па'че же 
пома'зана и= на'га, w=ста'вльша и= во гро'бjь 
погреба^льная своя^; не прельща'йтеся i=уде'_е, 
навы'кните рече'нi_емъ пр\оро'ч_ескимъ, и= 
о_у=разумjь'йте, jа='кw то'й _е='сть вои'стинну 
и=зба'витель мi'ра, и= всеси'льный.

Стi'хъ: W\т стра'жи о_у='треннiя до но'щи, w\т стра'жи 
о_у='треннiя, да о_у=пова'етъ i=и~ль на гд\са.

Гд\си, а='дъ плjьни'вый, и= сме'рть попра'вый сп~се 
на'шъ, просвjьти'вый мi'ръ кр\сто'мъ ч\стны'мъ, поми'луй 
на'съ. 

И='ны стiхи^ры, бц\дjь, па'vла а=морре'йскагw, гла'съ 
_е~:

Подо'бенъ: Ра'дуйся:

Стi'хъ: Jа='кw о_у= гд\са мл\сть, и= мно'гое о_у= 
негw` и=збавле'нiе, и= то'й и=зба'витъ i=и~ля w\т всjь'хъ 
беззако'нiй _е=гw`.

Пр\сто'лъ херувi'мскiй вои'стинну, jа='кw превы'шши 
тва'рей бы'вши: въ тебе' бо бж~iе сло'во, на'шъ зра'къ 
назда'сти хотя`, всели'ся, и= проше'дъ съ пло'тiю и=з\ъ 
тебе` всеч\стая: кр\стную же стр\сть на'съ ра'ди 
воспрiя'тъ, и= воскр\снiе jа='кw бг~ъ дарова`, 
и=змjь'ншемуся на'шему w=сужде'нному _е=стеству`. тjь'мъ 
jа='кw содjь'телю, сн~у твоему` мо'лимся бг~ома'ти, 
о_у=лучи'ти проще'нiе и= мл\сть въ ча'съ су'дный. 

Стi'хъ: Хвали'те гд\са вси` jа=зы'цы, похвали'те 
_е=го` вси` лю'дiе. 

Что` твою` бг~ороди'тельнице ч\стая, нареку` цр~ковь 
бг~осла'вную; вертогра'дъ _е=де'мскiй и=мену'ю, и= 
нw'евъ, ч\стая, ковче'гъ прореку`, сп~сшiй бг~у цр\ское 
сщ~е'нiе, ве'сь ст~ъ jа=зы'къ, хр\ста` бг~а на'шегw 
собо'ръ: мwv"се'ову же кiвw'ту о_у=подобля'ю тя`, въ 
не'мже w=чисти'лище и= же'злъ прозя'бшiй, въ не'мже 
свjь'щникъ и= ру'чка, кади'льница всезлата'я, во'ньже 
прибjьга'етъ вся'къ вjь'рный, и= и=спроша'етъ ве'лiю 
мл\сть.

Стi'хъ: Jа='кw о_у=тверди'ся мл\сть _е=гw` на на'съ, 
и= и='стина гд\сня пребыва'етъ во вjь'къ.

_Е=ди'на безнад_е'жнымъ наде'ждо, безпомw'щнымъ 
гото'вая по'моще, мл\сти ро'ждшая воли'теля i=и~са, мою` 
поми'луй ны'нjь не'мощь, ч\стая, и= пода'ждь ми` 
помыслw'мъ о_у=миле'нiе. струя'ми сле'зъ са'мыхъ 
согрjьше'нiй мои'хъ, непобjьди'мую потопи` пучи'ну: 
w\тжени` безмjь'рныхъ мои'хъ страсте'й бу'рю: и= 
и=спо'лни тишины` бж~е'ственныя смуще'нное се'рдце мое`, 
хр\ста` моля'щи, пода'ти ми` согрjьше'нiй соверше'нное 
w=ставле'нiе.

Сла'ва, и= ны'нjь, бг~оро'диченъ: Въ чермнjь'мъ мо'ри, 
неискусобра'чныя невjь'сты w='бразъ написа'ся и=ногда`: 
та'мw мwv"се'й, раздjьли'тель воды`: здjь' же гаврiи'лъ, 
служи'тель чудесе`. тогда` глубину` ше'ствова немо'креннw 
i=и~ль: ны'нjь же хр\ста` роди` безсjь'меннw дв~а. мо'ре 
по проше'ствiи i=и~левjь, пребы'сть непрохо'дно: 
непоро'чная по рж\ствjь` _е=мману'илевjь, пребы'сть 
нетлjь'нна. сы'й, и= пре'жде сы'й, jа=вле'йся jа='кw 
чл~вjь'къ, бж~е поми'луй на'съ.

Та'же, Свjь'те ти'хiй: Прокi'менъ: Гд\сь воцр~и'ся: И= 
про'чее по _о=бы'чаю. 

На стiхо'внjь стiхи^ры воскр\сны, гла'съ _е~:

Тебе` воплоще'ннаго сп~са хр\ста`, и= нб~съ не 
разлучи'вшася, во гла'сjьхъ пjь'нiй велича'емъ: jа='кw 
кр\стъ и= сме'рть прiя'лъ _е=си` за ро'дъ на'шъ, jа='кw 
чл~вjьколю'бецъ гд\сь, и=спрове'ргiй а='дwва врата`, 
тридне'внw воскре'слъ _е=си`, сп~са'я ду'шы на'шя.

И='ны стiхи^ры по а=лфави'ту. 

Стi'хъ: Гд\сь воцр~и'ся, въ лjь'поту w=блече'ся. 

Пробод_е'нымъ твои^мъ ре'брwмъ жизнода'вче, то'ки 
w=ставле'нiя всjь^мъ и=сточи'лъ _е=си`, жи'зни и= 
сп~се'нiя: пло'тiю же сме'рть воспрiя'лъ _е=си`, 
безсме'ртiе на'мъ да'руя: всели'въ же ся во гро'бъ на'съ 
свободи'лъ _е=си`, совоскр~си'въ съ собо'ю сла'внw jа='кw 
бг~ъ. сегw` ра'ди вопiе'мъ: чл~вjьколю'бче гд\си, сла'ва 
тебjь`.

Стi'хъ: И='бо о_у=тверди` вселе'нную, jа='же не 
подви'жится. 

Стра'нно твое` распя'тiе, и= _е='же во а='дъ 
соше'ствiе чл~вjьколю'бче _е='сть: плjьни'въ бо _е=го`, 
и= др_е'внiя ю='зники совоскр~си'въ съ собо'ю сла'внw 
jа='кw бг~ъ, ра'й w\тве'рзъ, воспрiя'ти сего` сподо'билъ 
_е=си`. тjь'мже и= на'мъ сла'вящымъ твое` тридне'вное 
воста'нiе, да'руй w=чище'нiе грjьхw'въ: рая` жи'тели 
сподобля'я, jа='кw _е=ди'нъ бл~гоутро'бенъ. 

Стi'хъ: До'му твоему` подоба'етъ ст~ы'ня гд\си, въ 
долготу` днi'й. 

На'съ ра'ди пло'тiю стр\сть прiи'мый, и= тридне'венъ 
и=з\ъ ме'ртвыхъ воскр~сы'й, плотскi^я на'шя стра^сти 
и=сцjьли`, и= возста'ви w\т прегрjьше'нiй лю'тыхъ 
чл~вjьколю'бче, и= сп~си` на'съ.

Сла'ва, и= ны'нjь, бг~оро'диченъ: Хра'мъ и= две'рь 
_е=си`, пала'та и= пр\сто'лъ цр~е'въ, дв~о всеч\стна'я, 
_е='юже и=зба'витель мо'й, хр\сто'съ гд\сь, во тьмjь` 
спя'щымъ jа=ви'ся, сл~нце сы'й пра'вды, просвjьти'ти 
хотя`, jа=`же созда` по w='бразу своему` руко'ю свое'ю. 
тjь'мже всепjь'тая, jа='кw мт~рне дерзнове'нiе къ нему` 
стяжа'вшая, непреста'ннw моли` сп~сти'ся душа'мъ на'шымъ. 

Та'же, Ны'нjь w\тпуща'еши: Трист~о'е. По _О='ч~е 
на'шъ:

Тропа'рь, гла'съ _е~:

Собезнача'льное сло'во _о=ц~у` и= дх~ови, w\т дв~ы 
ро'ждшееся на сп~се'нiе на'ше, воспои'мъ вjь'рнiи и= 
поклони'мся: jа='кw бл~говоли` пло'тiю взы'ти на кр\стъ, 
и= сме'рть претерпjь'ти, и= воскр~си'ти о_у=ме'ршыя 
сла'внымъ воскр\снiемъ свои'мъ.

Бг~оро'диченъ: Ра'дуйся две'ре гд\сня непроходи'мая: 
ра'дуйся стjьно` и= покро'ве притека'ющихъ къ тебjь`. 
ра'дуйся, неwбурева'емое приста'нище, и= 
неискусобра'чная, ро'ждшая пло'тiю творца` твоего` и= 
бг~а: моля'щи не w=скудjьва'й w= воспjьва'ющихъ, и= 
кла'няющихся рж\ству` твоему`.
