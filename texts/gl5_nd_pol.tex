Въ недjь'лю о_у='тра, на полу'нощницjь,

канw'нъ ст~jь'й и= живонача'льнjьй тр\оцjь, [_е=гw'же 
краестро'чiе: Пра'вило пя'тое свjь'ту трисо'лнечному.] 
Творе'нiе митрофа'ново, гла'съ _е~:

\cslemph{Пjь'снь а~}

I=рмо'съ: Коня` и= вса'дника въ мо'ре чермно'е, 
сокруша'яй бра^ни, мы'шцею высо'кою хр\сто'съ и=стрясе`: 
i=и~ля же сп~се`, побjь'дную пjь'снь пою'ща.

Припjь'въ: Прест~а'я тр\оце бж~е на'шъ, сла'ва тебjь`. 

Держа'ву _е=ди'нственнагw и= трисл~нечнагw зра'ка 
воспjьва'юще вопiе'мъ: о_у='мъ на'шъ w=зари` бж~е 
всеси'льне, и= къ твое'й вл\дко, возвы'си сла'вjь 
неизрече'ннjьй.

Горjь' тя а='гг~льская о_у=добр_е'нiя о_у='мная 
немо'лчнw пою'тъ трист~ы'ми пjь'сньми, _е=ди'ницу 
тричи'сленную, и= тр\оцу соwбра'зну, пресу'щественну, 
всеси'льну.

Сла'ва: Бж~е'ственное питiе` твоея` любве`, 
сладча'йшее, свjьтодjь'йственное, души` мое'й пода'ждь, 
тр\оце _е=ди'нице свjьтонача'льная, и= бж~е'ственное 
о_у=миле'нiе чисти'тельное, вл\дко многомл\стиве, всея` 
тва'ри.

И= ны'нjь, бг~оро'диченъ: Jа='коже на руно` сни'де 
без\ъ шу'ма съ нб~се` дв~о, до'ждь въ ложесна` твоя^ 
бж~е'ственный: и= сп~се` все` и=зсо'хшее человjь'ческое 
_е=стество` преч\стая.

\cslemph{Пjь'снь г~}

I=рмо'съ: Водрузи'вый на ничесо'мже зе'млю 
повелjь'нiемъ твои'мъ, и= повjь'сивый неwдержи'мw 
тяготjь'ющую, на недви'жимjьмъ хр\сте`, ка'мени 
за'повjьдей твои'хъ, цр~ковь твою` о_у=тверди`, _е=ди'не 
бл~же и= чл~вjьколю'бче.

О_у=мы'сливъ о_у='мная сущ_ества`, соста'вилъ _е=си` 
пjьвцы` непреста^нныя твоегw` бж~ества`, трисвjь'тлый 
бж~е и= вседjь'телю: но и= бре'нныхъ и= земноро'дныхъ 
прiими` моле'нiе и= мольбу`, jа='кw бл~гоутро'бенъ.

И='же вся'кагw по _е=стеству` премjьне'нiя 
непрiя'тенъ, и=змjьня'_емымъ на'мъ и= пою'щымъ, 
неизслjь'димый и=сто'чникъ твоея` бл~гости, согрjьше'нiй 
да'ждь проще'нiе, и= сп~се'нiе, jа='кw бл~гоутро'бенъ.

Сла'ва: _О=ц~а` и= сн~а, и= дх~а сла'вимъ, въ 
непремjь'ннjьмъ зра'цjь бж~ества` тебе` _е=ди'нственнаго 
и= трисiя'ннаго гд\са всjь'хъ, jа='коже пр\оро'цы, и= 
а=п\сли w\т тебе` jа='вjь научи'шася. 

Бг~оро'диченъ: Jа=ви'лся _е=си` мwv"се'ю въ купинjь`, 
jа='кw а='гг~лъ совjь'та вели'кагw вседержи'телева, твое` 
_е='же w\т дв~ы проявля'я воплоще'нiе, бж~iй сло'ве: 
и='мже на'съ претвори'лъ _е=си`, и= на нб~са` возве'лъ 
_е=си`.

 Гд\си поми'луй, три'жды.

Сjьда'ленъ, гла'съ _е~. Подо'бенъ: Собезнача'льное 
сло'во:

Мл\стива _е=си` тр\оце нераздjь'льная: ми'луеши бо 
всjь'хъ, jа='кw всеси'льная и= всеще'дра, сострада'тельна 
и= многомл\стива. тjь'мже прибjьга'емъ къ тебjь`, и=`же 
грjьхми` мно'гими w=тягча'еми взыва'юще: w=чи'сти твоя^ 
рабы^, и= и=зба'ви всjь'хъ вся'кiя му'ки.

Сла'ва, и= ны'нjь, бг~оро'диченъ: Всест~а'я дв~о, 
поми'луй на'съ прибjьга'ющихъ вjь'рою къ тебjь` 
бл~гоутро'бнjьй, и= прося'щихъ те'плагw твоегw` ны'нjь 
заступле'нiя: мо'жеши бо jа='кw бл~га'я всjь'хъ сп~сти`, 
jа='кw су'щи мт~и бг~а вы'шнягw, мт~рнiя твоя^ мл~твы 
при'снw о_у=потребля'ющи, бг~обл~года'тная.

\cslemph{Пjь'снь д~}

I=рмо'съ: Бж~е'ственное твое` разумjь'въ и=стоща'нiе, 
прозорли'вw а=вваку'мъ, хр\сте`, со тре'петомъ вопiя'ше 
тебjь`: во сп~се'нiе люде'й твои'хъ, сп~сти` пома^занныя 
твоя^ прише'лъ _е=си`.

Та'йнw науча'ется _е=ди'нагw гд\сьства трисвjь'тлому 
данiи'лъ, хр\ста` судiю` о_у=зрjь'въ ко _о=ц~у` и=ду'ща, 
и= дх~а проявля'юща видjь'нiе. 

Бре'нными пою'щихъ тя` о_у=сты`, пресу'щественнаго 
бг~а, тр\очна v=поста'сьми, _е=ди'нственна же 
_е=стество'мъ, сла'вы а='гг~льскiя сподо'би.

Сла'ва: _Е=ди'ну вла'сть, и= _е=ди'но гд\сьство, въ 
трiе'хъ сво'йствахъ неразлу'чнw сла'вимъ: _о='ч~е, и= 
сн~е, и= дш~е, просвjьти` ны` рабы^ твоя^.

Бг~оро'диченъ: Гора` ча'стая и= присjь'нная, ю='же 
ви'дjьвъ пре'жде а=вваку'мъ, и=з\ъ нея'же про'йде ст~ы'й: 
неудо'бь зри'мое рж\ство` jа=вля'ше твоегw` дв~о, 
зача'тiя. 

\cslemph{Пjь'снь _е~}

I=рмо'съ: W=дjья'йся свjь'томъ jа='кw ри'зою, къ 
тебjь` о_у='треннюю, и= тебjь` зову`: ду'шу мою` 
просвjьти` w=мраче'нную хр\сте`, jа='кw _е=ди'нъ 
бл~гоутро'бенъ.

И='же за бл~гость созда'вый человjь'ка, и= по w='бразу 
твоему` сотво'рь, во мнjь` w=бита'й трисвjь'тне бж~е 
мо'й, jа='кw бл~гъ и= бл~гоутро'бенъ.

Ты' мя наста'ви _е=ди'нице трисл~нечная, къ стезя'мъ 
бж~е'ств_еннымъ сп~се'нiя, и= твоегw` сiя'нiя и=спо'лни, 
jа='кw _е=стество'мъ бг~ъ неисчетноси'ленъ.

Сла'ва: Свjь'тъ нераздjь'льный _е=ди'нагw _е=стества`, 
раздjьле'нный начерта'ньми, трисiя'нный, невече'рнiй, 
мое` се'рдце луча'ми твои'ми w=зари`.

Бг~оро'диченъ: Jа='кw ви'дjь тя` дре'вле ч\стая 
пренепоро'чная, пр\оро'къ, зря^щая врата` къ свjь'ту 
незаходи'мому, а='бiе тя` позна` бж~iе жили'ще.

\cslemph{Пjь'снь s~}

I=рмо'съ: Неи'стовствующееся бу'рею душетлjь'нною, 
вл\дко хр\сте`, страсте'й мо'ре о_у=кроти`, и= w\т тли` 
возведи` мя`, jа='кw бл~гоутро'бенъ.

Трисвjь'тлое су'ще бг~онача'лiе v=поста'снjь, 
_е=ди'нственна _е=си`, jа='кw соwбра'зна и= 
равнодjь'тельна, и= по существу` и= хотjь'нiю. 

Дово'льнw и=з\ъяви` пр\оро'къ, поя` _о=ц~у` твоему` 
свjь'ту: о_у='зримъ дх~омъ, свjь'тъ сн~а, _е=ди'наго бг~а 
трисо'лнечнаго. 

Сла'ва: Несоста'вно существо` въ трiе'хъ сво'йствjьхъ, 
_е=ди'ну вла'сть и= держа'ву и=мы'й: тjь'мъ бо состои'тся 
тва'рь вся'ческая, и= w=бновля'ется.

Бг~оро'диченъ: Прегрjьше'нiй и=збавле'нiе и= бjь'дъ, 
вл\дко бж~е, _е=ди'нственне и= трисвjь'тне, низпосли` 
твои^мъ пjьвц_е'мъ, мл~твами бг~омт~ре.

Гд\си поми'луй, три'жды.

Сjьда'ленъ, гла'съ _е~. Подо'бенъ: Собезнача'льное 
сло'во:

Трисо'лнечный свjь'тъ славосло'вимъ, и= про'стjьй 
тр\оцjь ны'нjь поклони'мся, jа='кw просвjьти` на'съ и= 
поми'лова, и= и=зба'ви w\т тли` ве'сь ро'дъ 
человjь'ческiй, и=збавля'ющи w\т пре'лести i='дwльскiя 
ве'сь мi'ръ, и= ца'рство на'мъ подаде`.

Сла'ва, и= ны'нjь, бг~оро'диченъ: Недоумjь'вся w\т 
всjь'хъ, къ тебjь` прибjьго'хъ къ наде'жди всjь'хъ, и= 
прибjь'жищу грjь'шныхъ и= смире'нныхъ, зовы'й: 
согрjьши'хъ, и= пребыва'ю въ sлы'хъ, нечу'вствуя 
_о=кая'нный. поми'луй мя`, пре'жде конца` w=брати' мя, и= 
и=зба'ви вся'кiя му'ки, недосто'йнаго.

\cslemph{Пjь'снь з~}

I=рмо'съ: Превозноси'мый _о=тц_е'въ гд\сь, пла'мень 
о_у=гаси`, _о='троки w=роси` согла'снw пою'щыя: бж~е 
бл~гослове'нъ _е=си`.

Jа='кw и=мы'й бе'здну мл\сти, гд\си, и= пучи'ну 
неизче'тну щедро'тъ, поми'луй _е=ди'наго тя` пою'щихъ, 
трисвjь'тлаго бг~а всjь'хъ.

Неwбмы'слимаго, _е=ди'нственна и= трисвjь'тла, бг~а 
тя` и= гд\са пою'ще вопiе'мъ ти`: пода'ждь твои^мъ 
рабw'мъ w=чище'нiе грjьхw'въ.

Сла'ва: Ра'внw v=поста^си во _е=ди'ной держа'вjь 
чту'ще раздjьля'емъ нераздjьли'мо существо`, бг~а 
_о=ц~а`, и= сн~а, и= прест~а'го дх~а.

Бг~оро'диченъ: _О='трасль прозябла` _е=си`, _о=ц~у` 
собезнача'льную, цвjь'тъ бж~ества`: _о='трасль 
соприсносу'щную, дв~о, даю'щую жи'знь всjь^мъ 
человjь'кwмъ.

\cslemph{Пjь'снь и~}

I=рмо'съ: Тебjь` вседjь'телю, въ пещи` _о='троцы, 
всемi'рный ли'къ спле'тше, поя'ху: дjьла` вся^кая гд\са 
по'йте, и= превозноси'те во вся^ вjь'ки.

Да _е=ди'наго w\ткры'еши дре'вле jа='вjь гд\сьства 
тр\очную v=поста'сь, jа=ви'лся _е=си` бж~е мо'й, во 
w='бразjь человjь'кwвъ а=враа'му, пою'щу твою` держа'ву 
_е=ди'нственную.

Ты' мя къ твои^мъ бл~годjь'т_ельнымъ луча'мъ взира'ти 
сподо'би свjь'те непристу'пный, _о='ч~е ще'дрый, и= 
сло'ве, и= дш~е, _е='же бл~гоугожда'ти тебjь` при'снw, 
гд\си всjь'хъ.

Сла'ва: Ст~ъ _о=ц~ъ бг~ъ превjь'чный: ст~ъ же сн~ъ 
и=з\ъ _о=ц~а` рожде'нъ: ст~ъ же и= животворя'щiй дх~ъ, 
и=сходя` и=з\ъ _о=ц~а`, сн~омъ же jа=вля'емь.

Бг~оро'диченъ: W=блиста'ла _е=си` на'мъ w\т 
трисл~нечныя сла'вы, _е=ди'наго всепjь'тая хр\ста` гд\са, 
всjь'хъ тайнонауча'юща _е=ди'ному бг~онача'лiю въ трiе'хъ 
ли'цjьхъ, пjь'ти во вjь'ки.

\cslemph{Пjь'снь f~}

I=рмо'съ: И=са'iе лику'й, дв~а и=мjь` во чре'вjь, и= 
роди` сн~а _е=мману'ила, бг~а же и= человjь'ка, восто'къ 
и='мя _е=му`: _е=го'же велича'юще, дв~у о_у=бл~жа'емъ.

Глагw'ланiя человjь'ч_еская по достоя'нiю, 
безнача'льная _е=ди'нице, не мо'гутъ пjь'ти тя`: _о=ба'че 
jа='коже мо'щно дерза'юще w\т вjь'ры, бг~онача'льная, 
сопресто'льная тр\оце, сла'ву прино'симъ твое'й 
держа'вjь, и= хвалу`.

Равноста'тною сла'вою, тя` _е=динонача'льнаго 
трисвjь'тлагw бг~а сла'вятъ херувi'ми и= серафi'ми 
преч\стыми о_у=сты`: съ ни'миже и= на'съ грjь'шныхъ 
прiими` гд\си, твою` держа'ву велича'ющихъ. 

Сла'ва, И=са'iа тя` ви'дjь на пр\сто'лjь херувi'мстjь, 
и= серафi'мы стоя'щыя _о='крестъ тебе`, кри'лы ли'ца 
закрыва'ющыя, и= вопiю'щыя: ст~ъ, ст~ъ, ст~ъ, трист~ы'й 
бж~е, сла'вимый въ трiе'хъ сво'йствjьхъ.

Бг~оро'диченъ: Jа='кw ч\стая и= непоро'чная и= дв~а, 
родила` _е=си` сн~а, и=збавля'ющаго на'съ w\т 
и=скуше'нiй, бг~а неизмjь'нна: но и= ны'нjь w=ставле'нiе 
на'мъ прегрjьше'нiй да'ти сего` моли`. 

Посе'мъ припjь'вы григо'рiа сiнаи'та: Досто'йно 
_е='сть: И= про'чее полу'нощницы, пи'сано въ концjь` 
кни'ги сея`.
