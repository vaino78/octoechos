%<%[Въ суббw'ту на повече'рiи%],%>

%<глаго'летъ i=ере'й: Б%>л~гослове'нъ бг~ъ на'шъ: %<И= 
мы`: Сла'ва тебjь`, бж~е на'шъ, сла'ва тебjь`. Ц%>р~ю` 
нб\сный: Т%<рист~о'е. И= по _О='%>ч~е на'шъ: %<Г%>д\си 
поми'луй, %<в~_i. Сла'ва, и= ны'нjь: П%>рiиди'те, 
поклони'мся: %<три'жды. _Псало'мъ н~: П%>оми'луй мя` 
бж~е: %<_Псало'мъ _кс~f: Б%>ж~е, въ по'мощь мою` вонми`: 
%<_Псало'мъ рм~в: Г%>д\си, о_у=слы'ши мл~тву мою`: 
%<Посе'мъ: С%>ла'ва въ вы'шнихъ бг~у: %<Г%>д\си, 
прибjь'жище бы'лъ _е=си` на'мъ: %<С%>подо'би гд\си, въ 
но'щь сiю`: Т%<а'же: В%>jь'рую во _е=ди'наго бг~а: 
%<Посе'мъ пое'мъ канw'нъ прест~jь'й бц\дjь, гла'съ а~.%>

%<%[Пjь'снь а~%]%>

%<I=рмо'съ: %>Го'рькiя рабо'ты и=зба'влься i=и~ль, 
непроходи'мое про'йде jа='кw су'шу, врага` зря` 
потопля'ема, пjь'снь jа='кw бл~годjь'телю пое'тъ бг~у, 
чудодjь'ющему мы'шцею высо'кою, jа='кw просла'вися.

%<Припjь'въ: П%>рест~а'я бц\де сп~си` на'съ.

%<О_у=%>жасо'шася стра'хомъ всецр~и'це а='гг~льская 
чинонача^лiя, хва'ляще тя`: пое'тъ же бл~гости ра'ди 
вся'къ о_у='мъ, jа='кw мт~рь зижди'теля: похвалы' бо 
вся'кiя превозшла` _е=си` чи'нъ, хр\ста` ро'ждши.

%<Л%>ю'тыми напа'стьми томи'мь, и= w\т вра^гъ стужа'емь 
_о=кая'нный, съ пла'чемъ приглаша'ю: свы'ше мнjь` 
простри` (с. 28) пребога'тая ру'ку твою`, и=збавля'ющи 
мя`, и= безбjь'днw пожи'ти сподо'би мл~твами твои'ми.

%<Сла'ва: Д%>уши` та^йная прегрjьш_е'нiя, лjьчбо'ю 
мл\срдiя и=сцjьли`, и= о_у=тиши` плотско'е стремле'нiе 
бц\де: и= врагw'въ кw'пiя и= стрjь'лы возвра'щши, 
крjь'пкw въ сердца` и='хъ вонзи`.

%<И= ны'нjь: П%>а'жить дре'внюю потреби` 
чл~вjькоубi'йственную, о_у=тро'ба дв~и'ча хр\ста` 
ро'ждши: тjь'мже вся` тва'рь ра'дуется ны'нjь, преч\стая, 
jа='кw w=жи'вльшися: и= согла'снw воспjьва'етъ твоего` 
сн~а и= бг~а.

%<%[Пjь'снь г~%]%>

%<I=рмо'съ: %>Не му'дростiю и= бога'тствомъ да хва'лится 
сме'ртный свои'мъ, но вjь'рою гд\снею, правосла'внw 
взыва'я хр\сту` бг~у, и= поя` при'снw: на ка'мени твои'хъ 
за'повjьдей о_у=тверди' мя вл\дко.

%<I=%>а'кwвъ вели'кiй на пути` спя` и=ногда`, низходя'щыя 
свы'ше лjь'ствицею, дв~о, на зе'млю а='гг~лы ви'дя, 
дивля'шеся: и= возбну'въ, прописа` jа='вjь две'рь тя` 
нб\сную.

%<В%>ре'меннымъ о_у=держа'нiемъ, въ напа^сти вве'рженъ 
_о=кая'нный а='зъ, и= бу'рями напа'стными томи'мь, 
о_у=вы` мнjь`, вопiю`: jа='же бг~а ро'ждшая, и= 
возне'сшая ро'гъ на'шъ, сп~си' мя мл~твами твои'ми.

%<Сла'ва: С%>ъ нб~се` ру'ку крjь'пкую просте'ръ, _w 
всjь'хъ цр~ю` хр\сте`, чу'вственныхъ и= мы'сленныхъ, 
i=и~се мо'й, главы^ врагw'въ под\ъ но'зjь положи`, бц\ду 
мт~рь твою` вjь'рнw проповjь'дающихъ.

%<И= ны'нjь: I=%>са'iя дре'вле, w=чи'щся о_у='глемъ дх~а, 
w\т твоея` о_у=тро'бы jа='вjь пребога'тая дв~о, сн~у 
взыва'ше роди'тися: _е=го'же въ послjь^дняя времена`, 
мене` ра'ди, без\ъ му'жа родила' _е=си.

%<%[Пjь'снь д~%]%>

%<I=рмо'съ: %>Слы'ша дре'вле а=вваку'мъ хр\сте` чу'дный 
тво'й слу'хъ, и= стра'хомъ взыва'ше: w\т ю='га бг~ъ 
прiи'детъ, и= ст~ы'й w\т горы` прiwсjьне'нныя ча'щи, 
сп~сти` пома^занныя: сла'ва си'лjь твое'й гд\си. (с. 29)

%<Т%>ебе` jа='кw су'щую въ жена'хъ красну` и= прече'стну, 
jа='кw и=з\ъ пусты'ни восходя'щу, и= _о='трасль хр\ста` 
на руку^ нося'щу, мт~и бж~iя, проwбража'ше 
мiронача'льникъ вопiя`: сла'ва си'лjь твое'й гд\си.

%<П%>риклони` ми` о_у='хо твое` бл~га'я, и= ви'ждь мое` 
w=sлобле'нiе, и= бjь'дъ о_у=множе'нiе: къ тебjь' бо 
вл\дчце, душе'вное _о='ко просте'ръ, и= съ пла'чемъ 
колjь^на прикло'нь, ны'нjь молю` вопiя`: о_у=ста'ви ми` 
напа'стей смуще'нiе.

%<Сла'ва: С%>тjь'ну тя` неwбори'мую вjь'дый, привлече'нъ 
мольбо'ю, тво'й ра'бъ ны'нjь къ тебjь` прибjьга'ю, и= 
стрjь'лы вра^жiя, jа='кw младе'нческiя вмjьня'ю 
недjь'йственны, _w пребога'тая! тjь'мже и= ра'дуяся 
вопiю`: сла'ва бг~ома'ти, рж\ству` твоему`.

%<И= ны'нjь: С%>и'ла вы'шнягw тя` дв~о w=сjьни` наи'тiемъ 
бж~е'ственнагw дх~а: и= тогда` па'че _е=стества` 
вся'ческихъ гд\сь, пло'ть и= ду'шу w=животвори'въ, 
прiедини` себjь`, пребы'въ въ то'мжде _е=стествjь`.

%<%[Пjь'снь _е~%]%>

%<I=рмо'съ: %>Свjь'тъ тво'й незаходи'мый возсiя'й 
хр\сте`, въ сердца` вjь'рнw пою'щихъ тя`, ми'ръ подава'яй 
на'мъ, па'че о_у=ма`. тjь'мъ w\т но'щи невjь'дjьнiя ко 
дню` свjь'томъ твои'мъ теку'ще, славосло'вимъ тя` 
чл~вjьколю'бче.

%<Б%>ж~е'ственную тя`, прови'дjьвъ и=ногда` данiи'лъ, 
несjько'мую го'ру, препjь'тая, вопiя'ше проявле'ннjь, 
о_у=сjьщи'ся w\т тебе` ка'меню бг~оро'дiя, хр\сту` сп~су 
мi'ра: _е=го'же вjь'рнiи ны'нjь чту'ще, восхваля'емъ тя` 
бг~оневjь'сто.

%<Н%>апа'стьми _о=кая'нный мно'гими па'дся, съ болjь'знiю 
и= пла'чемъ се'рдца моля'ся, безсту'днw служи'тель тво'й 
вопiю`: и=зба'ви бг~ороди'тельнице, w\т бjь'дъ 
w=держа'щихъ мое` смире'нiе, и= весе'лiя мя` и=спо'лни.

%<Сла'ва: С%>трасте'й мои'хъ свирjь'пую пучи'ну 
о_у=ти'ши, крjь'пкою твое'ю мл~твою бл~га'я, jа='же 
страсте'й кромjь` (с. 30) ро'ждшая хр\ста`: jа='кw да въ 
тишинjь` души` ны'нjь живы'й, про'чее жи'зни моея`, въ 
пjь'снехъ воспjьва'ю тя`.

%<И= ны'нjь: Б%>г~а ка'кw на руку` но'сиши, рцы`; и= 
ка'кw дои'ши, вся'ч_еская руко'ю содержа'щаго, _w дв~о 
всебл~же'нная, а='зъ, рече`, чиста` пребыва'ю и= по 
рж\ствjь`, хр\ста` бг~а ро'ждши, а=да'мовъ до'лгъ, и= 
прама'тернiй w\те'мши.

%<%[Пjь'снь s~%]%>

%<I=рмо'съ: %>Ве'сь w\т страсте'й безмjь'рныхъ 
содержу'ся, и= ки'томъ sw'лъ сниспадо'хъ: но возведи` 
и=з\ъ и=стлjь'нiя мя` бж~е, jа='коже пре'жде i=w'ну, и= 
вjь'рою безстра'стiе ми` да'руй, jа='кw да во гла'сjь 
хвале'нiя, и= сп~се'нiя дх~омъ пожру' ти.

%<С%>jьдjь'нiя не w\тсту'пль нjь'дръ роди'теля, въ 
нjь'дрjьхъ мт~рнихъ сн~ъ предлjь'тный: напослjь'докъ же, 
и='же пре'жде вjь^къ сы'й со _о=ц~е'мъ, и=з\ъ о_у=тро'бы 
дв~ы произы'де, вся^ къ жи'зни безсме'ртнjьй возводя`, 
неизрече'нною бл~гостiю.

%<В%>ери'гами свя'занъ вра'жiями w\т w=sлобле'нiя, во 
а='дскiя вер_еи`, о_у=вы` мнjь`, низверго'хся: но 
предста'ни, съ нб~се` jа='вльшися, бг~о_отрокови'це 
чи'стая, возставля'ющи мя` твои'ми мл~твами служи'теля 
твоего`, и= ру'ку по'мощи да'ждь пою'щему бж~е'ственное 
твое` рж\ство`.

%<Сла'ва: В%>ъ ро'въ поги'белей _о=кая'нный впадо'хъ, и= 
sвjь'рiе мно'зи w=кроча'ютъ мя`: но вл\дчце, мл~твами 
твои'ми ка'м_енiя w\тврати`, и= невреди'ма твоего` раба` 
соблюди`: ка'мень бо краеуго'льный, хр\ста` въ ложесна'хъ 
твои'хъ носи'ла _е=си`.

%<И= ны'нjь: Д%>ре'вле пр\орw'къ бж~е'ственныхъ ли'къ 
провозгласи` дв~о, рж\ства` твоегw` w='бразы: свjь'телъ 
_о='блакъ тя`, и= свjь'щникъ, и= ста'мну, и= трапе'зу, и= 
нб\сную ро'су, хлjь'бъ, ма'нну же и= две'рь, пр\сто'лъ, 
и= пала'ту, же'злъ, ра'й, jа='кw хр\ста` ро'ждшую.

%<Г%>д\си поми'луй, %<три'жды. Сла'ва, и= ны'нjь:%>

%<Сjьда'ленъ, гла'съ а~: М%>т~рь тя` бж~iю свjь'мы вси`, 
дв~у вои'стинну, и= по рж\ствjь` jа='вльшуюся, любо'вiю 
прибjьга'юще (с. 31) къ твое'й бл~гости: тебе' бо и='мамы 
грjь'шнiи предста'тельницу, и= тебе` стяжа'хомъ въ 
напа'стехъ сп~се'нiе, _е=ди'ну всенепоро'чную.

%<%[Пjь'снь з~%]%>

%<I=рмо'съ: %>Проидо'ша jа='кw черто'гъ пе'щный пла'мень 
нестерпи'мый, и=`же за бл~гоче'стiе и=ногда` _о='троцы 
ст~i'и показа'вшеся jа='вjь, и= согла'снw воспjьва'юще, 
пjь'снь поя'ху: _о=тц_е'въ бж~е бл~гослове'нъ _е=си`.

%<Ш%>е'ствуя превjь'чный две'рьми непрохо'дными твои'ми 
всецр~и'це, чи'ста твоя^ зна'м_енiя и= цjь'ла сохрани`, 
чи'ста и= по рж\ствjь`. тjь'мже вопiе'мъ: _о=т_е'цъ 
на'шихъ бж~е, бл~гослове'нъ _е=си`.

%<В%>ъ пе'щь вве'рженъ седмочи'сленными пла'меньми 
w=паля'юся w\т душеубi'йственныхъ бjь'дъ: но сама` ро'су 
ми` w=дожди` вл\дчце бл~га'я, твои'ми мольба'ми, да 
вопiю`: бл~гослове'нъ бг~ъ _о=т_е'цъ на'шихъ.

%<Сла'ва: С%>трастьми` престарjь'вся, и= неwсла'бными 
напа'стьми и= скорбьми`, и= доспjь'въ къ за'падwмъ житiя` 
моегw`, добродjь'телемъ неприча'стенъ, лjь'ностiю 
снjьде'нъ бы'въ, вопiю' ти, вл\дчце: о_у=тjьше'нiе 
земны^мъ, поми'луй мя`!

%<И= ны'нjь: Т%>р\оцjь во _е=ди'нствjь правосла'внw 
служа'ще, тя` мт~и дв~о ч\стая, jа='кw бг~а пло'тiю 
ро'ждшую проповjь'дающе земнi'и, бж~е'ственнjь пое'мъ: 
_о=т_е'цъ на'шихъ бж~е, бл~гослове'нъ _е=си`.

%<%[Пjь'снь и~%]%>

%<I=рмо'съ: Ч%>у'да преесте'ственнагw показа` w='бразъ, 
_о=гнеро'сная пе'щь дре'вле: _о='гнь бо не w=пали` ю='ныя 
дjь'ти, хр\сто'во jа=вля'я безсjь'менное w\т дв~ы 
бж~е'ственное рж\ство`. тjь'мъ воспjьва'юще воспои'мъ: да 
бл~гослови'тъ тва'рь вся` гд\са, и= превозно'ситъ во вся^ 
вjь'ки.

%<С%>ло'во всеи'стинное сщ~е'нника воwбрази` рж\ство` 
твое`, дв~о: вои'стинну бо родила' _е=си сло'во бж~iе, и= 
ложесна` не разве'рзе дв~о, и='миже про'йде бг~ъ. тjь'мъ 
ра'дующеся бц\ду (л. 32) тя` по до'лгу согла'снw 
воспjьва'емъ, и= превозно'симъ ч\стую во вся^ вjь'ки.

%<Т%>е'рнiе возра'стшее въ души` мое'й недjь'йственно, 
бж~е'ственнымъ _о=гне'мъ попали` преч\стая, и= твои'ми 
мл~твами на добродjь'т_ели возста'ви мя`, твори'ти 
плодоно'сiе хр\сту`: и=з\ъ тебе' бо цвjь'тъ при'снw 
живы'й прора'стши, тва'рь всю` о_у=краси`. тjь'мъ тя` 
бг~ороди'тельницу ч\стую почита'емъ въ вся^ вjь'ки.

%<Сла'ва: И=%>сцjьле'нiе въ лю'тыхъ бг~ороди'тельнице, 
неболjь'зненнw пода'ждь ми` вско'рjь: впа'дъ бо въ 
скw'рби и= бjьды^, твою` ско'рость въ заступле'нiе 
_о=кая'нный призыва'ю, рыда'я. тjь'мже преч\стая, 
о_у=скори` и=з\ъя'ти мя`, и= сп~си` w\т вся'кiя му'ки, 
jа='кw да бл~гословя` воспjьва'ю твое` рж\ство`.

%<И= ны'нjь: Д%>ре'вле тя` прозя'бшiй а=арw'новъ же'злъ 
воwбрази` дв~о: ты' бо _е=ди'на родила` _е=си` 
прозябе'нiемъ без\ъ му'жа, нб\сный ны'нjь до'ждь прiе'мши 
во чре'вjь. тjь'мъ веселя'щеся, тя` бц\ду по до'лгу 
согла'снw вси` пое'мъ, и= превозно'симъ во вся^ вjь'ки.

%<%[Пjь'снь f~%]%>

%<I=рмо'съ: %>Неизглаго'ланное дв~ы та'инство: нб~о бо 
сiя`, и= пр\сто'лъ херувi'мскiй, и= свjьтоно'сный 
черто'гъ показа'ся хр\ста` бг~а вседержи'теля. сiю` 
бл~гоче'стнw jа='кw бц\ду велича'емъ.

%<П%>ресла'вно дв~ы та'инство: _е=го'же бо не вмjьсти'ша 
вышенб\сная вели^чiя, вмjьсти` во чре'вjь. тjь'мже 
соше'дшеся о_у=бл~жа'емъ ю=`, и= вjь'рнw веселя'щеся 
велича'емъ.

%<В%>и'дяще тя` _е=ди'ну, jа='кw вы'шшую нб~съ, бж~iю 
зарю`, нескве'рная, пр\сто'лъ херувi'мскiй, и= черто'гъ, 
и= _о='дръ ст~ъ, земнi'и восхваля'юще, хр\ста` бг~а 
на'шего велича'емъ: _е=го'же w\т чре'слъ чи'стыхъ родила` 
_е=си`.

%<Сла'ва: _О='%>крестъ мене` скw'рби мнw'ги, и= sлы` 
напа^сти ны'нjь напа'дающе, болjь^зни же и= лю^тыя 
грjьхи` въ ро'въ (с. 33) вверго'ша. тjь'мже тя` молю` въ 
го'рести души` моея`, прест~а'я бц\де, и=збавле'нiе 
w=брjьсти' ми.

%<И= ны'нjь: О_у=%>мири` мi'ръ, хр\сте`, мольба'ми ч\стыя 
бг~о_отрокови'цы, низлага'я вра'жiю си'лу и= тишину` 
неизглаго'ланную на'мъ о_у=строя'я, во вjь'ки сохрани`.

%<Та'же: Д%>осто'йно _е='сть: Т%<рист~о'е. И= по 
_О='%>ч~е на'шъ: %<конда'къ гла'са. Г%>д\си поми'луй, 
%<м~. И= мл~тва: И='%>же на вся'кое вре'мя: %<Г%>д\си 
поми'луй, %<три'жды. Сла'ва, и= ны'нjь: Ч\с%>тнjь'йшую 
херувi^мъ: %<И='%>менемъ гд\снимъ бл~гослови` _о='тче. 
%<Та'же, мл~тва: Н%>ескве'рная, небла'зная: %<и= про'чее 
_о=бы'чно, и= w\тпу'стъ, и= проще'нiе.%>
