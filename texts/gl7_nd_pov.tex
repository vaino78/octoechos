%[Въ суббw'ту на повече'рiи%], канw'нъ прест~jь'й бц\дjь. 
Гла'съ з~:

%[Пjь'снь а~%]

I=рмо'съ: Сокруши'вшему бра^ни мы'шцею свое'ю, и= 
вса'дники триста'ты погрузи'вшему, пои'мъ _е=му`, jа='кw 
и=зба'вителю на'шему бг~у: jа='кw просла'вися.

Припjь'въ: Прест~а'я бц\де, сп~си` на'съ.

Jа='кw о_у=краше'нiе пjь'нiемъ, и= пjь'снь 
бг~оподо'бну бл~годаре'нiя ти` да'ры прино'симъ: _е='же 
ра'дуйся ны'нjь ч\стая, jа='кw w\т печа'ли ра'дость на'мъ 
подала` _е=си`.

Мл\сти твоея` бл~года'ть, и= покро'ва крjь'пость, не 
премолчи'мъ дв~о преч\стая: jа='кw w\т бjь'дъ лю'тыхъ 
на'съ сп~сла` _е=си`.

Сла'ва: W\т разли'чныхъ и=зба'вльшеся и=скуше'нiй и= 
скорбе'й, мт~рними твои'ми мл~твами преч\стая, согла'снw 
и= те'плjь бл~годаре'нiя тебjь` вси` гла'сы пои'мъ.

И= ны'нjь: Ря'снами и=спещре'на jа='кw златы'ми 
добродjь'тельми, и= дх~а бл~года'тьми преч\стая, jа='кw 
невjь'ста _о='ч~а о_у=краси'вшися, вои'стинну jа=ви'лася 
_е=си` мт~и бг~у.

%[Пjь'снь г~%]

I=рмо'съ: Въ нача'лjь нб~са` всеси'льнымъ сло'вомъ 
твои'мъ о_у=твержде'й гд\си сп~се, и= вседjь'тельнымъ и= 
бж~iимъ дх~омъ, всю` си'лу и='хъ, на недви'жимjьмъ мя` 
ка'мени и=сповjь'данiя твоегw` о_у=тверди`.

Ра'дости бж~е'ственныя вели'чiе, jа='кw w\т нача'ла 
на'мъ весе'лiю, бл~годаре'нiя гла'сы прино'симъ 
прилjь'жнw, почита'юще ю=` jа='кw предста'тельницу на'шу.

И=`же тобо'ю и=зба'вльшеся w\т бjь'дъ, и= тебе` ра'ди 
ра'дость получи'вше, мт~и безневjь'стная, jа='кw 
пода'тельницу бл~гу'ю, и= засту'пницу бл~гохва'льну вси` 
прославля'емъ.

Сла'ва: И=змjьне'нiе прегрjьше'нiй и= напа'стей, 
бж~е'ственною твое'ю мл~твою прiе'млюще, мт~и хр\ста` 
бг~а, jа='кw вину` бл~ги'хъ воспjьва'емъ тя` вjь'рнw 
бл~года'рными гла'сы.

И= ны'нjь: Jа='же ра'дости и=сто'чникъ безсме'ртныя, 
то'читъ струи^ при'снw, и= сп~са'етъ всjь'хъ сама`: 
jа='кw присножи'зненное на'мъ бл~годая'нiе, мт~и хр\ста` 
бг~а.

%[Пjь'снь д~%]

I=рмо'съ: Покры'ла _е='сть нб~са` хр\сте` бж~е, 
смотре'нiемъ твои'мъ, добродjь'тель неизрече'нныя 
му'дрости твоея` чл~вjьколю'бче.

Досто'йну ти` пjь'снь ра'дованiя ч\стая дв~о, 
веселя'щеся прино'симъ, и=зба'вльшеся w\т бjь'дъ твои'ми 
мл~твами.

Бл~года'рственнw тебjь` дв~о ч\стая, душе'вныма 
рука'ма вно'симъ пjь'нiе, бж~е'ственными пjь'сньми 
и=гра'юще, w\т печа'ли и=зба'вльшеся мно'гiя.

Сла'ва: Воздвиго'ша на ны` бjьды^ мнw'ги страстнi'и 
грjьси`: но бж~е'ственнымъ твои'мъ покро'вомъ ч\стая, 
на'съ и=зба'ви.

И= ны'нjь: Бл~же'ни вои'стинну чту'щiи тя`, бц\де 
преч\стая, jа='кw тобо'ю грjьха` и= печа'ли 
и=зба'вихомся.

%[Пjь'снь _е~%]

I=рмо'съ: Но'щь несвjьтла` невjь^рнымъ хр\сте`, 
вjь^рнымъ же просвjьще'нiе въ сла'дости слове'съ твои'хъ: 
сегw` ра'ди къ тебjь` о_у='тренюю, и= воспjьва'ю твое` 
бж~ество`.

Грjьха` потреби'теля хр\ста` дв~о родила` _е=си`, 
и='мже w\т напа'стей и= болjь'зней сп~се'ся мi'ръ: 
тjь'мже ра'дуйся, и= мы` вопiе'мъ ти`, и=зба'вльшiися 
печа'лей.

W=б\ъе'мшеся разли'чными напа'стьми ч\стая вл\дчце, и= 
печа'лiю и= ско'рбiю, и= лю'тымъ w=бстоя'нiемъ, 
w\тча'явшеся мы` весе'лiя, наде'жду тя` w=брjьто'хомъ.

Сла'ва: Jа='кw сп~се'нiя храни'лище на'мъ рабw'мъ 
твои^мъ ч\стая, w\тгоня'еши напа^сти, и= соблюда'еши 
невреди'мы. тjь'мже прича^стницы мно'гихъ твои'хъ бла^гъ, 
бл~годари'мъ тя` пjь'сньми.

И= ны'нjь: Бл~годари'мъ тя` и=зба'вльшеся тобо'ю 
грjьхw'въ мно'гихъ, и= не'мощей и= болjь'зней и= неду^гъ 
лю'тыхъ преч\стая вл\дчце: наде'жда бо _е=си` тверда` 
твои^мъ вjь^рнымъ рабw'мъ.

%[Пjь'снь s~%]

I=рмо'съ: Во глубину` грjьхо'вную впа'дся бл~же, 
jа='кw i=w'на w\т ки'та вопiю' ти: возведи` w\т тли` 
живо'тъ мо'й, и= сп~си' мя чл~вjьколю'бче.

А='гг~льстiи я=зы'цы досто'йнw твоя^ хвалы^ воспjь'ти 
не мо'гутъ ч\стая: мы' же ны'нjь ра'бски взе'мшеся, 
ра'дованiе гаврiи'лово прино'симъ ти`.

Во глубину` печа'ли и= w=бстоя'нiя грjьхи` на'шими 
впа'дше, и=зба'вихомся тобо'ю w\т ну'жды и= напа'стей, 
дв~о бц\де ч\стая.

Сла'ва: Под\ъ до'лгомъ содjь'лася ве'сь мi'ръ, ч\стая, 
бл~годари'ти, и= хвали'ти, и= сла'вити бл~года'ть твою` 
бл~гоче'стнw: тобо'ю бо бjь'дъ и= печа'лей и=зба'влени 
бы'хомъ.

И= ны'нjь: Въ нощи` и= во дни`, jа='вjь и= вта'й, 
под\ъ тво'й прибjьга'емъ покро'въ, преч\стая дв~о, и=`же 
тя` вjь'рнw славосло'вящiи.

Гд\си поми'луй, три'жды. Сла'ва, и= ны'нjь:

Сjьда'ленъ, гла'съ з~.

Гд\си, мы` _е=смы` лю'дiе твои`, и= _о='вцы па'жити 
твоея`, заблу'ждшихъ jа='кw па'стырь w=брати` на'съ, 
тле'ю расточе'ныхъ собери` на'съ: поми'луй ста'до твое`, 
о_у=мл\срдися на лю'ди твоя^, мл~твами бц\ды, _е=ди'не 
безгрjь'шный.

%[Пjь'снь з~%]

I=рмо'съ: Въ пе'щь _о='гненную вве'ржени прп\дбнiи 
_о='троцы, _о='гнь въ ро'су преложи'ша, воспjьва'нiемъ 
си'це вопiю'ще: бл~гослове'нъ _е=си` гд\си бж~е _о=т_е'цъ 
на'шихъ.

Ра'дованiе бл~года'рственное тебjь` прино'симъ, мт~и 
бж~iя: jа='кw вои'стинну тобо'ю и=зба'влени вся'кагw 
sла'гw начина'нiя, и= согла'снw тебjь` пое'мъ: 
бл~гослове'нна _е=си`.

Водвори'хомся въ ве'черъ въ пла'чи печа'лей, и= 
ча'янiи sлы'хъ: но твои'мъ бг~олjь'пнымъ кро'вомъ дв~о, 
w=боже'ни бы'вше, w=брjьто'хомъ ра'дость зау'тра: ты' бо 
сп~сла` _е=си` на'съ.

Сла'ва: Jа='кw бж~е'ственное прибjь'жище, мы` покро'въ 
тво'й вси` стяжа'вше къ бг~у, въ напа'стехъ и= гоне'нiихъ 
и= грjьсjь'хъ, къ тебjь` прибjьга'емъ, и= тобо'ю 
и=змjьне'нiе прiе'млемъ, преч\стая.

И= ны'нjь: Бл~года'ть мл~твы твоея` ч\стая, 
проповjь'дуемъ о_у=сты` и= дх~омъ, сла'вная: jа='кw тебе` 
ра'ди, напа'сти и= бу'ри, и= тя'жкихъ печа'лей, и= 
грjьха` страсте'й вси` и=збавля'емся.

%[Пjь'снь и~%]

I=рмо'съ: Стра'шнаго херувi'мwмъ, и= чу'днаго 
серафi'мwмъ, и= мi'ру творца`, сщ~е'нницы и= раби`, и= 
ду'си пра'ведныхъ, по'йте, бл~гослови'те и= превозноси'те 
_е=го` во вjь'ки.

Jа='кw и=зба'вльшеся бу'ри грjьхо'вныя и= страсте'й, 
и= напа'стей твои'ми мл~твами бц\де бл~га'я, 
бл~года'рственнымъ гла'сомъ, ра'дуйся, вопiе'мъ ти`: 
jа='кw тобо'ю w\т печа'ли въ ра'дость преше'дше.

Въ болjь'знехъ и= напа'стехъ w=держи^мыя не пре'зри 
бл~га'я, но мольбу` худу'ю о_у=слы'шавши, свободи` на'съ 
w\т скорбе'й вели'кихъ: да вjь'рнw твою` мл~тву ч\стая, 
воспjьва'емъ.

Сла'ва: Прегрjьше'нiй разруше'нiю вина`, ны'нjь 
возста'ви на'съ w\т печа'лей, напа'стей и= страсте'й 
человjь^къ, и= и=скуше'нiя неподо'бна: твои'ми же бц\де 
бж~е'ственными мольба'ми, и=зба'ви на'съ w\т ни'хъ 
пресла'внw.

И= ны'нjь: Щедрw'ты твоя^ на всjь'хъ всегда` 
низпосыла'ются, хр\сте`, бл~года'тiю вои'стинну и= 
мольба'ми ро'ждшiя тя`: w\т тебе' бо прiе'млемъ 
хр\стiа'не мл\сть твою`, сп~се мл\стиве.

%[Пjь'снь f~%]

I=рмо'съ: Па'че _е=стества` мт~рь, и= по _е=стеству` 
дв~у, _е=ди'ну въ жена'хъ бл~гослове'нную, пjь'сньми 
вjь'рнiи бц\ду велича'емъ.

Ра'дованiе по достоя'нiю, со гла'сомъ восклица'нiя 
тебjь` дв~о, ны'нjь прино'симъ со а='гг~ломъ гаврiи'ломъ, 
и=зба'вльшеся w\т разли'чныхъ и=скуше'нiй бц\де, мл~твами 
твои'ми.

Ра'дость и= весе'лiе, и= ра'дованiе бж~е'ственное, 
дв~о неискусому'жная, на'мъ о_у=мно'жися бж~iе: се' бо 
пла'чущiися лю'тjь, ра'дуемся мл~твами твои'ми.

Сла'ва: Же'ртву хвале'нiя пожру`, и= я=зы'комъ и= 
гла'сомъ, пjь'снь бл~года'рственную воздаю' ти 
прилjь'жнw, дв~о: jа='кw да тебjь` моля'ся, въ де'нь 
печа'ли тобо'ю и=зба'влюся.

И= ны'нjь: Сра'дуемся преч\стая, бж~е'ственному 
твоему` рж\ству` бл~гоче'стнw: ра'дость бо на'мъ 
и=сточи'ла _е=си` w\т напа'стей и= печа'лей. тjь'мже ти` 
бл~года'рнw и= мы` пjь'снь соверша'емъ, вjь'рнw 
воспjьва'юще тя`.

Та'же, Досто'йно _е='сть: Трист~о'е. По _О='ч~е на'шъ: 
тропари`, и= про'чее по _о=бы'чаю, и= w\тпу'стъ.
