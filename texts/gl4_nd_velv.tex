%<%[Въ суббw'ту на вели'цjьй вече'рни%],%>

%<на Г%>д\си воззва'хъ, %<стiхи^ры воскр\сны, гла'съ 
д~.%>

%<Стi'хъ: И=%>зведи` и=з\ъ темни'цы ду'шу мою`, 
и=сповjь'датися и='мени твоему`.

%<Ж%>ивотворя'щему твоему` кр\сту`, непреста'ннw 
кла'няющеся хр\сте` бж~е, тридне'вное воскр\снiе твое` 
сла'вимъ: тjь'мъ бо w=бнови'лъ _е=си` и=стлjь'вшее 
человjь'ческое _е=стество` всеси'льне, и= и='же на нб~са` 
восхо'дъ w=бнови'лъ _е=си` на'мъ, jа='кw _е=ди'нъ бл~гъ 
и= чл~вjьколю'бецъ.

%<Стi'хъ: М%>ене` жду'тъ пра'в_едницы, до'ндеже 
возда'си мнjь`.

%<Д%>ре'ва преслуша'нiя запреще'нiе разрjьши'лъ _е=си` 
сп~се, на дре'вjь кр\стнjьмъ во'лею пригвозди'вся, и= во 
а='дъ соше'дъ си'льне, см_е'ртныя о_у='зы jа='кw бг~ъ 
растерза'лъ _е=си`. тjь'мже кла'няемся _е='же и=з\ъ 
ме'ртвыхъ твоему` воскр\снiю, ра'достiю вопiю'ще: 
всеси'льне гд\си, сла'ва тебjь`.

%<Стi'хъ: И=%>з\ъ глубины` воззва'хъ къ тебjь` гд\си, 
гд\си, о_у=слы'ши гла'съ мо'й. (с. 500)

%<В%>рата` а='дwва сокруши'лъ _е=си` гд\си, и= твое'ю 
сме'ртiю сме'ртное ца'рство разруши'лъ _е=си`: ро'дъ же 
человjь'ческiй w\т и=стлjь'нiя свободи'лъ _е=си`, живо'тъ 
и= нетлjь'нiе мi'ру дарова'въ, и= ве'лiю мл\сть.

%<И='ны стiхи^ры, творе'нiе а=нато'лiево.%>

%<Стi'хъ: Д%>а бу'дутъ о_у='ши твои` вне'млющjь гла'су 
моле'нiя моегw`.

%<П%>рiиди'те воспои'мъ лю'дiе, сп~сово тридне'вное 
воста'нiе, и='мже и=зба'вихомся а='довыхъ нерjьши'мыхъ 
о_у='зъ: и= нетлjь'нiе и= жи'знь вси` воспрiя'хомъ 
зову'ще: распны'йся, и= погребы'йся, и= воскр~сы'й, 
сп~си' ны воскр\снiемъ твои'мъ _е=ди'не чл~вjьколю'бче.

%<Стi'хъ: А='%>ще беззакw'нiя на'зриши гд\си, гд\си, 
кто` постои'тъ; jа='кw о_у= тебе` w=чище'нiе _е='сть.

%<А='%>гг~ли и= человjь'цы сп~се, твое` пою'тъ 
тридне'вное воста'нiе, и='мже w=зари'шася вселе'нныя 
концы`, и= рабо'ты вра'жiя вси` и=зба'вихомся, зову'ще: 
животво'рче всеси'льне сп~се сп~си' ны воскр\снiемъ 
твои'мъ, _е=ди'не чл~вjьколю'бче.

%<Стi'хъ: И='%>мене ра'ди твоегw` потерпjь'хъ тя` 
гд\си, потерпjь` душа` моя` въ сло'во твое`, о_у=пова` 
душа` моя` на гд\са.

%<В%>рата` мjь^дная сте'рлъ _е=си`, и= вер_еи` 
сокруши'лъ _е=си` хр\сте` бж~е, и= ро'дъ человjь'ческiй 
па'дшiй воскр~си'лъ _е=си`. сегw` ра'ди согла'снw 
вопiе'мъ: воскр~сы'й и=з\ъ ме'ртвыхъ гд\си, сла'ва 
тебjь`.

%<Стi'хъ: W\т%> стра'жи о_у='треннiя до но'щи, w\т 
стра'жи о_у='треннiя да о_у=пова'етъ i=и~ль на гд\са.

%<Г%>д\си, _е='же w\т _о=ц~а` твое` рж\ство`, 
безлjь'тно _е='сть и= присносу'щно: _е='же w\т дв~ы 
воплоще'нiе, неизрече'нно человjь'кwмъ и= несказа'нно: и= 
_е='же во а='дъ соше'ствiе стра'шно дiа'волу и= 
а='ггелwмъ _е=гw`: сме'рть бо попра'въ, тридне'венъ 
воскр\слъ _е=си`, нетлjь'нiе подава'я человjь'кwмъ, и= 
ве'лiю мл\сть.

%<И='ны стiхи^ры бц\дjь, па'vла а=морре'йскагw. Гла'съ 
и~.%>

%<Подо'бенъ: _W%> пресла'внаго чудесе`!

%<Стi'хъ: Jа='%>кw о_у= гд\са мл\сть, и= мно'гое о_у= 
негw` и=збавле'нiе: и= то'й и=зба'витъ i=и~ля w\т всjь'хъ 
беззако'нiй _е=гw`. (с. 501)

%<Т%>ебе` вjь^рнымъ покро'въ показа`, и='же 
вся'ческихъ бг~ъ, вопло'щься w\т крове'й твои'хъ бц\де 
всеч\стая, и= предста'тельницу и= побо'рницу су'щымъ въ 
ну'ждахъ и= w=бстоя'нiихъ, и= въ бу'ри приста'нище 
бл~гоути'шное: ты` о_у='бw сп~си` w\т вся'кiя ско'рби и= 
туги`, всjь'хъ притека'ющихъ къ бж~е'ственному покро'ву 
твоему`.

%<Стi'хъ: Х%>вали'те гд\са вси` jа=зы'цы, похвали'те 
_е=го` вси` лю'дiе.

%<Д%>а прославля'ю и= почита'ю, да чту` и= пою`, да 
воспjьва'ю всегда` твое` бж~е'ственное и='мя, 
пребл~же'нная вл\дчце, да мя` не w=ста'виши врагw'мъ 
ра'дованiе бы'ти, покро'ву твоему` притека'ющаго: но 
крилы` честны'хъ мл~твъ твои'хъ всегда` цjь'ла мя` 
сохрани` w\т всjь'хъ и=скуше'нiй.

%<Стi'хъ: Jа='%>кw о_у=тверди'ся мл\сть _е=гw` на 
на'съ, и= и='стина гд\сня пребыва'етъ во вjь'къ.

%<Р%>а'дуйся, бг~омт~и преч\стая. ра'дуйся, вjь^рнымъ 
наде'жда. ра'дуйся, мi'ру w=чище'нiе. ра'дуйся, 
и=збавля'ющая вся'кихъ скорбе'й рабы^ твоя^, jа='же 
сме'рти разруши'тельница. ра'дуйся, животоно'сная. 
ра'дуйся, о_у=тjь'шительнице. ра'дуйся, засту'пнице. 
ра'дуйся прибjь'жище.

%<Сла'ва, и= ны'нjь, бг~оро'диченъ:%> И='же тебе` 
ра'ди бг~о_оте'цъ пр\оро'къ дв~дъ пjь'сненнw w= тебjь` 
провозгласи`, вели^чiя тебjь` сотво'ршему: предста` 
цр~и'ца w=десну'ю тебе`. тя' бо мт~рь, хода'таицу живота` 
показа`, без\ъ _о=ц~а` и=з\ъ тебе` вочеловjь'читися 
бл~говоли'вый бг~ъ, да сво'й па'ки w=бнови'тъ w='бразъ, 
и=стлjь'вшiй страстьми`, и= заблу'ждшее горохи'щное 
w=брjь'тъ _о=вча`, на ра'мо воспрiи'мъ, ко _о=ц~у` 
принесе'тъ, и= своему` хотjь'нiю, съ нб\сными совокупи'тъ 
си'лами, и= сп~се'тъ бц\де, мi'ръ, хр\сто'съ и=мjь'яй 
ве'лiю и= бога'тую мл\сть.

%<Та'же, С%>вjь'те ти'хiй: %<Прокi'менъ: Г%>д\сь 
воцр~и'ся: %<и= про'чее по _о=бы'чаю.%>

%<На стiхо'внjь стiхи^ры воскр\сны, гла'съ д~:%>

%<Г%>д\си, возше'дъ на кр\стъ, пра'дjьднюю на'шу 
кля'тву потреби'лъ _е=си`, и= соше'дъ во а='дъ, вjь^чныя 
о_у='зники (с. 502) свободи'лъ _е=си`, нетлjь'нiе да'руя 
человjь'ческому ро'ду: сегw` ра'ди пою'ще сла'вимъ 
животворя'щее и= сп~си'тельное твое` воста'нiе.

%<И='ны стiхи^ры по а=лфави'ту.%>

%<Стi'хъ: Г%>д\сь воцр~и'ся, въ лjь'поту w=блече'ся.

%<П%>овjь'шенъ на дре'вjь _е=ди'не си'льне, всю` 
тва'рь поколеба'лъ _е=си`: положе'нъ же во гро'бjь, 
живу'щыя во гробjь'хъ воскр~си'лъ _е=си`, нетлjь'нiе и= 
жи'знь да'руя человjь'ческому ро'ду. тjь'мже пою'ще 
сла'вимъ тридне'вное твое` воста'нiе.

%<Стi'хъ: И='%>бо о_у=тверди` вселе'нную, jа='же не 
подви'жится.

%<Л%>ю'дiе беззако'ннiи хр\сте`, тебе` преда'вше 
пiла'ту, распя'ти w=суди'ша, неблагода'рни w= 
бл~годjь'тели jа=ви'вшеся. но во'лею претерпjь'лъ _е=си` 
погребе'нiе: самовла'стнw воскр\снъ _е=си` тридне'внw 
jа='кw бг~ъ, да'руя на'мъ безконе'чный живо'тъ, и= ве'лiю 
мл\сть.

%<Стi'хъ: Д%>о'му твоему` подоба'етъ ст~ы'ня гд\си, въ 
долготу` днi'й.

%<С%>о слеза'ми ж_ены` доше'дшя гро'ба, тебе` 
и=ска'ху, не w=брjь'тшя же, рыда'ющя съ пла'чемъ вопiю'щя 
глаго'лаху: о_у=вы` на'мъ, сп~се на'шъ цр~ю` всjь'хъ, 
ка'кw о_у=кра'денъ бы'лъ _е=си`; ко'е же мjь'сто держи'тъ 
живоно'сное тjь'ло твое`; а='гг~лъ же къ ни^мъ 
w\твjьщава'ше, не пла'чите, глаго'летъ, но ше'дшя 
проповjь'дите, jа='кw воскр~се гд\сь, подая` на'мъ 
ра'дость, jа='кw _е=ди'нъ бл~гоутро'бенъ.

%<Сла'ва, и= ны'нjь, бг~оро'диченъ:%> При'зри на 
мол_е'нiя твои'хъ ра^бъ всенепоро'чная, о_у=толя'ющи 
лю^тая на ны` воста^нiя, вся'кiя ско'рби на'съ 
и=змjьня'ющи: тя' бо _е=ди'ну тве'рдое и= и=звjь'стное 
о_у=твержде'нiе и='мамы, и= твое` предста'тельство 
стяжа'хомъ. да не постыди'мся вл\дчце, тя` призыва'ющiи, 
потщи'ся на о_у=моле'нiе, тебjь` вjь'рнw вопiю'щихъ: 
ра'дуйся вл\дчце, всjь'хъ по'моще, ра'досте и= покро'ве, 
и= сп~се'нiе ду'шъ на'шихъ.

%<Та'же, Н%>ы'нjь w\тпуща'еши: %<и= Трист~о'е. По 
_О='%>ч~е на'шъ: (с. 503) 

%<Тропа'рь, гла'съ д~:%>

%<С%>вjь'тлую воскр\снiя про'повjьдь w\т а='гг~ла 
о_у=вjь'дjьвшя гд\сни о_у=ч~нцы, и= пра'дjьднее 
w=сужде'нiе w\тве'ргшя, а=п\слwмъ хва'лящяся глаго'лаху: 
и=спрове'ржеся сме'рть, воскр~се хр\сто'съ бг~ъ, да'руяй 
мi'рови ве'лiю мл\сть.

%<Сла'ва, и= ны'нjь, бг~оро'диченъ: _Е='%>же w\т 
вjь'ка о_у=тае'ное, и= а='гг~лwмъ несвjь'домое та'инство: 
тобо'ю бц\де су'щымъ на земли` jа=ви'ся бг~ъ, въ 
несли'тномъ соедине'нiи воплоща'емь, и= кр\стъ во'лею 
на'съ ра'ди воспрiи'мъ, и='мже воскр~си'въ 
первозда'ннаго, сп~се` w\т сме'рти ду'шы на'шя.
