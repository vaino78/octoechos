%<%[%[Въ недjь'лю на о_у='трени%], по шесто_пса'лмiи%]:%>

%<Б%>г~ъ гд\сь, и= jа=ви'ся на'мъ: %<на гла'съ г~. 
Тропа'рь воскр\снъ: Д%>а веселя'тся нб\сная: С%<ла'ва, и= 
ны'нjь, бг~оро'диченъ: Т%>я` хода'тайствовавшую сп~се'нiе 
ро'да на'шегw: %<Та'же _о=бы'чное стiхосло'вiе.%>

%<По а~-мъ стiхосло'вiи сjьда'льны воскр\сны, гла'съ 
г~:%>

%<Х%>р\сто'съ w\т ме'ртвыхъ воста`, нача'токъ 
о_у=со'пшихъ: перворожде'нъ тва'ри, и= содjь'тель всjь'хъ 
бы'вшихъ, и=стлjь'вшее _е=стество` ро'да на'шегw въ 
себjь` само'мъ w=бнови`. не ктому` сме'рте w=блада'еши: 
и='бо всjь'хъ вл\дка держа'ву твою` разруши`.

%<Стi'хъ: В%>оскр\сни` гд\си бж~е мо'й, да вознесе'тся 
рука` твоя`, не забу'ди о_у=бо'гихъ твои'хъ до конца`. 
(с. 361)

%<П%>ло'тiю сме'рти вкуси'въ гд\си, го'ресть сме'рти 
пресjь'клъ _е=си` воста'нiемъ твои'мъ, и= человjь'ка на 
ню` о_у=крjьпи'лъ _е=си`, пе'рвыя кля'твы w=долjь'нiе 
призыва'я: защи'тниче жи'зни на'шея гд\си, сла'ва тебjь`.

%<Сла'ва, и= ны'нjь, бг~оро'диченъ: К%>расотjь` 
дjь'вства твоегw`, и= пресвjь'тлой чистотjь` твое'й, 
гаврiи'лъ о_у=диви'вся, вопiя'ше ти` бц\де: ку'ю ти` 
похвалу` принесу` досто'йную; что' же воз\ъимену'ю тя`; 
недоумjьва'ю и= о_у=жаса'юся. тjь'мже jа='кw повелjь'нъ 
бы'хъ, вопiю' ти: ра'дуйся бл~года'тная.

%<По в~-мъ стiхосло'вiи, сjьда'льны воскр\сны, гла'съ 
г~.%>

%<Подо'бенъ: К%>расотjь` дв\ства:

%<Н%>еизмjь'ннагw бж~ества`, и= во'льныя стра'сти 
твоея` гд\си, о_у=жа'сся а='дъ, въ себjь` рыда'ше: 
трепе'щу @пло'ти нетлjь'нныя v=поста'си@{@плотска'гw 
неистлjь'вша соста'ва@}, ви'жду неви'димаго, та'йнw 
борю'ща мя`. тjь'мже и= и=`хже держу`, зову'тъ: сла'ва 
хр\сте` воскр\снiю твоему`.

%<Стi'хъ: И=%>сповjь'мся тебjь` гд\си, всjь'мъ 
се'рдцемъ мои'мъ, повjь'мъ вся^ чудеса` твоя^.

%<Н%>епостижи'мое распя'тiя, и= несказа'нное воста'нiя 
бг~осло'вствуемъ вjь'рнiи, та'инство неизрече'нное: 
дне'сь бо сме'рть и= а='дъ плjьни'ся, ро'дъ же 
человjь'ческiй въ нетлjь'нiе w=блече'ся. тjь'мъ 
бл~годаря'ще вопiе'мъ ти`: сла'ва хр\сте` воста'нiю 
твоему`.

%<Сла'ва, и= ны'нjь, бг~оро'диченъ: Н%>епостижи'маго 
и= неwпи'саннаго, _е=диносу'щнаго _о=ц~у` и= дх~ови, во 
о_у=тро'бу твою` та'йнw вмjьсти'ла _е=си` бц\де, 
@@_е=ди'но и= несмjь'сно w\т тр\оцы бж~ество`: позна'хомъ 
рж\ство` твое`@@{@@_е=ди'ну и= неслiя'нну тр\оцы си'лу 
позна'хомъ рж\ство'мъ твои'мъ@@} въ мi'рjь сла'вити. 
тjь'мже и= бл~года'рственнw вопiе'мъ ти`: ра'дуйся 
бл~года'тная.

%<По непоро'чнахъ v=пакои`, гла'съ г~:%>

%<О_у=%>дивля'я видjь'нiемъ, w=роша'я глаго'ланiи, 
блиста'яйся а='гг~лъ мv"роно'сицамъ глаго'лаше: что` 
жива'гw и='щете во (c. 362) гро'бjь; воста` и=стощи'вый 
гро'бы. тли` премjьни'теля разумjь'йте непремjь'ннаго. 
рцы'те бг~ови: ко'ль стра'шна дjьла` твоя^, jа='кw ро'дъ 
сп~слъ _е=си` человjь'ческiй!

%<Степ_е'нна, гла'съ г~. %[А=нтiфw'нъ%] а~-й, 
повторя'юще пое'мъ:%>

%<П%>лjь'нъ сiw'нь ты` и=з\ъя'лъ _е=си` w\т 
вавv"лw'на: и= мене` w\т страсте'й къ животу` привлецы` 
сло'ве.

%<В%>ъ ю='гъ сjь'ющiи слеза'ми бж~е'ственными, жну'тъ 
кла'сы ра'достiю присноживо'тiя.

%<Сла'ва: С%>т~о'му дх~у вся'кое бл~года'рiе, jа='коже 
_о=ц~у` и= сн~у соwблиста'етъ, въ не'мже вся^ живу'тъ и= 
дви'жутся.

%<И= ны'нjь, то'йже. %[А=нтiфw'нъ%] в~-й:%>

%<А='%>ще не гд\сь сози'ждетъ до'мъ добродjь'телей, 
всу'е тружда'емся: ду'шу же покрыва'ющу, никто'же на'шъ 
разори'тъ гра'дъ.

%<П%>лода` чре'вна дх~омъ сынотворе'ное тебjь` хр\сту` 
jа='коже и= _о=ц~у`, ст~i'и всегда` су'ть.

%<Сла'ва: С%>т~ы'мъ дх~омъ прозри'тся вся'кая ст~ы'ня, 
прему'дрость: w=существу'етъ бо вся'кую тва'рь: тому` 
послу'жимъ, бг~ъ бо, jа='кw _о=ц~у' же и= сло'ву.

%<И= ны'нjь, то'йже.%>

%<%[А=нтiфw'нъ%] г~-й:%>

%<Б%>оя'щiися гд\са бл~же'ни, въ пути` ходя'ще 
за'повjьдей: снjьдя'тъ бо живо'тное всепло'дiе.

%<_О='%>крестъ трапе'зы твоея` возвесели'ся, зря` 
твоя^ пастыренача'льниче и=сча^дiя, нося'ща вjь^тви 
бл~годjь'ланiя.

%<Сла'ва: С%>т~ы'мъ дх~омъ вся'кое бога'тство сла'вы, 
w\т негw'же бл~года'ть, и= живо'тъ вся'кой тва'ри: со 
_о=ц~е'мъ бо воспjьва'емь _е='сть, и= съ сло'вомъ.

%<И= ны'нjь, то'йже.%>

%<Прокi'менъ, гла'съ г~: Р%>цы'те во jа=зы'цjьхъ, 
jа='кw гд\сь воцр~и'ся, нб~о и=спра'ви вселе'нную, jа='же 
не подви'жится.

%<Стi'хъ: В%>оспо'йте гд\севи пjь'снь но'ву. 
%<В%>ся'кое дыха'нiе: (с. 363)

%<Стi'хъ: Х%>вали'те бг~а во ст~ы'хъ _е=гw`: 
%<_Е=v\глiе воскр\сно, и= про'чее по ря'ду. В%>оскр\снiе 
хр\сто'во: %<_псало'мъ н~.%>

%<Канw'нъ воскр\снъ. Гла'съ г~.%>

%<%[Пjь'снь а~%]%>

%<I=рмо'съ: В%>о'ды дре'вле, ма'нiемъ бж~е'ственнымъ, 
во _е=ди'но со'нмище совокупи'вый, и= раздjьли'вый мо'ре 
i=и~льт_ескимъ лю'демъ, се'й бг~ъ на'шъ, препросла'вленъ 
_е='сть: тому` _е=ди'ному пои'мъ, jа='кw просла'вися.

%<Припjь'въ: С%>ла'ва гд\си ст~о'му воскр\снiю 
твоему`.

%<И='%>же зе'млю w=суди'въ, престу'пльшему @по'томъ 
и=знести` плода` те'рнiе@{@по'та и=знести` пло'дъ 
т_е'рнiя@}, w\т те'рнiя вjьне'цъ и=з\ъ руки` 
законопресту'пныя, се'й бг~ъ на'шъ, пло'тски прiе'мый, 
кля'тву разруши'лъ _е='сть: jа='кw просла'вися.

%<П%>обjьди'тель и= w=долjь'тель сме'рти, и='же 
сме'рти о_у=боя'вся, jа=ви'ся: стра'стную бо пло'ть 
w=душевле'нную прiе'мъ, се'й бг~ъ на'шъ, и= бра'вся съ 
мучи'телемъ, вся^ совоскр~си`: jа='кw просла'вися.

%<Бг~оро'диченъ: И='%>стинную бц\ду вси` jа=зы'цы 
сла'вятъ тя`, без\ъ сjь'мене ро'ждшую: соше'дъ бо во 
о_у=тро'бу w=свяще'нную твою`, се'й бг~ъ на'шъ, _е='же по 
на'мъ w=существова'ся, бг~ъ же и= человjь'къ и=з\ъ тебе` 
роди'ся.

%<Другi'й канw'нъ кр\стовоскре'сенъ. Гла'съ г~.%>

%<%[Пjь'снь а~%]%>

%<I=рмо'съ: П%>jь'снь но'вую пои'мъ лю'дiе:

%<Р%>о'дъ человjь'чь порабоще'нъ мучи'телемъ 
грjьхолю'бнымъ, кро'вiю бж~е'ственною хр\сто'съ и=скупи`, 
и= w=бг~отвори'въ w=бнови'лъ _е='сть: jа='кw просла'вися.

%<Jа='%>кw о_у='бw сме'ртенъ, сме'рти хотя'й, и='же 
животу` сокро'вищникъ, хр\сто'съ вкуси`: а= jа='кw 
безсме'ртенъ сы'й _е=стество'мъ, м_е'ртвыя w=животвори'лъ 
_е='сть: jа='кw просла'вися. (с. 364)

%<И='нъ канw'нъ прест~jь'й бц\дjь, [_е=гw'же 
краестро'чiе: П%>ою` тре'тiе пjь'нiе тебjь` бц\де.%<] 
Гла'съ г~.%>

%<%[Пjь'снь а~%] %>

%<I=рмо'съ то'йже.%>

%<К%>олjь'но приклоня'етъ вся'ко _е=стество` нб\сныхъ 
дв~о, вопло'щшемуся и=з\ъ тебе`, со земны'ми же 
достолjь'пнw преиспw'дняя: jа='кw просла'вися.

%<_W%> примире'нiй jа=`же въ тебjь`! бл~ги'хъ бо 
незави'стнw пода'тель, jа='кw бг~ъ дх~а бж~е'ственнагw 
пода'вый на'мъ, пло'ть w\т тебе` _о=трокови'це прiя'тъ: 
jа='кw просла'вися.

%<Катава'сiа: W\т%>ве'рзу о_у=ста` моя^:

%<%[Пjь'снь г~%]%>

%<I=рмо'съ: И='%>же w\т не су'щихъ вся^ приведы'й, 
сло'вомъ созида'_емая, соверша'_емая дх~омъ, 
вседержи'телю вы'шнiй, въ любви` твое'й о_у=тверди` 
мене`.

%<К%>р\сто'мъ твои'мъ постыдjь'ся нечести'вый, 
содjь'ла бо jа='му, ю='же и=скопа`, впаде`: смире'нныхъ 
же вознесе'ся хр\сте` ро'гъ, во твое'мъ воскр\снiи.

%<Б%>л~гоче'стiя проповjь'данiе jа=зы'кwмъ, jа='кw 
вода` покры` моря` чл~вjьколю'бче: воскр~съ бо w\т 
гро'ба, тр\оческiй w\ткры'лъ _е=си` свjь'тъ.

%<Бг~оро'диченъ: П%>ресла^вная глагw'лана бы'ша w= 
тебjь`, гра'де w=душевле'нный при'снw цр\ствующагw: 
тобо'ю бо вл\дчце, су'щымъ на земли` бг~ъ спожи'лъ 
_е='сть.

%<И='нъ: I=рмо'съ: Л%>у'къ сокруши'лъ _е=си` вра'жiй:

%<W=%>чище'нiе преч\стый кр\сте, i='дwльскихъ 
jа=ви'лся _е=си` ты` нечисто'тъ: jа='кw i=и~съ 
пребж~е'ственный, на тебjь` ру'цjь распросте'рлъ _е='сть.

%<Д%>а вси` вjь'рнiи живопрiе'мный гро'бе, тебjь` 
покланя'емся: погребе'ся бо въ тебjь` и= воста` хр\сто'съ 
вои'стинну бг~ъ на'шъ.

%<И='нъ. I=рмо'съ то'йже%>

%<Ж%>е'злъ и=з\ъ ко'рене i=ессе'ова, пр\оро'чески 
прозя'бши дв~а, цвjь'тъ тя` хр\сте`, возсiя` на'мъ: ст~ъ 
_е=си` гд\си. (с. 365)

%<Д%>а бж~е'ственному @прича'стiю@{прича'стны} 
земнорw'дныя содjь'лаеши, w\т дв~ы пло'ть на'шу, ты` 
w=бнища'лъ _е=си`, прiе'мъ: ст~ъ _е=си` гд\си.

%<%[Пjь'снь д~%]%>

%<I=рмо'съ: П%>оложи'лъ _е=си` къ на'мъ тве'рдую 
любо'вь гд\си, _е=диноро'днаго бо твоего` сн~а за ны` на 
сме'рть да'лъ _е=си`. тjь'мже ти` зове'мъ бл~годаря'ще: 
сла'ва си'лjь твое'й гд\си.

%<Jа='%>звы хр\сте`, и= ра'ны мл\стивнw под\ъя'лъ 
_е=си`, доса'ду по лани'тома о_у=даре'нiй терпя`, и= 
долготерпjьли'внjь w=плева^нiя нося`, и='миже содjь'лалъ 
_е=си` мнjь` сп~се'нiе: сла'ва си'лjь твое'й гд\си.

%<Т%>jь'ломъ сме'ртнымъ животе`, сме'рти причасти'лся 
_е=си`, стра'сти ра'ди ни'щихъ, и= воздыха'нiя 
о_у=бо'гихъ твои'хъ: и= растли'въ @@тлjь'ющаго@@ 
{растли'вшаго} препросла'вленне, всjь'хъ совоскреси'лъ 
_е=си`, jа='кw просла'вися.

%<Бг~оро'диченъ: П%>омяни` хр\сте`, _е='же стяжа'лъ 
_е=си` ста'до стр\стiю твое'ю: препросла'вленныя твоея` 
мт~ре мл\стивныя мольбы^ прiе'мъ, и= посjьти'въ 
w=sло'бленное, и=зба'ви си'лою твое'ю гд\си.

%<И='нъ. I=рмо'съ: С%>тра'нна и= неизрече'нна:

%<С%>озда'вый по w='бразу твоему` чл~вjьколю'бче 
человjь'ка, и= о_у=мерщвле'на грjьхо'мъ преступле'нiя 
ра'ди, распе'нся на ло'бнjьмъ, сп~слъ _е=си`.

%<М%>_е'ртвыя о_у='бw, и=`хже пожре` сме'рть, 
w\тдаде`: разори'ся же и= а='дово тлетво'рное ца'рство, 
воскр\сшу ти` и=з\ъ гро'ба гд\си.

%<Бг~оро'диченъ: М%>р~i'е ч\стая, злата'я кади'льнице, 
неслiя'ннw jа='кw _е=ди'нъ w\т тр\оцы въ тя` соше'дъ, 
вопло'щься бг~ъ сло'во, мi'ръ w=благоуха'лъ _е='сть.

%<И='нъ. I=рмо'съ то'йже%>

%<И='%>же поста'вивый го'ры вл\дко, мjь'риломъ 
бж~е'ственнагw ра'зума, w\тсjьче'нъ _е=си` w\т дв~ы 
ка'мень, кромjь` ру'къ: си'лjь твое'й сла'ва 
чл~вjьколю'бче. (с. 366)

%<Н%>еду'гующее и=сцjьли'лъ _е=си` на'ше _е=стество` 
вл\дко, скорjь'йшую @и=з\ъ дв~ы сiю`@{@въ дв~jь тому`@} 
соедини'въ цjьльбу`, твое` преч\стое сло'ве бж~ество`.

%<Ч%>а'сть моя` _е=си` гд\си, и= наслjь'дiе жела'нное, 
соедини'вый мя` и=з\ъ дв~ы, сло'ве, v=поста'си твое'й, 
пло'тiю бы'въ v=поста'сь.

%<%[Пjь'снь _е~%]%>

%<I=рмо'съ: К%>ъ тебjь` о_у='тренюю всjь'хъ творцу`, 
преиму'щему вся'къ о_у='мъ ми'рови, зане` свjь'тъ 
повелjь^нiя твоя^: въ ни'хже наста'ви мя`.

%<С%>удiи` непра'ведному, _е=вре'йскою за'вистiю 
пре'данъ бы'въ всеви'дче, и= все'й пра'веднjь судя'й 
земли`, а=да'ма дре'внягw и=зба'вилъ _е=си` w=сужде'нiя.

%<Т%>во'й ми'ръ цр~квамъ твои^мъ хр\сте`, 
непобjьди'мою си'лою кр\ста` твоегw`, и=з\ъ ме'ртвыхъ 
воскр~сы'й пода'ждь, и= сп~си` ду'шы на'шя.

%<Бг~оро'диченъ: С%>ки'нiа ст~а'я, и= простра'ннjьйши 
нб~съ, jа='кw и='же во все'й тва'ри невмjьсти'маго сло'ва 
бж~iя прiе'мши, _е=ди'на jа=ви'лася _е=си` приснодв~о.

%<%[И='нъ%]%>

%<I=рмо'съ: Н%>а земли` неви'димый jа=ви'лся _е=си`, 
и= человjь'кwмъ во'лею сожи'лъ _е=си` непостижи'мый, и= 
къ тебjь` о_у='тренююще, воспjьва'емъ тя` чл~вjьколю'бче.

%<К%>опiе'мъ въ ребро` твое`, _w хр\сте` мо'й, 
прободе'нъ бы'въ, w\т ребра` человjь'ча созда'нную, 
губи'тельства всjь^мъ человjь'кwмъ бы'вшую хода'таицу, 
кля'твы свободи'лъ _е=си`.

%<Р%>а'венъ _о=ц~у` по существу` сы'й, сщ~е'нный 
хра'мъ преч\стагw твоегw` и= всеч\стна'гw тjьлесе`, и=з\ъ 
ме'ртвыхъ воскр~си'лъ _е=си` хр\сте` сп~се на'шъ.

%<И='нъ. I=рмо'съ то'йже%>

%<С%>ло'во бж~iе сн~ъ тво'й дв~о, содjь'тель а=да'ма 
первозда'ннагw, не созда'нiе, а='ще и= пло'ть 
w=душевле'ну себjь` и=з\ъ тебе` созда'лъ _е='сть.

%<Р%>а'венъ _о=ц~у` сн~ъ тво'й дв~о, сло'во бж~iе, 
v=поста'сь соверше'нна во двою` _е=ст_еству`, i=и~съ 
гд\сь, бг~ъ соверше'нъ и= чл~вjь'къ. (с. 367)

%<%[Пjь'снь s~%]%>

%<I=рмо'съ: Б%>е'здна послjь'дняя грjьхw'въ w=бы'де 
мя`, и= и=счеза'етъ ду'хъ мо'й: но простры'й вл\дко 
высо'кую твою` мы'шцу, jа='кw петра' мя о_у=пра'вителю 
сп~си`.

%<Б%>е'здна мл\сти и= щедро'тъ w=бы'де мя`, 
бл~гоутро'бнымъ сни'тiемъ твои'мъ: вопло'щься бо вл\дко, 
и= бы'въ въ ра'бiи зра'цjь w=божи'лъ _е=си`, съ собо'ю 
сопросла'вивъ.

%<О_у=%>мерщвле'нiе под\ъя'тъ о_у=мертви'тель, 
о_у=мерщвле'наго w=живле'нно ви'дя: твоегw` воскр\снiя 
сi'и су'ть хр\сте` w='бразы, и= стр\сти твоея` преч\стыя 
побjьди'т_ельная.

%<Бг~оро'диченъ: П%>реч\стая, jа='же _е=ди'на 
созда'телю и= человjь'кwмъ, па'че о_у=ма` 
и=схода'таившая, сн~а твоего` мл\стива прегрjь'шшымъ 
рабw'мъ твои^мъ, и= побо'рника бы'ти, помоли'ся.

%<%[И='нъ%]%>

%<I=рмо'съ: С%>еле'нiя i=w'на, _е='же въ 
преиспо'днjьмъ а='дjь, _е=сте'ственный w='бразъ бы'въ, 
вопiя'ше: возведи` w\т тли` живо'тъ мо'й, чл~вjьколю'бче.

%<Р%>а'нами ты` и=скуси'вся, w\т а='да о_у=ра'неныхъ, 
стр\стiю кр\ста` совоскр~си'лъ _е=си`. тjь'мже ти` зову`: 
возведи` w\т тли` живо'тъ мо'й, чл~вjьколю'бче.

%<W\т%>верза'ются ти` хр\сте` стра'хомъ врата` 
а='дwва, сосу'ды же вра^жiя восхища'ются: тjь'мже тя` 
ж_ены` срjьто'ша, вмjь'стw печа'ли ра'дость прiи'мшя.

%<И='нъ. I=рмо'съ то'йже%>

%<В%>оwбража'ется _е='же по на'мъ, w\т нетлjь'нныя 
дв~ы, и='же @w='бразомъ неразлу'чный@{@w='бразу 
неприча'стный@} w='бразомъ бы'въ и= ве'щiю, не прело'жся 
бж~ество'мъ человjь'къ.

%<Б%>е'здны грjьхw'въ, и= бу'ри страсте'й преч\стая 
и=зба'ви мя`: _е=си' бо приста'нище, и= бе'здна чуде'съ, 
вjь'рою притека'ющымъ къ тебjь`.

%<Конда'къ, гла'съ г~. Подо'бенъ: Д%>в~а дне'сь:

%<В%>оскр\слъ _е=си` дне'сь и=з\ъ гро'ба ще'дре, и= 
на'съ возве'лъ _е=си` w\т вра'тъ сме'ртныхъ: дне'сь 
а=да'мъ лику'етъ, и= ра'дуется (с. 368) _е='vа, вку'пjь 
же и= пр\оро'цы съ патрiа'рхи воспjьва'ютъ непреста'ннw 
бж~е'ственную держа'ву вла'сти твоея`.

%<I='косъ: Н%>б~о и= земля` дне'сь да ликовству'ютъ, 
и= хр\ста` бг~а _е=диному'дреннw да воспjьва'ютъ, jа='кw 
о_у='зники w\т гробw'въ воскр~си`. сра'дуется вся` 
тва'рь, принося'щи достw'йныя пjь^сни созда'телю всjь'хъ, 
и= и=зба'вителю на'шему: jа='кw человjь'ки и=з\ъ а='да 
дне'сь jа='кw жизнода'тель совозве'дъ, на нб~са` 
совозвыша'етъ, и= низлага'етъ вра^жiя вознош_е'нiя, и= 
врата` а='дwва сокруша'етъ бж~е'ственною держа'вою 
вла'сти своея`.

%<%[Пjь'снь з~%]%>

%<I=рмо'съ: Jа='%>коже дре'вле бл~гочести^выя три` 
_о='троки w=роси'лъ _е=си` въ пла'мени халде'йстjьмъ, 
свjь'тлымъ бж~ества` _о=гне'мъ и= на'съ w=зари`, 
бл~гослове'нъ _е=си`, взыва'ющыя, бж~е _о=т_е'цъ на'шихъ.

%<Р%>аздра'ся цр~ко'вная свjь'тлая 
@катапета'сма@{@завjь'са@}, въ распя'тiи содjь'теля, 
сокрове'нную въ писа'нiи jа=вля'ющи вjь^рнымъ и='стину, 
бл~гослове'нъ _е=си`, зову'щымъ, бж~е _о=т_е'цъ на'шихъ.

%<П%>робод_е'ннымъ твои^мъ ре'брwмъ, ка'плями 
бг~ото'чныя животворя'щiя кро'ве хр\сте`, смотри'тельнw 
ка'плющiя на зе'млю, су'щихъ w\т земли` возсозда'лъ 
_е=си`, бл~гослове'нъ _е=си`, зову'щихъ, бж~е _о=т_е'цъ 
на'шихъ.

%<Тр\оченъ: Д%>х~а бл~га'го со _о=ц~е'мъ просла'вимъ, 
и= съ сн~омъ _е=диноро'днымъ, _е=ди'но въ трiе'хъ 
вjь'рнiи чту'ще нача'ло, и= _е=ди'но бж~ество`: 
бл~гослове'нъ _е=си`, зову'ще, бж~е _о=т_е'цъ на'шихъ.

%<%[И='нъ%]%>

%<I=рмо'съ: %>Го'рдый мучи'тель, но дjьте'й бы'сть 
и=гра'лище: jа='коже бо пе'рсть попра'вше седмери'чный 
пла'мень, поя'ху: бл~гослове'нъ _е=си` гд\си, бж~е 
_о=т_е'цъ на'шихъ.

%<Н%>е про'ста со'лнце на кр\стjь` ви'сяща чл~вjь'ка, 
но бг~а воплоще'нна зря` помрача'ется. _е=му'же и= 
пое'мъ: бл~гослове'нъ _е=си` гд\си, бж~е _о=т_е'цъ 
на'шихъ. (с. 369)

%<К%>рjь'пкаго бж~ество'мъ прiе'мъ а='дъ страшли'вый, 
нетлjь'нiя пода'теля, ду'шы пра'ведныхъ вопiю'щыя 
и=зблева`: бл~гослове'нъ _е=си` гд\си, бж~е _о=т_е'цъ 
на'шихъ.

%<Бг~оро'диченъ: С%>окро'вище многоцjь'нное 
бл~гослове'нiя jа=ви'лася _е=си` преч\стая, чи'стымъ 
се'рдцемъ тебе` и=сповjь'дающымъ бг~ороди'тельницу: нб~о 
и=з\ъ тебе` воплоти'ся бг~ъ _о=т_е'цъ на'шихъ.

%<И='нъ. I=рмо'съ то'йже%>

%<И='%>же сла'вы гд\сь, и= держа'й гw'рнiя си^лы, 
и='же со _о=ц~е'мъ сjьдя'й, дjь'вственныма рука'ма 
носи'мь: бл~гослове'нъ _е=си` гд\си, бж~е _о=т_е'цъ 
на'шихъ.

%<Jа='%>ра сме'рть, но сiю` тебjь` @бесjь'довавшую@ 
{@соедини'вшуся@} погуби'лъ _е=си`, w\т дв~ы 
бг~оv"поста'сная пло'ть бы'въ: бл~гослове'нъ _е=си` 
гд\си, бж~е _о=т_е'цъ на'шихъ.

%<Б%>ц\ду вси' тя бг~а ро'ждшую о_у=вjь'дjьхомъ, 
_е=ди'наго бо w\т тр\оцы воплоти'вшагося и=з\ъ тебе` 
родила` _е=си`: бл~гослове'нъ преч\стая, пло'дъ твоегw` 
чре'ва.

%<%[Пjь'снь и~%]%>

%<I=рмо'съ: Н%>естерпи'мому _о=гню` соедини'вшеся, 
бг~оче'стiя предстоя'ше ю='нwши, пла'менемъ же 
неврежде'ни, бж~е'ственную пjь'снь поя'ху: бл~гослови'те 
вся^ дjьла` гд\сня гд\са, и= превозноси'те во вся^ 
вjь'ки.

%<Р%>аздра'ся цр~ко'вная свjь'тлость, _е=гда` кр\стъ 
тво'й водрузи'ся на ло'бнjьмъ, и= тва'рь преклоня'шеся 
стра'хомъ, воспjьва'ющи: бл~гослови'те вся^ дjьла` гд\сня 
гд\са, по'йте и= превозноси'те _е=го` во вjь'ки.

%<В%>оскр\слъ _е=си` хр\сте` и=з\ъ гро'ба, и= па'дшаго 
прельще'нiемъ, дре'вомъ и=спра'вилъ _е=си` бж~е'ственною 
си'лою, зову'ща и= глаго'люща: бл~гослови'те вся^ дjьла` 
гд\сня гд\са, по'йте и= превозноси'те _е=го` во вjь'ки.

%<Бг~оро'диченъ: Х%>ра'мъ бж~iй jа=ви'лася _е=си` 
вмjьсти'лище w=душевле'нное, и= ковче'гъ: творца' бо 
человjь'кwмъ, (с. 370) бг~ороди'тельнице преч\стая, 
примири'ла _е=си`, и= досто'йнw вся^ дjьла` пое'мъ тя`, 
и= превозно'симъ во вся^ вjь'ки.

%<%[И='нъ%]%>

%<I=рмо'съ: В%>еще'ственнагw _о=гня` пла'мень 
невеще'ственнымъ о_у=вяди'ша, бг~озри'мiи _о='троцы, и= 
поя'ху: бл~гослови'те вся^ дjьла` гд\сня гд\са.

%<С%>ло'во нестра'стное, безстра'стно о_у='бw 
бж~ество'мъ, стра'ждетъ же пло'тiю бг~ъ, _е=му'же и= 
пое'мъ: бл~гослови'те вся^ дjьла` гд\сня гд\са, по'йте и= 
превозноси'те _е=го` во вjь'ки.

%<О_у=%>сну'вый о_у='бw jа='кw сме'ртенъ, воскр\снъ 
_е=си` jа='кw безсме'ртенъ сп~се, и= сп~са'еши w\т 
сме'рти пою'щихъ: бл~гослови'те вся^ дjьла` гд\сня гд\са, 
по'йте и= превозноси'те _е=го` во вjь'ки.

%<Тр\оченъ: С%>лу'жимъ бл~гоче'стнw трiv"поста'сному 
бж~еству`, соединя'ему неизрече'ннw, и= пое'мъ: 
бл~гослови'те вся^ дjьла` гд\сня гд\са, по'йте и= 
превозноси'те _е=го` во вjь'ки.

%<И='нъ. I=рмо'съ то'йже%>

%<Ч%>и'ны о_у='мныя jа='кw мт~и превозшла` _е=си`, и= 
бли'зъ бг~а бы'вши: бл~гослови'мъ бл~гослове'нная дв~о, 
рж\ство` твое`, и= превозно'симъ во вся^ вjь'ки.

%<Д%>обро'ту _е=сте'ственную, краснjь'йшу показа'ла 
_е=си`, w=блистава'ющую пло'ть бж~ества`. бл~гослови'мъ 
бл~гослове'нная дв~о, рж\ство` твое`, и= превозно'симъ во 
вся^ вjь'ки.

%<Та'же, пjь'снь бц\ды: В%>ели'читъ душа` моя` гд\са: 
%<съ припjь'вомъ: Ч\с%>тнjь'йшую херувi^мъ:

%<%[Пjь'снь f~%]%>

%<I=рмо'съ: Н%>о'вое чу'до и= бг~олjь'пное, 
дв~и'ческую бо две'рь затворе'ную jа='вjь прохо'дитъ 
гд\сь, на'гъ во вхо'дjь, и= плотоно'сецъ jа=ви'ся во 
и=схо'дjь бг~ъ, и= пребыва'етъ две'рь затворе'на: сiю` 
неизрече'ннw, jа='кw бг~омт~рь велича'емъ.

%<С%>тра'шно _е='сть зрjь'ти тебе` творца` на дре'вjь 
воздви'жена сло'ве бж~iй, пло'тски же стра'ждуща бг~а за 
рабы^, и= во гро'бjь бездыха'нна лежа'ща, м_е'ртвыя же 
и=з\ъ а='да разрjьши'вша: тjь'мже тя` хр\сте`, jа='кw 
всеси'льна велича'емъ. (с. 371)

%<И=%>з\ъ тли` сме'ртныя сп~слъ _е=си` хр\сте` 
пра'_отцы, положе'нъ бы'въ во гро'бjь ме'ртвъ, и= живо'тъ 
процвjь'лъ _е=си`, м_е'ртвыя воскр~си'въ, руководи'въ 
_е=стество` человjь'ческое ко свjь'ту, и= въ 
бж~е'ственное w=бле'къ нетлjь'нiе. тjь'мже и=сто'чника 
тя` свjь'та при'снw жива'гw велича'емъ.

%<Бг~оро'диченъ: Х%>ра'мъ и= пр\сто'лъ jа=ви'лася 
_е=си` бж~iй, во'ньже всели'ся и='же въ вы'шнихъ сы'й, 
рожде'йся неискусому'жнw всеч\стая, пло'ти твоея` не 
w\тве'рзъ вся'чески врата`. тjь'мже непреста'нными ч\стая 
мл~твами твои'ми, jа=зы'ки ва^рварскiя ско'рw до конца` 
покори`.

%<%[И='нъ%]%>

%<I=рмо'съ: С%>ла'дкою преч\стагw твоегw` рж\ства` 
стрjьло'ю о_у=я'звлени ч\стая, твое'й достожела'ннjьй 
добро'тjь дивя'щеся, пjь'сньми а='гг~льскими досто'йнw 
тя`, jа='кw мт~рь бж~iю велича'емъ.

%<Ч%>е'сть человjь'кwмъ w\т безче'стныя сме'рти 
всjь^мъ и=сточи'лъ _е=си`: _е=я'же распя'тiемъ твои'мъ 
сп~се вкуси'въ, существо'мъ сме'ртнымъ нетлjь'нiе мнjь` 
дарова'лъ _е=си` хр\сте`, jа='кw чл~вjьколю'бецъ.

%<С%>п~слъ мя` _е=си` воскр~съ и=з\ъ гро'ба хр\сте`, 
@воскр\слъ же@{@и= возне'слъ@} _е=си`, и= _о=ц~у` 
приве'лъ _е=си` твоему` роди'телю: w=десну'ю же _е=гw` 
спосади'лъ _е=си` за бл~гоутро'бiе мл\сти твоея` гд\си.

%<И='нъ. I=рмо'съ то'йже%>

%<С%>ы'тость твои'хъ похва'лъ дв~о, бл~гоч_ести'вымъ 
вjь^рнымъ w\тню'дъ не быва'етъ: жела'нiемъ бо жела'нiе 
при'снw бж~е'ственное, и= духо'вное прiе'млюще, jа='кw 
мт~рь бж~iю велича'емъ.

%<П%>оложи'лъ _е=си` на'мъ непосты'дную мл~твенницу, 
тебе` ро'ждшую хр\сте`. тоя` мольба'ми мл\стива подае'ши 
на'мъ дх~а, пода'теля бл~гости, w\т _о=ц~а` тобо'ю 
происходя'ща.

%<По катава'сiи _е=ктенiа`. Та'же, С%>т~ъ гд\сь бг~ъ 
на'шъ. %<И= свjьти'ленъ.%>

%<На хвали'техъ стiхи^ры воскр\сны, гла'съ г~.%>

%<Стi'хъ: С%>отвори'ти въ ни'хъ су'дъ напи'санъ: 
сла'ва сiя` бу'детъ всjь^мъ прп\дбнымъ _е=гw`. (с. 372)

%<П%>рiиди'те вси` jа=зы'цы, о_у=разумjь'йте стра'шныя 
та'йны си'лу: хр\сто'съ бо сп~съ на'шъ, _е='же въ 
нача'лjь сло'во, распя'тся на'съ ра'ди, и= во'лею 
погребе'ся, и= воскр~се и=з\ъ ме'ртвыхъ, _е='же сп~сти` 
вся'ч_еская: тому` поклони'мся.

%<Стi'хъ: Х%>вали'те бг~а во ст~ы'хъ _е=гw`, хвали'те 
_е=го` во о_у=тверже'нiи си'лы _е=гw`.

%<П%>овjь'даша вся^ чудеса` стра'жiе твои` гд\си: но 
собо'ръ суеты` и=спо'лни мздо'ю десни'цу и='хъ, скры'ти 
мня'ше воскр\снiе твое`, _е='же мi'ръ сла'витъ: поми'луй 
на'съ.

%<Стi'хъ: Х%>вали'те _е=го` на си'лахъ _е=гw`, 
хвали'те _е=го` по мно'жеству вели'чествiя _е=гw`.

%<Р%>а'дости вся^ и=спо'лнишася воскр\снiя и=ску'съ 
прiи^мша: марi'а бо магдали'на ко гро'бу прiи'де, 
w=брjь'те а='гг~ла на ка'мени сjьдя'ща, ри'зами 
блиста'ющася и= глаго'люща: что` и='щете жива'гw съ 
ме'ртвыми: нjь'сть здjь`, но воста`, jа='коже рече`, 
предваря'я вы` въ галiле'и.

%<Стi'хъ: Х%>вали'те _е=го` во гла'сjь тру'бнjьмъ: 
хвали'те _е=го` въ _псалти'ри и= гу'слехъ.

%<В%>о свjь'тjь твое'мъ вл\дко, о_у='зримъ свjь'тъ 
чл~вjьколю'бче: воскр\снъ бо _е=си` и=з\ъ ме'ртвыхъ, 
сп~се'нiе ро'ду человjь'ческому да'руя: да тя` вся` 
тва'рь славосло'витъ _е=ди'наго безгрjь'шнаго, поми'луй 
на'съ.

%<И='ны стiхи^ры а=нато'лiевы, гла'съ то'йже.%>

%<Стi'хъ: Х%>вали'те _е=го` въ тv"мпа'нjь и= ли'цjь, 
хвали'те _е=го` въ стру'нахъ и= _о=рга'нjь.

%<П%>jь'снь о_у='треннюю мv"ронw'сицы ж_ены` со 
слеза'ми приноша'ху тебjь` гд\си, бл~гоуха'нiя бо 
а=рwма'ты и=му'щя, гро'ба твоегw` достиго'ша, преч\стое 
тjь'ло твое` пома'зати тща'щяся. а='гг~лъ сjьдя'й на 
ка'мени тjь^мъ бл~говjьсти`: что` и='щете жива'гw съ 
ме'ртвыми; сме'рть бо попра'въ воскр~се jа='кw бг~ъ, 
подая` всjь^мъ ве'лiю мл\сть.

%<Стi'хъ: Х%>вали'те _е=го` въ кv"мва'лjьхъ 
доброгла'сныхъ, хвали'те _е=го` въ кv"мва'лjьхъ 
восклица'нiя: вся'кое дыха'нiе да хва'литъ гд\са. (с. 
373)

%<Б%>листа'яся а='гг~лъ на гро'бjь твое'мъ 
животво'рнjьмъ, мv"роно'сицамъ глаго'лаше: и=стощи'въ 
гро'бы и=зба'витель плjьни` а='да, и= воскр~се 
тридне'венъ, jа='кw _е=ди'нъ бг~ъ и= всеси'ленъ.

%<Стi'хъ: В%>оскр\сни` гд\си бж~е мо'й, да вознесе'тся 
рука` твоя`, не забу'ди о_у=бо'гихъ твои'хъ до конца`.

%<В%>о гро'бjь тя` и=ска'ше, прише'дши во _е=ди'ну w\т 
суббw'тъ марi'а магдали'на, не w=брjь'тши же рыда'ше съ 
пла'чемъ вопiю'щи: о_у=вы` мнjь` сп~се мо'й! о_у=кра'денъ 
бы'лъ _е=си` всjь'хъ цр~ю`. супру^гъ же живоно'сныхъ 
а='гг~лъ вну'трь гро'ба вопiя'ше: что` пла'чеши _w же'но; 
пла'чу, глаго'летъ, jа='кw взя'ша гд\са моего` w\т 
гро'ба, и= не вjь'мъ гдjь` положи'ша _е=го`. сiя' же 
w=бра'щшися вспя'ть, jа='кw ви'дjь тя`, а='бiе возопи`: 
гд\сь мо'й и= бг~ъ мо'й, сла'ва тебjь`.

%<Стi'хъ: И=%>сповjь'мся тебjь` гд\си всjь'мъ 
се'рдцемъ мои'мъ, повjь'мъ вся^ чудеса` твоя^.

%<_Е=%>вре'и затвори'ша во гро'бjь живо'тъ, 
разбо'йникъ же w\тве'рзе я=зы'комъ наслажде'нiе, зовы'й 
и= глаго'ля: и='же со мно'ю мене` ра'ди распны'йся: 
соwбjь'си ми ся на дре'вjь, и= jа=ви'ся мнjь` на 
пр\сто'лjь со _о=ц~е'мъ сjьдя`: то'й бо _е='сть хр\сто'съ 
бг~ъ на'шъ, и=мjь'яй ве'лiю мл\сть.

%<Сла'ва, стiхи'ра _е=v\гльская. И= ны'нjь, 
бг~оро'диченъ: П%>ребл~гослове'нна _е=си` бц\де дв~о: 
%<Славосло'вiе вели'кое.%>

%<По славосло'вiи тропа'рь:%>

%<Д%>не'сь сп~се'нiе мi'ру бы'сть, пое'мъ воскр\сшему 
и=з\ъ гро'ба, и= нача'льнику жи'зни на'шея: разруши'въ бо 
сме'ртiю сме'рть, побjь'ду даде` на'мъ и= ве'лiю мл\сть.

%<И= _е=кт_енiи`, и= w\тпу'стъ.%>
