%<%[Въ суббw'ту на повече'рiи%],%>

%<канw'нъ прест~jь'й бц\дjь. Гла'съ г~%>

%<%[Пjь'снь а~%]%>

%<I=рмо'съ: В%>о'ды дре'вле, ма'нiемъ бж\ственнымъ, во 
_е=ди'но со'нмище совокупи'вый, и= раздjьли'вый мо'ре 
i=и~льт_ескимъ лю'демъ, се'й бг~ъ на'шъ, препросла'вленъ 
_е='сть: тому` _е=ди'ному пои'мъ, jа='кw просла'вися.

%<Jа='%>кw мт~рь бж~iю тя` дх~омъ бж~е'ственнымъ, 
_о=ц~а' же благоволе'нiемъ jа='вльшуюся, вjь'рнiи 
пjь'сньми чи'стыми вjьнча'имъ, пречи'стая бг~оневjь'сто: 
и= цjьлу'емъ тя` пjь'ньми со а=рха'гг~ломъ, во сп~се'нiе 
на'ше.

%<W\т%> ре'бръ о_у='бw а=да'мовыхъ созида'ется _е='vа, 
дре'внимъ бг~озда'нiемъ: w\т о_у=тро'бы же бц\ды jа=ви'ся 
хр\сто'съ, сы'й бг~ъ на'шъ, вочл~вjь'чся непрело'жнw, 
вре'мененъ бы'сть превjь'чный.

%<Сла'ва: И='%>же _е='vину о_у=тро'бу w=су'ждь въ 
печа'лехъ, роди'ти плоды` въ болjь'зни, во чре'во твое` 
всели'ся дв~о ч\стая, сы'й бг~ъ на'шъ, пло'тiю jа=ви'ся 
па'че сло'ва, и= прама'тернiй разрjьши` до'лгъ.

%<И= ны'нjь: В%>ъ неча'янiя глубину` впа'дше 
тягота'ми, прилjь'жнw зове'мъ бг~ороди'тельнице къ 
тебjь`: вл\дчце, помози` потопля'_емымъ дjь'лы лука'выхъ 
прегрjьше'нiй, тя' бо _е=ди'ну и='мамы наде'жду по бз~jь. 
(с. 351)

%<%[Пjь'снь г~%]%>

%<I=рмо'съ: И='%>же w\т не су'щихъ вся^ приведы'й, 
сло'вомъ созида'_емая, соверша'_емая дх~омъ, 
вседержи'телю вы'шнiй, въ любви` твое'й о_у=тверди` 
мене`.

%<Т%>я` проявля'ше же'злъ а=арw'новъ, без\ъ напое'нiя 
процвjь'тшiй, бц\де преч\стая, без\ъ сjь'мене ро'ждши 
вопло'щшагося бг~а непремjь'ннw.

%<Б%>ж~е'ственный _о='гнь тя` нося'щу, jа='кw 
свjьти'льникъ преч\стая, прови'дjь пр\оро'къ дх~омъ, 
благоуха'нiе нося'щую су'щымъ въ мi'рjь, и= жи'знь 
вjь'чную.

%<Сла'ва: К%>о а=рха'гг~лу присту'пимъ гаврiи'лу, 
ра'дуйся, вjьща'юще дв~jь пjь'сньми: тобо'ю бо 
разрjьши'тся w=сужде'нiя пра'дjьдняя кля'тва.

%<И= ны'нjь: Т%>я` прест~а'я бц\де, jа='кw стjь'ну 
сп~се'нiя и=му'ще грjь'шнiи сп~са'емся: не пре'зри 
вл\дчце, и= не посрами` на^ша мол_е'нiя.

%<%[Пjь'снь д~%]%>

%<I=рмо'съ: П%>оложи'лъ _е=си` къ на'мъ тве'рдую 
любо'вь гд\си, _е=диноро'днаго бо твоего` сн~а за ны` на 
сме'рть да'лъ _е=си`. тjь'мже ти` зове'мъ бл~годаря'ще: 
сла'ва си'лjь твое'й гд\си.

%<Г%>о'ру мы'сленную, добродjь'телей сjь'нь, дх~омъ 
дре'вле а=вваку'мъ бг~ови'днw тя` ви'дjьвъ, проповjь'даше 
преч\стая: jа='кw w\т ю='га прише'дша w\т тебе` сло'во 
пло'ть прiе'мша.

%<Г%>о'ру тя` ве'лiю несjько'мую данiи'лъ дх~омъ 
ви'дjь, не новосjько'му jа=вля'я ч\стоту` дв~ства твоегw` 
вл\дчце: w\т нея'же ка'мень о_у=сjьче'ся, хр\сто'съ 
сло'во, л_е'сти низлага'я i='дwльскiя.

%<Сла'ва: Г%>о'ру тя` дв~дъ чу'дну и= ту'чну прорече`: 
_е=диноро'дный бо w\т _о=ц~а` сн~ъ въ тя` бл~говоли` 
ч\стая, всели'тися воплоща'емь. тjь'мъ ти`, ра'дуйся, 
ду'хомъ зове'мъ.

%<И= ны'нjь: В%>ся` бл~га'я и= те'плая засту'пница 
су'щи грjь'шникwмъ и= смир_е'ннымъ, бг~ороди'тельнице 
преч\стая вл\дчце, w\т бjь'дъ, и= скорбе'й, и= грjьхw'въ 
сп~са'й рабы^ твоя^. (с. 352)

%<%[Пjь'снь _е~%]%>

%<I=рмо'съ: Jа='%>кw ви'дjь и=са'iа w=бра'знw на 
пр\сто'лjь превознесе'на бг~а, w\т а='гг~лъ сла'вы 
дорv"носи'ма, _w _о=кая'нный, вопiя'ше, а='зъ: 
прови'дjьхъ бо воплоща'ема бг~а, свjь'та невече'рня, и= 
ми'ромъ влады'чествующа.

%<И=%>зрасти` же'злъ i=ессе'евъ цвjь'тъ неувяда'емый, 
дв~а мр~i'а бг~а безнача'льнаго, без\ъ сjь'мене, w\т дх~а 
бж~е'ственнагw и= _о='ч~а, ца^рства держа'вная 
jа=зы'ч_еская w=блада'вшаго, на него'же jа=зы'цы 
о_у=пова'ша.

%<К%>ня'зь ми'рный прiи'де на пр\сто'лъ дв~довъ, 
ца'рствуя и=з\ъ тебе` бц\де бг~ъ воплоща'емь: _w чу'до! 
и= бра'нь разрjь'шъ, потче` кня'зи мwавi^тскiя, и= тебе` 
ро'ждшую цр~и'цу показа`.

%<Сла'ва: Н%>епоро'чную твою` добро'ту, w\т нея'же 
хр\сто'съ въ пло'ть w=блече'ся без\ъ сjь'мене дв~о, 
и=са'iа jа=вля'я, зове'тъ вопiя`: гд\сь сла'вы гряде'тъ 
на _о='блацjь ле'гцjь, и= пре'лести тьму` w\тгоня'я, 
просвjьти'тъ на'съ.

%<И= ны'нjь: W\т%> дх~а ст~а заче'нши сло'во 
_е=диносу'щное _о=ц~у`, дв~о сiе` родила` _е=си` во дву` 
_е=ст_еству` су'ща, бг~а соверше'нна и= соверше'нна 
чл~вjь'ка: _е=гw'же вjь'рою почита'емъ плотско'е 
jа=вле'нiе.

%<%[Пjь'снь s~%]%>

%<I=рмо'съ: Б%>е'здна послjь'дняя грjьхw'въ w=бы'де 
мя`, и= и=счеза'етъ ду'хъ мо'й: но простры'й вл\дко 
высо'кую твою` мы'шцу, jа='кw петра' мя о_у=пра'вителю 
сп~си`.

%<_О='%>дръ тя` пjь'сньми прему'дрый преднапису'етъ 
всепjь'тая, на не'мже почи` бг~ъ воплоща'емь во 
v=поста'си и=з\ъ тебе`, и= просла'ви тя`, несмjь'снw 
ражда'емь.

%<Б%>ы'сть сло'ву jа='кw и=збра'нная, дв~о всепjь'тая, 
и=збра'нная _о=де'жда: jа='кw порфv'ру бж~е'ственную 
и=з\ъ тебе` пло'ть прiе'мъ воцари'ся, бг~олjь'пнw 
w=дjь'явся.

%<Сла'ва: Б%>ы'сть соедине'нiя бж~е'ственнагw 
вмjьсти'лище, бг~оневjь'сто, зла'та свjьтлjь'йши: тобо'ю 
бо бы'сть бг~ъ jа='кw человjь'къ, и= бесjь'дова 
человjь'кwмъ бг~ъ, jа='кw человjь'къ. (с. 353)

%<И= ны'нjь: О_у=%>мертви` нече'ствующихъ въ тя`, дв~о 
всепjь'тая, sло'бы _е='ресь: jа='кw за'вистiю та'ютъ 
пресла'вное твоегw` w='браза ви'дяще подо'бiе.

%<Г%>д\си поми'луй, %<три'жды. Сла'ва, и= ны'нjь:%>

%<Сjьда'ленъ, гла'съ г~:%>

%<W\т%> теплоты` вjь'ры вопiю' ти бц\де, недосто'йными 
о_у=сты`, и= скве'рнымъ се'рдцемъ: сп~си' мя 
погруже'ннаго грjьха'ми, о_у=ще'дри о_у=мерщвле'ннаго 
w\тча'янiемъ, да зову' ти сп~са'емь: ра'дуйся дв~о, 
хр\стiа'нwмъ помо'щнице.

%<%[Пjь'снь з~%]%>

%<I=рмо'съ: П%>ре'жде w='бразу злато'му, персi'дскому 
чти'лищу, _о='троцы не поклони'шася, трiе` пою'ще 
посредjь` пе'щи: _о=тц_е'въ бж~е бл~гослове'нъ _е=си`.

%<К%>упина` и= пла'мень соwбра'знjь совоку'пльшеся, и= 
нетлjь^нна _о=боя` jа='вльшеся, тя` дв~о jа='снw 
проявля'ютъ: и='бо родила` _е=си` бг~а, и= дjь'вствуеши.

%<Р%>уно` и= роса` гедеw'ну во и=змjьне'нiи 
w=бразу'ема, рж\ство` твое` предпи'шутъ: ты' бо _е=ди'на 
бж~е'ственное сло'во но'сиши во чре'вjь, jа='кw до'ждь, 
дв~о мт~и.

%<Сла'ва: Г%>рjьха` моегw` _о='гнь па'че гее'нскагw, 
пла'мень мнjь` содjьва'етъ, сi'й о_у=гаси` ч\стая, твое'ю 
мл\стiю, покая'нiемъ мя` ко свjь'ту наставля'ющи.

%<И= ны'нjь: Т%>воегw` зра'ка jа=вле'нiе бц\де 
преч\стая, чту'ще jа='кw пе'рвый w='бразъ тя`, 
засту'пницу при'снw, и= покро'въ благопремjь'ненъ къ бг~у 
и='мамы вси`.

%<%[Пjь'снь и~%]%>

%<I=рмо'съ: В%>авv"лw'нская пе'щь _о='троки не 
w=пали`, ниже` бж~ества` _о='гнь дв~у растли`. тjь'мъ со 
_о='троки вjь'рнiи возопiи'мъ: бл~гослови'те дjьла` 
гд\сня гд\са.

%<Т%>я` _о=ц~ъ и=зво'ли невjь'сту неискусобра'чную, 
jа='кw крi'нъ пресвjь'тлый посредjь` те'рнiя w=брjь'тъ, 
свjь'тлостiю добро'ты блиста'ющую дх~омъ бж~е'ственнымъ, 
сн~у въ жили'ще. (с. 354)

%<С%>т~jь'йшую вы'шнихъ си'лъ дв~у пренепоро'чную, 
без\ъ разсужде'нiя сла'влю jа='вjь: творца' бо си'хъ во 
чре'вjь понесе`, несмjь'снымъ соедине'нiемъ пло'ть 
прiе'мша w\т тебе`.

%<Сла'ва: Д%>jь'вство некра'домо дв~о сохра'нши, мт~и 
jа=ви'лася _е=си` сн~а бж~iя вои'стинну, _о=ц~а` 
бл~говоле'нiемъ невjь'ста бы'вши, и= прiя'телище сла'вы 
дх~а нетлjь'нно.

%<И= ны'нjь: И='%>же _е=стество'мъ невеще'ственный 
бг~ъ и= неви'димый, про'стw и= неизрече'ннw па'че 
_е=стества` ражда'ется человjь'къ w\т ст~ы'я дв~ы, 
сугу'бъ зри'мь во _е=ди'ной v=поста'си, въ не'йже ви'димь 
_е='сть писа'ньми.

%<%[Пjь'снь f~%]%>

%<I=рмо'съ: Н%>о'вое чу'до и= бг~олjь'пное: 
дjьви'ческую бо две'рь затворе'ную jа='вjь прохо'дитъ 
гд\сь, на'гъ во вхо'дjь, и= плотоно'сецъ jа=ви'ся во 
и=схо'дjь бг~ъ, и= пребыва'етъ две'рь затворе'на. сiю` 
неизрече'ннw jа='кw бг~ома'терь велича'емъ.

%<С%>т~а'гw ко'рене ст~ъ пло'дъ: непло'дныя бо, 
w=свяще'нныя зако'номъ бж~iимъ _о='ч~iимъ, неувяда'емую 
жи'знь соверша'ющую бц\да _о=трокови'ца процвjьте`: и= 
ра'дуется а='нна, прiе'мши _о=троча` въ ста'рости, бж~iю 
мт~рь, ю='же сла'вимъ.

%<Н%>о'во рж\ство` и= бг~ови'дно ст~а'гw твоегw` и= 
бг~оно'снагw чре'ва ч\стая: въ не'мъ бо написа'ся 
w='бразъ въ человjь'ка _о='ч~iимъ пе'рстомъ, сн~ъ 
вопло'щься ст~ъ дх~омъ ст~ы'мъ. _е=го'же несмjь'снw 
jа='кw бг~а и= человjь'ка велича'емъ.

%<Сла'ва: В%>опiе'тъ вся'къ су'дъ на мя`: дjьла' бо 
грjьхо'вныхъ дjья'нiй во'пль содjьва'ютъ мнjь` претя'ще, 
и= душа` вся^ вjь'сть, и='миже препрjь'на бу'детъ, и= 
трепе'щетъ пла'мене гее'нскагw: w\т негw'же пре'жде 
конца`, мл~твами твои'ми и=зба'ви мя` вл\дчце.

%<И= ны'нjь: П%>о рж\ствjь` jа=ви'ся нетлjь'нна 
ч\стая: неизмjь'ннw бо зижди'теля всjь'хъ пло'тiю, бц\де, 
родила` _е=си` (с. 355) человjь'ка па'че _е=стества`, 
_о='ч~а существа` не w\тсту'пльша, и= а='бiе пребы'сть 
дв~а дх~омъ бж~е'ственнымъ. тjь'мже сла'вяще тя`, пjь'снь 
соверша'емъ.

%<Та'же, Д%>осто'йно _е='сть: %<Трист~о'е. Конда'къ 
воскр\снъ, и= w\тпу'стъ.%>