%<%[Въ суббw'ту на вели'цjьй вече'рни%],%>

%<на Г%>д\си воззва'хъ, %<поста'вимъ стiхw'въ _i~, и= 
пое'мъ стiхи^ры воскре'сны _о=смогла'сника, г~: и= 
а=нато'лiевы, д~: и= мине'и г~, и=ли` д~, и=ли` s~, а='ще 
пра'зднуется ст~ы'й, Сла'ва, мине'и: И= ны'нjь, 
бг~оро'диченъ пе'рвый, гла'са.%>

%<Стiхи^ры воскр\сны, гла'съ в~.%>

%<Стi'хъ: И=%>зведи` и=з\ъ темни'цы ду'шу мою`, 
и=сповjь'датися и='мени твоему`.

%<П%>ре'жде вjь^къ w\т _о=ц~а` ро'ждшемуся бж~iю 
сло'ву, вопло'щшемуся w\т дв~ы мр~i'и, прiиди'те 
поклони'мся: кр\стъ бо претерпjь'въ, погребе'нiю 
предаде'ся, jа='кw са'мъ восхотjь`: и= воскр~съ и=з\ъ 
ме'ртвыхъ, сп~се' мя заблужда'ющаго человjь'ка.

%<Стi'хъ: М%>ене` жду'тъ прв\дницы, до'ндеже возда'си 
мнjь`.

%<Х%>р\сто'съ сп~съ на'шъ, _е='же на ны` рукописа'нiе 
пригвозди'въ на кр\стjь` загла'ди, и= сме'ртную держа'ву 
о_у=праздни`: покланя'емся _е=гw` тридне'вному 
воскр\снiю.

%<Стi'хъ: И=%>з\ъ глубины` воззва'хъ къ тебjь` гд\си, 
гд\си, о_у=слы'ши гла'съ мо'й.

%<С%>о а=рха'гг~лы воспои'мъ хр\сто'во воскр\снiе: 
то'й бо _е='сть и=зба'витель и= сп~съ ду'шъ на'шихъ, и= 
въ сла'вjь стра'шнjьй и= крjь'пцjьй си'лjь, па'ки 
гряде'тъ суди'ти мi'ру, _е=го'же созда`. (с. 197)

%<И='ны стiхи^ры а=нато'лiевы, гла'съ то'йже%>

%<Стi'хъ: Д%>а бу'дутъ о_у='ши твои`, вне'млющjь 
гла'су моле'нiя моегw`.

%<Т%>ебе` распе'ншагося и= погребе'ннаго, а='гг~лъ 
проповjь'да вл\дку, и= глаго'лаше жена'мъ: прiиди'те 
ви'дите, и=дjь'же лежа'ше гд\сь: воскр~се бо, jа='коже 
рече`, jа='кw всеси'ленъ. тjь'мже тебjь` покланя'емся 
_е=ди'ному безсме'ртному: жизнода'вче хр\сте`, поми'луй 
на'съ.

%<Стi'хъ: А='%>ще беззакw'нiя на'зриши гд\си, гд\си, 
кто` постои'тъ; jа='кw о_у= тебе` w=чище'нiе _е='сть.

%<К%>р\сто'мъ твои'мъ о_у=праздни'лъ _е=си`, ю='же w\т 
дре'ва кля'тву, погребе'нiемъ твои'мъ о_у=мертви'лъ 
_е=си` сме'рти держа'ву: воста'нiемъ же твои'мъ 
просвjьти'лъ _е=си` ро'дъ человjь'ческiй. сегw` ра'ди 
вопiе'мъ ти`: бл~годjь'телю хр\сте` бж~е на'шъ, сла'ва 
тебjь`.

%<Стi'хъ: И='%>мене ра'ди твоегw`, потерпjь'хъ тя` 
гд\си, потерпjь` душа` моя` въ сло'во твое`, о_у=пова` 
душа` моя` на гд\са.

%<W\т%>верзо'шася тебjь` гд\си, стра'хомъ врата` 
см_е'ртная, вра'тницы же а='дwвы ви'дjьвше тя`, 
о_у=боя'шася: врата' бо мjь^дная сокруши'лъ _е=си`, и= 
вер_еи` желjь^зныя сте'рлъ _е=си`, и= и=зве'лъ _е=си` 
на'съ w\т тьмы` и= сjь'ни сме'ртныя, и= о_у='зы на^ша 
растерза'лъ _е=си`.

%<Стi'хъ: W\т%> стра'жи о_у='треннiя до но'щи, w\т 
стра'жи о_у='треннiя, да о_у=пова'етъ i=и~ль на гд\са.

%<С%>п~си'тельную пjь'снь пою'ще, w\т о_у='стъ 
возсле'мъ: прiиди'те вси` въ дому` гд\снемъ припаде'мъ, 
глаго'люще: на дре'вjь распны'йся, и= и=з\ъ ме'ртвыхъ 
воскр~сы'й, и= сы'й въ нjь'дрjьхъ _о='ч~ихъ, w=чи'сти 
грjьхи` на'шя.

%<И='ны стiхи^ры прест~jь'й бц\дjь, пое'мъ и=`хъ, 
и=дjь'же нjь'сть мине'и. Творе'нiе па'vла а=морре'йскагw. 
Гла'съ s~%>

%<Подо'бенъ: W\т%>ча'янная житiя` ра'ди:

%<Стi'хъ: Jа='%>кw о_у= гд\са мл\сть, и= мно'гое о_у= 
негw` и=збавле'нiе: и= то'й и=зба'витъ i=и~ля w\т всjь'хъ 
беззако'нiй _е=гw`. (с. 198)

%<Jа='%>же ненаде'жныхъ о_у=пова'нiе и=звjь'стное, и= 
согрjьша'ющихъ сп~се'нiе, мр~i'е всепjь'тая ч\стая бц\де, 
прiими` моле'нiе мое` сiе`, и= и=спроси' ми разрjьше'нiя 
всjь'хъ, jа=`же согрjьши'хъ въ житiи`, мт~рними твои'ми 
мл~твами, и= сп~си` w\т бjь'дъ и= бу'дущагw суда` 
вл\дчце, вели'кiя ра'ди твоея` мл\сти.

%<Стi'хъ: Х%>вали'те гд\са вси` jа=зы'цы, похвали'те 
_е=го` вси` лю'дiе.

%<Л%>ука'во вре'мя живота` моегw`, лука'во и= 
и=спо'лнь вся'кiя sло'бы, сатанjь` лука'вому лю'тjь мя` 
смуща'ющу, ты' мя бг~ороди'тельнице, и=зба'ви тогw` 
па'кости, ты' мя w\т о_у='стъ _о='нагw и=сто'ргни 
прест~а'я: на тя' бо все` возложи'хъ ча'янiе мое`, сп~си' 
мя твои'ми бо'дрыми мл~твами.

%<Стi'хъ: Jа='%>кw о_у=тверди'ся мл\сть _е=гw` на 
на'съ, и= и='стина гд\сня пребыва'етъ во вjь'къ.

%<Р%>а'дуйся, непосты'дная предста'тельнице. ра'дуйся, 
пребл~га'я бц\де. ра'дуйся, w=чисти'лище мi'ра. ра'дуйся, 
ра'досте скорбя'щихъ, и= приста'нище w=бурева'_емымъ. 
ра'дуйся, jа='же всjь^мъ помога'ющая су'щымъ въ ну'ждахъ. 
ты` о_у='бw сохрани` и= мене` дв~о, всjь'хъ ско'рбныхъ, 
пренепоро'чная вл\дчце.

%<Сла'ва, и= ны'нjь, бг~оро'диченъ: П%>ре'йде сjь'нь 
зако'нная, бл~года'ти прише'дши: jа='коже бо купина` не 
сгара'ше w=паля'ема, та'кw дв~о родила` _е=си`, и= дв~а 
пребыла` _е=си`. вмjь'стw столпа` _о='гненнагw, 
пра'ведное возсiя` со'лнце: вмjь'стw мwv"се'а, хр\сто'съ, 
сп~се'нiе ду'шъ на'шихъ.

%<Та'же вхо'дъ съ кади'ломъ. С%>вjь'те ти'хiй: 
%<Прокi'менъ и= _е=кт_енiи`.%>

%<На стiхо'внjь стiхи^ры воскр\сны, гла'съ в~:%>

%<В%>оскр\снiе твое` хр\сте` сп~се, всю` просвjьти` 
вселе'нную и= призва'лъ _е=си` твое` созда'нiе: 
всеси'льне гд\си сла'ва тебjь`.

%<И='ны стiхи^ры по а=лфави'ту.%>

%<Стi'хъ: Г%>д\сь воцр~и'ся, въ лjь'поту w=блече'ся.

%<Д%>ре'вомъ сп~се о_у=праздни'лъ _е=си`, ю='же w\т 
дре'ва кля'тву, держа'ву сме'рти погребе'нiемъ твои'мъ 
о_у=мертви'лъ _е=си`, (с. 199) просвjьти'лъ же _е=си` 
ро'дъ на'шъ воста'нiемъ твои'мъ. тjь'мже вопiе'мъ тебjь`: 
животода'вче хр\сте` бж~е на'шъ, сла'ва тебjь`.

%<Стi'хъ: И='%>бо о_у=тверди` вселе'нную, jа='же не 
подви'жится.

%<Н%>а кр\стjь` jа='влься хр\сте` пригвожда'емъ, 
и=змjьни'лъ _е=си` добро'ту зда'нiй: и= безчеловjь'чiе 
о_у='бw во'ини показу'юще, копiе'мъ ре'бра твоя^ 
прободо'ша, _е=вре'и же печа'тати гро'ба проси'ша, твоея` 
вла'сти не вjь'дуще. но за мл\срдiе о_у=тро'бъ твои'хъ, 
прiе'мый погребе'нiе, и= тридне'венъ воскр~сы'й гд\си 
сла'ва тебjь`.

%<Стi'хъ: Д%>о'му твоему` подоба'етъ ст~ы'ня гд\си, въ 
долготу` днi'й.

%<Ж%>ивотода'вче хр\сте`, во'лею стр\сть претерпjь'вый 
сме'ртныхъ ра'ди, во а='дъ же снизше'дъ jа='кw си'ленъ, 
та'мw твоегw` прише'ствiя w=жида'ющыя, и=схи'тивъ jа='кw 
@w\т sвjь'ря@{@и=з\ъ руки`@} крjь'пкагw, ра'й вмjь'стw 
а='да жи'ти дарова'лъ _е=си`. тjь'мже и= на'мъ сла'вящымъ 
тридне'вное твое` воста'нiе, да'руй w=чище'нiе грjьхw'въ, 
и= ве'лiю мл\сть.

%<Сла'ва, и= ны'нjь, бг~оро'диченъ: _W%> чудесе` 
но'вагw всjь'хъ дре'внихъ чуде'съ! кто' бо позна` мт~рь 
без\ъ му'жа ро'ждшую, и= на руку` нося'щую, всю` тва'рь 
содержа'щаго; бж~iе _е='сть и=зволе'нiе, ро'ждшееся. 
_е=го'же jа='кw мл\днца преч\стая, твои'ма рука'ма 
носи'вшая, и= мт~рне дерзнове'нiе къ нему` и=му'щая, не 
преста'й моля'щи w= чту'щихъ тя`, о_у=ще'дрити и= сп~сти` 
ду'шы на'шя.

%<Та'же: Н%>ы'нjь w\тпуща'еши: %<Трист~о'е. По 
_О='%>ч~е на'шъ:

%<Тропа'рь воскр\снъ, гла'съ в~:%>

%<_Е=%>гда` снизше'лъ _е=си` къ сме'рти, животе` 
безсме'ртный, тогда` а='дъ о_у=мертви'лъ _е=си` 
блиста'нiемъ бж~ества`. _е=гда' же и= о_у=ме'ршыя w\т 
преиспо'днихъ воскр~си'лъ _е=си`, вся^ си^лы нб\сныя 
взыва'ху: жизнода'вче хр\сте` бж~е на'шъ, сла'ва тебjь`.

%<Бг~оро'диченъ: В%>ся^ па'че смы'сла, вся^ 
пресла^вная твоя^ бц\де, та^инства, @@чистотjь` 
запеча'танной, и= дjь'вству храни'му@@{@@чистото'ю 
запеча'тана, и= дjь'вствомъ храни'ма@@}, мт~и позна'лася 
_е=си` нело'жна, бг~а ро'ждши и='стиннаго: того` моли` 
сп~сти'ся душа'мъ на'шымъ. (с. 200)
