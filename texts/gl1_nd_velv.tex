%<%[Въ суббw'ту на вели'цjьй вече'рни%],%>

%<по предначина'тельномъ _псалмjь`, _о=бы'чное 
стiхосло'вiе _псалти'ря. На Г%>д\си воззва'хъ, 
%<поста'вимъ стiхw'въ _i~: и= пое'мъ стiхи^ры воскр\сны 
_о=смогла'сника г~, и= а=нато'лiевы д~, и= мине'и г~, 
и=ли` s~. А='ще пра'зднуемый ст~ы'й: Сла'ва, мине'и: И= 
ны'нjь, бг~оро'диченъ а~-й, гла'са.%>

%<Стiхи^ры воскр\сны _о=смогла'сника, гла'съ а~%>

%<Стi'хъ: И=%>зведи` и=з\ъ темни'цы ду'шу мою`, 
и=сповjь'датися и='мени твоему`.

%<В%>еч_е'рнiя на'шя мл~твы прiими` ст~ы'й гд\си, и= 
пода'ждь на'мъ w=ставле'нiе грjьхw'въ, jа='кw _е=ди'нъ 
_е=си` jа=вле'й въ мi'рjь воскр\снiе.

%<Стi'хъ: М%>ене` жду'тъ прв\дницы, до'ндеже возда'си 
мнjь`.

%<W=%>быди'те лю'дiе сiw'нъ, и= w=б\ъими'те _е=го`, и= 
дади'те сла'ву въ не'мъ воскр\сшему и=з\ъ ме'ртвыхъ: 
jа='кw то'й _е='сть бг~ъ на'шъ, и=збавле'й на'съ w\т 
беззако'нiй на'шихъ.

%<Стi'хъ: И=%>з\ъ глубины` воззва'хъ къ тебjь` гд\си, 
гд\си, о_у=слы'ши гла'съ мо'й.

%<П%>рiиди'те лю'дiе, воспои'мъ, и= поклони'мся хр\сту`, 
сла'вяще _е=гw` и=з\ъ ме'ртвыхъ воскр\снiе: jа='кw то'й 
_е='сть бг~ъ на'шъ, w\т пре'лести вра'жiя мi'ръ 
и=збавле'й.

%<И='ны стiхи^ры а=нато'лiевы, гла'съ а~%>

%<Стi'хъ: Д%>а бу'дутъ о_у='ши твои`, вне'млющjь гла'су 
моле'нiя моегw`.

%<В%>есели'теся нб~са`, воструби'те w=снова^нiя земли`, 
возопi'йте го'ры весе'лiе: се' бо _е=мману'илъ грjьхи` 
на'шя на кр\стjь` пригвозди`, и= живо'тъ дая'й, сме'рть 
о_у=мертви`, а=да'ма воскр~си'вый, jа='кw 
чл~вjьколю'бецъ. (с. 24)

%<Стi'хъ: А='%>ще беззакw'нiя на'зриши гд\си, гд\си, кто` 
постои'тъ; jа='кw о_у= тебе` w=чище'нiе _е='сть.

%<П%>ло'тiю во'лею распе'ншагося на'съ ра'ди, 
пострада'вша и= погребе'нна, и= воскр\сша и=з\ъ 
ме'ртвыхъ, воспои'мъ глаго'люще: о_у=тверди` 
правосла'вiемъ цр~ковь твою` хр\сте`, и= о_у=мири` жи'знь 
на'шу, jа='кw бл~гъ и= чл~вjьколю'бецъ.

%<Стi'хъ: И='%>мене ра'ди твоегw`, потерпjь'хъ тя` гд\си, 
потерпjь` душа` моя` въ сло'во твое`, о_у=пова` душа` 
моя` на гд\са.

%<Ж%>ивопрiе'мному твоему` гро'бу предстоя'ще 
недосто'йнiи, славосло'вiе прино'симъ неизрече'нному 
твоему` бл~гоутро'бiю, хр\сте` бж~е на'шъ: jа='кw кр\стъ 
и= сме'рть прiя'лъ _е=си` безгрjь'шне, да мi'рови 
да'руеши воскр\снiе, jа='кw чл~вjьколю'бецъ.

%<Стi'хъ: W\т%> стра'жи о_у='треннiя до но'щи, w\т 
стра'жи о_у='треннiя, да о_у=пова'етъ i=и~ль на гд\са.

%<И='%>же _о=ц~у` собезнача'льна и= соприсносу'щна 
сло'ва, w\т дв~и'ческа чре'ва произше'дшаго неизрече'ннw, 
и= кр\стъ и= сме'рть на'съ ра'ди во'лею прiе'мшаго, и= 
воскр\сша во сла'вjь, воспои'мъ глаго'люще: живода'вче 
гд\си сла'ва тебjь`, сп~се ду'шъ на'шихъ.

%<И='ны стiхи^ры прест~jь'й бц\дjь, творе'нiе па'vла 
а=морре'йскагw, по_е'мыя и=дjь'же нjь'сть мине'и, и=ли` 
на лiтi'и. [Вjь'домо о_у='бw бу'ди jа='кw въ гре'ческихъ 
прево'дjьхъ си'хъ стiхи'ръ на ряду` нjь'сть.]%>

%<Стiхи^ры, гла'съ а~. Подо'бенъ: Н%>б\сныхъ чинw'въ.

%<Стi'хъ: Jа='%>кw о_у= гд\са мл\сть, и= мно'гое о_у= 
негw` и=збавле'нiе: и= то'й и=зба'витъ i=и~ля w\т всjь'хъ 
беззако'нiй _е=гw`.

%<С%>т~jь'йшая ст~ы'хъ всjь'хъ си'лъ, ч\стнjь'йшая 
вся'кiя тва'ри, бц\де вл\дчце мi'ра, сп~си` на'съ jа='же 
сп~са ро'ждшая, w\т всjь'хъ прегрjьше'нiй, и= болjь'зней 
и= бjь'дъ, мл~твами твои'ми.

%<Стi'хъ: Х%>вали'те гд\са вси` jа=зы'цы, похвали'те 
_е=го` вси` лю'дiе.

%<Jа='%>же бл~гоутро'бiя две'ре, смире'нную мою` ду'шу да 
не пре'зриши, вjь'рнw молю' тя _о=трокови'це: но 
о_у=ще'дри вско'рjь, и= сп~си` ю=` w\т пучи'ны 
согрjьше'нiй мои'хъ, и= w=бно'вльши бл~года'ть твою` на 
мнjь`, просвjьти` дв~о ч\стая. (с. 25)

%<Стi'хъ: Jа='%>кw о_у=тверди'ся мл\сть _е=гw` на на'съ, 
и= и='стина гд\сня пребыва'етъ во вjь'къ.

%<Т%>ы` бг~а человjь'кwмъ соедини'ла _е=си` вл\дчце: ты` 
ме'ртвеное существо` возвела` _е=си` _е=ди'на, къ 
бж~е'ственному нетлjь'нiю: ты` земны^мъ сп~се'нiе 
и=сточи'ла _е=си`: ты` бц\де, свободи` на'съ w\т всjь'хъ 
муче'нiй.

%<Сла'ва, мине'и. И= ны'нjь, бг~оро'диченъ:%>

%<В%>семi'рную сла'ву, w\т человjь^къ прозя'бшую, и= 
вл\дку ро'ждшую, нб\сную две'рь воспои'мъ мр~i'ю дв~у, 
безпло'тныхъ пjь'снь, и= вjь'рныхъ о_у=добре'нiе: сiя' бо 
jа=ви'ся нб~о, и= хра'мъ бж\ства`: сiя` прегражде'нiе 
вражды` разруши'вши, мi'ръ введе`, и= цр\ствiе w\тве'рзе. 
сiю` о_у='бw и=му'ще вjь'ры о_у=твержде'нiе, побо'рника 
и='мамы и=з\ъ нея` ро'ждшагося гд\са. дерза'йте о_у='бw, 
дерза'йте лю'дiе бж~iи: и='бо то'й побjьди'тъ враги`, 
jа='кw всеси'ленъ.

%<Вхо'дъ. С%>вjь'те ти'хiй: %<По вхо'дjь же положи'въ 
о_у=чине'нный мона'хъ _о=бы'чный покло'нъ предстоя'телю, 
пое'тъ дневны'й прокi'менъ: Г%>д\сь воцр~и'ся, въ 
лjь'поту w=блече'ся. %<Стi'хъ: W=%>блече'ся гд\сь въ 
си'лу, и= препоя'сася. %<Стi'хъ: И='%>бо о_у=тверди` 
вселе'нную, jа='же не подви'жится. %<Стi'хъ: Д%>о'му 
твоему` подоба'етъ ст~ы'ня гд\си, въ долготу` днi'й.

%<Та'же _о=бы'чная _е=ктенiа`. С%>подо'би гд\си: 
%<И=%>спо'лнимъ вече'рнюю. %<И= прw'чая. И= по возгла'сjь 
пое'мъ самогла'сну стiхи'ру ст~а'гw _о=би'тели: лiтi'ю 
творя'ще въ притво'рjь, на не'йже пое'мъ стiхи^ры па'vла 
а=морре'йскагw, и=ли` что` настоя'тель и=зво'литъ.%>

%<По _о=бы'чныхъ же мл~твахъ вхо'димъ во хра'мъ, пою'ще 
стiхи'ру, гла'съ а~:%>

%<С%>тр\стiю твое'ю хр\сте`, w\т страсте'й свободи'хомся, 
и= воскр\снiемъ твои'мъ и=з\ъ и=стлjь'нiя и=зба'вихомся, 
гд\си сла'ва тебjь`.

%<И='ны стiхи^ры, по а=лфави'ту%>

%<Стi'хъ: Г%>д\сь воцр~и'ся, въ лjь'поту w=блече'ся.

%<Д%>а ра'дуется тва'рь, нб~са` да веселя'тся, рука'ми да 
воспле'щуть jа=зы'цы съ весе'лiемъ: хр\сто'съ бо сп~съ 
на'шъ, (с. 26) на кр\стjь` пригвозди` грjьхи` на'шя: и= 
сме'рть о_у=мертви'въ живо'тъ на'мъ дарова`, па'дшаго 
а=да'ма всеро'днаго воскр~си'вый, jа='кw чл~вjьколю'бецъ.

%<Стi'хъ: И='%>бо о_у=тверди` вселе'нную, jа='же не 
подви'жится.

%<Ц%>р~ь сы'й нб~се` и= земли` непостижи'ме, во'лею 
распя'лся _е=си` за чл~вjьколю'бiе. _е=го'же а='дъ 
срjь'тъ до'лjь, w=горчи'ся, и= пра'ведныхъ ду'шы прiе'мшя 
возра'довашася: а=да'мъ же, ви'дjьвъ тя` зижди'теля въ 
преиспо'днихъ, воскр~се. _W чудесе`! ка'кw сме'рти вкуси` 
всjь'хъ жи'знь; но jа='коже восхотjь` мi'ръ просвjьти'ти 
зову'щiй, и= глаго'лющiй: воскр~сы'й и=з\ъ ме'ртвыхъ, 
гд\си сла'ва тебjь`.

%<Стi'хъ: Д%>о'му твоему` подоба'етъ ст~ы'ня гд\си, въ 
долготу` днi'й.

%<Ж%>ены` мv"ронw'сицы, мv'ра нося'ща, со тща'нiемъ и= 
рыда'нiемъ гро'ба твоегw` достиго'ша, и= не w=брjь'тшя 
преч\стагw тjь'ла твоегw`, w\т а='гг~ла же о_у=вjь'дjьвшя 
но'вое и= пресла'вное чу'до, а=п\слwмъ глаго'лаху: 
воскр~се гд\сь подая` мi'рови ве'лiю мл\сть.

%<Сла'ва, мине'и, а='ще и='мать. А='ще ли нjь'сть:%>

%<Сла'ва, и= ны'нjь, бг~оро'диченъ:%>

%<С%>е` и=спо'лнися и=са'iино прорече'нiе, дв~а бо 
родила` _е=си`, и= по рж\ствjь` jа='кw пре'жде рж\ства` 
пребыла` _е=си`: бг~ъ бо бjь` рожде'йся, тjь'мже и= 
@_е=ст_ества` новопресjьче`@{@_е=стество` 
новопремjьни'ся@}. но _w бг~ома'ти, мол_е'нiя твои'хъ 
рабw'въ, въ твое'мъ хра'мjь приноси^мая тебjь` не 
пре'зри: но jа='кw бл~гоутро'бнаго твои'ма рука'ма 
нося'щи, на твоя^ рабы^ о_у=милосе'рдися, и= моли` 
сп~сти'ся душа'мъ на'шымъ.

%<Та'же: Н%>ы'нjь w\тпуща'еши: Т%<рист~о'е. По _О='%>ч~е 
на'шъ:

%<Тропа'рь воскр\сный, гла'съ а~:%>

%<К%>а'мени запеча'тану w\т i=уд_е'й, и= во'инwмъ 
стрегу'щымъ преч\стое тjь'ло твое`, воскр\слъ _е=си` 
тридне'вный сп~се, да'руяй мi'рови жи'знь. сегw` ра'ди 
си^лы нб\сныя вопiя'ху ти`, жизнода'вче: сла'ва 
воскр\снiю твоему` хр\сте`: сла'ва цр\ствiю твоему`: 
сла'ва смотре'нiю твоему`, _е=ди'не чл~вjьколю'бче. (с. 
27)

%<Бг~оро'диченъ: Г%>аврiи'лу вjьща'вшу тебjь`, дв~о, 
ра'дуйся, со стра'хомъ воплоща'шеся всjь'хъ вл\дка, въ 
тебjь` ст~jь'мъ кiвw'тjь, jа='коже рече` прв\дный дв~дъ: 
jа=ви'лася _е=си` ши'ршая нб~съ, поноси'вши зижди'теля 
твоего`. сла'ва все'льшемуся въ тя`: сла'ва проше'дшему 
и=з\ъ тебе`: сла'ва свободи'вшему на'съ рж\ство'мъ 
твои'мъ.

%<То'йже тропа'рь и= на Б%>г~ъ гд\сь: %<И= про'чее 
послjь'дованiе.%>
