%<%[Въ недjь'лю о_у='тра, нача'ло полу'нощницы%]%>

%<I=ере'й: Б%>л~гослове'нъ бг~ъ на'шъ: %<И= мы` 
глаго'лемъ: А=%>ми'нь. %<С%>ла'ва тебjь`, бж~е на'шъ, 
сла'ва тебjь`. %<Ц%>р~ю` нб\сный: Т%<рист~о'е. 
П%>рест~а'я тр\оце: %<_О='%>ч~е на'шъ: %<I=ере'й: 
Jа='%>кw твое` _е='сть цр\ство: %<Г%>д\си поми'луй, 
%<в~_i. Сла'ва, и= ны'нjь: П%>рiиди'те поклони'мся: 
%<три'жды. _Псало'мъ н~. П%>оми'луй мя` бж~е:

%<Та'же, Канw'нъ ко ст~jь'й и= живонача'льнjьй тр\оцjь, 
_е=гw'же краестро'чiе: _Е=%>ди'но тя` пою` трисо'лнечное 
_е=стество`.

%<Творе'нiе митрофа'ново%>

%<%[Пjь'снь а~%]%>

%<I=рмо'съ: Т%>воя` побjьди'тельная десни'ца бг~олjь'пнw 
въ крjь'пости просла'вися: та' бо безсме'ртне, jа='кw 
всемогу'щая, проти^вныя сотре`, i=и~льтянwмъ пу'ть 
глубины` новосодjь'лавшая.

%<Припjь'въ: П%>рест~а'я тр\оце бж~е на'шъ, сла'ва 
тебjь`.

%<_Е=%>ди'но трiv"поста'сное нача'ло, серафi'ми немо'лчнw 
сла'вятъ: безнача'льное, присносу'щное, твори'тельное 
всjь'хъ, непостижи'мое, _е='же и= вся'къ jа=зы'къ вjь'рнw 
почита'етъ пjь'сньми. (с. 34)

%<Д%>а человjь'кwмъ _е=ди'нственное, трисiя'тельное твое` 
jа=ви'ши бж~ество`, созда'вый пре'жде человjь'ка, по 
твоему` w='бразу воwбрази'лъ _е=си`, о_у='мъ _е=му`, и= 
сло'во и= ду'хъ да'въ, jа='кw чл~вjьколю'бецъ.

%<Сла'ва: С%>вы'ше показу'я _е=ди'нственную 
бг~онача'льныхъ въ трiе'хъ v=поста'сjьхъ держа'ву, 
_о='ч~е, ре'клъ _е=си` равнодjь'тельному твоему` сн~у, и= 
дх~у: прiиди'те соше'дше, я=зы'ки и='хъ слiя'имъ.

%<И= ны'нjь, бг~оро'диченъ: О_у='%>мъ о_у='бw _е='сть 
нерожде'нный _о=ц~ъ, w=бра'знw прему'дрыми предрече'ся, 
сло'во же собезнача'льно, соесте'ственный сн~ъ и= дх~ъ 
ст~ы'й, и='же въ дв~jь сло'ва созда'вшiй воплоще'нiе.

%<%[Пjь'снь г~%]%>

%<I=рмо'съ: _Е=%>ди'не вjь'дый человjь'ческагw существа` 
не'мощь, и= ми'лостивнw въ не` воwбра'жся, препоя'ши мя` 
съ высоты` си'лою, _е='же вопи'ти тебjь` @ст~ы'й: 
w=душевле'нный хра'ме@{@ст~ъ w=душевле'нный хра'мъ@} 
неизрече'нныя сла'вы твоея`, чл~вjьколю'бче.

%<Т%>ы` дре'вле jа='вjь а=враа'му jа='кw jа=ви'лся _е=си` 
трiv"поста'сный, _е=ди'нственъ же _е=стество'мъ 
бж~ества`, бг~осло'вiя и='стиннjьйшее w=бра'зно jа=ви'лъ 
_е=си`: и= вjь'рнw пое'мъ тя`, _е=динонача'льнаго бг~а, 
и= трисо'лнечнаго.

%<И=%>з\ъ тебе` роди'выйся бг~олjь'пнw нетлjь'ннw 
_о='ч~е, возсiя` свjь'тъ w\т свjь'та, сн~ъ непремjь'ненъ, 
и= дх~ъ бж~е'ственный, свjь'тъ и=зы'де: _е=ди'нагw 
бж~ества` свjь'тлости трiv"поста'снjьй покланя'емся 
вjь'рнw, и= сла'вимъ.

%<Сла'ва: _Е=%>ди'ница тр\оца преесте'ственнjь, 
неизрече'ннw па'че смы'сла, о_у='мными существы` 
сла'вится, трист~ы'ми гла'сы немо'лчную вопiю'щими 
хвалу`: и='миже согла'снw пое'тся и= на'ми 
трiv"поста'сный гд\сь.

%<Бг~оро'диченъ: И=%>з\ъ тебе` вре'меннjь без\ъ сjь'мене 
произы'де вышшелjь'тный, о_у=подо'бивыйся на'мъ 
неви'димый, и= (с. 35) _е=ди'ному _е=стеству` и= 
гд\сьству _о='ч~у научи'въ, и= сн~о'вню, и= дх~ову, 
бц\де: тjь'мже тя` сла'вимъ.

%<Г%>д\си поми'луй, %<три'жды%>

%<Сjьда'ленъ, гла'съ а~. Подо'бенъ: Г%>ро'бъ тво'й сп~се:

%<_О=%>ц~у` и= сн~у поклони'мся вси`, и= дх~у пра'вому и= 
равноче'стну, сла'ва тр\оцjь несозда'ннjьй, и= 
пребж~е'ственнjьй си'лjь, ю='же сла'вятъ безпло'тныхъ 
чи'нове: сiю` дне'сь и= земноро'днiи со стра'хомъ вjь'рнw 
восхва'лимъ.

%<Сла'ва, и= ны'нjь, бг~оро'диченъ: Н%>аста'ви ны` на 
пу'ть покая'нiя, о_у=клоня'ющыяся при'снw къ безпу'тi_емъ 
sw'лъ, и= пребл~га'го прогнjь'вающыя гд\са, 
неискусобра'чная бл~гослове'нная мр~i'е, прибjь'жище 
w\тча'янныхъ человjь'кwвъ, бж~iе пребыва'нiе.

%<%[Пjь'снь д~%]%>

%<I=рмо'съ: %>Го'ру тя` бл~года'тiю бж~iею 
прiwсjьне'нную, прозорли'выма а=вваку'мъ о_у=смотри'въ 
_о=чи'ма, и=з\ъ тебе` и=зы'ти i=и~леву провозглаша'ше 
ст~о'му, во сп~се'нiе на'ше и= w=бновле'нiе.

%<В%>озсiя'й ми` бг~онача'лiе трисо'лнечное, сiя'ньми 
твои'хъ бг~одjь'тельныхъ w=заре'нiй, въ серде'чныхъ 
_о=чесjь'хъ добро'тjь мечта'тися, _е='же па'че о_у=ма` 
бг~онача'льныя твоея` свjь'тлости, и= свjьтодjь'тельнагw 
и= сла'дкагw прича'стiя.

%<П%>е'рвjье нб~са` о_у=тверди'лъ _е=си` гд\си, и= всю` 
си'лу и='хъ сло'вомъ твои'мъ вседjь'тельнымъ, и= дх~омъ 
о_у='стъ соесте'ственнымъ, съ ни'миже влады'чествуеши 
вся'ческими въ трисiя'тельнjь _е=динонача'лiи бж~ества`.

%<Сла'ва: Jа='%>кw созда'лъ _е=си` мя` по w='бразу 
твоему` и= подо'бiю, бг~онача'льная и= вседjь'тельная 
тр\оце, неслiя'нная _е=ди'нице вразуми`, просвjьти`, во 
_е='же твори'ти во'лю твою` ст~у'ю, бл~гу'ю въ 
крjь'пости, и= соверше'нную.

%<Бг~оро'диченъ: Р%>одила` _е=си` w\т тр\оцы _е=ди'наго 
преч\стая, бг~онача'льнjьйша сн~а, вопло'щшася на'съ 
ра'ди и=з\ъ тебе`, w=заря'юща земноро'дныхъ, 
трисо'лнечнагw бж~ества` невече'рнимъ свjь'томъ и= 
сiя'ньми. (с. 36)

%<%[Пjь'снь _е~%]%>

%<I=рмо'съ: %>Просвjьти'вый сiя'нiемъ прише'ствiя твоегw` 
хр\сте`, и= w=свjьти'вый кр\сто'мъ твои'мъ мi'ра концы`, 
сердца` просвjьти` свjь'томъ твоегw` бг~оразу'мiя, 
правосла'внw пою'щихъ тя`.

%<Jа='%>же пе'рвой а='гг~льской непосре'дственнjь 
о_у='твари, непристу'пными твоея` добро'ты луча'ми 
w=сiява'емой бы'ти бл~говоли'вши, твои'ми свjь'тлостьми 
просвjьти`, тр\оце _е=динонача'льственнjьйшая, 
правосла'внw тебе` пою'щихъ.

%<Н%>ы'нjь _е=стество`, _е=ди'нственное бг~онача'лiя 
трисо'лнечное, воспjьва'етъ тя`._е='же w=существова'ла 
_е=си` за бл~гость, прегрjьше'нiй и=збавле'нiе и= 
напа'стей и=спроша'ющи, и= бjь'дъ и= скорбе'й.

%<Сла'ва: _О=%>ц~а`, и= сн~а, и= дх~а ст~а'го, _е=ди'но 
_е=стество` и= бж~ество` вjь'рою сла'вимъ, дjьли'тельное, 
нераздjь'льное, _е=ди'нагw бг~а неви'димыя и= ви'димыя же 
тва'ри.

%<Бг~оро'диченъ: Р%>еч_е'нiя вся^ пр\оро'ч_еская 
преднаписа'ша, преч\стая, твое` рж\ство`, неизрече'нное 
и= неудо'бь сказу'емое, _е='же мы` позна'хомъ, 
тайноводи'тельное _е=ди'нственнагw и= трисо'лнечнагw 
бж~ества`.

%<%[Пjь'снь s~%]%>

%<I=рмо'съ: W=%>бы'де на'съ послjь'дняя бе'здна, нjь'сть 
и=збавля'яй, вмjьни'хомся jа='кw _о='вцы заколе'нiя, 
сп~си` лю'ди твоя^, бж~е на'шъ: ты' бо крjь'пость 
немощству'ющихъ и= и=справле'нiе.

%<Р%>авноста'тную си'лу jа='кw и=му'щи, тр\оце 
пресу'щественная, въ то'ждествjь хотjь'нiя, _е=ди'ница 
_е=си` проста` и= нераздjь'льна: ты` о_у='бw на'съ си'лою 
твое'ю соблюди`. %<[Два'жды.]%>

%<Сла'ва: Т%>ы` вся^ вjь'ки хотjь'нiемъ твои'мъ, jа='кw 
бл~га соста'вила _е=си` w\т не су'щихъ, непостижи'мая 
тр\оце, та'же и= человjь'ка создала` _е=си`: но и= ны'нjь 
w\т вся'кагw и=зба'ви мя` w=бстоя'нiя.

%<Бг~оро'диченъ: С%>о'лнца незаходи'магw до'мъ была` 
_е=си`, созда'вшагw и= въ чину` поста'вльшагw свjьти^ла 
вели^кая (с. 37) всеси'льнjь, преч\стая дв~о 
бг~оневjь'стная: но и= ны'нjь страсте'й мя` и=зба'ви 
помраче'нiя.

%<Г%>д\си поми'луй, %<три'жды%>

%<Сjьда'ленъ, гла'съ а~. Подо'бенъ: Г%>ро'бъ тво'й сп~се:

%<Т%>р\оцjь ст~jь'й, и= нераздjь'льному _е=стеству`, въ 
тре'хъ ли'цjьхъ сjько'мjьй несjь'ченw, и= пребыва'ющей 
нераздjь'льнjьй по существу` бж~ества`, поклони'мся 
земноро'днiи со стра'хомъ, и= сла'вимъ jа='кw творца` и= 
вл\дку, бг~а пребл~га'го.

%<Сла'ва, и= ны'нjь, бг~оро'диченъ: О_у=%>пра'ви ч\стая, 
_о=кая'нную мою` ду'шу, и= о_у=ще'дри ю=` w\т мно'жества 
прегрjьше'нiй, во глубину` попо'лзшуюся поги'бели, 
всенепоро'чная, и= въ ча'съ мя` стра'шный сме'ртный ты` 
и=схити` w=глаго'лующихъ де'мwнwвъ, и= вся'кiя му'ки.

%<%[Пjь'снь з~%]%>

%<I=рмо'съ: Т%>ебе` о_у='мную бц\де, пе'щь разсмотря'емъ 
вjь'рнiи: jа='коже бо _о='троки сп~се` три` 
превозноси'мый, мi'ръ w=бнови`, во чре'вjь твое'мъ 
всецjь'лъ, хва'льный _о=тц_е'въ бг~ъ, и= препросла'вленъ.

%<С%>ло'ве бж~iй, соесте'ственное сiя'нiе вседержи'теля 
бг~а, jа='коже w=бjьща'лъ _е=си` _е='же о_у= тебе`, 
бг~одjь'тельное вселе'нiе сотвори`, jа='кw 
бл~гоутро'бенъ, со _о=ц~е'мъ твои'мъ и= дх~омъ: и= 
стра'шна бjьсw'мъ мя` покажи`, и= страсте'мъ. 
%<[Два'жды.]%>

%<Сла'ва: Д%>а твоегw` бл~гоутро'бiя вл\дко пока'жеши 
пучи'ну на'мъ, сн~а твоегw` посла'въ къ на'шему 
смире'нiю, па'ки воwбрази'лъ _е=си` на пе'рвую 
свjь'тлость, но и= ны'нjь бж~е'ственнымъ мя` вразуми` 
дх~омъ.

%<Бг~оро'диченъ: И='%>же на херувi'мстjьмъ пр\сто'лjь 
носи'мый, и= всjь'хъ цр~ь, во чре'вjь твое'мъ 
дв~ственнjьмъ всели'ся преч\стая, всjь'хъ и=збавля'я w\т 
и=стлjь'нiя, jа='кw чл~вjьколю'бецъ: но и= ны'нjь твои'ми 
мя` мл~твами сохрани`.

%<%[Пjь'снь и~%]%>

%<I=рмо'съ: Ч%>у'да преесте'ственнагw показа` w='бразъ, 
_о=гнеро'сная пе'щь дре'вле: _о='гнь бо не w=пали` ю='ныя 
дjь'ти, хр\сто'во jа=вля'я (с. 38) безсjь'менное w\т дв~ы 
бж~е'ственное рж\ство`. тjь'мъ воспjьва'юще воспои'мъ: да 
бл~гослови'тъ тва'рь вся` гд\са, и= превозно'ситъ во вся^ 
вjь'ки.

%<М%>а'нiемъ бг~одjь'тельнымъ, гд\си всjь'хъ, 
трiv"поста'сне и= вседержи'телю, нб~са` просте'рлъ _е=си` 
jа='кw ко'жу: та'же и= земли` повjь'силъ _е=си` глубину`, 
всемогу'щею твое'ю дла'нiю. тjь'мже и= рабы^ твоя^ 
о_у=крjьпи` любо'вiю и= вjь'рою твое'ю чл~вjьколю'бче: да 
тя` сла'вимъ жела'нiемъ во вjь'ки.

%<Сла'ва: П%>росвjьти` бг~онача'льнымъ свjь'томъ пою'щихъ 
тя`, трисо'лнечный свjь'те ли'цы, _е=ди'нственный же 
па'ки существо'мъ, и= къ твои^мъ свjьтода^тельнымъ 
луча'мъ взира'ти при'снw: и='миже насы'щуся сла'вы твоея` 
сла'дкiя, и= свjьтода'тельныя, и= пребога'тыя: и= 
превозношу' тя вjь'рнw во вjь'ки.

%<Бг~оро'диченъ: В%>ознесе` на нб~са`, человjь'ческое 
прiе'мъ _е=стество` непрело'жнjь сн~ъ тво'й преч\стая 
бц\де, превосхожде'нiемъ бл~гости и=зба'вль дре'внiя 
тли`. _е=му'же бл~года'рственнw воспjьва'емъ: да 
бл~гослови'тъ тва'рь вся` гд\са, и= превозно'ситъ во вся^ 
вjь'ки.

%<%[Пjь'снь f~%]%>

%<I=рмо'съ: W='%>бразъ чи'стагw рж\ства` твоегw`, 
_о=гнепали'мая купина` показа` неwпа'льная: и= ны'нjь на 
на'съ напа'стей свирjь'пjьющую о_у=гаси'ти мо'лимся 
пе'щь, да тя` бц\де непреста'ннw велича'емъ.

%<С%>п~си` сп~си'телю тва'ри, чу'вственныя же и= 
о_у='мныя рабы^ твоя^ w\т вра'жiя навjь'та и= 
w=sлобле'нiя, прест~а'я тр\оце _е=диносу'щная, и= 
соблюда'й ста'до твое` вы'ну ненавjь'тнw. %<[Два'жды.]%>

%<Сла'ва: Д%>а глубину` неизче'тную су'щественныя 
пока'жеши твоея` бл~гости, да'лъ _е=си` на'мъ w=бjь'ты: 
трисо'лнечный, и= _е=динонача'льный бж~е всеси'льный, 
сп~си'тельныя твои^мъ рабw'мъ, и=`хже соверши'ти 
сподо'би.

%<Бг~оро'диченъ: П%>ри'зри на на^ша мол_е'нiя, и='же въ 
трiе'хъ бг~онача'льныхъ v=поста'сjьхъ, _е=ди'нъ сы'й бг~ъ 
во и='стинjь (с. 40) вjь'руемый: и= пода'ждь твои^мъ 
рабw'мъ о_у=тjьше'нiе, мл~твами преч\стыя и= препjь'тыя 
бг~ома'тере.

%<Посе'мъ припjь'вы григо'рiа сiнаи'та, глаго'лемы по 
тр\очныхъ канw'нjьхъ по вся^ недjь^ли: Д%>осто'йно 
_е='сть, jа='кw вои'стинну: %<Пи'саны въ концjь` кни'ги 
сея`. И= про'чее полу'нощницы. И= w\тпу'стъ.%>
