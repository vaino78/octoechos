Въ недjь'лю о_у='тра, на полу'нощницjь,

канw'нъ прест~jь'й и= живонача'льнjьй тр\оцjь, 
[_е=го'же краестро'чiе: Тр\оце _е=ди'нице, сп~си' мя 
твоего` раба`.] Творе'нiе митрофа'ново. Гла'съ и~:

\cslemph{Пjь'снь а~}

I=рмо'съ: Колесницегони'теля фараw'ня погрузи`, 
чудотворя'й и=ногда` мwv"се'йскiй же'злъ, кр\стоwбра'знw 
порази'въ, и= раздjьли'въ мо'ре: i=и~ля же бjьглеца` 
пjьшехо'дца сп~се`, пjь'снь бг~ови воспjьва'юща.

Припjь'въ: Прест~а'я тр\оце, бж~е на'шъ, сла'ва 
тебjь`.

Трисо'лнечному цр~ю`, и= строи'телю, и= промысли'телю 
вся'ческихъ, и= бл~го'му _е=ди'ному, _е=сте'ственнjь 
су'щему, и= _е=ди'нственную и=му'щему бж~ества` сла'ву, 
бг~у _е=динонача'льному припа'даемъ, пjь'снь трист~у'ю 
пою'ще.

Теч_е'нiя бж~е'ств_енная, и= пр\оро'ч_ествiя jа=`же 
свы'ше по'мняще jа='вjь, бг~онача'льное _е=стество` 
_е=ди'нственное сла'вимъ, присносу'щное, собезнача'льное, 
въ трiе'хъ ли'цjьхъ, _о=ц~jь`, и= сн~jь, и= дс~jь, 
содjь'тельное, всеси'льное.

Сщ~еннотаи'нникъ а=враа'мъ бы'въ, сщ~енноwбра'знw 
дре'вле, творца` всjь'хъ и= бг~а и= гд\са, въ трiе'хъ 
о_у='бw v=поста'сехъ прiя'тъ ра'дуяся, и= тре'хъ 
v=поста'сей держа'ву _е=ди'нственную позна`.

Бг~оро'диченъ: Неискусобра'чнw хр\ста` родила` _е=си`, 
_е='же по на'мъ, на'съ ра'ди воспрiе'мша _е=стество` 
преч\стая, и= непрело'жна по _о=бою` пребы'вша. _е=го'же 
моли` непреста'ннw, грjьхw'въ ми` дарова'ти и= 
и=скуше'нiй и=збавле'нiе.

\cslemph{Пjь'снь г~}

I=рмо'съ: О_у=твержде'й въ нача'лjь нб~са` ра'зумомъ, 
и= зе'млю на вода'хъ w=снова'вый, на ка'мени мя` хр\сте` 
@за'повjьдей твои'хъ@{цр~кве} о_у=тверди`, jа='кw 
нjь'сть ст~ъ па'че тебе`, _е=ди'не чл~вjьколю'бче.

Тебе` непристу'пнаго бг~а и= цр~я` сла'вы на 
пр\сто'лjь и=са'iа ви'дjь высо'цjь, и= херувi'мы и= 
серафi'мы сла'вящыя непреста'нными пjь'сньми, 
_е=ди'нственнаго трiv"поста'снаго.

_Е=ди'наго w\т _о=ц~а` jа='кw w\т о_у=ма` рожде'наго 
сло'ва, и= дх~а происходя'ща неизглаго'ланнw, по 
коему'ждо помышле'нiи, и= кни'жными о_у=че'нiи пости'гше, 
_е=ди'наго бг~а трисо'лнечнаго почита'емъ.

И='же сы'й нерожде'нный _о=ц~ъ, и= своегw` существа` 
сiя'нiе роди'въ нетлjь'ннw сн~а, свjь'тъ w\т свjь'та: 
и=схо'днjь предлага'етъ сра'сленный свjь'тъ дх~а, 
вседjь'тельна, и= _е=диноче'стна.

Бг~оро'диченъ: Хра'мъ jа=ви'лася _е=си` чи'стъ, дв~о 
мт~и мр~i'е, вся^ всеси'льнw и= прему'дрw соста'вльшему 
хр\сту`, и= въ чи'нjь поло'жшему и= нося'щему: _е=го'же 
мл\стива ми` сотвори` мт~рними твои'ми мл~твами.

Гд\си поми'луй, три'жды.

Та'же, сjьда'ленъ, гла'съ и~: Подо'бенъ: Повеле'нное 
та'йнw:

Трисо'лнечнагw и= честна'гw бг~онача'лiя си'лу 
вjь'рнiи ны'нjь восхва'лимъ, jа='кw ма'нiемъ то'кмw вся^ 
соста'вилъ вы^шняя ликостоя^нiя а='гг~льская, и= ни^жняя 
сщ~еннонача^лiя цр~кw'вная, _е='же взыва'ти: ст~ъ, ст~ъ, 
ст~ъ _е=си` бж~е пребл~гi'й: сла'ва и= пjь'нiе держа'вjь 
твое'й.

Сла'ва, и= ны'нjь, бг~оро'диченъ: Jа='же 
непремjь'ннаго бг~а поро'ждшая, премjьня'емое при'снw 
се'рдце мое` грjьхо'мъ и= лjь'ностiю, прило'гми 
льсти'вагw, о_у=тверди` бл~га'я мл~твами мт~рними 
твои'ми: jа='кw да и= а='зъ бл~года'рнw сла'влю тя` 
бг~ороди'тельнице мр~i'е, поми'луй ста'до твое`, _е='же 
стяжа'ла _е=си`, всенепоро'чная.

\cslemph{Пjь'снь д~}

I=рмо'съ: Ты` моя` крjь'пость гд\си, ты` моя` и= 
си'ла, ты` мо'й бг~ъ, ты` мое` ра'дованiе, не w=ста'вль 
нjь'дра _о='ч~а, и= на'шу нищету` посjьти'въ. тjь'мъ съ 
пр\оро'комъ а=вваку'момъ зову' ти: си'лjь твое'й сла'ва 
чл~вjьколю'бче.

Восто'къ jа='вльшися бж~ества` су'щымъ во тьмjь`, всю` 
разгна` несвjь'тлую но'щь страсте'й: и= пра'вды сл~нце 
возсiя`, про'сто о_у='бw по существу`, трисiя'нно же 
ли'цы: _е='же пое'мъ при'снw, и= сла'вимъ.

Серафi'мскими о_у=сты` воспjьва'емаго, бре'нными 
о_у=стна'ми сла'вимъ _е=ди'нственнаго, и= тр\очнаго гд\са 
сла'вы, _е=стество'мъ и= v=поста'сьми, вопiю'ще: _w 
всецр~ю`, твои^мъ рабw'мъ пода'ждь разли'чныхъ 
прегрjьше'нiй проще'нiе!

Содержи'тельная всjь'хъ су'щихъ, неви'димая, 
всеще'драя, бл~гоутро'бная, чл~вjьколюби'вая тр\оце 
ч\стна'я, и= бг~онача'льная, не забу'ди мене` твоегw` 
раба` въ коне'цъ: ниже` разори` и='же завjьща'лъ _е=си` 
твои^мъ рабw'мъ завjь'тъ, за неизрече'нную мл\сть.

Бг~оро'диченъ: Кра'сную тя` всеч\стая, _е=ди'ну 
w=брjь'тъ w\т вjь'ка добро'ту i=а'кwвлю, пребезнача'льное 
сло'во, и= всели'ся въ тя` бл~гоутро'бiя ра'ди, w=бнови` 
человjь'ческое _е=стество`: _е=го'же моли` непреста'ннw, 
w\т вся'кiя мнjь` и=зба'витися ско'рби.

\cslemph{Пjь'снь _е~}

I=рмо'съ: Вску'ю мя` w\три'нулъ _е=си` w\т лица` 
твоегw` свjь'те незаходи'мый, и= покры'ла мя` _е='сть 
чужда'я тьма` _о=кая'ннаго: но w=брати' мя, и= къ свjь'ту 
за'повjьдей твои'хъ пути^ моя^ напра'ви, молю'ся.

Соприсносу'щная три` ли'ца сла'вимъ, _е=ди'наго же 
гд\са, тя` бж~е'ственное _е=стество`, раздjьля'юще 
про'стw, и= совокупля'юще, и= вjь'рнw вопiе'мъ: 
бг~онача'льная тр\оце ст~а'я, твоя^ рабы^ w\т ско'рби 
и=зба'ви.

Рыда'ю sjь'льнjь за не'мощь мы'сли моея`, ка'кw не 
хотя` стражду` нево'льное вои'стинну и=змjьне'нiе; сегw` 
ра'ди зову`: живонача'льная тр\оце ст~а'я, до'брыхъ въ 
стоя'нiи мя` о_у=чини`.

Дрема'нiемъ w=тягче'на мя` грjьхо'внымъ, и= 
порjьва'ема въ со'нъ сме'ртный, jа='кw чл~вjьколюби'вая 
и= пребл~га'я, и= всемл\стивая, бг~онача'льная тр\оце 
ст~а'я, о_у=ще'дри и= возста'ви мя`.

Бг~оро'диченъ: Мт~и дв~о _о=трокови'це, преч\стая, 
всенепоро'чная, бг~обл~года'тная, твои'ми мл~твами сн~а 
и= бг~а твоего` и= гд\са мл\стива сотвори` мнjь`: и= 
страсте'й, и= прегрjьше'нiй твоего` раба` и=зба'ви 
вско'рjь.

\cslemph{Пjь'снь s~}

I=рмо'съ: W=чи'сти мя` сп~се, мнw'га бо беззакw'нiя 
моя^, и= и=з\ъ глубины` sw'лъ возведи`, молю'ся: къ 
тебjь' бо возопи'хъ, и= о_у=слы'ши мя`, бж~е сп~се'нiя 
моегw`.

Нб\сныхъ о_у=мw'въ чинонача^лiя подража'юще, 
_е=динонача'льная всjь'хъ тр\оце пресу'щественная, 
трист~ы'ми пjь'сньми тя` славосло'вимъ, бре'нными на'шими 
о_у=сты`.

И='же по w='бразу твоему` человjь'ка созда'вшему, и= 
w\т не су'щихъ все` прему'дрw соста'вльшему, бг~у 
трiv"поста'сному покланя'юся, и= чту`, и= пою`, и= 
велича'ю тя`.

Вседержи'телю бж~е, и= _е=ди'не неwпредjьле'нный, 
всели'ся въ мя` за неизрече'нную мл\сть, трисо'лнечный 
вл\дко: и= w=зари' мя, и= вразуми`, jа='кw 
бл~гоутро'бенъ.

Бг~оро'диченъ: Хра'мъ jа=ви'лася _е=си` бг~а 
невмjьсти'магw преч\стая, хра'мъ и= мене` тогw` покажи`, 
бж~е'ственныя бл~года'ти прест~а'я вл\дчце, твои'ми 
мольба'ми, и= соблюди` невреди'ма.

Гд\си поми'луй, три'жды.

Сjьда'ленъ, гла'съ и~: Подо'бенъ: Повелjь'нное та'йнw:

_О=ц~а` безнача'льна вjь'рнiи, сн~а собезнача'льна, и= 
дх~а бж~е'ственнаго вои'стинну пjьсносло'вимъ, 
неслiя'ннjь, непревра'тнjь и= неизмjь'ннjь, тр\оцу 
про'сту, и= ст~у, и= сра'слену, вопiю'ще со а='гг~лы: 
ст~ъ _е=си` _о='ч~е, сн~е, со дх~омъ прест~ы'мъ и= 
ч\стны'мъ. поми'луй, jа=`же созда'лъ _е=си` по w='бразу 
твоему`, вл\дко.

Сла'ва, и= ны'нjь, бг~оро'диченъ: Бл~годари'мъ тя` 
при'снw бц\де, и= велича'емъ ч\стая, и= покланя'емся, 
воспjьва'юще рж\ство` твое` бл~года'тная, вопiю'ще 
непреста'ннw: сп~си` на'съ дв~о премл\стивая, jа='кw 
бл~га'я въ ча'съ и=спыта'нiя, да не посра'млени бу'демъ 
раби` твои`.

\cslemph{Пjь'снь з~}

I=рмо'съ: Бж~iя снизхожде'нiя _о='гнь о_у=стыдjь'ся въ 
вавv"лw'нjь и=ногда`, сегw` ра'ди _о='троцы въ пещи` 
ра'дованною ного'ю, jа='кw во цвjь'тницjь лику'юще 
поя'ху: бл~гослове'нъ _е=си` бж~е _о=т_е'цъ на'шихъ.

Прему'дростiю неизглаго'ланною твое'ю, и= пучи'ною 
бл~госты'ни, ту'не твоего` раба` поми'лованна покажи' мя: 
и= ны'нjь, jа='коже дре'вле, и=зба'ви w=sлобле'нiя, 
тр\оце _е=ди'нице бж~е, грjьхw'въ и= страсте'й. 
[Два'жды.]

О_у='мъ нерожде'нный _о=ц~ъ, и= сло'во ро'ждшееся w\т 
негw`, и= дх~ъ бж~е'ственный, непости'жнjь и=схо'денъ 
сы'й, бж~е _е=динонача'льне, трисо'лнечне, пою` тебjь`: 
бл~гослове'нъ бг~ъ _о=т_е'цъ на'шихъ.

Бг~оро'диченъ: О_у=мерщвле'нъ бы'хъ преч\стая, 
грjьхо'внымъ jа='домъ напое'нъ, и= притека'ю къ тебjь` 
вjь'рою, ро'ждшей нача'льника живота`: твои'ми мл~твами 
раба` твоего` w=живи`, и= и=скуше'нiй и= страсте'й 
и=зба'ви, _е=ди'на ч\стая.

\cslemph{Пjь'снь и~}

I=рмо'съ: Седмери'цею пе'щь халде'йскiй мучи'тель 
бг~очести^вымъ неи'стовнw разжже`, си'лою же лу'чшею 
сп~се'ны сiя^ ви'дjьвъ, творцу` и= и=зба'вителю вопiя'ше: 
_о='троцы бл~гослови'те, сщ~е'нницы воспо'йте, лю'дiе 
превозноси'те во вся^ вjь'ки.

Свjь'тъ сы'й незаходи'мъ, трисiя'ненъ и= трисо'лнечный 
и= _е=динонача'ленъ, самодержа'венъ, простjь'йшiй, бг~ъ 
непостижи'мый, и= самодержа'венъ гд\сь, ны'нjь те'мное и= 
w=мраче'нное мое` се'рдце w=зари`: и= покажи` 
свjьтоза'рнw и= свjьтоно'снw пjь'ти тя`, и= сла'вити во 
вся^ вjь'ки.

Сщ~е'ннjьйшими крилы^, серафi'ми бж~е'ственнjьйшiи, 
ли'ца и= но'ги бл~гоговjь'йнw покрыва'ютъ, сла'вы не 
терпя'ще непостижи'мыя добро'ты твоея`, бл~гонача'льная, 
бг~онача'льная, _е=динонача'льная тр\оце прест~а'я: 
_о=ба'че и= мы` воспjьва'ти тя` дерза'емъ, и= сла'вити 
вjь'рнw во вjь'ки.

Гд\сонача'лiе безнача'льное, всеси'льную и= пребл~гу, 
совершеннонача'льну, бл~годjь'йственну, неwпредjь'льну, 
вину` невино'вну, твори'тельну, присносу'щну, 
промысли'тельну и= сп~си'тельную всjь^мъ _е=дини'цу по 
существу`, и= тр\оцу ли'цы, сла'влю тя` бж~е мо'й, 
вjь'рнw во вjь'ки.

Бг~оро'диченъ: На земли` возсiя` невече'рнее сл~нце, 
рж\ство'мъ _е='же и=з\ъ тебе` дв~ственнымъ, преч\стая 
вл\дчце, и= человjь'ки и=зба'ви w\т i='дwльскагw 
помраче'ннаго мра'ка. тjь'мже и= ны'нjь мя` па'че тогw` 
бг~онача'лiя w=зари` луча'ми, и= соблюди` твоего` раба`.

\cslemph{Пjь'снь f~}

I=рмо'съ: О_у=жасе'ся w= се'мъ нб~о, и= земли` 
о_у=диви'шася концы`, jа='кw бг~ъ jа=ви'ся человjь'кwмъ 
пло'тски, и= чре'во твое` бы'сть простра'ннjьйшее нб~съ. 
тjь'мъ тя` бц\ду, а='гг~лwвъ и= человjь^къ чинонача^лiя 
велича'ютъ.

И='же всjь'ми цр\ствующее, и= вседjь'тельное 
пренача'льное _е=стество`, вышшелjь'тную, живонача'льную, 
бл~гоутро'бну, чл~вjьколюби'вую, бл~гу'ю, 
_е=динонача'льную тр\оцу тя` ны'нjь славосло'вяще, 
грjьхw'въ проще'нiя про'симъ, мi'рови ми'ра, и= цр~квамъ 
_е=диномы'слiя. [Два'жды.]

_Е=ди'но гд\сьство и= трисiя'нное, _е=ди'нственное 
бг~онача'лiе трисо'лнечное, пjьвцы` прiими` твоя^ 
бг~олjь'пнw, и= прегрjьше'нiй и=зба'ви, и= и=скуше'нiй и= 
лю'тыхъ: и= вско'рjь ми'ръ пода'ждь чл~вjьколю'бнw, и= 
цр~квамъ соедине'нiе.

Бг~оро'диченъ: Во о_у=тро'бу хр\сте` сп~се мо'й, 
дв~и'ческую все'лься, jа=ви'лся _е=си` мi'ру твоему` 
бг~ому'жнw, непревра'тенъ, неслiя'ненъ вои'стинну: и= 
w=бjьща'лъ _е=си` всегда` съ твои'ми рабы^ бы'ти jа='вjь. 
тjь'мже тебе` ро'ждшiя мл~твами, ми'ръ всему` ста'ду 
твоему` о_у=стро'й.

Посе'мъ припjь'вы григо'рiа сiнаи'та: Досто'йно 
_е='сть: И= про'чее полу'нощницы, пи'сано въ концjь` 
кни'ги сея`.