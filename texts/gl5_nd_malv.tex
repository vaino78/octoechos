%[Въ суббw'ту ве'чера,  на ма'лjьй вече'рни%],

на Гд\си воззва'хъ, поста'вимъ стiхw'въ д~: и= пое'мъ 
стiхи^ры воскре'сны _о=смогла'сника г~, повторя'юще а~-ю, 
гла'съ _е~:

Стi'хъ: W\т стра'жи о_у='треннiя до но'щи, w\т стра'жи 
о_у='треннiя, да о_у=пова'етъ i=и~ль на гд\са.

Ч\стны'мъ твои'мъ кр\сто'мъ хр\сте`, дiа'вола 
посрами'лъ _е=си`, и= воскр\снiемъ твои'мъ жа'ло 
грjьхо'вное притупи'лъ _е=си`, и= сп~слъ _е=си' ны w\т 
вра'тъ сме'ртныхъ: сла'вимъ тя` _е=диноро'дне.

Воскр\снiе дая'й ро'ду человjь'ческому: jа='кw _о=вча` 
на заколе'нiе веде'ся: о_у=страши'шася сегw` кня'зи 
а='дстiи, и= взя'шася врата` плач_е'вная. вни'де бо цр~ь 
сла'вы хр\сто'съ, глаго'ля су'щымъ во о_у='захъ, 
и=зыди'те: и= су'щымъ во тьмjь`, w\ткры'йтеся.

Ве'лiе чу'до, неви'димыхъ содjь'тель, за 
чл~вjьколю'бiе пло'тiю пострада'въ, воскр~се 
безсме'ртный. прiиди'те _о=те'ч_ествiя jа=зы^къ, тому` 
поклони'мся: бл~гоутро'бiемъ бо _е=гw` w\т пре'лести 
и=зба'вльшеся, въ трiе'хъ v=поста'сjьхъ _е=ди'наго бг~а 
пjь'ти навыко'хомъ.

Сла'ва, и= ны'нjь, бг~оро'диченъ: Бг~олjь'пную и= 
ч\стну'ю _о=трокови'цу почти'мъ, преч\стну'ю херувi'мwвъ: 
содjь'тель бо всjь'хъ вочл~вjь'читися восхотjь'вый, въ 
ту'ю всели'ся неизрече'ннw. _w стра'нныхъ веще'й, и= 
пресла'вныхъ та'инствъ! кто` не о_у=диви'тся w= се'мъ 
внуши'вый, jа='кw бг~ъ человjь'къ быва'етъ, и= 
преложе'нiе въ не'мъ не бjь`; и= дв~ства врата` про'йде, 
и= о_у=мале'нiе въ не'мъ не w=ста'вися, jа='коже 
пр\оро'къ глаго'летъ: человjь'къ сiя^ не про'йдетъ 
когда`, то'кмw _е=ди'нъ гд\сь бг~ъ i=и~левъ, и=мjь'яй 
ве'лiю мл\сть. 

Та'же, свjь'те ти'хiй:

Посе'мъ прокi'менъ: Гд\сь воцр~и'ся: три'жды. Стi'хъ: 
W=блече'ся гд\сь въ си'лу, и= препоя'сася. 

Та'же, Сподо'би гд\си въ ве'черъ се'й:

I=ере'й же _е=ктенiи` не глаго'летъ, но пое'мъ на 
стiхо'внjь стiхи'ру воскр\сну, гла'съ _е~:

Тебе` воплоще'ннаго сп~са хр\ста`, и= нб~съ не 
разлучи'вшася, во гла'сjьхъ пjь'нiй велича'емъ: jа='кw 
кр\стъ и= сме'рть прiя'лъ _е=си` за ро'дъ на'шъ, jа='кw 
чл~вjьколю'бецъ гд\сь, и=спрове'ргiй а='дwва врата`, 
тридне'внw воскре'слъ _е=си`, сп~са'я ду'шы на'шя.

И='ны стiхи^ры подо'бны прест~jь'й бц\дjь, гла'съ _е~:

Подо'бенъ: Ра'дуйся по'стникwмъ:

Стi'хъ: Помяну` и='мя твое` во вся'комъ ро'дjь и= 
ро'дjь. 

Ру'цjь простира'ю къ тебjь`, и= скве'рнjьи о_у=стнjь` 
w\тверза'ю къ моле'нiю, и= преклоня'ю серде'чное 
колjь'но, и= о_у='мнw нога'мъ твои^мъ пречи^стымъ ны'нjь 
прикаса'юся, ч\стая: и= припа'даю къ тебjь`: бwлjь'зни 
моя^ и=сцjьли`, лjь^тныя моя^ мнw'гiя и= неисцjь^льныя 
бл~гостiю твое'ю и=сцjьли` стру'пы. и=зба'ви w\т 
ви'димыхъ врагw'въ и= неви'димыхъ: w=блегчи` 
_о=трокови'це, тяготу` лjь'ности моея`, jа='кw да тя` 
пою` и= сла'влю, _е='юже w=брjь'те мi'ръ ве'лiю мл\сть. 

Стi'хъ: Слы'ши дщи` и= ви'ждь, и= приклони` о_у='хо 
твое`. 

Ра'дуйся сн~а бж~iя несказа'ннw заче'ншая 
всенепоро'чная, и= сего` ро'ждшая, пло'ть по на'мъ 
вои'стинну w\т крове'й твои'хъ прiи'мшаго, ду'шу 
о_у='мную же и= самовла'стную и=му'щаго: неwску'днw бо во 
а=да'ма w=бле'кшагося, ми'лости ра'ди и= бл~гости 
неизрече'нныя. w\т_ону'дуже во двою` _е=ст_еству` на'мъ 
возвjьща'ется, _о=бои'хъ показу'яй въ себjь` дjь'йство 
хр\сто'съ: _е=го'же моли`.душа'мъ на'шымъ да'ти ве'лiю 
мл\сть.

Стi'хъ: Лицу` твоему` помо'лятся бога'тiи лю'дстiи. 

Ра'дуйся добро'то i=а'кwвля, ю='же и=збра` бг~ъ, ю='же 
возлюби`, две'ре сп~са'емыхъ, пламеноно'сная клеще`, 
кля'твы разрjьше'нiе, всебл~гослове'нная: чре'во 
бг~овмjьсти'мое, па'дшихъ возведе'нiе: ст~jь'йшая 
херувi'мwвъ, и= тва'рей преиму'щая: неудобозри'мое 
видjь'нiе, слы'шанiе новjь'йшее, неизрече'нное 
глаго'ланiе, колесни'це сло'ва: _о='блаче, и=з\ъ негw'же 
возсiя` со'лнце, и= мене` w=заря'я, и= су'щымъ во тьмjь` 
подая` ве'лiю мл\сть.

Сла'ва, и= ны'нjь, бг~оро'диченъ: Jа=`же w= тебjь` 
пр\оро'ч_ествiя, и=спо'лнишася дв~о ч\стая: _о='въ 
о_у='бw w\т прорw'къ две'рь тя` прорече`, во _е=де'мjь на 
восто'къ зря'щую, ю='же никто'же про'йде, то'чiю 
зижди'тель тво'й, и= всегw` мi'ра: _о='въ же купину` 
_о=гне'мъ жего'му, jа='кw въ тебjь` w=бита` _о='гнь 
бж~ества`, и= неwпали'ма пребы'сть. и='нъ го'ру ст~у'ю, 
w\т нея'же w\тсjьче'ся ка'мень краеуго'льный, кромjь` 
ру'ки человjь'ческiя, и= порази` w='бразъ мы'сленнагw 
навуходоно'сора: вои'стинну ве'лiе и= пресла'вное, _е='же 
въ тебjь` та'инство _е='сть бг~ома'ти. тjь'мже тя` 
сла'вимъ: тобо'ю бо бы'сть сп~се'нiе душа'мъ на'шымъ.

Та'же, Ны'нjь w\тпуща'еши: Трист~о'е. И= по _О='ч~е 
на'шъ: Тропа'рь: Собезнача'льное сло'во: Бг~оро'диченъ: 
Ра'дуйся две'ре гд\сня непроходи'мая: _Е=ктенiа`. И= 
w\тпу'стъ.
