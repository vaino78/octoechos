\cslemph{Въ суббw'ту ве'чера,  на ма'лjьй вече'рни},

на Гд\си воззва'хъ, стiхи^ры воскр\сны г~, повторя'юще 
пе'рвую. Гла'съ и~:

Стi'хъ: W\т стра'жи о_у='треннiя до но'щи, w\т стра'жи 
о_у='треннiя, да о_у=пова'етъ i=и~ль на гд\са.

Вече'рнюю пjь'снь, и= слове'сную слу'жбу, тебjь` 
хр\сте` прино'симъ: jа='кw бл~говоли'лъ _е=си` 
поми'ловати на'съ воскр\снiемъ.

Гд\си, гд\си, не w\тве'ржи на'съ w\т твоегw` лица`: но 
бл~говоли` поми'ловати на'съ воскр\снiемъ.

Ра'дуйся сiw'не ст~ы'й, мт~и цр~кве'й, бж~iе жили'ще: 
ты' бо прiя'лъ _е=си` пе'рвый, w=ставле'нiе грjьхw'въ, 
воскр\снiемъ.

Сла'ва, и= ны'нjь, бг~оро'диченъ, догма'тiкъ. Гла'съ 
и~.

Ка'кw тя` о_у=бл~жи'мъ бц\де; ка'кw же воспои'мъ 
пребл~гослове'нная, непостижи'мое та'инство рожде'нiя 
твоегw`; вjькw'въ бо творе'цъ, и= на'шегw содjь'тель 
_е=стества`, сво'й w='бразъ о_у=ще'дривъ, низведе` 
самаго` себе` во и=стоща'нiе неизслjь'димое, сы'й въ 
невеще'ственныхъ нjь'дрjьхъ _о='ч~ихъ, во о_у=тро'бjь 
твое'й ч\стая всели'ся, и= пло'ть непрело'жнw бы'сть, w\т 
тебе` неискусобра'чная, пребы'въ о_у='бw, _е='же бjь` 
_е=стество'мъ бг~ъ. тjь'мже _е=му` покланя'емся бг~у 
соверше'нну, и= человjь'ку соверше'нну, тому` во 
_о=бое'мъ зра'цjь: и='бо _о=бое` _е=стество` въ не'мъ 
_е='сть вои'стинну: сугу'ба же вся^ проповjь'дуемъ 
_е=сте'ств_енная _е=гw` свw'йства, по сугу'бому 
существу`, два` почита'юще дjь^йства и= хотjь^нiя. 
_е=диносу'щенъ бо сы'й бг~у и= _о=ц~у`, самовла'стнw 
хо'щетъ и= дjь'йствуетъ jа='кw бг~ъ: _е=диносу'щенъ же 
сы'й и= на'мъ, самовла'стнw хо'щетъ, и= дjь'йствуетъ 
jа='кw человjь'къ. того` моли` ч\стая всебл~же'нная, 
сп~сти'ся душа'мъ на'шымъ.

Та'же, Свjь'те ти'хiй: Посе'мъ прокi'менъ: Гд\сь 
воцр~и'ся, въ лjь'поту w=блече'ся. Три'жды. Стi'хъ: 
W=блече'ся гд\сь въ си'лу, и= препоя'сася. Та'же, 
Сподо'би гд\си въ ве'черъ се'й:

I=ере'й же _е=ктенiи` не глаго'летъ, но пое'мъ на 
стiхо'внjь стiхи'ру воскр\сну, гла'съ и~:

Возше'лъ _е=си` на кр\стъ i=и~се, снизше'дый съ 
нб~се`: прише'лъ _е=си` на сме'рть животе` безсме'ртный, 
къ су'щымъ во тьмjь`, свjь'тъ и='стинный: къ па'дшымъ, 
всjь'хъ воскр\снiе, просвjьще'нiе, и= сп~се на'шъ, сла'ва 
тебjь`.

Та'же, бг~оро'дичны подw'бны, три`. Гла'съ и~:

Подо'бенъ: _W пресла'внагw чудесе`!

Стi'хъ: Помяну` и='мя твое` во вся'комъ ро'дjь и= 
ро'дjь.

Ра'дуйся бц\де всепjь'тая. ра'дуйся и=сто'чниче 
живота` вjь^рнымъ и=сточа'ющь. ра'дуйся всjь'хъ вл\дчце, 
и= госпоже` тва'ри бл~гослове'нная. ра'дуйся 
всенепоро'чная, препросла'вленная. ра'дуйся всепреч\стая. 
ра'дуйся пала'то. ра'дуйся бж~е'ственное селе'нiе. 
ра'дуйся ч\стая. ра'дуйся мт~и дв~о. ра'дуйся 
бг~оневjь'сто.

Стi'хъ: Слы'ши дщи` и= ви'ждь, и= приклони` о_у='хо 
твое`.

Ра'дуйся, бг~ома'ти преч\стая. ра'дуйся, вjь'рныхъ 
наде'жде. ра'дуйся, мi'ра w=чище'нiе. ра'дуйся, вся'кiя 
ско'рби и=збавля'ющи рабы^ твоя^. ра'дуйся, человjь'кwвъ 
о_у=тjьше'нiе живоно'сное. ра'дуйся, заступле'нiе. 
ра'дуйся, предсто'лпiе призыва'ющихъ тя`. ра'дуйся, бж~iе 
бж~е'ственное пребыва'нiе, и= горо` ст~а'я.

Стi'хъ: Лицу` твоему` помо'лятся бога'тiи лю'дстiи.

Ра'дуйся бц\де, мт~и хр\сто'ва. ра'дуйся, _е=ди'на 
наде'жде, человjь'кwвъ заступле'нiе. ра'дуйся, 
прибjь'жище. ра'дуйся, свjь'щниче свjь'та свjь'тлый. 
ра'дуйся, свjьще` w=свяще'нная. ра'дуйся пала'то. 
ра'дуйся раю`. ра'дуйся бж~е'ственное селе'нiе. ра'дуйся 
и=сто'чниче, и=сточа'ющь во'ды притека'ющымъ къ тебjь`.

Сла'ва, и= ны'нjь, бг~оро'диченъ догма'тiкъ:

_Е=гw'же нб~о не вмjьсти`, дв~о бц\де, во чре'вjь 
твое'мъ нетjьсномjь'стнw вмjьсти'ся: и= пребыла` _е=си` 
ч\стая, сло'вомъ неизрече'ннымъ, ничи'мже дв~ству 
w=скве'рншуся. ты' бо _е=ди'на была` _е=си` въ жена'хъ, 
и= мт~и, и= дв~а: и= ты` _е=ди'на преч\стая, воздои'ла 
_е=си` сн~а живода'вца, и= на w=б\ъя'тiяхъ твои'хъ 
носи'ла _е=си` недре'млющее _о='ко: но не w=ста'ви 
нjь'дра _о='ч~а, jа='коже пре'жде вjь^къ предбы'сть. но 
горjь` ве'сь бг~ъ со а='гг~лы, до'лjь ве'сь и=з\ъ тебе` 
съ человjь'ки, и= вездjь` несказа'ннw. того` моли`, 
всест~а'я вл\дчце сп~сти'ся, правосла'внw бц\ду ч\стую 
и=сповjь'дающымъ тя`.

Та'же, Ны'нjь w\тпуща'еши: Трист~о'е, и= по _О='ч~е 
на'шъ: Тропа'рь воскр\снъ. Сла'ва, и= ны'нjь, 
бг~оро'диченъ _е=гw`, и= w\тпу'стъ.
