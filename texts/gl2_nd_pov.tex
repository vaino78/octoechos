%<%[Въ суббw'ту на повече'рiи%],%>

%<канw'нъ прест~jь'й бц\дjь. [то'же и= w\т 
w=клевета'нiя, въ печа'лехъ, и= напа'стехъ су'щiи да 
пою'тъ.] Гла'съ в~.%>

%<%[Пjь'снь а~%]%>

%<I=рмо'съ: В%>о глубинjь` постла` и=ногда`, 
фараwни'тское всево'инство преwруже'нная си'ла, 
вопло'щшееся же сло'во всеsло'бный грjь'хъ потреби'ло 
_е='сть: препросла'вленный гд\сь сла'внw бо просла'вися.

%<Jа='%>же всjь^мъ о_у=до'бь послу'шная въ ско'рби, и= 
въ печа'ли помога'ющая, бц\де бл~га'я, бл~года'ть 
пода'ждь пjь'ти тя` дерза'ющымъ, вл\дчце скорбя'щихъ 
ра'досте.

%<М%>ногобога'тную бл~года'ть стяжа'вши вл\дчце, 
дерзнове'нною твое'ю мл~твою, пресла'внw и=зми' мя w\т 
напа'стей, о_у=бо'гаго раба` твоего`, скорбя'щихъ 
ра'досте.

%<Сла'ва: W\т%> вра^гъ ви'димыхъ и= неви'димыхъ 
и=зба'ви, мо'лимся, къ тебjь` прибjьга'ющихъ бц\де, и= 
разори` вся'къ совjь'тъ борю'щихъ на'съ.

%<И= ны'нjь: W\т%>ими` поноше'нiе человjь'ческое, и= 
клев_еты` навjь^тникъ w\т мене`, бц\де, молю' тя: да 
о_у=се'рднw сла'влю гд\са, _е=го'же пита'ла _е=си`.

%<%[Пjь'снь г~%]%>

%<I=рмо'съ: О_у=%>тверди` на'съ въ тебjь` гд\си, 
дре'вомъ о_у=мерщве'й грjь'хъ, и= стра'хъ тво'й всади` въ 
сердца` на'съ пою'щихъ тя`.

%<В%>рагw'въ навjь'ты разори` су'_етныя, всепjь'тая 
бц\де: не w=скудjь'й мл~твами твои'ми, снабдя'щи на'съ 
восхваля'ющихъ тя`.

%<П%>ри'зри ч\стая, ми'лостивнымъ твои'мъ _о='комъ, и= 
и=зба'ви мя` вся'кагw навjь'та ви'димыхъ и= неви'димыхъ 
врагw'въ, w=слjьпи'вши зjь^ницы _о=че'съ и='хъ.

%<Сла'ва: S%>ло'е нападе'нiе врагw'въ, jа='кw _о='гнь 
паля'щее, и='щущихъ всегда` погуби'ти на'съ дв~о, росо'ю 
мл~твъ твои'хъ о_у=гаси`. (с. 201)

%<И= ны'нjь: Н%>еугаси'мая лампа'до, заре` 
присносiя'тельная, но'щiю мя` скорбе'й w=держи'ма, 
просвjьти` мл~твами твои'ми, ро'ждшая сл~нце сла'вы 
хр\ста`.

%<%[Пjь'снь д~%]%>

%<I=рмо'съ: О_у=%>слы'шахъ гд\си гла'съ тво'й, 
_е=го'же ре'клъ _е=си`, гла'съ вопiю'щагw въ пусты'ни, 
jа='кw возгремjь'лъ _е=си` над\ъ вода'ми мно'гими, 
твоему` свидjь'тельствуяй сн~у, ве'сь бы'въ соше'дшагw 
дх~а, возопи`: ты` _е=си` хр\сто'съ, бж~iя му'дрость и= 
си'ла.

%<С%>п~се'нiя тя` мо'стъ, мл~тву неусы'пну, 
заступле'нiе тве'рдо мо'лимъ: о_у=милосе'рдися и= ви'ждь 
на'шу нестерпи'мую печа'ль, болjь^зни, скw'рби, стра^сти, 
и= благопремjь'ннw посjьти` мт~и бж~iя, ра'дость ско'рую 
подаю'щи.

%<Н%>е неприча'стни _е=смы` вл\дчце, твоегw` 
заступле'нiя въ ско'рбехъ: тjь'мже и= ны'нjь на'мъ 
помози` вско'рjь, лю'тjь w=бурева'_емымъ ру'ку 
простира'ющи ч\стая: бу'ди ми'лостива болjь'знемъ 
на'шымъ, мт~и бж~iя, ра'дость ско'рую подаю'щи.

%<Сла'ва: Н%>е о_у=пова'ша вл\дчце на тя` 
беззако'ннующiи, но о_у=пова'ша на я=зы'къ велерjь'чивъ, 
на я=зы'къ человjь'чь, при'снw зави'стнw и=злiя'вшiйся, 
непра'веднw пролiя'ти кро'вь и='скреннягw, рыка'юще: ты' 
же ч\стая, ч_е'люсти и='хъ сокруши`.

%<И= ны'нjь: С%>мири` вл\дчце, врагw'въ вознесе'нную 
вы'ю велерjь'чующихъ, совjь'ты и= sлонра^вiя, и= сердца` 
поуча^ющаяся sлы^мъ на мя` по вся^ дни^: крjь'пость же и= 
побjь'ду пода'ждь призыва'ющымъ тя`, мт~и бж~iя, ра'дость 
ско'рую подаю'щи.

%<%[Пjь'снь _е~%]%>

%<I=рмо'съ: О_у='%>гль и=са'iи проявле'йся, сл~нце 
и=з\ъ дв~ственныя о_у=тро'бы возсiя`, во тьмjь` 
заблу'ждшымъ, бг~оразу'мiя просвjьще'нiе да'руя. (с. 202)

%<М%>л~твеннице нело'жная, о_у=пова'нiе хр\стiа'нъ, 
w=бра'дованная, прiими` мольбы^ w\т на'съ прилjь'жнw 
призыва'ющихъ и= моля'щихся тебjь`.

%<И=%>сто'чникъ тя` жи'зни су'щи ч\стая, безсме'ртiя 
и=сточа'ющiй во'ды, земноро'днiи позна'вше о_у=блажа'емъ.

%<Сла'ва: В%>оwружи'ся на ны` лука'внw вра'гъ, 
я=зы'комъ jа='коже мече'мъ погуби'ти хотя`: но твое'ю, 
бг~ороди'тельнице, крjь'постiю предвари`.

%<И= ны'нjь: П%>учи'ну, ч\стая, заступле'нiю кто` 
и=зочте'тъ си'лы твоея`; тjь'мже на'съ су'щихъ въ ну'ждjь 
ско'рw предвари`.

%<%[Пjь'снь s~%]%>

%<I=рмо'съ: В%>ъ бе'зднjь грjьхо'внjьй валя'яся, 
неизслjь'дную милосе'рдiя твоегw` призыва'ю бе'здну: w\т 
тли` бж~е мя` возведи`.

%<Ц%>jьлому'дрiя су'щи хода'таица, призыва'ющымъ тя` 
ны'нjь jа=ви'ся, и= w\т вся'кихъ и=зба'ви напа'стей и= 
бjь'дъ, бг~оневjь'сто.

%<S%>лодjь^йства вра^жiя разори`, и= клеветы^ 
непра'в_едныя о_у=ста'ви, бл~гослове'нная преч\стая, 
и=збавля'ющи непови^нныя w\т ско'рби.

%<Сла'ва: Л%>ю'тыми грjьхи` w=круже'ни, и= напа'стными 
бjьда'ми потопля'еми, под\ъ бж~е'ственный покро'въ тво'й 
прибjьга'емъ, мт~и хр\ста` бг~а.

%<И= ны'нjь: Н%>еискусому'жнw ро'ждши гд\са, jа=ви'ся 
по рж\ствjь` па'ки дjь'вствующи: _w пресла'внагw чудесе`, 
въ тебjь` содjь'ланнагw, бг~оневjь'сто!

%<Та'же: Г%>д\си, поми'луй, %<три'жды. Сла'ва, и= 
ны'нjь:%>

%<Сjьда'ленъ, гла'съ в~:%>

%<М%>оле'нiе те'плое, и= стjьна` неwбори'мая, мл\сти 
и=сто'чниче, мi'рови прибjь'жище, прилjь'жнw вопiе'мъ 
ти`: бц\де вл\дчце предвари`, и= w\т бjь'дъ и=зба'ви 
на'съ, _е=ди'на вско'рjь предста'тельствующая. (с. 203)

%<%[Пjь'снь з~%]%>

%<I=рмо'съ: %>Да пресла'вное рж\ство` твое` хр\сте`, 
_е='же w\т дв~ы проwбрази'ши jа='вjь, неwпали^мы въ пещи` 
_о='троки сохрани'лъ _е=си`, пjь'сньми пою'щыя тебjь`: 
_о=тц_е'въ бж~е бл~гослове'нъ _е=си`.

%<_W%> благоутро'бiя твоегw` дв~о ч\стая! рjьши'ши бо 
безмjь^рныя печа^ли и= напа^сти, призыва'ющымъ въ ну'ждjь 
и= w=бстоя'нiи: тjь'мже и= ны'нjь помози` 
бл~гослове'нная, тя` восхваля'ющымъ.

%<П%>окажи` ско'рое твое` заступле'нiе, покажи` jа='кw 
мо'жеши, jа='кw мт~и бг~у, призыва'емъ тя` w\т се'рдца со 
слеза'ми припа'дающе, ско'рбь о_у=кроти` и= болjь'знь 
рабw'въ твои'хъ вско'рjь.

%<Сла'ва: О_у=%>ста` человjь'кwмъ jа='кw львw'мъ 
лю^тымъ, гро'ба тяжча'йше w\тверзо'шася, го'рькw 
поглоти'ти мя`: но ты` бц\де, jа=ви'ся наде'жда 
ненадjь'ющымся, и= низложи` крjь'пость и='хъ 
бл~гослове'нная.

%<И= ны'нjь: Д%>а о_у='зрятъ врази` и= постыдя'тся, да 
разумjь'ютъ и= ви'дятъ твою`, _е='же по на'мъ сопроти'въ 
борю'щую си'лу, и= въ про'пасть преиспо'днюю низложи` 
и=`хъ бл~гослове'нная, наде'ждо ненадjь'ющихся.

%<%[Пjь'снь и~%]%>

%<I=рмо'съ: П%>е'щь и=ногда` _о='гненная въ 
вавv"лw'нjь дjь^йства раздjьля'ше, бж~iимъ велjь'нiемъ 
халд_е'и w=паля'ющая, вjь^рныя же w=роша'ющая, пою'щыя: 
бл~гослови'те вся^ дjьла` гд\сня гд\са.

%<П%>рибjь'жище на'ше, и= мi'ру ра'дованiе, 
бг~ороди'тельнице, о_у=милосе'рдитися потщи'ся, и= 
спjь'шнw бл~года'ть твою` пода'ждь на'мъ 
w=скорбл_е'ннымъ, и= заступи` бл~га'я, твоя^ рабы^.

%<С%>овjь'тъ су'етенъ совjьща'ша лю'тjь на ны` 
безбо'жныхъ сw'нмища, jа='коже пре'жде а=хiтофе'лъ. но 
вопiе'мъ: твои'ми мл~твами се'й разори`, низло'жши си'хъ 
крjь'пость бц\де. (с. 204)

%<Сла'ва: П%>ризыва'ющихъ тя` бц\де, w\т души` 
нело'жнw, во вся'кой ско'рби и= въ болjь'знехъ 
разли'чныхъ, и= бjьда'хъ лю'тыхъ, спjь'шнw о_у=слы'ши 
вл\дчце, си'хъ и=збавля'ющи при'снw мл~твами твои'ми.

%<И= ны'нjь: Д%>а и='мя твое` бц\де, на земли` 
славосло'вится, и='же и=з\ъ тебе` возсiя'въ, грjь'шникwмъ 
держа'вную наде'жду и= стjь'ну тя` дарова`: тобо'ю бо 
вся'кое дыха'нiе къ бг~у притека'етъ.

%<%[Пjь'снь f~%]%>

%<I=рмо'съ: Н%>едоумjь'етъ вся'къ я=зы'къ 
бл~гохвали'ти по достоя'нiю, и=з\ъумjьва'етъ же о_у='мъ 
и= премi'рный пjь'ти тя` бц\де: _о=ба'че бл~га'я су'щи, 
вjь'ру прiими`, и='бо любо'вь вjь'си бж~е'ственную на'шу, 
ты' бо хр\стiа'нъ _е=си` предста'тельница, тя` 
велича'емъ.

%<Д%>а загради'тся вся'къ я=зы'къ, лука^вымъ 
поуча'яйся: да молча'тъ о_у=стна` льсти^вая, и= о_у=ста` 
непра'веднw глагw'лющая на пра'веднаго беззако'нiе, 
горды'нею и= за'вистiю вражде'бною вку'пjь, бц\ды 
мл~твою, и= ст~ы'хъ хр\сто'выхъ.

%<И=`%>же въ мл~твахъ бо'друю, ч\стую бц\ду, 
призыва'емъ вси` болjь'знiю и= ско'рбiю w=держи'ми, 
взыва'юще: вл\дчце ч\стая, ско'рw и=зба'ви w\т 
w=держа'щихъ болjь'зней рабы^ твоя^ при'снw, jа='кw по 
бз~jь засту'пницы и=ны'я не и='мамы.

%<Сла'ва: W\т%>ча'явшихся ве'лiе прибjь'жище, 
w=бурева'емыхъ ти'хое приста'нище, ты` _е=си` бц\де. 
тjь'мже и= мы` притека'емъ, взыва'юще: да не постыди'мся 
мт~и и='стиннагw живота`, но да благодаря'ще о_у=се'рднw 
тя` велича'емъ.

%<И= ны'нjь: П%>рiими`, _о=трокови'це преч\стая, 
бж~е'ственную пjь'снь, воздава'ющи бл~года'ть на тя` 
надjь'ющымся, и= ми'ръ и=спроси` цр~квамъ при'снw 
посла'ти, да хр\стiа'нскiй вся'къ я=зы'къ тя` велича'етъ.

%<Та'же: Д%>осто'йно _е='сть: %<И=%> про'чее 
_о=бы'чно, и= w\тпу'стъ. (с. 205)
