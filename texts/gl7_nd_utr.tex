На о_у='трени по шесто_пса'лмiи:

Бг~ъ гд\сь, и= jа=ви'ся на'мъ: на гла'съ з~. И= 
глаго'лемъ тропа'рь воскр\снъ: Разруши'лъ _е=си` 
кр\сто'мъ твои'мъ сме'рть: Два'жды. Сла'ва, и= ны'нjь, 
бг~оро'диченъ: Jа='кw на'шегw воскр\снiя сокро'вище:

Та'же, _о=бы'чное стiхосло'вiе _псалти'ра.

По а~-мъ стiхосло'вiи сjьда'льны воскр\сны, гла'съ з~:

Жи'знь во гро'бjь возлежа'ше, и= печа'ть на ка'мени 
надлежа'ше, jа='кw цр~я` спя'ща во'ини стрежа'ху хр\ста`: 
и= а='гг~ли сла'вляху jа='кw бг~а безсме'ртна, ж_ены' же 
взыва'ху: воскр~се гд\сь, подая` мi'рови ве'лiю мл\сть.

Стi'хъ: Воскр\сни` гд\си бж~е мо'й, да вознесе'тся 
рука` твоя`, не забу'ди о_у=бо'гихъ твои'хъ до конца`.

Тридне'внымъ погребе'нiемъ твои'мъ плjьни'вый сме'рть, 
и= и=стлjь'вша человjь'ка живоно'снымъ воста'нiемъ 
твои'мъ воскр~си'вый хр\сте` бж~е, jа='кw 
чл~вjьколю'бецъ, сла'ва тебjь`.

Сла'ва, и= ны'нjь, бг~оро'диченъ: Распе'ншагося на'съ 
ра'ди, и= воскр\сшаго хр\ста` бг~а на'шего, и= 
низложи'вшаго сме'рти держа'ву, непреста'ннw моли` бц\де 
дв~о: да сп~се'тъ ду'шы на'шя.

По в~-мъ стiхосло'вiи, сjьда'льны воскр\сны, гла'съ 
з~:

Запеча'тану гро'бу, живо'тъ w\т гро'ба возсiя'лъ 
_е=си` хр\сте` бж~е: и= две'ремъ заключ_е'ннымъ, 
о_у=ч~нкw'мъ предста'лъ _е=си` всjь'хъ воскр\снiе, дх~ъ 
пра'вый тjь'ми w=бновля'я на'мъ, по вели'цjьй твое'й 
мл\сти.

Стi'хъ: И=сповjь'мся тебjь` гд\си, всjь'мъ се'рдцемъ 
мои'мъ, повjь'мъ вся^ чудеса` твоя^.

На гро'бъ теча'ху ж_ены`, со слеза'ми мv'ра нося'щя: 
и= во'инwмъ стрегу'щымъ тя` всjь'хъ цр~я`, глаго'лаху къ 
себjь`: кто` w\твали'тъ на'мъ ка'мень; воскр~се вели'ка 
совjь'та а='гг~лъ, попра'вый сме'рть: всеси'льне гд\си, 
сла'ва тебjь`.

Сла'ва, и= ны'нjь, бг~оро'диченъ: Ра'дуйся 
бл~года'тная бц\де дв~о, приста'нище и= предста'тельство 
ро'да человjь'ческагw, и=з\ъ тебе' бо воплоти'ся 
и=зба'витель мi'ра: _е=ди'на бо _е=си` мт~и и= дв~а, 
при'снw бл~гослове'на и= препросла'влена. моли` хр\ста` 
бг~а ми'ръ дарова'ти все'й вселе'ннjьй.

V=пакои`, гла'съ з~:

И='же на'шъ зра'къ воспрiе'мый, и= претерпjь'вый 
кр\стъ пло'тски, сп~си' мя воскр\снiемъ твои'мъ хр\сте` 
бж~е, jа='кw чл~вjьколю'бецъ.

Степ_е'нна, гла'съ з~. \cslemph{А=нтiфw'нъ} а~, и='хже стiхи` 
повторя'юще пое'мъ:

Плjь'нъ сiw'нь w\т ле'сти w=брати'въ, и= мене` сп~се, 
w=живи`, и=з\ъима'я рабо'тныя стра'сти.

Въ ю='гъ сjь'яй скw'рби пw'стныя со сле'зами, се'й 
ра'достныя по'жнетъ рукоя^ти присноживопита'нiя.

Сла'ва: Ст~ы'мъ дх~омъ и=сто'чникъ бж~е'ственныхъ 
сокро'вищъ, w\т негw'же прему'дрость, ра'зумъ, стра'хъ: 
тому` хвала` и= сла'ва, че'сть и= держа'ва.

И= ны'нjь, то'йже.

\cslemph{А=нтiфw'нъ} в~:

А='ще не гд\сь сози'ждетъ до'мъ душе'вный, всу'е 
тружда'емся: ра'звjь бо тогw` ни дjья'нiе, ни сло'во 
соверша'ется.

Плода` чре'вна, ст~i'и дх~одви'жнw прозяба'ютъ 
_о=те'ч_еская преда^нiя, сн~оположе'нiя.

Сла'ва: Ст~ы'мъ дх~омъ, вся'ч_еская _е='же бы'ти 
и='мутъ: пре'жде бо всjь'хъ бг~ъ, всjь'хъ гд\сьство, 
свjь'тъ непристу'пенъ, живо'тъ всjь'хъ.

И= ны'нjь, то'йже.

\cslemph{А=нтiфw'нъ} г~:

Боя'щiися гд\са, пути^ живота` w=брjь'тше, ны'нjь и= 
при'снw о_у=бл~жа'ются сла'вою нетлjь'нною.

_О='крестъ трапе'зы твоея`, jа='кw сте'блiе ви'дя 
и=сча^дiя твоя^, ра'дуйся и= весели'ся, приводя` сiя^ 
хр\сто'ви @пастыренача'льнику@{пастыренача'льниче}.

Сла'ва: Ст~ы'мъ дх~омъ глубина` дарова'нiй, бога'тство 
сла'вы, суде'бъ глубина` ве'лiя: _е=диносла'венъ _о=ц~у`, 
и= сн~у, служи'мь бо.

И= ны'нjь, то'йже.

Прокi'менъ, гла'съ з~: Воскр\сни` гд\си бж~е мо'й, да 
вознесе'тся рука` твоя`, не забу'ди о_у=бо'гихъ твои'хъ 
до конца`. Стi'хъ: И=сповjь'мся тебjь` гд\си, всjь'мъ 
се'рдцемъ мои'мъ. Та'же, Вся'кое дыха'нiе: Стi'хъ: 
Хвали'те бг~а во ст~ы'хъ _е=гw`. _Е=v\глiе воскр\сно: 
Воскр\снiе хр\сто'во: _псало'мъ н~: и= прw'чая поря'ду.

Канw'нъ воскр\снъ, гла'съ з~:

\cslemph{Пjь'снь а~}

I=рмо'съ: Ма'нiемъ твои'мъ на земны'й w='бразъ 
преложи'ся, пре'жде о_у=доборазлива'емое водно'е 
_е=стество` гд\си: тjь'мже немо'креннw пjьшеше'ствовавъ 
i=и~ль, пое'тъ тебjь` пjь'снь побjь'дную.

Припjь'въ: Сла'ва гд\си, ст~о'му воскр\снiю твоему`.

W=суди'ся сме'ртное мучи'тельство дре'вомъ, 
непра'ведною сме'ртiю w=сужде'ну ти` гд\си: w\тню'дуже 
кня'зь тьмы` тебjь` не w=долjь'въ, пра'веднw и=згна'нъ 
бы'сть.

А='дъ тебjь` прибли'жися, и= зубы` не возмо'гъ сте'рти 
тjь'ло твое`, челюстьми` сокруши'ся. w\тню'дуже сп~се, 
бwлjь'зни разру'шъ см_е'ртныя, воскр\слъ _е=си` 
тридне'венъ.

Бг~оро'диченъ: Разрjьши'шася бwлjь'зни прама'тере 
_е='vы: болjь'зни бо и=збjьжа'вше, неискусому'жнw родила` 
_е=си`. w\тню'дуже jа='вjь бц\ду, преч\стая, вjь'дуще 
тя`, вси` сла'вимъ.

И='нъ канw'нъ кр\стовоскр\снъ, гла'съ з~:

\cslemph{Пjь'снь а~}

I=рмо'съ: По'нтомъ покры` фараw'на съ колесни'цами:

Два` и=стw'чника на'мъ на кр\стjь` и=сточи` сп~съ 
живонw'сная w\т своегw` прободе'ннагw ребра`. пои'мъ 
_е=му`: jа='кw просла'вися.

Во гро'бъ всели'вся, и= воскр~съ тридне'венъ, 
нетлjь'нiе подаде`, ча'янiе хр\сто'съ см_е'ртнымъ. пои'мъ 
_е=му`: jа='кw просла'вися.

Бг~оро'диченъ: _Е=ди'на дв~а и= по рж\ствjь` 
показа'лася _е=си`: зижди'теля бо мi'рови воплоще'нна 
родила` _е=си`: тjь'мже ра'дуйся, тебjь` вси` зове'мъ.

И='нъ канw'нъ прест~jь'й бц\дjь, [_е=гw'же 
краегране'сiе по а=лфави'ту.] Гла'съ з~.

I=рмо'съ: Ма'нiемъ твои'мъ на земны'й:

Бе'здну ро'ждшая бл~гоутро'бiя дв~о, ду'шу мою` 
w=зари` свjьтоно'снымъ твои'мъ сiя'нiемъ, jа='кw да 
досто'йнw воспою` твои'хъ чуде'съ бе'здну.

Стрjьло'ю грjьхо'вною о_у=я'звлены на'съ сло'во 
ви'дjьвъ, jа='кw бл~годjь'тель о_у=ще'дри: w\тню'дуже 
несказа'ннw соедини'ся всеч\стая, @пло'тiю, и='же и=з\ъ 
тебе`@{пло'ти jа='же и=з\ъ тебе`} пребж~е'ственный.

Бы'сть сме'ртiю jа='то _е=стество`, тлjь'нное же и= 
ме'ртвенное человjь'ческое, вл\дчце: ты' же жи'знь 
заче'нши, сiе` w\т и=стлjь'нiя къ животу` возвела` 
_е=си`.

Катава'сiа: W\тве'рзу о_у=ста` моя^:

\cslemph{Пjь'снь г~}

I=рмо'съ: Въ нача'лjь нб~са` всеси'льнымъ сло'вомъ 
твои'мъ о_у=твержде'й гд\си сп~се, и= вседjь'тельнымъ и= 
бж~iимъ дх~омъ всю` си'лу и='хъ, на недви'жимjьмъ мя` 
ка'мени и=сповjь'данiя твоегw` о_у=тверди`.

Ты` возше'дъ на дре'во, w= на'съ болjь'знуеши во'лею 
бл~гоутро'бне сп~се, и= терпи'ши jа='зву, примире'нiя 
хода'тайственну, и= сп~се'нiя вjь^рнымъ: _е='юже твоему` 
мл\стиве, вси` примири'хомся роди'телю.

Ты' мя w=чи'стивъ w\т jа='звы, душе'ю о_у=я'звленнаго 
sмi'евымъ о_у=грызе'нiемъ хр\сте`, и= показа'лъ _е=си` 
свjь'тъ, во тьмjь` дре'вле мнjь` слежа'щу и= во тлjь'нiи: 
кр\сто'мъ бо во а='дъ соше'дъ, мене` совоскр~си'лъ 
_е=си`.

Бг~оро'диченъ: Неискусому'жныя твоея` мт~ре мольба'ми 
ми'ръ мi'рови пода'ждь, сп~се, и= несказа'нныя твоея` 
сла'вы славосло'вящыя тя` сподо'би.

\cslemph{И='нъ} I=рмо'съ: Нб~са` о_у=тверди'вый сло'вомъ:

И='же на кр\стjь` стр\сти претерпjь'вый, и= 
разбо'йнику ра'й w\тве'рзый, jа='кw бл~годjь'тель и= 
бг~ъ, о_у=тверди` мо'й о_у='мъ въ во'лю твою`, _е=ди'не 
чл~вjьколю'бче.

Воскр~сы'й тридне'венъ и=з\ъ гро'ба, и= живо'тъ 
мi'рови возсiя'вый, jа='кw жизнода'вецъ и= бг~ъ, 
о_у=тверди` мо'й о_у='мъ въ во'лю твою`, _е=ди'не 
чл~вjьколю'бче.

Бг~оро'диченъ: Jа='кw бг~а безсjь'меннw заче'нши, и= 
w\т кля'твы _е='vу и=зба'вльши, дв~о мт~и марiа'мъ, моли` 
w\т тебе` вопло'щшагося бг~а, сп~сти` ста'до твое`.

\cslemph{И='нъ} I=рмо'съ: Въ нача'лjь нб~са`:

Sмi'й поползы'й и=з\ъ _е=де'ма, мене` w=боже'нiя 
жела'нiемъ прельсти'въ, ве'рже въ зе'млю: но и='же 
мл\стивъ, и= _е=стество'мъ бл~гоутро'бенъ, о_у=ще'дривъ 
бг~осодjь'ла, во чре'во твое` все'лься, и= подо'бенъ 
мнjь` бы'въ мт~и дв~о.

Бл~гослове'нъ пло'дъ твоегw` чре'ва, дв~о бц\де, 
всjь'хъ ра'досте, ра'дость бо всему` мi'ру родила` 
_е=си`, и= весе'лiе вои'стинну разгоня'ющее печа'ль 
грjьхо'вную, бг~оневjь'сто.

Жи'знь вjь'чную, и= свjь'тъ бг~ороди'тельнице дв~о, и= 
ми'ръ родила` _е=си` на'мъ, дре'внихъ человjь^къ бра'нь 
jа='же ко _о=ц~у` и= бг~у, о_у=кротjьва'ющiй, вjь'рою и= 
и=сповjь'данiемъ бл~года'ти.

\cslemph{Пjь'снь д~}

I=рмо'съ: _О='ч~а нjь'дра не w=ста'вль, и= соше'дъ на 
зе'млю хр\сте` бж~е, та'йну о_у=слы'шахъ смотре'нiя 
твоегw`, и= просла'вихъ тя` _е=ди'не чл~вjьколю'бче.

Своя^ плещи` да'въ, и='же w\т дв~ы вопло'щься, на 
ра^ны, рабу` прегрjьши'вшу, бiе'нъ быва'етъ вл\дка 
непови'ненъ, разрjьша'я моя^ согрjьш_е'нiя.

Предстоя'въ суди'щу судi'й законопресту'пныхъ, jа='кw 
w=суди'мый @пови'ненъ быва'етъ@{и=стязу'ется}, и= 
зауша'ется бре'нною руко'ю, созда'вый человjь'ка jа='кw 
бг~ъ, и= судя'й пра'веднw земли`.

Бг~оро'диченъ: Jа='кw вои'стинну мт~и бж~iя, творца` 
твоего` и= сн~а моли`, къ сп~си'тельному напра'вити мя` 
приста'нищу всенепоро'чная сла'внагw _е=гw` хотjь'нiя.

\cslemph{И='нъ} I=рмо'съ: Смотря'яй пр\оро'къ:

Не вjь'дый грjьха`, и= _е=гw` ра'ди бы'въ гд\си, 
_е='же не бы'лъ _е=си` воwбража'ешися, прiе'мъ чу'ждее, 
да сп~се'ши мi'ръ, и= о_у=бiе'ши прельсти'въ мучи'теля.

На кр\стjь` воздви'женъ бы'лъ _е=си`, и= пра'_отца 
а=да'ма разрjьши'лъ _е=си` грjь'хъ [_е=гw'же ра'ди твою` 
о_у=слы'шахъ си'лу:] jа='кw вся^ сп~сти` пома^занныя 
твоя^ прише'лъ _е=си`.

Бг~оро'диченъ: Ты` рожде'йся w\т дв~ы, о_у=мира'еши, 
w=живля'еши же а=да'ма мы'слiю заблу'ждшаго: и='бо 
о_у=боя'ся сме'рть крjь'пости твоея`, jа='кw вся^ сп~сти` 
растлjь'вшыяся прише'лъ _е=си`.

\cslemph{И='нъ} I=рмо'съ: _О='ч~а нjь'дра не w=ста'вль:

И=збра'нная вся` и= до'брая, jа='вльшися бг~у пре'жде 
созда'нiя свjь'тлостiю всепjь'тая, @свjьтоли'тiемъ 
твои'мъ@{свjьтоли'тiя твоегw`} пою'щыя тя` просвjьти`.

Бг~а человjь'кwмъ родила` _е=си` ч\стая, воплоще'нна 
w\т ч\стыхъ крове'й твои'хъ, и=збавля'юща согрjьше'нiй 
мно'гихъ, любо'вiю сла'вящыя и= почита'ющыя тя` мт~и 
дв~о.

Сщ~еннодjь'йствуетъ _е=стество` слове'сное, 
возсiя'вшему и=з\ъ тебе` всепjь'тая, та'инству 
неизрече'нному рж\ства` твоегw` научи'вшееся ны'нjь, 
пребл~же'нная.

\cslemph{Пjь'снь _е~}

I=рмо'съ: Но'щь не свjьтла` невjь^рнымъ хр\сте`, 
вjь^рнымъ же просвjьще'нiе въ сла'дости слове'съ твои'хъ: 
сегw` ра'ди къ тебjь` о_у='тренюю, и= воспjьва'ю твое` 
бж~ество`.

За твоя^ рабы^ продае'шися хр\сте`, и= по лани'тjь 
о_у=даре'нiе терпи'ши, свобо'дjь хода'тайственно 
пою'щымъ: къ тебjь` о_у='тренюю, и= воспjьва'ю твое` 
бж~ество`.

Бж~е'ственною твое'ю си'лою хр\сте`, не'мощiю 
плотско'ю крjь'пкаго низложи'лъ _е=си`, и= побjьди'теля 
мя` сме'рти, сп~се, воскр\снiемъ показа'лъ _е=си`.

Бг~оро'диченъ: Бг~а родила` _е=си` мт~и ч\стая, 
воплоще'ннаго и=з\ъ тебе` бг~олjь'пнw всепjь'тая: поне'же 
не позна'ла _е=си` му'жеска по'ла, но w\т ст~а'гw 
ражда'еши дх~а.

\cslemph{И='нъ} I=рмо'съ: Гд\си бж~е мо'й, w\т но'щи:

_Е=гда` со беззако'нными вмjьни'вся, возне'слся _е=си` 
на ло'бнjьмъ, свjьти^ла сокрыва'хуся, и= земля` 
колеба'шеся, и= цр~ко'вная свjь'тлость раздра'ся, 
_е=вре'йское jа=вля'ющи w\тпаде'нiе.

Тебе` разруши'вшаго мучи'телеву всю` си'лу крjь'постiю 
непостижи'магw твоегw` бж~ества`, и= м_е'ртвыя твои'мъ 
воскр\снiемъ воздви'гшаго пjь'сньми сла'вимъ.

Бг~оро'диченъ: Мт~и цр~я` и= бг~а всепjь'тая бц\де, 
вjь'рою и= любо'вiю тя` пjь'сньми восхваля'ющымъ при'снw, 
w=чище'нiе прегрjьше'нiй твои'ми мольба'ми низпосли`.

\cslemph{И='нъ} I=рмо'съ: Но'щь не свjьтла` невjь^рнымъ:

Лjь'ствицу о_у=зрjь'въ i=а'кwвъ, къ высотjь` 
о_у=твержде'ну, w='бразу научи'ся, неискусобра'чная 
тебе`: тобо'ю бо бг~ъ чл~вjь'кwмъ прiwбщи'ся, всеч\стая 
вл\дчце.

И=збавле'нiе вjь'чное тобо'ю дв~о, ны'нjь w=брjь'тше 
о_у=се'рднw зове'мъ ти`: _е='же ра'дуйся бг~оневjь'стная: 
и= твои'мъ свjь'томъ возра'довавшеся всепjь'тая, 
пjь'сньми тя` пое'мъ.

_Е=ди'ну тя` жени'хъ посредjь` те'рнiя крi'нъ дв~о, 
w=брjь'тъ, чистоты` блиста'нiемъ свjьтя'щуся, и= 
свjь'томъ дв~ства всенепоро'чная, невjь'сту воспрiя'тъ.

\cslemph{Пjь'снь s~}

I=рмо'съ: Пла'вающаго въ молвjь` жите'йскихъ 
попече'нiй, съ корабле'мъ потопля'ема грjьхи`, и= 
душетлjь'нному sвjь'рю примета'ема, jа='кw i=w'на 
хр\сте`, вопiю' ти: и=з\ъ смертоно'сныя глубины` возведи' 
мя.

Воспомина'ху тя` заключ_е'нныя во а='дjь ду'шы, и= 
w=ста'вльшыяся прв\дныхъ, и= w\т тебе` сп~се'нiя 
моля'хуся: _е='же кр\сто'мъ хр\сте`, по'далъ _е=си` 
преиспw'днимъ, прише'дъ jа='кw бл~гоутро'бенъ.

Ко w=душевле'нному твоему` и= нерукотворе'нному 
хра'му, разруше'ну бы'вшу страда'ньми, воззрjь'ти па'ки 
ли'къ а=п\сльскiй w\тча'яся: но па'че наде'жды 
покло'нься, воскре'сша повсю'ду проповjь'да.

Бг~оро'диченъ: Неизрече'ннагw рж\ства` твоегw` 
всенепоро'чнагw w='бразъ, дв~о бг~оневjь'стная, и='же 
на'съ ра'ди, кто` w\т человjь^къ сказа'ти возмо'жетъ; 
jа='кw бг~ъ неwпи'саннjь, сло'во соедини'вся тебjь`, 
пло'ть и=з\ъ тебе` бы'сть.

\cslemph{И='нъ} I=рмо'съ: I=w'на и=з\ъ чре'ва:

На кр\стjь` возне'сся сп~се во'лею, вра'жiю плjьни'лъ 
_е=си` держа'ву, на се'мъ пригвозди'въ грjьхо'вное бл~же, 
рукописа'нiе.

И=з\ъ ме'ртвыхъ воскр~съ сп~се вла'стiю, совоздви'глъ 
_е=си` человjь'ческiй ро'дъ, живо'тъ и= нетлjь'нiе 
дарова'вый на'мъ, jа='кw чл~вjьколю'бецъ.

Бг~оро'диченъ: _Е=го'же родила` _е=си` бц\де 
несказа'ннw бг~а на'шего, моля'щи не преста'й, 
и=зба'витися w\т бjь'дъ пою'щымъ тя`, ч\стая приснодв~о.

\cslemph{И='нъ} I=рмо'съ: Пла'вающаго въ молвjь`:

Зако'ннiи тя` w='бразы, и= пр\оро'ч_еская 
пр\ореч_е'нiя jа='вjь предвозвjьща'ху, хотя'щую роди'ти 
бл~годjь'теля ч\стая всея` тва'ри, многоча'стнjь и= 
многоwбра'знjь бл~годjь'йствовавшаго вjь'рнw 
воспjьва'ющихъ тя`.

О_у=стра'ншагося дре'вле навjь'томъ человjькоубi'йцы, 
а=да'ма первозда'ннаго ра'йскiя бж~е'ственныя сла'дости, 
неискусобра'чная па'ки возвела` _е=си`, ро'ждши и='же w\т 
преступле'нiя на'съ и=зба'вившаго.

И='же хотjь'нiемъ бж~е'ственнымъ, содjь'тельною же 
си'лою все` составле'й w\т не су'щихъ, и=з\ъ чре'ва 
твоегw` ч\стая, произы'де, и= су'щыя во тьмjь` 
сме'ртнjьй, бг~онача'льнjьйшими мо'лнiями w=сiя`.

Конда'къ, гла'съ з~: Не ктому` держа'ва сме'ртная 
возмо'жетъ держа'ти человjь'ки: хр\сто'съ бо сни'де 
сокруша'я и= разоря'я си^лы _е=я`. связу'емь быва'етъ 
а='дъ, пр\оро'цы согла'снw ра'дуются: предста`, 
глаго'люще, сп~съ су'щымъ въ вjь'рjь, и=зыди'те вjь'рнiи 
въ воскр\снiе.

I='косъ: Вострепета'ша до'лjь преиспw'дняя дне'сь, 
а='дъ и= сме'рть _е=ди'нагw w\т тр\оцы: земля` 
поколеба'ся, вра'тницы же а='довы ви'дjьвше тя` 
о_у=жасо'шася: вся' же тва'рь со пр\оро'ки ра'дующися 
пое'тъ тебjь` побjь'дную пjь'снь, и=зба'вителю бг~у 
на'шему, разруши'вшему ны'нjь сме'ртную си'лу. да 
воскли'кнемъ и= возопiи'мъ ко а=да'му, и= къ су'щымъ 
и=з\ъ а=да'ма: дре'во сего` па'ки введе`. и=зыди'те 
вjь'рнiи въ воскр\снiе.

\cslemph{Пjь'снь з~}

I=рмо'съ: Пе'щь _о='троцы _о=гнепа'льну дре'вле 
росоточа'щу показа'ша, _е=ди'наго бг~а воспjьва'юще, и= 
глаго'люще: превозноси'мый _о=тц_е'въ бг~ъ, и= 
препросла'вленъ.

Дре'вомъ о_у=мерщвля'ется а=да'мъ, во'лею преслуша'нiе 
содjь'лавъ: послуша'нiемъ же хр\сто'вымъ па'ки 
w=бновля'емь _е='сть. мене' бо ра'ди распина'ется сн~ъ 
бж~iй, препросла'вленный.

Тебе` воскре'сшаго хр\сте`, и=з\ъ гро'ба, тва'рь вся` 
воспjь`: ты' бо жи'знь су'щымъ во а='дjь процвjь'лъ 
_е=си`, м_е'ртвымъ воскр\снiе, и=`же во тьмjь`, свjь'тъ 
препросла'вленный.

Бг~оро'диченъ: Ра'дуйся дщи` а=да'ма тлjь'ннагw. 
ра'дуйся _е=ди'на бг~оневjь'сто. ра'дуйся, _е='юже тля` 
и=згна'на бы'сть, jа='же бг~а ро'ждши: _е=го'же моли` 
ч\стая, сп~сти'ся всjь^мъ на'мъ.

\cslemph{И='нъ} I=рмо'съ: Въ пе'щь _о='гненную вве'ржени:

И='же на дре'вjь кр\стнjьмъ грjьхо'вное жа'ло 
притупи'въ, и= а=да'мова преступле'нiя рукописа'нiе 
разруши'въ копiе'мъ ребра` твоегw`, бл~гослове'нъ _е=си` 
гд\си бж~е _о=т_е'цъ на'шихъ.

И='же въ ребро` прободе'нъ бы'въ, и= кропле'ньми 
кро'ве бж~е'ственныя зе'млю w=чи'стивъ, кровьми` 
i=дwлобjь'сiя w=скверне'нную, бл~гослове'нъ _е=си` гд\си 
бж~е _о=т_е'цъ на'шихъ.

Бг~оро'диченъ: _Е='же пре'жде сл~нца просвjьще'нiя, 
возсiя'вши мi'рови бг~ороди'тельнице хр\ста`, w\т тьмы` 
и=зба'вльшаго, и= просвjьща'юща вся^ бг~овjь'дjьнiемъ: 
бл~гослове'нъ _е=си` зову'щыя, бж~е _о=т_е'цъ на'шихъ.

\cslemph{И='нъ} I=рмо'съ: Пе'щь _о='троцы:

Преиспещре'ну, позлаще'ну о_у='тварь тя` и=му'щу 
возлюби` созда'тель тво'й дв~о, и= гд\сь: превозноси'мый 
_о=тц_е'въ бг~ъ и= препросла'вленъ.

W=чища'ется _о=трокови'це, о_у='гль дре'вле и=са'iа 
прiе'мъ: знамена'тельнjь твое` рж\ство` ви'дjьвъ, 
превозноси'маго _о=тц_е'въ бг~а, и= препросла'влена.

W='бразы знамена'тельная дре'вле бж~е'ственнагw 
твоегw` рж\ства`, бж~е'ственнiи пр\оро'цы зря'ще, 
ра'достнw воспjьва'юще взыва'ху: превозноси'мый 
_о=тц_е'въ бг~ъ, и= препросла'вленъ.

\cslemph{Пjь'снь и~}

I=рмо'съ: Неwпа'льная _о=гню` въ сiна'и прича'щшаяся 
купина`, бг~а jа=ви` медленоязы'чному и= гугни'вому 
мwv"се'ови, и= _о='троки ре'вность бж~iя три` 
непребори'мыя во _о=гни` пjьвцы` показа`: вся^ дjьла` 
гд\сня гд\са по'йте, и= превозноси'те во вся^ вjь'ки.

Преч\стый а='гнецъ слове'сный за мi'ръ закла'нъ бы'въ, 
преста'ви jа=`же по зако'ну приноси^мая, w=чи'стивъ сего` 
кромjь` прегрjьше'нiй jа='кw бг~ъ, при'снw зову'ща: вся^ 
дjьла` гд\сня гд\са по'йте, и= превозноси'те во вся^ 
вjь'ки.

Нетлjь'нна не су'щи пре'жде стр\сти, воспрiя'тая w\т 
созда'теля пло'ть на'ша, по стр\сти и= воскр\снiи 
неприкоснове'нна тлjь'нiю о_у=стро'ися, и= см_е'ртныя 
w=бновля'етъ, зову'щыя: вся^ дjьла` гд\сня гд\са по'йте, 
и= превозноси'те во вся^ вjь'ки.

Бг~оро'диченъ: Твое` чисто'тное и= всенепоро'чное, 
дв~о, скве'рное и= ме'рзское вселе'нныя w=чи'сти, и= 
была` _е=си` на'шегw примире'нiя къ бг~у вина`, 
преч\стая: тjь'мже тя` дв~о вся^ дjьла` бл~гослови'мъ, и= 
превозно'симъ во вся^ вjь'ки.

\cslemph{И='нъ} I=рмо'съ: _Е=ди'наго безнача'льнаго цр~я`:

Претерпjь'вшаго стра^сти во'лею, и= на кр\стjь` 
пригвожде'на хотjь'нiемъ, и= разру'шшаго си^лы а='довы, 
по'йте сщ~е'нницы, лю'дiе превозноси'те во вся^ вjь'ки.

О_у=праздни'вшаго сме'рти держа'ву, и= w\т гро'ба 
воскр\сшаго со сла'вою, и= сп~сшаго чл~вjь'ческiй ро'дъ, 
по'йте сщ~е'нницы, лю'дiе превозноси'те во вся^ вjь'ки.

Бг~оро'диченъ: _Е=ди'наго бл~гоутро'бнаго, 
превjь'чнаго сло'ва, напослjь'докъ w\т дв~ы рожде'на, и= 
разрjь'шшаго дре'внюю кля'тву, по'йте сщ~е'нницы, лю'дiе 
превозноси'те во вся^ вjь'ки.

\cslemph{И='нъ} I=рмо'съ: Неwпа'льная _о=гню` въ сiна'и:

Свjь'томъ рж\ства` твоегw`, страннолjь'пнw вселе'нную 
просвjьти'ла _е=си` бг~ороди'тельнице, су'ща бо 
вои'стинну бг~а на w=б\ъя'тiяхъ но'сиши твои'хъ, 
просвjьща'юща вjь^рныя при'снw зову'щыя: вся^ дjьла` 
гд\сня гд\са по'йте, и= превозноси'те во вся^ вjь'ки.

Пое'мъ ч\стая, бл~гоче'стнw твое` чре'во, бг~а 
вмjьсти'вшее несказа'ннw воплоща'ема, да'вшаго всjь'мъ 
вjь^рнымъ бг~оразу'мiя просвjьще'нiе, при'снw зову'щымъ: 
вся^ дjьла` гд\сня гд\са по'йте, и= превозноси'те во вся^ 
вjь'ки.

Свjь'та твоегw` блиста'ньми, тебе` пою'щыя 
свjьтови'дны содjь'лала _е=си`, свjьтороди'тельнице бц\де 
ч\стая: свjь'та бо jа=ви'лася _е=си` селе'нiе, 
о_у=ясня'ющи свjь'томъ зову'щыя: вся^ дjьла` гд\сня гд\са 
по'йте, и= превозноси'те во вся^ вjь'ки.

Та'же, пое'мъ пjь'снь бц\ды: Вели'читъ душа` моя` 
гд\са: съ припjь'вомъ: Ч\стнjь'йшую херувi^мъ:

\cslemph{Пjь'снь f~}

I=рмо'съ: Нетлjь'нiя и=скуше'нiемъ ро'ждшая, и= 
всехитрецу` сло'ву пло'ть взаимода'вшая, мт~и 
неискусому'жная дв~о бц\де: прiя'телище нестерпи'магw, 
село` невмjьсти'магw зижди'теля твоегw`, тя` велича'емъ.

И=`же бж~еству` стра'сть прилага'ющiи, заусти'теся 
вси` чуждему'дреннiи: гд\са бо сла'вы пло'тiю распя'та, 
не распя'та же _е=стество'мъ бж~е'ственнымъ, jа='кw во 
двою` _е=ст_еству` _е=ди'наго велича'емъ.

И=`же тjьлес_е'мъ воста'нiю невjь'рующiи, ко хр\сто'ву 
ше'дше гро'бу, научи'теся: jа='кw о_у=мерщвле'на бы'сть, 
и= воскр~се па'ки пло'ть жизнода'вца, во о_у=вjьре'нiе 
послjь'днягw воскр\снiя, на не'же о_у=пова'емъ.

Тр\оченъ: Не бж\ствъ тр\оцу, но v=поста'сей, ниже` 
_е=ди'ницу ли'цъ, но бж\ства` чту'ще, ссjьца'емъ же сiю` 
дjьля'щихъ: слива'емъ же па'ки, слiя'нiе дерза'ющихъ на 
сiю`, ю='же велича'емъ.

\cslemph{И='нъ} I=рмо'съ: Мт~и бж~iя и= дв~а:

Свjь'тъ w\т свjь'та, _о='ч~ее сiя'нiе сла'вы 
безлjь'тнw w=сiя'вый, jа='коже во тьмjь` чл~вjь'ческому 
житiю` хр\сто'съ возсiя`, и= гоня'щую прогна` тьму`: 
_е=го'же непреста'ннw вjь'рнiи велича'емъ.

Стра^сти плwтскi'я, и= крjь'пость бж~ества`, во 
хр\стjь` ви'дяще, му'дрствующiи _е=ди'но сло'жное 
_е=стество`, да посра'мятся: то'й бо jа='кw человjь'къ 
о_у='бw о_у=мира'етъ, jа='коже всегw` содjь'тель 
востае'тъ.

Мv'ро м_е'ртвымъ, живо'му же пjь'нiе: сле'зы 
о_у=мира'ющымъ, животу' же всjь'хъ пjь'снь ж_ены` 
принеси'те, и='же воста'нiя проповjь'дникъ вопiя'ше, 
бл~говjьству'я хр\сто'во воскр\снiе.

Бг~оро'диченъ: Бг~а ра'звjь тебе` и=но'гw не зна'ю, 
цр~ковь вопiе'тъ ти`: w\т невjь'рныхъ мя` jа=зы^къ 
невjь'сту свою` и=збра'вый, да'ждь о_у='бw сло'ве, 
вjь^рнымъ сп~се'нiе, ро'ждшiя тя` мл~твами, jа='кw 
бл~гоутро'бенъ.

\cslemph{И='нъ} I=рмо'съ то'йже.

Ра'дости на'мъ вjь'чныя хода'таица и= весе'лiя 
jа=ви'лася _е=си` приснодв~о _о=трокови'це, и=зба'вителя 
ро'ждши, и='стиною и= дх~омъ бж~е'ственнымъ того` 
чту'щихъ, jа='кw бг~а и=збавля'ющаго.

Поя` дв~дъ, тво'й пра'_от_ецъ, преч\стая, тя` ковче'гъ 
и=мену'етъ ст~ы'ни бж~е'ственныя, преесте'ственнjь бг~а 
вмjьсти'вшую, во _о=ч~ескихъ сjьдя'щаго нjь'дрjьхъ, 
_е=го'же непреста'ннw вjь'рнiи велича'емъ.

Jа='кw вои'стинну превы'шши _е=си` всея` тва'ри 
_о=трокови'це: зижди'теля во всjь'хъ тjьле'снjь на'мъ 
родила` _е=си`. тjь'мже jа='кw мт~и _е=ди'нагw вл\дки, 
но'сиши проти'ву всjь'хъ нача'льнjь побjьжде'нiе.

По катава'сiи, _е=ктенiа` ма'лая. Та'же, Ст~ъ гд\сь 
бг~ъ на'шъ. Посе'мъ _е=_ксапостiла'рiй о_у='треннiй.

На хвали'техъ стiхи^ры воскр\сны. Гла'съ з~:

Стi'хъ: Сотвори'ти въ ни'хъ су'дъ напи'санъ: сла'ва 
сiя` бу'детъ всjь^мъ прп\дбнымъ _е=гw`.

Воскр~се хр\сто'съ и=з\ъ ме'ртвыхъ, разру'шъ 
см_е'ртныя о_у='зы: бл~говjьсти` земле`, ра'дость ве'лiю, 
по'йте нб~са` бж~iю сла'ву.

Стi'хъ: Хвали'те бг~а во ст~ы'хъ _е=гw`, хвали'те 
_е=го` во о_у=тверже'нiи си'лы _е=гw`.

Воскр\снiе хр\сто'во ви'дjьвше, поклони'мся ст~о'му 
гд\су i=и~су: _е=ди'ному безгрjь'шному.

Стi'хъ: Хвали'те _е=го` на си'лахъ _е=гw`, хвали'те 
_е=го` по мно'жеству вели'чествiя _е=гw`.

Хр\сто'ву воскр\снiю кла'няющеся не преста'емъ: то'й 
бо сп~слъ _е='сть на'съ w\т беззако'нiй на'шихъ, ст~ы'й 
гд\сь i=и~съ, jа=вле'й воскр\снiе.

Стi'хъ: Хвали'те _е=го` во гла'сjь тру'бнjьмъ, 
хвали'те _е=го` во _псалти'ри и= гу'слехъ.

Что` возда'мы гд\севи w= всjь'хъ, jа=`же воздаде` 
на'мъ; на'съ ра'ди бг~ъ въ человjь'цjьхъ, за и=стлjь'вшее 
_е=стество` сло'во пло'ть бы'сть, и= всели'ся въ ны`, къ 
неблагода^рнымъ, бл~годjь'тель: къ плjь'нникwмъ, 
свободи'тель: ко и=`же во тьмjь` сjьдя'щымъ, сл~нце 
пра'вды: на кр\стjь`, безстра'стный: во а='дjь, свjь'тъ: 
въ сме'рти, живо'тъ: воскр\снiе, па'дшихъ ра'ди. къ 
нему'же возопiи'мъ: бж~е на'шъ, сла'ва тебjь`.

И='ны стiхи^ры а=натw'лiевы, гла'съ то'йже.

Стi'хъ: Хвали'те _е=го` въ тv"мпа'нjь и= ли'цjь, 
хвали'те _е=го` во стру'нахъ и= _о=рга'нjь.

Врата` а='дова сокруши'лъ _е=си` гд\си, и= сме'ртную 
держа'ву о_у=праздни'лъ _е=си` крjь'пкою си'лою твое'ю, 
и= совоздви'глъ _е=си` м_е'ртвыя, и=`же w\т вjь'ка во 
тьмjь` спя'щыя, бж~е'ственнымъ и= сла'внымъ воскр\снiемъ 
твои'мъ, jа='кw цр~ь всjь'хъ и= бг~ъ всеси'ленъ.

Стi'хъ: Хвали'те _е=го` въ кv"мва'лjьхъ 
доброгла'сныхъ, хвали'те _е=го` въ кv"мва'лjьхъ 
восклица'нiя: вся'кое дыха'нiе да хва'литъ гд\са.

Прiиди'те возра'дуемся гд\севи, и= возвесели'мся w= 
воскр\снiи _е=гw`, jа='кw совоздви'же м_е'ртвыя w\т 
а='довыхъ нерjьши'мыхъ о_у='зъ: и= дарова` мi'рови jа='кw 
бг~ъ жи'знь вjь'чную, и= ве'лiю мл\сть.

Стi'хъ: Воскр\сни` гд\си бж~е мо'й, да вознесе'тся 
рука` твоя`, не забу'ди о_у=бо'гихъ твои'хъ до конца`.

Блиста'яйся а='гг~лъ, на ка'мени сjьдя'ше 
живопрiе'мнагw гро'ба, и= жена'мъ мv"роно'сицамъ 
бл~говjьствова'ше, глаго'ля: воскр~се гд\сь, jа='коже 
пре'жде рече` ва'мъ, возвjьсти'те о_у=ч~нкw'мъ _е=гw`, 
jа='кw предваря'етъ вы` въ галiле'и: мi'рови же подае'тъ 
жи'знь вjь'чную, и= ве'лiю мл\сть.

Стi'хъ: И=сповjь'мся тебjь` гд\си всjь'мъ се'рдцемъ 
мои'мъ, повjь'мъ вся^ чудеса` твоя^.

Почто` небрего'ма сотвори'сте ка'мене краеуго'льнагw. 
_w пребеззако'ннiи i=уд_е'и; се` то'й _е='сть, _е=го'же 
положи` бг~ъ въ сiw'нjь: и='же и=з\ъ ка'мене и=сточи'вый 
въ пусты'ни во'ду, и= на'мъ и=сточа'яй w\т ре'бръ свои'хъ 
безсме'ртiе. се'й _е='сть ка'мень, и='же w\т горы` 
дв~и'ческiя w\тсjь'кся без\ъ хотjь'нiя му'жеска: сн~ъ 
чл~вjь'чь гряды'й на _о='блацjьхъ нб\сныхъ, къ ве'тхому 
де'ньми, jа='коже рече` данiи'лъ, и= вjь'чно _е=гw` 
цр\ство.

Сла'ва, стiхи'ра _е=v\гльская. И= ны'нjь, 
бг~оро'диченъ: Пребл~гослове'нна _е=си`: Славосло'вiе 
вели'кое. Та'же тропа'рь воскр\снъ:

Дне'сь сп~се'нiе мi'ру бы'сть, пое'мъ воскр~сшему 
и=з\ъ гро'ба, и= нача'льнику жи'зни на'шея: разруши'въ бо 
сме'ртiю сме'рть, побjь'ду даде` на'мъ, и= ве'лiю мл\сть.

И= _е=ктенiи`, и= w\тпу'стъ.

