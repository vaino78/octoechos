%[Въ суббw'ту на повече'рiи%],

канw'нъ моле'бный прест~jь'й бц\дjь. Гла'съ и~:

%[Пjь'снь а~%]

I=рмо'съ: Пои'мъ гд\севи, прове'дшему лю'ди своя^ 
сквозjь` чермно'е мо'ре, jа='кw _е=ди'нъ сла'внw 
просла'вися.

Припjь'въ: Прест~а'я бц\де, сп~си` на'съ.

Прiиди'те та'йнw бра'тiе, jа='кw w\т нача'ла ст~jь'й 
бц\дjь, внесе'мъ вjь'рнiи пjь'снь но'ву, дне'сь 
похваля'юще вели^чiя _е=я`.

Дре'вле бг~овидjь'нiемъ мwv"се'й w=зари'вся о_у=мо'мъ, 
твоему` jа='снw науча'шеся ч\стая, бг~олjь'пному 
зача'тiю, па'че _е=стества` дв~о, jа='вльшуся _е=му` въ 
купинjь` саду`.

Сла'ва: Тебjь` предлага'ю серд_е'чная дjья^нiя, и= 
прiя'тнw подаю` писа'нiе, бли'зъ су'щiй бж~е'ственный 
заступле'нiя кро'въ, ко вл\дцjь хр\сту` тебе` 
предложи'въ.

И= ны'нjь: Приклони' ми о_у='хо твое` ч\стая, 
правосла'вною вjь'рою, въ сjь'ни лица` твоегw`, че'стнw 
любо'вiю ти` притека'ющему, и= стра'хомъ покланя'ющуся, 
моле'бный мо'й гла'съ о_у=слы'ши.

%[Пjь'снь г~%]

I=рмо'съ: Ты` _е=си` о_у=твержде'нiе притека'ющихъ къ 
тебjь`, гд\си: ты` _е=си` свjь'тъ w=мраче'нныхъ, и= 
пое'тъ тя` ду'хъ мо'й.

Лjь'ствица дре'вле патрiа'рхова тя` проwбража'ше, 
пренепоро'чная: а='гг~льско во jа=вля'ше сни'тiе бж~iе къ 
на'мъ, бж~е'ственное соше'ствiе во о_у=тро'бjь твое'й.

I=у'дово колjь'но всели'ся, ю='же i=а'кwвъ прорече`, 
w\т колjь'на _е=гw` прорасти'ти и=збавле'нiе, i=и~са 
хр\ста`: _е=го'же ты` ро'ждши преч\стая, просла'вилася 
_е=си`.

Сла'ва: Грjьхми` w\тча'янъ, w=брjьто'хъ тя` 
приста'нище сп~се'нiя преч\стая бц\де, о_у=пова'нiе на'ше 
и= по'моще: тjь'мже мя` къ покая'нiю наста'ви.

И= ны'нjь: Бли'зъ тя` су'щу вл\дки и='мамъ, 
пресла'вная вл\дчце, дjья'нiй мои'хъ кни'гу возложи'хъ на 
тя` вjь'рою: не премолчи` о_у=ще'дрити мя`.

%[Пjь'снь д~%]

I=рмо'съ: И=з\ъ горы` прiwсjьне'нныя сло'ве, 
пр\оро'къ, _е=ди'ныя бц\ды, хотя'ща воплоти'тися, 
бг~ови'днw о_у=смотри`, и= со стра'хомъ славосло'вяше 
си'лу твою`.

Ты` jа='кw мони'сты златы'ми w=дjь'яна невjь'ста 
_о='ч~а, бл~года'ть прiе'мши, о_у=кра'шшися добро'тою 
дjь'вства, мт~и jа=ви'лася _е=си` сн~а бж~iя.

Тя` и='стинный сiw'нъ, хр\сто'съ сло'во, и=зво'ли 
себjь` въ бж~е'ственное селе'нiе, jа='кw и=збра'нну 
и=збра'въ бц\де, на w=бновле'нiе всегw` мi'ра.

Сла'ва: Ра'дуйся, кра'сная пала'то сло'ва, 
дjь'вственный черто'же цр~я`: ра'дуйся, похвало` всjь'хъ 
безпло'тныхъ: ра'дуйся, человjь'кwвъ по'моще.

И= ны'нjь: О_у=даля'ются w\т бг~а, тjь'мже и= 
погиба'ютъ, w\тмета'ющiи w='бразы сн~а твоегw`, мт~и 
бж~iя дв~о бц\де: и='миже сп~са'ются чту'щiи тя`.

%[Пjь'снь _е~%]

I=рмо'съ: Мра'къ души` моея` разжени` свjьтода'вче 
хр\сте` бж~е, началоро'дную тьму` и=згна'въ бе'здны: и= 
да'руй ми` свjь'тъ повелjь'нiй твои'хъ сло'ве, да 
о_у='тренюя сла'влю тя`.

Съ бж~е'ственнымъ соше'дшеся гаврiи'ломъ, возопiи'мъ 
бц\дjь вjь'рнw: ра'дуйся дв~о ст~а'я, бл~года'тная, гд\сь 
съ тобо'ю, и='же тебе` ра'ди потреби'въ печа'ль, подаде` 
ра'дованiе.

Преч\стое твое` чре'во гедеw'нъ ви'дjь дв~о ч\стая, въ 
не'же jа='кw до'ждь сло'во соше'дъ, воплоти'ся 
бж~е'ственнымъ дх~омъ, _о='ч~а неразлу'ченъ сы'й 
бж~ества`.

Сла'ва: Помо'щница мi'ра, и= засту'пница _е=си` 
человjь'кwвъ грjь'шныхъ, бг~ороди'тельнице дв~о: и= 
вjь'рою и= любо'вiю прибjьга'ющымъ къ тебjь`, 
премjьне'нiе сп~си'тельное, и= рjьши'ло прегрjьше'нiй 
мно'гихъ.

И= ны'нjь: Прорасти'ла _е=си` без\ъ сjь'мене, и='же 
пре'жде вся'кiя тва'ри, прозябе'нiе сн~а _о='ч~а, 
безлjь'тно же и= безнача'льно, дх~омъ бж~е'ственнымъ, 
бг~ороди'тельнице ч\стая: _е=гw'же подо'бiе ви'да вси` 
почита'емъ.

%[Пjь'снь s~%]

I=рмо'съ: Содержи'ма мя` прiими` чл~вjьколю'бче, 
грjьхи` мно'гими, и= припа'дающа щедро'тамъ твои^мъ, 
jа='кw пр\оро'ка гд\си, и= сп~си' мя.

Дв~ства тя` зерца'ло су'що, и= прiя'телище чи'сто 
бж~ества` восхваля'емъ, дв~о неискусобра'чная, пjь'сньми.

Бг~ъ во о_у=тро'бjь твое'й воплоти'ся, безстра'стнw и= 
о_у=жа'снw бг~оневjь'сто, jа='коже въ сви'тцjь но'вjь 
пи'санъ пе'рстомъ _о='ч~имъ.

Сла'ва: W=чище'нiе и='мамы покро'въ тво'й, и= 
и=звjь'стную наде'жду и= заступле'нiе, дв~о ч\стая: не 
посрами` вл\дчце рабы^ твоя^.

И= ны'нjь: Страсте'й неуста'вное бjьше'нiе, 
предста'нiемъ твои'мъ бг~оневjь'сто, о_у=ста'ви въ 
тишину`: и= ко приста'нищу наста'ви на'съ тишины`.

Та'же: Гд\си поми'луй, три'жды. Сла'ва, и= ны'нjь:

Конда'къ, гла'съ и~:

Взбра'нной воево'дjь побjьди'т_ельная, jа='кw 
и=зба'вльшеся w\т sлы'хъ, бл~года'рств_енная воспису'емъ 
ти` раби` твои` бц\де: но jа='кw и=му'щая держа'ву 
непобjьди'мую, w\т вся'кихъ на'съ бjь'дъ свободи`, да 
зове'мъ ти`: ра'дуйся невjь'сто неневjь'стная.

%[Пjь'снь з~%]

I=рмо'съ: Бж~iя снизхожде'нiя _о='гнь о_у=стыдjь'ся въ 
вавv"лw'нjь и=ногда`, сегw` ра'ди _о='троцы въ пещи` 
ра'дованною ного'ю, jа='кw во цвjь'тницjь лику'юще, 
поя'ху: бл~гослове'нъ _е=си` бж~е _о=т_е'цъ на'шихъ.

Ра'дости на'шея хода'таица jа=ви'лася _е=си` дв~о, и= 
бл~года'ти принося'ще вjьне'цъ любо'вiю, ра'дуйся, 
вопiе'мъ ти`, бл~гослове'нная ч\стая, похваля'юще.

Гора` ст~а'я _е=си` бж~iя прiwсjьне'нная, гора` ту'чна 
пренепоро'чная: гора` о_у=сыре'нна бж~е'ственными 
сiя'нiи: гора`, въ не'йже бг~ъ бл~говоли` жи'ти.

Сла'ва: Побjьжда'яй бл~года'ть твою` нjь'сть грjь'хъ, 
мт~рне бо дерзнове'нiе и= во'лю и='маши, и= рjьши'ши 
прегрjьш_е'нiя мл~твами твои'ми, и= прево'диши вся^ 
стремл_е'нiя.

И= ны'нjь: W\т тр\оцы родила` _е=си` _е=ди'наго, 
бц\де, бы'вша непрело'жна плотски'мъ соедине'нiемъ, 
сугу'ба су'ща _е=стество'мъ: _е=го'же ви'да w='бразъ 
почита'емъ.

%[Пjь'снь и~%]

I=рмо'съ: На горjь` ст~jь'й просла'вльшася, и= въ 
купинjь` _о=гне'мъ приснодв~ы мwv"се'ови та'йну 
jа='вльшаго, гд\са по'йте, и= превозноси'те во вся^ 
вjь'ки.

Кади'льница jа=ви'лася _е=си` пр\оро'ку, 
бж~е'ственнагw о_у='гля су'щи, грjьхи` w\те'млющагw, 
бц\де дв~о мт~и бг~а на'шегw.

Данiи'лъ прови'дjь тя` го'ру ве'лiю, бц\де дв~о: и=з\ъ 
нея'же честны'й ка'мень хр\сто'съ, пло'тiю w=блече'ся, и= 
ле'сти низложи` i='дwльскiя хра'мы.

Сла'ва: Вели'кiй ки'тъ и='щетъ пожре'ти мя`, лю'тагw 
грjьха` и= страсте'й мои'хъ w\тча'янiя: но предвари` и= 
сп~си` раба` твоего` вл\дчце.

И= ны'нjь: И='же тобо'ю бесjь'довавый къ человjь'кwмъ, 
бг~ъ сы'й вся'ческихъ, зра'къ человjь'ка воспрiя'тъ: 
_е=гw'же взо'ръ почита'емъ дв~о, въ писа'нiихъ.

%[Пjь'снь f~%]

I=рмо'съ: Вои'стинну бц\ду тя` и=сповjь'дуемъ 
сп~се'ннiи тобо'ю дв~о ч\стая, съ безпло'тными ли'ки тя` 
велича'юще.

Вертогра'дъ затворе'нъ тя` дв~о бц\де, и= запеча'танъ 
и=сто'чникъ дх~омъ бж~е'ственнымъ, прему'дрый въ 
пjь'снехъ пое'тъ: тjь'мже jа='кw са'дъ жи'зни, 
воплоща'ется хр\сто'съ.

Твоегw` неска'заннагw рж\ства` пропису'я пр\оро'къ, 
кни'гу запеча'тану прови'дjь, _е=я'же никто'же та'инство 
разумjь`, вочеловjь'ченiя рж\ства` твоегw`.

Сла'ва: Тебjь` во о_у=миле'нiи души` мо'лимся вси`: не 
пре'зри вл\дчце на'шя мольбы^, но бу'ди бл~гоувjь'тливъ 
на'мъ покро'въ, и= мл~тву на'шу о_у=слы'ши.

И= ны'нjь: Твоему` и= сн~а твоегw` припа'даю 
w=бразw'мъ: и= сомня'щихся почита'ти, jа='кw ма'нентовы 
лжы` w\тмета'ю, бц\де дв~о: тjь'мже правосла'внw пjь'снь 
скончава'ю.

Та'же, досто'йно _е='сть: Трист~о'е. И= по _О='ч~е 
на'шъ: конда'къ, и= про'чее по _о=бы'чаю, и= w\тпу'стъ.
