Въ недjь'лю о_у='тра на полу'нощницjь,

канw'нъ прест~jь'й и= живонача'льнjьй тр\оцjь, 
[_е=го'же краестро'чiе: Хвалю` тр\оце тя`, 
_е=динонача'льное _е=стество`.] Творе'нiе митрофа'ново. 
Гла'съ з~:

\cslemph{Пjь'снь а~}

I=рмо'съ: Ма'нiемъ твои'мъ на земны'й w='бразъ 
преложи'ся, пре'жде о_у=доборазлива'емое водно'е 
_е=стество` гд\си: тjь'мже немо'креннw пjьшеше'ствовавъ 
i=и~ль, пое'тъ тебjь` пjь'снь побjь'дную.

Припjь'въ: Прест~а'я тр\оце бж~е на'шъ, сла'ва тебjь`.

W\тве'рзи ми` о_у=ста` о_у='мная се'рдца, и= 
о_у=стнjь` веще'ств_енныя ко хвалjь` твое'й, трисвjь'тлое 
_е=ди'но бж~ество` всjь'хъ, пjь'ти тебjь` пjь'снь 
свjьтодjь'телю, бл~года'рственную.

Да преизли'шнее твоея` бл~гости пока'жеши, создала` 
_е=си` человjь'ка тр\оце безмjьрноси'льная: то'кмw 
w='бразъ бре'нный твоегw` нача'льнагw содjь'телю, 
гд\сьства.

О_у='мъ безнача'льный, сло'во соприсносу'щное 
роди'вый, и= дх~а собезнача'льнаго просiя'вый, сподо'би 
_е=ди'ному бг~у по существу` соwбра'зну на'съ 
покланя'тися трiv"поста'сному.

Бг~оро'диченъ: Jа=ви'лся _е=си` при купинjь` мwv"се'ю 
бж~iй сло'ве, jа='кw _о='гнь чисти'тельный, не w=паля'я 
же w\тню'дъ, _е='же w\т дв~ы проwбразу'я твое` 
воплоще'нiе, и='мже человjь'ки воwбрази'лъ _е=си`.

\cslemph{Пjь'снь г~}

I=рмо'съ: Въ нача'лjь нб~са`, всеси'льнымъ сло'вомъ 
твои'мъ о_у=твержде'й гд\си сп~се, и= вседjь'тельнымъ и= 
бж~iимъ дх~омъ всю` си'лу и='хъ: на недви'жимjьмъ мя` 
ка'мени и=сповjь'данiя твоегw` о_у=тверди`.

И='же _е=ди'нственнаго и= трисiя'ннаго, и= 
вседjь'тельнаго тя` вл\дку воспjьва'юще, грjьхw'въ и= 
и=скуше'нiй про'симъ и=збавле'нiя безмjьрноси'льный бж~е: 
да не о_у='бw пре'зриши, вjь'рою твою` бл~гость 
сла'вящихъ.

_О='трасль jа=ви'ся w\т _о=ц~а`, jа='кw w\т ко'рене 
безнача'льна бг~ъ сло'во, и= равномо'щенъ со сра'сленымъ 
и= бж~е'ственнымъ дх~омъ: и= сегw` ра'ди вjь'рнiи, 
тр\очное ли'цы _е=ди'но гд\сьство сла'вимъ.

Равносла'вно и= сра'слено, трiv"поста'сное _е=стество` 
нераздjь'льнjь и= раздjь'льнjь, и= _е=динонача'льную 
тр\оцу вси' тя славосло'вимъ вjь'рнiи: и= покланя'ющеся 
про'симъ согрjьше'нiй проще'нiя.

Бг~оро'диченъ: Неизмjь'ннw человjь'кwмъ о_у=подо'блься 
по всему` бж~iй сло'ве, w\т ч\стыя _о=трокови'цы проше'лъ 
_е=си` jа='вjь: и= всjь^мъ показа'лъ _е=си` бг~онача'лiе 
трисiя'нное, и= _е=ди'нственное существо'мъ неизмjь'нныхъ 
v=поста'сей.

Гд\си поми'луй, три'жды.

Сjьда'ленъ, гла'съ з~. Подо'бенъ: И='же мене` ра'ди:

Согрjьши'вшыя поми'луй тр\оце ст~а'я, рабы^ твоя^: и= 
прiими` ка'ющыяся тебjь` бл~гоутро'бне, и= проще'нiя 
сподо'би.

Сла'ва, и= ны'нjь, бг~оро'диченъ: W=sло'бл_енныя ду'шы 
на'шя грjьхми`, о_у=бл~жи` всеч\стая бц\де, и= и=зба'ви 
w\т прегрjьше'нiй тебе` пою'щыя бг~оневjь'сто.

\cslemph{Пjь'снь д~}

I=рмо'съ: _О='ч~а нjь'дра не w=ста'вль, и= соше'дъ на 
зе'млю хр\сте` бж~е, та'йну о_у=слы'шахъ смотре'нiя 
твоегw`, и= просла'вихъ тя` _е=ди'не чл~вjьколю'бче.

Содержи'тельная трисвjь'тлая _е=ди'нице, 
бг~онача'льная и= сп~си'тельная всjь'хъ, твоя^ пjьвцы` 
ны'нjь w=гради`, и= сп~си` w\т ско'рби, и= страсте'й, и= 
вся'кагw w=sлобле'нiя.

Назна'менательныхъ рече'нiй, твоегw` непостижи'магw 
трисвjь'тлагw бж~ества`, недоумjьва'юще воспjьва'емъ тя` 
чл~вjьколю'бче гд\си, и= прославля'емъ твою` си'лу.

На земли` jа='кw на нб~сjь'хъ со безпло'тными ли'ки, 
_е=ди'нице и= тр\оце, нераздjь'льнw тя` раздjьля'емъ, и= 
любо'вiю сла'вимъ, jа='кw су'щими всjь'ми 
влады'чествующую.

Бг~оро'диченъ: _О=ч~скiя сла'вы не w\тсту'пль, къ 
на'шей ху'дости снизше'лъ _е=си` во'лею, вопло'щься 
пресу'щественный: и= возне'слъ _е=си` къ бж~е'ственнjьй 
сла'вjь, jа='кw бл~гоутро'бенъ.

\cslemph{Пjь'снь _е~}

I=рмо'съ: Но'щь несвjьтла` невjь^рнымъ хр\сте`, 
вjь^рнымъ же просвjьще'нiе въ сла'дости слове'съ твои'хъ: 
сегw` ра'ди къ тебjь` о_у='тренюю, и= воспjьва'ю твое` 
бж~ество`.

Свjьтонача'льное _е=стество` тр\оце начерта'ньми, и= 
_е=ди'нице въ совjь'тjь и= сла'вjь и= че'сти, о_у=тверди` 
на'съ въ твое'й любви`.

О_у='мъ, и= сло'во, и= дх~ъ, _е=ди'но бг~онача'льное 
и= трисо'лнечное _е=стество` сла'вяще, про'симъ 
и=зба'витися и=скуше'нiй, и= вся'кихъ скорбе'й.

Бг~оро'диченъ: Воwбра'жься бж~iй сло'ве, въ 
_е=стество` человjь'ческое, w\т ст~ы'я дв~ы, тр\оцу во 
_е=ди'ницjь научи'лъ _е=си` пjь'ти, соwбра'зну и= 
сопресто'льну.

\cslemph{Пjь'снь s~}

I=рмо'съ: Пла'вающаго въ молвjь` жите'йскихъ 
попече'нiй, съ корабле'мъ потопля'ема грjьхи`, и= 
душетлjь'нному sвjь'рю примета'ема, jа='кw i=w'на хр\сте` 
вопiю' ти: и=з\ъ смертонw'сныя глубины` возведи' мя.

_Е=диносла'вную держа'ву, и= _е=динонача'льное 
бг~онача'лiе, треми` v=поста'сьми су'щее, непремjь'ннjь 
съ дру'гъ дру'гомъ сла'вимъ, разли'чiе и=му'ще то'кмw, 
бытiе'мъ ко'йждо сво'йства.

О_у='мнiи тя` чи'нове а='гг~льскихъ красо'тъ хва'лятъ, 
трисо'лнечное бг~онача'лiе: съ ни'миже и= мы` бре'нными 
о_у=сты`, jа='кw твори'тельну всjь'хъ, _е=ди'ну 
пjьсносло'вимъ, и= вjь'рнw сла'вимъ.

Бг~оро'диченъ: Недоумjь'ннымъ сло'вомъ, сло'во 
ро'ждшееся w\т сл~нца _о=ц~а`, и='но сл~нце, пре'жде 
вjь^къ, w\т дв~ы напослjь'докъ возсiя`: и= 
_е=ди'нственнаго треми` ли'цы, недомы'слимаго бг~а 
проповjь'да.

Сjьда'ленъ, гла'съ з~:

Подо'бенъ: _О=гня` свjьтлjь'йши:

Тр\оце _е=диносу'щная, _е=ди'нице всjь'хъ 
трiv"поста'сная, поми'луй, jа=`же создала` _е=си`, 
безсме'ртная: попаля'ющи sлw'бы прегрjьша'ющихъ, и= 
просвjьща'ющи сердца` пою'щихъ бл~гоутро'бiе твое`, бж~е 
на'шъ, сла'ва тебjь`.

Сла'ва, и= ны'нjь, бг~оро'диченъ: _О=гня` 
свjьтлjь'йши, свjь'та дjь'йственнjьйши мл\сть бл~года'ти 
твоея` вл\дчце, попаля'ющи грjьхи` человjь'кwвъ, и= 
w=роша'ющи мы^сли хва'лящихъ вели^чiя твоя^, бц\де 
пренепоро'чная.

\cslemph{Пjь'снь з~}

I=рмо'съ: Въ пе'щь _о='гненную вве'ржени прп\дбнiи 
_о='троцы, _о='гнь въ ро'су преложи'ша, воспjьва'нiемъ 
си'це вопiю'ще: бл~гослове'нъ _е=си` гд\си бж~е _о=т_е'цъ 
на'шихъ.

Свjьтолу'чными блиста'нiи при'снw сiя'я, трисвjь'те 
бж~е, _е=ди'нице непристу'пная и= пресу'щная сп~си`, 
и=`же въ тя` вл\дко, вjь'рующыя бл~гоче'стнw, и= 
покланя'ющыяся тебjь`.

Рече'нi_емъ бж~е'ственныхъ пр\орw'къ повину'ющеся, 
_е=ди'наго тя` то'чiю бг~а всjь'хъ сла'вимъ въ трiе'хъ 
начерта'нiихъ, си'це вопiю'ще: бл~гослове'нъ _е=си` гд\си 
бж~е _о=т_е'цъ на'шихъ.

Пе'рстными о_у=стна'ми мы` съ невеще'ственными чи'нми, 
пjь'сньми тр\оце ст~а'я, тя` пое'мъ во _е=ди'нствjь 
существа`, вопiю'ще: бл~гослове'нъ _е=си` гд\си бж~е 
_о=т_е'цъ на'шихъ.

Бг~оро'диченъ: Да и='же созда'вый а=да'ма, 
возсози'ждетъ па'ки, всеч\стая, и=з\ъ тебе` jа='вjь 
вочеловjь'чися, человjь'ки w=божи'въ си'це вопiю'щыя: 
бл~гослове'нъ преч\стая, пло'дъ твоегw` чре'ва.

\cslemph{Пjь'снь и~}

I=рмо'съ: Неwпа'льная _о=гню` въ сiна'и прича'щшаяся 
купина`, бг~а jа=ви` медленоязы'чному и= гугни'вому 
мwv"се'ови, и= _о='троки ре'вность бж~iя три` 
непребwри'мыя во _о=гни` пjьвцы` показа`: вся^ дjьла` 
гд\сня гд\са по'йте, и= превозноси'те во вся^ вjь'ки.

Со'лнца луча'ми трисiя'ннагw w=зари'тися 
свjьтодjь'тельными сподо'би, сердца'мъ пjьвц_е'въ 
твои'хъ: и= ны'нjь зрjь'ти добро'ту твою` тр\оце, 
_е=ди'нице, jа='кw мо'щно да'руй всегда`, всjь^мъ вjь'рою 
подо'бною, твое` вели'чiе пjьсносло'вящымъ во вся^ 
вjь'ки.

Держи'ши вся'ч_еская, тр\очное и= _е=ди'нственное 
гд\сонача'лiе безнача'льное, и= о_у=правля'еши нб~о и= 
зе'млю. тjь'мже мя` любо'вiю твое'ю привлачи'ма при'снw, 
сохрани` пjь'ти тебjь`: вся^ дjьла` гд\сня гд\са по'йте, 
и= превозноси'те во вся^ вjь'ки.

Хра'мъ мя` твоея` трисвjь'тныя сотвори` зари`, 
бл~годjь'телю чл~вjьколю'бче: и= прича'стiя и= 
прiwбще'нiя непристу'пна врагw'мъ неви^димымъ, и= 
плотски^мъ страсте'мъ вл\дко, покажи`, _е=динонача'льный 
бж~е мо'й и= гд\си сла'вы, пjьсносло'вити тя` во вся^ 
вjь'ки.

Бг~оро'диченъ: Свjь'тъ бг~онача'льный w\т чре'ва 
твоегw` возсiя'вый, бг~ома'ти преч\стая, ве'сь мi'ръ 
трисо'лнечнымъ свjь'томъ w=зари`, и= зе'млю jа='коже 
друго'е нб~о показа`, пою'щу: вся^ дjьла` гд\сня гд\са 
по'йте, и= превозноси'те во вся^ вjь'ки.

\cslemph{Пjь'снь f~}

I=рмо'съ: Мт~и бж~iя и= дв~а, ро'ждшая и= 
дjь'вствующая па'ки, не _е=стества` дjь'ло, но бж~iя 
снизхожде'нiя: тjь'мъ jа='кw _е=ди'ну бж~iихъ чуде'съ 
сподо'бльшуюся, тя` при'снw велича'емъ.

Высокосло'вити и= пjь'ти тя` досто'йнw, _е=го'же въ 
вы'шнихъ непреста'ннw серафi'ми воспjьва'ютъ, не 
возмога'емъ бре'ннiи: _о=ба'че jа='кw вл\дку всjь'хъ 
дерза'юще, и= чл~вjьколю'бнjьйшаго бг~а велича'емъ.

Тjьле'сныя болjь'зни и=зба'ви, и= душе'внагw 
пристра'стiя пjьвцы` твоя^, _е=ди'нственная тр\оце 
нераздjь'льная, и= сохрани'тися неврежд_е'ннымъ w\т 
всjь'хъ жите'йскихъ и=скуше'нiй сподо'би.

Равноси'льная, бг~онача'льная, трисвjь'тная, 
всеси'льная держа'во, неизмjь'нная добро'то су'щественныя 
бл~гости, да'ждь прегрjьше'нiй разрjьше'нiе твои^мъ 
рабw'мъ, и= и=скуше'нiй, и= страсте'й и=зба'ви.

Бг~оро'диченъ: О_у='мъ и= ду'шу, и= плотско'е 
смjьше'нiе прiе'мъ бц\де, w\т твои'хъ ложе'снъ 
преч\стыхъ, бг~ъ сло'во пои'стиннjь человjь'къ jа=ви'ся: 
и= бж~е'ственнагw существа` _о='бщника человjь'ка jа='вjь 
показа'лъ _е='сть.

Посе'мъ, припjь'вы григо'рiа сiнаи'та: Досто'йно 
_е='сть: И= про'чее полу'нощницы, пи'сано въ концjь` 
кни'ги сея`.
