\cslemph{Въ недjь'лю на о_у='трени},

по _е=_кса` _псалмjь'хъ, Бг~ъ гд\сь, и= jа=ви'ся: на 
гла'съ s~, и= глаго'лемъ тропа'рь: А='гг~льскiя си^лы на 
гро'бjь твое'мъ: [Два'жды.] Сла'ва, и= ны'нjь: 
Бл~гослове'нную нарекi'й твою` мт~рь: И= _о=бы'чное 
стiхосло'вiе _псалти'ра.

По а~-мъ стiхосло'вiи, сjьда'льны воскр\сны, гла'съ 
s~: 

Гро'бу w\тве'рсту, а='ду пла'чущуся, мр~i'а вопiя'ше 
ко скры'вшымся а=п\слwмъ: и=зыди'те вiногра'да 
дjь'лателiе, проповjь'дите воскр\снiя сло'во: воскр~се 
гд\сь, подая` мi'рови ве'лiю мл\сть.

Стi'хъ: Воскр\сни` гд\си бж~е мо'й, да вознесе'тся 
рука` твоя`, не забу'ди о_у=бо'гихъ твои'хъ до конца`.

Гд\си, предстоя'ше гро'бу твоему` мр~i'а магдали'на, 
и= пла'каше вопiю'щи, вертогра'даря тя` мня'щи, 
глаго'лаше: гдjь` сокры'лъ _е=си` вjь'чный живо'тъ; гдjь` 
положи'лъ _е=си`, и='же на пр\сто'лjь @херувi'мстjьмъ@ 
{херувi'мстjьмъ сjьдя'щаго}; стрегу'щiи бо сего`, w\т 
стра'ха w=мертвjь'ша. и=ли` гд\са моего` дади'те ми`, 
и=ли` со мно'ю возопi'йте: и='же въ ме'ртвыхъ бы'въ, и= 
м_е'ртвыя воскр~си'въ, сла'ва тебjь`.

Сла'ва, и= ны'нjь, бг~оро'диченъ: Преднапису'етъ 
гедеw'нъ зача'тiе, и= сказу'етъ дв~дъ рж\ство` твое` 
бц\де: сни'де бо jа='кw до'ждь на руно`, сло'во во чре'во 
твое`, и= прозябла` _е=си` без\ъ сjь'мене, земле` ст~а'я, 
мi'рови сп~се'нiе, хр\ста` бг~а на'шего, бл~года'тная.

По в~-мъ стiхосло'вiи сjьда'льны воскр\сны, гла'съ s~: 

Живо'тъ во гро'бjь возлежа'ше, и= печа'ть на ка'мени 
надлежа'ше: jа='кw цр~я` спя'ща во'ини стрежа'ху хр\ста`, 
и= враги` своя^ неви'димw порази'вый, воскр~се гд\сь.

Стi'хъ: И=сповjь'мся тебjь` гд\си, всjь'мъ се'рдцемъ 
мои'мъ, повjь'мъ вся^ чудеса` твоя^.

Преднапису'етъ i=w'на гро'бъ тво'й, и= сказу'етъ 
сv"меw'нъ воста'нiе бж~е'ственное, безсме'ртне гд\си: 
соше'лъ бо _е=си` jа='кw ме'ртвъ во гро'бъ, разруши'вый 
а='дwва врата`. воскр\слъ же _е=си` кромjь` тлjь'нiя 
jа='кw вл\дка, во сп~се'нiе мi'ра, хр\сте` бж~е на'шъ, 
просвjьти'вый су'щыя во тьмjь`.

Сла'ва, и= ны'нjь, бг~оро'диченъ: Бц\де дв~о, моли` 
сн~а твоего`, во'лею пригвожде'ннаго на кр\стjь`, и= 
воскр\сшаго и=з\ъ ме'ртвыхъ хр\ста` бг~а на'шего, 
сп~сти'ся душа'мъ на'шымъ.

По непоро'чныхъ, v=пакои`, гла'съ s~:

Во'льною и= животворя'щею твое'ю сме'ртiю хр\сте`, 
врата` а='дwва сокруши'въ jа='кw бг~ъ, w\тве'рзлъ _е=си` 
на'мъ дре'внiй ра'й, и= воскр~съ и=з\ъ ме'ртвыхъ, 
и=зба'вилъ _е=си` w\т тлjь'нiя живо'тъ на'шъ. 

Та'же степ_е'нна, гла'съ s~. \cslemph{А=нтiфw'нъ} а~, и='хже 
повторя'юще пое'мъ.

На нб~о _о='чи мои` возвожу`, къ тебjь` сло'ве: 
о_у=ще'дри мя`, да живу` тебjь`.

Поми'луй на'съ о_у=ничиже'нныхъ, о_у=строя'я 
бл~гопотр_е'бныя твоя^ сосу'ды сло'ве.

Сла'ва: Ст~о'му дх~у, вся'кая всесп~си'тельная вина`, 
а='ще ко'ему се'й по достоя'нiю дхне'тъ, ско'рw взе'млетъ 
w\т земны'хъ: восперя'етъ, возраща'етъ, о_у=строя'етъ 
горjь`. 

И= ны'нjь, то'йже. 

\cslemph{А=нтiфw'нъ} в~:

А='ще не гд\сь бы бы'лъ въ на'съ, никто'же w\т на'съ 
проти'ву возмо'глъ бы вра^жiимъ бра'немъ w=долjь'ти: 
побjьжда'ющiи бо w\т здjь` возно'сятся.

Зубы` и='хъ да не jа='та бу'детъ душа` моя` jа='кw 
птене'цъ, сло'ве: о_у=вы` мнjь`, ка'кw и='мамъ w\т врага` 
и=збы'ти грjьхолюби'въ сы'й!

Сла'ва: Ст~ы'мъ дх~омъ w=боже'нiе всjь^мъ, 
бл~говоле'нiе, ра'зумъ, ми'ръ и= бл~гослове'нiе: 
равнодjь'теленъ бо _е='сть _о=ц~у` и= сло'ву. 

И= ны'нjь, то'йже. 

\cslemph{А=нтiфw'нъ} г~:

Надjь'ющiися на гд\са врагw'мъ стра'шни, и= всjь^мъ 
ди'вни: горjь' бо зря'тъ.

Въ беззакw'нiя ру'къ свои'хъ, пра'ведныхъ жре'бiй, 
помо'щника тя` и=мjь'я сп~се, не простира'етъ.

Сла'ва: Ст~а'гw дх~а держа'ва на всjь'хъ: _е=му'же 
вы^шняя вw'инства покланя'ются, со вся'кимъ дыха'нiемъ 
до'льнымъ.

И= ны'нjь, то'йже: Прокi'менъ гла'съ s~: Гд\си, 
воздви'гни си'лу твою`, и= прiиди` во _е='же сп~сти` 
на'съ. Стi'хъ: Пасы'й i=и~ля вонми`, наставля'яй jа='кw 
_о=вча` i=w'сифа. Вся'кое дыха'нiе: Стi'хъ: Хвали'те бг~а 
во ст~ы'хъ _е=гw`: _Е=v\глiе воскр\сно: Воскр\снiе 
хр\сто'во: _Псало'мъ пятидеся'тый. И= прw'чая по ря'ду.

Канw'нъ воскр\снъ. Гла'съ s~.

\cslemph{Пjь'снь а~}

I=рмо'съ: Jа='кw по су'ху пjьшеше'ствовавъ i=и~ль, по 
бе'зднjь стопа'ми, гони'теля фараw'на ви'дя потопля'ема, 
бг~у побjь'дную пjь'снь пои'мъ, вопiя'ше.

Припjь'въ: Сла'ва гд\си, ст~о'му воскр\снiю твоему`. 

Распросте'ртыма дла'ньма на кр\стjь`, _о=те'ческагw 
и=спо'лнилъ _е=си` бл~говоле'нiя, бл~гi'й i=и~се, 
вся'ч_еская. тjь'мже побjь'дную пjь'снь тебjь` вси` 
пои'мъ. 

Стра'хомъ къ тебjь` jа='кw рабы'ня, сме'рть повелjь'на 
приступи` вл\дцjь живота`, то'ю подаю'щему на'мъ 
безконе'чный живо'тъ и= воскр\снiе.

Бг~оро'диченъ: Своего` прiе'мши содjь'теля, jа='кw 
са'мъ восхотjь`, w\т безсjь'меннагw твоегw` чре'ва, па'че 
о_у=ма` воплоща'ема, ч\стая, тва'рей вои'стинну 
jа=ви'лася _е=си` вл\дчца.

И='нъ канw'нъ, кр\стовоскр\снъ.

\cslemph{Пjь'снь а~} I=рмо'съ: Волно'ю морско'ю:

Суди'лищу пiла'тову предстои'тъ хотя` беззако'нному 
суду`, jа='кw суди'мь судiя`, и= w\т руки` непра'вды по 
лицу` зауша'ется бг~ъ, _е=го'же трепе'щутъ земля`, и= 
нб\сная.

Просте'рлъ _е=си` бж~е'ственнjьи дла^ни твои` сп~се, 
на преч\стjьмъ твое'мъ и= живоно'снjьмъ кр\стjь`: и= 
собра'лъ _е=си` jа=зы'ки въ позна'нiе твое` вл\дко, 
покланя'ющыяся гд\си, сла'вному твоему` распя'тiю.

Кр\стобг~оро'диченъ: Стоя'ше сле'зъ и=сто'чники 
и=спуща'ющи, пренепоро'чная при кр\стjь` твое'мъ сп~се, 
jа=`же w\т ребра` твоегw` ка^пли крове'й зря'щи хр\сте`, 
и= твою` безприкла'дную мл\сть сла'вящи.

И='нъ канw'нъ прест~jь'й бц\дjь, [_е=гw'же 
краестро'чiе: Мт~и бж~iя, незави'стную ми` да'ждь 
бл~года'ть.]

I=рмо'съ: Jа='кw по су'ху пjьшеше'ствовавъ i=и~ль:

Прича'щшися _е='vа @саду`, преслуша'ннагw бра'шна@ 
{_е='же са'да преслуша'ннагw бра'шна}, кля'тву введе`: но 
сiю` разрjь'шила _е=си` ч\стая, бл~гослове'нiя нача'токъ 
хр\ста` ро'ждши.

Jа='же би'сера w\т бж~е'ственныя мо'лнiи хр\ста` 
ро'ждши, страсте'й мои'хъ мглу`, и= прегрjьше'нiй 
смуще'нiе ч\стая, разжени` свjь'томъ твоея` свjь'тлости.

Jа=зы'кwвъ ча'янiе, i=а'кwвъ и=з\ъ тебе` воплоща'емаго 
предзря'ше та'йнw о_у='мныма _о=чи'ма бг~а, и=зба'вльшаго 
на'съ хода'тайствомъ твои'мъ. 

W=скудjь'вшымъ княз_е'мъ w\т колjь'на i=у'дова 
преч\стая, сн~ъ тво'й и= бг~ъ, проше'дъ во'ждь, над\ъ 
концы` земны'ми ны'нjь вои'стинну воцр~и'ся. 

Катава'сiа: W\тве'рзу о_у=ста` моя^:

\cslemph{Пjь'снь г~}

I=рмо'съ: Нjь'сть ст~ъ, jа='коже ты` гд\си бж~е мо'й, 
вознесы'й ро'гъ вjь'рныхъ твои'хъ бл~же, и= 
о_у=тверди'вый на'съ на ка'мени и=сповjь'данiя твоегw`.

Бг~а распина'ема пло'тiю зря'щи тва'рь растаява'шеся 
стра'хомъ: но содержи'тельною дла'нiю на'съ ра'ди 
распя'тагw, крjь'пкw держи'ма бjь`.

Сме'ртiю сме'рть разоре'на лежи'тъ _о=кая'нная без\ъ 
дыха'нiя: живота' бо не терпя'щи бж~е'ственнагw 
прираже'нiя, о_у=мерщвля'ется @крjь'пкiй@{крjь'пкая}, и= 
да'руется всjь^мъ воскр\снiе.

Бг~оро'диченъ: Бж~е'ственнагw рж\ства` твоегw` ч\стая, 
вся'кiй _е=стества` чи'нъ превосхо'дитъ чу'до: бг~а бо 
преесте'ственнjь зачала` _е=си` во чре'вjь, и= ро'ждши 
пребыва'еши при'снw дв~а.

\cslemph{И='нъ} I=рмо'съ: Тебе` на вода'хъ:

Во гро'бjь тридне'вствовавый воскр~си'лъ _е=си` 
животворя'щимъ воста'нiемъ твои'мъ пре'жде 
о_у=мерщвл_е'нныя, и= w=сужде'нiя разрjьши'вшеся 
ра'достнw веселя'хуся, се` и=збавле'нiе прише'лъ _е=си` 
гд\си, взыва'юще.

Сла'ва твоему` воста'нiю сп~се на'шъ, jа='кw на'съ w\т 
а='да тлjь'нiя, и= сме'рти и=зба'вилъ _е=си` jа='кw 
всеси'ленъ, и= пою'ще глаго'лемъ: нjь'сть ст~ъ ра'звjь 
тебе` гд\си чл~вjьколю'бче.

Бг~оро'диченъ: Ты` w\т тебе` ро'ждшагося jа='кw 
ви'дjьла _е=си` о_у=я'звлена копiе'мъ, о_у=язви'лася 
_е=си` се'рдцемъ прест~а'я всенепоро'чная, и= 
о_у=жаса'ющися глаго'лала _е=си`: что` тебjь` воздаде`, 
ча'до, наро'дъ пребеззако'нный; 

\cslemph{И='нъ} I=рмо'съ: Нjь'сть ст~ъ, jа='коже ты`:

Тлjь'нную мою` пло'ть и= сме'ртную, всеч\стая 
бг~ома'ти, и=з\ъ чре'ва твоегw` несказа'ннw прiе'мъ 
бл~гi'й, и= w=безтлjь'нивъ сiю`, вjь'чнjь себjь` связа'лъ 
_е='сть. 

Бг~а воплоща'ема и=з\ъ тебе` зря'ще дв~о, 
о_у=жаса'хуся стра'хомъ ли'цы а='гг~льстiи, и= jа='кw 
мт~рь бж~iю немо'лчными пjь'сньми тя` почита'ютъ.

О_у=жасе'ся, го'ру о_у='мную ви'дjьвъ тя` пр\оро'къ 
данiи'лъ, и=з\ъ нея'же ка'мень w\тсjьче'ся кромjь` ру'къ: 
и= де'мwнская ка^пища, бг~омт~и, крjь'пкw сокруши`.

Не мо'жетъ сло'во тя` чл~вjь'ческое, ниже` я=зы'къ 
дв~о похвали'ти досто'йнw: и=з\ъ тебе' бо без\ъ сjь'мене 
жизнода'вецъ хр\сто'съ воплоти'тися, преч\стая, 
бл~говоли`.

\cslemph{Пjь'снь д~}

I=рмо'съ: Хр\сто'съ моя` си'ла, бг~ъ и= гд\сь, 
ч\стна'я цр~ковь бг~олjь'пнw пое'тъ взыва'ющи, w\т 
смы'сла чи'ста w= гд\сjь пра'зднующи.

Дре'во процвjьло` _е='сть хр\сте`, и='стинныя жи'зни: 
кр\стъ бо водрузи'ся, и= напое'нъ бы'въ кро'вiю и= водо'ю 
w\т нетлjь'ннагw твоегw` ребра`, живо'тъ на'мъ прозябе`.

Не ктому` sмi'й мнjь` ло'жнjь w=боже'нiе подлага'етъ: 
хр\сто'съ бо бг~одjь'латель чл~вjь'ческагw _е=стества`, 
ны'нjь невозбра'ннw стезю` живота` мнjь` w\тве'рзе.

Бг~оро'диченъ: Jа='кw вои'стинну неизвjьща^нна и= 
непостижи^ма, jа=`же твоегw` бг~олjь'пнагw бц\де су'ть 
рж\ства`, су'щымъ на земли` и= на нб~си`, приснодв~о, 
та^инства.

\cslemph{И='нъ} I=рмо'съ: На кр\стjь` твое` бж~е'ственное 
и=стоща'нiе: 

Ч\стны'й кр\стъ тво'й почита'емъ, и= гво'зди хр\сте`, 
и= ст~о'е копiе` съ тро'стiю, вjьне'цъ и='же w\т те'рнiй, 
и='миже w\т а='дова и=стлjь'нiя и=зба'вихомся.

Гро'бъ сп~се тя` под\ъя'тъ во'лею ме'ртва w= на'съ 
jа='вльшагося, но ника'коже возмо'же сло'ве, 
о_у=держа'ти: jа='кw бг~ъ бо воскр\слъ _е=си`, сп~са'я 
ро'дъ на'шъ.

Кр\стобг~оро'диченъ: Бг~ороди'тельнице приснодв~о, 
сп~са хр\ста` человjь'кwмъ ро'ждшая, w\т бjь'дъ и= му'къ 
и=зба'ви на'съ, прибjьга'ющихъ вjь'рою къ бж~е'ственному 
покро'ву твоему`. 

\cslemph{И='нъ} I=рмо'съ: Хр\сто'съ моя` си'ла: 

Пое'мъ преч\стая тебе` всенепоро'чную, и=`же тобо'ю 
сп~сшiися, и= бл~гоче'стнw пою'ще взыва'емъ: 
бл~гослове'нна, jа='же бг~а приснодв~о ро'ждшая.

Свjь'тъ незаходи'мый дв~о родила` _е=си`, су'щымъ во 
тьмjь` житiя`, пло'тiю свjьтя'щъ всебл~же'нная, и= 
пою'щымъ тя`, ра'дость приснодв~о и=сточи'ла _е=си`.

Бл~года'ть процвjьте`, зако'нъ преста`, тобо'ю 
всест~а'я: ты' бо ч\стая, родила` _е=си` гд\са, 
подаю'щаго на'мъ приснодв~о, w=ставле'нiе.

Ме'ртва мя` показа` сада` вкуше'нiе, жи'зни же дре'во 
и=з\ъ тебе` jа='вльшееся преч\стая, воскр~си`, и= 
ра'йскiя сла'дости наслjь'дника мя` о_у=стро'и.

\cslemph{Пjь'снь _е~}

I=рмо'съ: Бж~iимъ свjь'томъ твои'мъ бл~же, 
о_у='тренюющихъ ти` ду'шы любо'вiю w=зари`, молю'ся, тя` 
вjь'дjьти сло'ве бж~iй, и='стиннагw бг~а, w\т мра'ка 
грjьхо'внагw взыва'юща.

О_у=ступа'ютъ мнjь` херувi'ми ны'нjь, и= пла'менное 
_о=ру'жiе, вл\дко, плещы` мнjь` дае'тъ, тя` ви'дjьвше 
сло'ве бж~iй и='стиннаго бг~а, разбо'йнику пу'ть 
сотво'ршаго въ ра'й.

Не ктому` бою'ся, _е='же въ зе'млю вл\дко хр\сте` 
возвраще'нiя: ты' бо w\т земли` мя` возве'лъ _е=си` 
забве'нна, бл~гоутро'бiя ра'ди мно'гагw, къ высотjь` 
нетлjь'нiя воскр\снiемъ твои'мъ.

Бг~оро'диченъ: И=`же бц\ду тя` w\т души`, вл\дчце 
мi'ра бл~га'я, и=сповjь'дающихъ сп~си`: тебе' бо 
предста'тельство непобори'мое и='мамы, и='стинную 
бг~ороди'тельницу. 

\cslemph{И='нъ} I=рмо'съ: Бг~оявле'нiя твоегw` хр\сте`:

Снjь'дiю дре'ва во _е=де'мjь прельсти'выйся въ тлю` 
поползе` родонача'льникъ, преслу'шавый гд\си за'пwвjьди 
твоя^ пребл~гi'й: но сего` кр\сто'мъ па'ки возве'лъ 
_е=си` въ пе'рвую добро'ту, послушли'въ _о=ц~у` сп~се 
бы'вый.

Твое'ю сме'ртiю бл~же, сме'рти потреби'ся держа'ва, 
и=сто'чникъ жи'зни на'мъ и=сточи`, и= безсме'ртiе 
дарова'ся: сегw` ра'ди погребе'нiю твоему` и= воскр\снiю 
вjь'рою покланя'емся, и='мже jа='кw бг~ъ мi'ръ ве'сь 
просвjьти'лъ _е=си`.

Кр\стобг~оро'диченъ: Живы'й на нб~сjь'хъ и= творе'цъ 
всjь'хъ гд\сь, во твою` всенепоро'чная, всели'ся 
неизрече'ннw о_у=тро'бу, просла'вивый тя` превы'шшую 
нб~съ, и= ст~jь'йшую чинw'въ невеще'ственныхъ: тjь'мже 
ны'нjь и=`же на земли' тя о_у=бл~жа'емъ. 

\cslemph{И='нъ} I=рмо'съ: Бж~iимъ свjь'томъ твои'мъ:

Чистото'ю возсiя'вши свjь'тлw бж~е'ственное 
пребыва'нiе вл\дки всепjь'тая была` _е=си`. ты' бо 
_е=ди'на мт~и бж~iя jа=ви'лася _е=си`, во w=б\ъя'тiихъ 
jа='кw младе'нца сего` носи'вши.

Нося'щи добро'ту о_у='мную, краснjь'йшiя твоея` души`, 
невjь'ста бж~iя была` _е=си`, запеча'тствована дв~ствомъ 
чи'стая, и= свjь'томъ чистоты` мi'ръ просвjьща'ющи.

Да рыда'етъ собра'нiе sлочести'выхъ, не 
проповjь'дающихъ тя` jа='вjь чи'стую бг~ома'терь: ты' бо 
врата` бж~iя свjь'та jа=ви'лася _е=си` на'мъ, мра'къ 
прегрjьше'нiй разгоня'ющи.

\cslemph{Пjь'снь s~}

I=рмо'съ: Жите'йское мо'ре воздвиза'емое зря` 
напа'стей бу'рею, къ ти'хому приста'нищу твоему` прите'къ 
вопiю' ти: возведи` w\т тли` живо'тъ мо'й, многомл\стиве.

Распина'емь вл\дко, гвоздьми` о_у='бw кля'тву ю='же на 
на'съ потреби'лъ _е=си`: копiе'мъ же пробода'емь въ 
ребро`, а=да'мово рукописа'нiе растерза'въ, мi'ръ 
свободи'лъ _е=си`.

А=да'мъ низведе'ся, ле'стiю запя'тъ бы'въ, ко а='довjь 
про'пасти: но и='же _е=стество'мъ, бг~ъ же и= мл\стивъ, 
сше'лъ _е=си` на взыска'нiе, и= на ра'му поне'съ, 
совоскр~си'лъ _е=си`.

Бг~оро'диченъ: Преч\стая вл\дчце, ро'ждшая 
человjь'кwмъ ко'рмчiю гд\са, страсте'й мои'хъ 
непостоя'нное и= лю'тое о_у=толи` смуще'нiе, и= тишину` 
пода'ждь се'рдцу моему`. 

\cslemph{И='нъ} I=рмо'съ: Jа='тъ бы'сть:

Хр\стоубi'йца и= пр\орокоубi'йца бы'сть _е=вре'йское 
мно'жество: jа='кw бо пр\оро'ки дре'вле и='стины су'щыя 
та^йныя лучи`, о_у=би'ти не о_у=боя'ся: си'це и= ны'нjь 
гд\са, _е=го'же проповjь'даху _о=ни` тогда`, за'вистiю 
влеко'ми о_у=би'ша. но на'мъ бы'сть живо'тъ 
о_у=мерщвле'нiе _е=гw`.

Jа='тъ бы'лъ _е=си`, но не о_у=держа'нъ сп~се во 
гро'бjь, а='ще бо и= во'лею вкуси'лъ _е=си` сме'рти 
сло'ве, но воскр\слъ _е=си` jа='кw бг~ъ безсме'ртенъ, 
совоздви'гнувый о_у='зники су'щыя во а='дjь, и= ра'дость 
жена'мъ вмjь'стw печа'ли пре'жнiя, па'ки премjьни'вый.

Бг~оро'диченъ: Безче'стенъ тво'й и= ску'денъ jа=ви'ся 
ви'дъ плотскi'й па'че человjь'кwвъ, во вре'мя стра'сти: 
и='бо бж~ества` существо'мъ, красе'нъ добро'тою дв~ду 
показа'ся: но жезло'мъ твоегw` цр\ствiя врагw'въ сотры'й 
крjь'пость, глаго'лаше ч\стая: _w сн~е мо'й и= бж~е, w\т 
гро'ба воста'ни.

\cslemph{И='нъ} I=рмо'съ: Жите'йское мо'ре:

Вели'кiй преднаписа` во пр\оро'цjьхъ мwv"се'й тя`, 
ковче'гъ, и= трапе'зу, и= свjь'щникъ, и= ру'чку, 
w=бра'знw назна'менуя воплоще'нiе, _е='же и=з\ъ тебе` 
вы'шнягw, мт~и дв~о.

О_у=мерщвля'ется сме'рть, и= тлjь'нiе о_у=пражня'ется 
а=да'мова w=сужде'нiя, _w вл\дчце! плоду` твоему` 
прирази'вшееся: жи'знь бо родила` _е=си`, w\т и=стлjь'нiя 
и=збавля'ющую пою'щихъ тя`.

Зако'нъ и=знемо'же, и= сjь'нь ми'мw и='де, па'че 
о_у=ма` и= смы'сла jа='вльшейся ми` бл~года'ти, _е='же 
w\т тебе` дв~о рж\ства`, бг~а и= сп~са, многопjь'тая.

Конда'къ, гла'съ s~: Живонача'льною дла'нiю, 
о_у=ме'ршыя w\т мра'чныхъ о_у=до'лiй, жизнода'вецъ 
воскр~си'въ всjь'хъ хр\сто'съ бг~ъ, воскр\снiе подаде` 
человjь'ческому ро'ду: _е='сть бо всjь'хъ сп~си'тель, 
воскр\снiе и= живо'тъ, и= бг~ъ всjь'хъ.

I='косъ: Кр\стъ и= погребе'нiе твое` жизнода'вче, 
воспjьва'емъ вjь'рнiи, и= покланя'емся, jа='кw а='дъ 
связа'лъ _е=си` безсме'ртне, jа='кw бг~ъ всеси'льный: и= 
м_е'ртвыя совоскр~си'лъ _е=си`, и= врата` см_е'ртная 
сокруши'лъ _е=си`, и= держа'ву а='дову низложи'лъ _е=си`, 
jа='кw бг~ъ. тjь'мже земноро'днiи славосло'вимъ тя` 
любо'вiю, воскр~сшаго, и= низложи'вшаго вра'жiю держа'ву 
всепа'губную, и= всjь'хъ воскр~си'вшаго въ тя` 
вjь'ровавшихъ, и= мi'ръ и=зба'вльшаго w\т стрjь'лъ 
sмiи'ныхъ, и= w\т пре'лести вра'жiя, jа='кw бг~ъ всjь'хъ.

\cslemph{Пjь'снь з~}

I=рмо'съ: Росода'тельну о_у='бw пе'щь содjь'ла 
а='гг~лъ прп\дбнымъ _о=трокw'мъ, халд_е'и же w=паля'ющее 
велjь'нiе бж~iе, мучи'теля о_у=вjьща` вопи'ти: 
бл~гослове'нъ _е=си` бж~е _о=т_е'цъ на'шихъ.

Рыда'ющее во стр\сти твое'й со'лнце, во мра'къ 
w=блече'ся, и= во дни` по все'й вл\дко, земли` свjь'тъ 
поме'рче, вопiя`: бл~гослове'нъ _е=си` бж~е _о=т_е'цъ 
на'шихъ.

W=блеко'шася хр\сте` схожде'нiемъ твои'мъ во свjь'тъ 
преиспw'дняя, пра'_отецъ же весе'лiя и=спо'лнь jа=ви'ся 
ликовству'я, взыгра'ся вопiя`: бл~гослове'нъ _е=си` бж~е 
_о=т_е'цъ на'шихъ.

Бг~оро'диченъ: Тобо'ю мт~и дв~о, свjь'тъ возсiя` все'й 
вселе'ннjьй свjь'тлый: зижди'теля бо ты` всjь'хъ родила` 
_е=си` бг~а. _е=го'же проси` всеч\стая, на'мъ низпосла'ти 
вjь^рнымъ ве'лiю мл\сть. 

\cslemph{И='нъ} I=рмо'съ: Неизрече'нное чу'до:

_W стра'ннагw w='браза! i=и~ля и=збавле'й w\т рабо'ты 
фараw'нскiя, распина'ется во'лею w\т негw`, и= 
разрjьша'етъ вери^ги согрjьше'нiй. _е=му'же вjь'рою 
пое'мъ: и=зба'вителю бж~е, бл~гослове'нъ _е=си`.

Тебе` сп~са на ло'бнjьмъ распя'ша пребеззако'нныхъ 
_о='троцы нечести'вiи, врата` мjь^дная и= вер_еи` 
сломи'вшаго, во сп~се'нiе на'съ пою'щихъ: и=зба'вителю 
бж~е, бл~гослове'нъ , _е=си`.

Бг~оро'диченъ: _Е='vы дре'внiя свобожде'нiе ро'ждшая 
w\т кля'твы, разрjьша'еши а=да'ма дв~о ч\стая. тjь'мже со 
а='гг~лы тя`, съ сн~омъ твои'мъ пое'мъ и= вопiе'мъ: 
и=зба'вителю бж~е, бл~гослове'нъ _е=си`.

\cslemph{И='нъ} I=рмо'съ: Росода'тельну о_у='бw пе'щь:

Ю='нwшъ трiе'хъ пе'щь не w=пали`, рж\ство` 
проwбразу'ющихъ твое`: бж~е'ственный бо _о='гнь тебе` не 
w=пали'въ, всели'ся въ тя`, и= вся^ научи` вопи'ти: 
бл~гослове'нъ _е=си` бж~е _о=т_е'цъ на'шихъ.

Бл~жа'тъ концы' тя всеч\стая мт~и, jа='коже прорекла` 
_е=си`, просвjьща'еми свjьтолу'чными сiя'ньми твои'ми, и= 
бл~года'тiю пою'ще, вопiю'тъ: бл~гослове'нъ _е=си` бж~е 
_о=т_е'цъ на'шихъ.

Па^губныя о_у='бw зу'бы въ мя` вонзе` sмi'й 
лука'внjьйшiй: но @са'мъ@{сiя`} тво'й бг~ома'ти, 
сокруши` сн~ъ, крjь'пость же мнjь` даде` вопи'ти: 
бл~гослове'нъ _е=си` бж~е _о=т_е'цъ на'шихъ.

W=чисти'лище _е=стества` ты` _е=си` _е=ди'на 
бг~обл~же'нная, на ра^му бо херувi^мску сjьдя'щаго бг~а 
бо w=б\ъя'тiихъ нося'щи вопiе'ши: бл~гослове'нъ _е=си` 
бж~е _о=т_е'цъ на'шихъ.

\cslemph{Пjь'снь и~}

I=рмо'съ: И=з\ъ пла'мене прп\дбнымъ ро'су и=сточи'лъ 
_е=си`, и= пра'веднагw же'ртву водо'ю попали'лъ _е=си`: 
вся^ бо твори'ши хр\сте`, то'кмw _е='же хотjь'ти, тя` 
превозно'симъ во вся^ вjь'ки.

I=уд_е'йскiя дре'вле пр\орокоубi^йцы лю'ди, 
бг~оубi^йцы за'висть ны'нjь содjь'ла, тебе` на кр\стъ 
возне'сшыя, сло'ве бж~iй: _е=го'же превозно'симъ во вся^ 
вjь'ки.

Нб\снагw кру'га не w=ста'вилъ _е=си`, и= во а='дъ 
соше'дъ, всего` совоздви'глъ _е=си` лежа'щаго во гно'ищи 
хр\сте` человjь'ка, тя` превознося'ща во вся^ вjь'ки.

Бг~оро'диченъ: W\т свjь'та свjьтода'вца сло'ва зачала` 
_е=си`, и= ро'ждши неизрече'ннw сего`, просла'вилася 
_е=си`: дх~ъ бо въ тя` _о=трокови'це бж~iй всели'ся. 
тjь'мже тя` пое'мъ во вся^ вjь'ки. 

\cslemph{И='нъ} I=рмо'съ: О_у=жасни'ся боя'йся нб~о:

О_у=жасе'ся вся'къ слу'хъ, ка'кw вы'шнiй во'лею 
прiи'де на зе'млю, а='дову крjь'пость разруши'ти 
кр\сто'мъ и= погребе'нiемъ, и= вся^ воздви'гнути зва'ти: 
_о='троцы бл~гослови'те, сщ~е'нницы воспо'йте, лю'дiе 
превозноси'те во вся^ вjь'ки.

Преста` а='дово мучи'тельство, и= ца'рство 
о_у=ничижи'ся про'чее, на кр\стjь' бо на земли` 
водру'зився, и='же над\ъ всjь'ми бг~ъ, сегw` могу'тство 
низложи`. _е=го'же _о='троцы бл~гослови'те, сщ~е'нницы 
воспо'йте, лю'дiе превозноси'те во вся^ вjь'ки.

_W неизрече'ннагw твоегw` хр\сте` чл~вjьколю'бiя, и= 
неизглаго'ланныхъ бла^гъ! мене' бо ви'дя погиба'юща во 
а='довjь темни'цjь, стра^сти претерпjь'вый и=зба'вилъ 
_е=си`. тjь'мже тя` бл~гослови'мъ всjь'хъ вл\дку, и= 
превозно'симъ во вся^ вjь'ки. 

\cslemph{И='нъ} I=рмо'съ: И=з\ъ пла'мене прп\дбнымъ:

Позлаще'ною ри'зою jа='коже цр~и'цу сн~ъ тво'й 
просвjьти'въ заре'ю дх~а, тебе` w=десну'ю себе` поста'ви 
преч\стая: _е=го'же превозно'симъ во вся^ вjь'ки.

И='же хотjь'нiемъ _е=ди'нымъ мi'ръ водрузи'вый, w\т 
преч\стыя твоея` о_у=тро'бы пло'ть взе'млетъ, свы'ше сiю` 
назда'ти хотя`: _е=го'же превозно'симъ во вся^ вjь'ки.

Счета'нiемъ сло'ва ко мнjь` человjь'ку, бж~е'ственное 
жили'ще была` _е=си` преч\стая, просiя'вши jа='вjь 
дв~ства свjь'тлостiю: тjь'мже тя` пое'мъ во вся^ вjь'ки.

Златоза'рный тя` свjь'щникъ предвоwбрази`, прiе'мшую 
несказа'ннw свjь'тъ непристу'пный, ра'зумомъ свои'мъ 
w=заря'ющъ вся'ч_еская. тjь'мже тя` пое'мъ ч\стая, во 
вjь'ки.

Та'же, пое'мъ пjь'снь бц\ды: Вели'читъ душа` моя` 
гд\са: Съ припjь'вомъ: Ч\стнjь'йшую херувi^мъ:

\cslemph{Пjь'снь f~}

I=рмо'съ: Бг~а человjь'кwмъ не возмо'жно ви'дjьти, на 
него'же не смjь'ютъ чи'ни а='гг~льстiи взира'ти: тобо'ю 
же всеч\стая, jа=ви'ся человjь'кwмъ сло'во воплоще'нно, 
_е=го'же велича'юще, съ нб\сными вw'и тя` о_у=бл~жа'емъ. 

Страсте'й неприча'стенъ ты` пребы'лъ _е=си` сло'ве 
бж~iй, пло'тiю прiwбщи'вся стр\ст_е'мъ: но рjьши'ши w\т 
страсте'й человjь'ка, стр\ст_е'мъ бы'въ стра'сть сп~се 
на'шъ: _е=ди'нъ бо _е=си` безстр\стенъ, и= всеси'ленъ.

Тлю` сме'ртную прiе'мъ, тлjь'нiя соблю'лъ _е=си` 
тjь'ло твое` невку'сно, твоя' же животворя'щая и= 
бж~е'ственная вл\дко душа`, во а='дjь не w=ста'влена 
бы'сть: но jа='коже w\т сна` воскр~съ, на'съ совоздви'глъ 
_е=си`.

Тр\оченъ: Бг~а _о=ц~а`, сн~а собезнача'льна, вси` 
человjь'цы о_у=стна'ми о_у='бw чи'стыми сла'вимъ, 
неизрече'нную же и= пресла'вную дх~а всест~а'гw си'лу 
чти'мъ: _е=ди'на бо _е=си` всеси'льная тр\оца 
неразлу'чная. 

\cslemph{И='нъ} I=рмо'съ: Не рыда'й мене` мт~и:

А='ще и= во гро'бъ соше'лъ _е=си` jа='кw ме'ртвъ 
жизнода'вче, но а='дову крjь'пость разруши'лъ _е=си` 
хр\сте`, совоздви'гнувъ м_е'ртвыя, jа=`же и= поглоти`, и= 
воскр\снiе всjь^мъ по'далъ _е=си`, jа='кw бг~ъ, вjь'рою 
и= любо'вiю тя` велича'ющымъ.

Да ра'дуется тва'рь, и= да процвjьте'тъ jа='кw крi'нъ: 
хр\сто'съ бо w\т ме'ртвыхъ воста` jа='кw бг~ъ. гдjь` 
твое`, сме'рте, ны'нjь жа'ло, воззове'мъ; гдjь` твоя` 
а='де, побjь'да; низложи' тя въ зе'млю, возвы'сивый ро'гъ 
на'шъ, jа='кw бл~гоутро'бенъ.

Кр\стобг~оро'диченъ: Но'сиши нося'щаго вся^, и= 
держи'ши jа='кw младе'нца въ рука'хъ, и=з\ъ руки` 
и=збавля'ющаго на'съ бори'теля врага`, преч\стая вл\дчце, 
и= ви'диши на кр\стъ дре'ва возвыша'ема, w\т ро'ва sло'бы 
на'съ возвы'сивша. 

\cslemph{И='нъ} I=рмо'съ: Бг~а человjь'кwмъ не возмо'жно:

Sвjьзда` сiя'ющи заря'ми бж~ества` w\т i=а'кwва, w\т 
мра'ка содержи^мымъ возсiя`: тобо'ю бо всеч\стая, 
хр\сто'съ бг~ъ сло'во воплоще'нно: и='мже просвjьща'еми, 
съ нб\сными вw'и тя` о_у=бл~жа'емъ.

О_у=крjь'плься си'лою твое'ю и= бл~года'тiю, тебjь` 
пjь'снь о_у=се'рднw w\т се'рдца возложи'хъ: но сiю` 
прiими` дв~о ч\стая, бл~года'ть воздаю'щи твою` 
многосвjь'тлую, w\т нетлjь'нныхъ сокро'вищъ, 
бг~обл~же'нная.

Поста'въ показа'лася _е=си` jа='вjь бж~ества`, въ 
не'мже ри'зу тjьлесе` сло'во и=стка`, бг~осодjь'лавъ мо'й 
дв~о зра'къ: въ него'же w=бле'кся всjь'хъ сп~се`, w\т 
смы'сла чи'ста тебе` велича'ющихъ.

М_е'ртвымъ воскр\снiе ны'нjь дарова'ся, твои'мъ 
несказа'ннымъ и= неизрече'ннымъ рж\ство'мъ, бц\де 
всеч\стая: жи'знь бо и=з\ъ тебе` пло'тiю w=бле'кшися, 
всjь^мъ возсiя`, и= сме'ртную дря'хлость jа='вjь 
разруши`.

По катава'сiи _е=ктенiа` ма'лая. Та'же: Ст~ъ гд\сь 
бг~ъ на'шъ: _Е=_ксапостiла'рiй о_у='треннiй. 

На хвали'техъ стiхи^ры воскр\сны, гла'съ s~:

Стi'хъ: Сотвори'ти въ ни'хъ су'дъ напи'санъ: сла'ва 
сiя` бу'детъ всjь^мъ прп\дбнымъ _е=гw`.

Кр\стъ тво'й гд\си, жи'знь и= воскр\снiе лю'демъ 
твои^мъ _е='сть, и= надjь'ющеся на'нь, тебе` воскр\сшаго 
бг~а на'шего пое'мъ: поми'луй на'съ.

Стi'хъ: Хвали'те бг~а во ст~ы'хъ _е=гw`, хвали'те 
_е=го` во о_у=тверже'нiи си'лы _е=гw`.

Погребе'нiе твое` вл\дко, ра'й w\тве'рзе ро'ду 
человjь'ческому: и= w\т и=стлjь'нiя и=зба'вльшеся, тебе` 
воскр\сшаго бг~а на'шего пое'мъ: поми'луй на'съ.

Стi'хъ: Хвали'те _е=го` на си'лахъ _е=гw`, хвали'те 
_е=го` по мно'жеству вели'чествiя _е=гw`.

Со _о=ц~е'мъ и= дх~омъ хр\ста` воспои'мъ, воскр\сшаго 
и=з\ъ ме'ртвыхъ, и= къ нему` вопiи'мъ: ты` живо'тъ _е=си` 
на'шъ, и= воскр\снiе, поми'луй на'съ.

Стi'хъ: Хвали'те _е=го` во гла'сjь тру'бнjьмъ, 
хвали'те _е=го` во _псалти'ри и= гу'слехъ.

Тридне'венъ воскр\слъ _е=си` хр\сте` w\т гро'ба 
jа='коже пи'сано _е='сть, совоздви'гнувый пра'_отца 
на'шего. тjь'мже тя` и= сла'витъ ро'дъ человjь'ческiй, и= 
воспjьва'етъ твое` воскр\снiе.

И='ны стiхи^ры а=натw'лiевы. 

Стi'хъ: Хвали'те _е=го` въ тv"мпа'нjь и= ли'цjь, 
хвали'те _е=го` во стру'нахъ и= _о=рга'нjь.

Гд\си, ве'лiе и= стра'шно _е='сть твоегw` воскр\снiя 
та'инство: та'кw бо произше'лъ _е=си` w\т гро'ба, jа='кw 
жени'хъ w\т черто'га, сме'ртiю сме'рть разруши'вый, да 
а=да'ма свободи'ши. тjь'мже на нб~сjь'хъ а='гг~ли 
ликовству'ютъ, и= на земли` человjь'цы сла'вятъ, _е='же 
на на'съ бы'вшее бл~гоутро'бiе твое`, чл~вjьколю'бче.

Стi'хъ: Хвали'те _е=го` въ кv"мва'лjьхъ 
доброгла'сныхъ, хвали'те _е=го` въ кv"мва'лjьхъ 
восклица'нiя: вся'кое дыха'нiе да хва'литъ гд\са.

_W пребеззако'ннiи i=уде'_е, гдjь` су'ть печа^ти и= 
сре'бр_еницы, jа=`же да'сте во'инwмъ; не о_у=кра'дено 
бы'сть сокро'вище, но воскр~се jа='кw си'ленъ: са'ми же 
посрами'стеся w\тве'ргшiися хр\ста` гд\са сла'вы, 
пострада'вша и= погребе'нна, и= воскр\сша и=з\ъ 
ме'ртвыхъ: тому` поклони'мся.

Стi'хъ: Воскр\сни` гд\си бж~е мо'й, да вознесе'тся 
рука` твоя`, не забу'ди о_у=бо'гихъ твои'хъ до конца`.

Запеча'тану гро'бу, ка'кw w=кра'дени бы'сте i=уде'_е, 
стра'жы поста'вльше, и= зна'м_енiя поло'жше, две'ремъ 
заключ_е'ннымъ про'йде цр~ь; и=ли` jа='кw ме'ртва 
предста'вите, и=ли` jа='кw бг~у поклони'теся, съ на'ми 
пою'ще: сла'ва гд\си кр\сту` твоему`, и= воскр\снiю 
твоему`.

Стi'хъ: И=сповjь'мся тебjь` гд\си, всjь'мъ се'рдцемъ 
мои'мъ, повjь'мъ вся^ чудеса` твоя^.

Живопрiя'тнагw твоегw` гро'ба, мv"ронw'сицы ж_ены` 
рыда'ющя, достиго'ша гд\си, и= мv^ра нося'щя, тjь'ло 
твое` преч\стое пома'зати и=ска'ху: w=брjьто'ша же 
свjьтоно'сна а='гг~ла на ка'мени сjьдя'ща, и= къ ни^мъ 
вjьща'юща, и= глаго'люща: что` слезите`, и=з\ъ ре'бръ 
и=сточи'вшаго жи'знь мi'рови; что` и='щете jа='кw ме'ртва 
во гро'бjь безсме'ртнаго; те'кшя же па'че возвjьсти'те 
тогw` о_у=ч~нкw'мъ, сла'внагw _е=гw` воскр\снiя 
всемi'рное ра'дованiе, и='мже и= на'съ сп~се 
просвjьти'вый, да'руй w=чище'нiе и= ве'лiю мл\сть. 

Сла'ва, стiхи'ра о_у='тренняя _е=v\гльская. И= ны'нjь: 
Пребл~гослове'нна _е=си`: Славосло'вiе вели'кое. 

Та'же, тропа'рь воскр\снъ.

Воскр~съ и=з\ъ гро'ба, и= о_у='зы растерза'лъ _е=си` 
а='да, разруши'лъ _е=си` w=сужде'нiе сме'рти гд\си, вся^ 
w\т сjьте'й врага` и=зба'вивый: jа=ви'вый же себе` 
а=п\слwмъ твои^мъ, посла'лъ _е=си` я=` на про'повjьдь, и= 
тjь'ми ми'ръ тво'й по'далъ _е=си` вселе'ннjьй, _е=ди'не 
многомл\стиве.

Та'же, _е=ктенiи`. И= w\тпу'стъ. 
