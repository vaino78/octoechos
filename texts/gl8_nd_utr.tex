На о_у='трени по _е=_кса_псалмjь'хъ,

Бг~ъ гд\сь, и= jа=ви'ся на'мъ: на гла'съ и~, и= 
глаго'лемъ тропа'рь воскр\снъ: Съ высоты` снизше'лъ 
_е=си`: два'жды. Сла'ва, и= ны'нjь, бг~оро'диченъ: И='же 
на'съ ра'ди рожде'йся w\т дв~ы: Та'же _о=бы'чное 
стiхосло'вiе _псалти'ра.

По а~-мъ стiхосло'вiи, сjьда'льны воскр\сны, гла'съ 
и~:

Воскр~слъ _е=си` и=з\ъ ме'ртвыхъ животе` всjь'хъ, и= 
а='гг~лъ свjь'телъ жена'мъ вопiя'ше: преста'ните w\т 
сле'зъ, а=п\слwмъ бл~говjьсти'те, возопi'йте пою'щя: 
jа='кw воскр~се хр\сто'съ гд\сь, бл~говоли'вый сп~сти` 
jа='кw бг~ъ ро'дъ человjь'ческiй.

Стi'хъ: Воскр\сни` гд\си бж~е мо'й, да вознесе'тся 
рука` твоя`, не забу'ди о_у=бо'гихъ твои'хъ до конца`.

Воскр~сы'й и=з\ъ гро'ба jа='кw вои'стинну, прп\дбнымъ 
повелjь'лъ _е=си` жена'мъ проповjь'дати воста'нiе 
а=п\слwмъ, jа='коже пи'сано _е='сть: и= ско'рый пе'тръ 
предста` гро'бу, и= свjь'тъ зря` во гро'бjь, 
о_у=жаса'шеся. тjь'мже и= о_у=ви'дjьвъ плащани^цы, 
кромjь` бж~е'ственнагw тjь'ла въ не'мъ лежа'щыя, @со 
стра'хомъ@{съ вjь'рою} возопи`: сла'ва тебjь` хр\сте` 
бж~е, jа='кw сп~са'еши вся^ сп~се` на'шъ: _о='ч~ее бо 
_е=си` сiя'нiе.

Сла'ва, и= ны'нjь, бг~оро'диченъ: Нб\сную две'рь, и= 
кiвw'тъ, всест~у'ю го'ру, свjьтоза'рный _о='блакъ 
воспои'мъ, нб\сную лjь'ствицу, слове'сный ра'й, _е='vино 
и=збавле'нiе, вселе'нныя всея` вели'кое сокро'вище, 
jа='кw сп~се'нiе въ не'й содjь'лася мi'рови, и= 
w=ставле'нiе дре'внихъ согрjьше'нiй. сегw` ра'ди вопiе'мъ 
ти`: моли` сн~а твоего` и= бг~а, прегрjьше'нiй 
w=ставле'нiе дарова'ти, бл~гоче'стнw покланя'ющымся 
прест~о'му рж\ству` твоему`.

По в~-мъ стiхосло'вiи сjьда'льны воскр\сны, гла'съ и~:

Человjь'цы сп~се, гро'бъ тво'й запеча'таша: а='гг~лъ 
ка'мень w\т двере'й w\твали`: ж_ены` ви'дjьша воста'вша 
w\т ме'ртвыхъ, и= ты'я бл~говjьсти'ша о_у=ч~нкw'мъ 
твои^мъ въ сiw'нjь, jа='кw воскре'слъ _е=си` животе` 
всjь'хъ, и= разрjьши'шася о_у='зы см_е'ртныя: гд\си 
сла'ва тебjь`.

Стi'хъ: И=сповjь'мся тебjь` гд\си, всjь'мъ се'рдцемъ 
мои'мъ, повjь'мъ вся^ чудеса` твоя^.

Мv^ра погреба'т_ельная ж_ены` прине'сшя, гла'съ 
а='гг~льскiй и=з\ъ гро'ба слы'шаху: преста'ните w\т 
сле'зъ, и= вмjь'стw печа'ли ра'дость прiими'те, 
возопi'йте пою'щя: jа='кw воскр~се хр\сто'съ гд\сь, 
бл~говоли'вый сп~сти` jа='кw бг~ъ ро'дъ человjь'ческiй.

Сла'ва, и= ны'нjь, бг~оро'диченъ, не сjьдя'ще пое'мъ, 
но стоя'ще, и= со стра'хомъ и= бл~гоговjь'нiемъ:

W= тебjь` ра'дуется бл~года'тная вся'кая тва'рь, 
а='гг~льскiй собо'ръ, и= чл~вjь'ческiй ро'дъ, w=сщ~е'нный 
хра'ме, и= раю` слове'сный: дв~ственная похвало`, и=з\ъ 
нея'же бг~ъ воплоти'ся, и= мл\днецъ бы'сть, пре'жде 
вjь^къ сы'й бг~ъ на'шъ: ложесна' бо твоя^ пр\сто'лъ 
сотвори`, и= чре'во твое` простра'ннjье нб~съ содjь'ла. 
w= тебjь` ра'дуется, бл~года'тная, вся'кая тва'рь, сла'ва 
тебjь`.

V=пакои`, гла'съ и~: Мv"ронw'сицы жизнода'вца 
предстоя'щя гро'бу, вл\дку и=ска'ху въ ме'ртвыхъ 
безсме'ртнаго, и= ра'дость бл~говjь'щенiя w\т а='гг~ла 
прiе'мшя, а=п\слwмъ возвjьща'ху: jа='кw воскр~се 
хр\сто'съ бг~ъ, подая'й мi'рови ве'лiю мл\сть.

Степ_е'нна, гла'съ и~: И='хже стiхи` повторя'юще 
пое'мъ.

\cslemph{А=нтiфw'нъ} а~:

W\т ю='ности моея` вра'гъ мя` и=скуша'етъ, сластьми` 
пали'тъ мя`: а='зъ же надjь'яся на тя` гд\си, побjьжда'ю 
сего`.

Ненави'дящiи сiw'на, да бу'дутъ о_у='бw пре'жде 
и=сторже'нiя jа='кw трава`: ссjь'четъ бо хр\сто'съ вы^я 
и='хъ, о_у=сjьче'нiемъ му'къ.

Сла'ва: Ст~ы'мъ дх~омъ, _е='же жи'ти вся'ч_ескимъ: 
свjь'тъ w\т свjь'та, бг~ъ вели'къ: со _о=ц~е'мъ пое'мъ 
_е=му`, и= съ сло'вомъ.

И= ны'нjь, то'йже.

\cslemph{А=нтiфw'нъ} в~:

Се'рдце мое` стра'хомъ твои'мъ да покры'ется 
смиреному'дрствующее: да не возне'сшееся w\тпаде'тъ w\т 
тебе` всеще'дре.

На гд\са и=мjь'вый наде'жду, не о_у=страши'тся тогда`, 
_е=гда` _о=гне'мъ вся^ суди'ти и='мать, и= му'кою.

Сла'ва: Ст~ы'мъ дх~омъ, вся'къ кто` бж~е'ственный 
ви'дитъ, и= предглаго'летъ, чудодjь'йствуетъ вы^шняя, въ 
трiе'хъ _е=ди'наго бг~а поя`: а='ще бо и= трисiя'етъ, 
_е=динонача'льствуетъ бж~ество`.

И= ны'нjь, то'йже.

\cslemph{А=нтiфw'нъ} г~:

Воззва'хъ тебjь` гд\си, вонми`, приклони' ми о_у='хо 
твое` вопiю'щу, и= w=чи'сти, пре'жде да'же не во'змеши 
мене` w\тсю'ду.

Въ ма'тери свое'й земли` w\тходя'й вся'къ, па'ки 
разрjьша'ется, прiя'ти му^ки, и=ли` по'ч_ести пожи'вшихъ.

Сла'ва: Ст~ы'мъ дх~омъ бг~осло'вiе, _е=ди'ница 
трист~а'я: _о=ц~ъ бо безнача'ленъ: w\т негw'же роди'ся 
сн~ъ безлjь'тнw, и= дх~ъ сопр\сто'ленъ, соwбра'зенъ, w\т 
_о=ц~а` спросiя'вшiй.

И= ны'нjь, то'йже.

\cslemph{А=нтiфw'нъ} д~:

Се` ны'нjь что` добро`, и=ли` что` красно`; но _е='же 
жи'ти бра'тiи вку'пjь: въ се'мъ бо гд\сь w=бjьща` живо'тъ 
вjь'чный.

W= ри'зjь свое'й, и='же крi'ны се'льныя о_у=краша'яй, 
повелjьва'етъ, jа='кw не подоба'етъ пещи'ся.

Сла'ва: Ст~ы'мъ дх~омъ, _е=динови'дною вино'ю, вся^ 
содержа'тся миропода'тельнjь: бг~ъ бо се'й _е='сть, 
_о=ц~у' же и= сн~ови _е=диносу'щенъ госпо'дственнjь.

И= ны'нjь, то'йже.

Прокi'менъ, гла'съ и~: Воцр~и'тся гд\сь во вjь'къ, 
бг~ъ тво'й сiw'не, въ ро'дъ и= ро'дъ. Стi'хъ: Хвали` 
душе` моя` гд\са, восхвалю` гд\са въ животjь` мое'мъ. 
Вся'кое дыха'нiе: и= _Е=v\глiе воскр\сно о_у='треннее. 
Воскр\снiе хр\сто'во. _псало'мъ н~. И= прw'чая поря'ду.

Канw'нъ воскр\снъ. Гла'съ и~:

\cslemph{Пjь'снь а~}

I=рмо'съ: Колесницегони'теля фараw'ня погрузи`, 
чудотворя'й и=ногда` мwv"се'йскiй же'злъ, кр\стоwбра'знw 
порази'въ, и= раздjьли'въ мо'ре: i=и~ля же бjьглеца`, 
пjьшехо'дца сп~се`, пjь'снь бг~ови воспjьва'юща.

Припjь'въ: Сла'ва гд\си, ст~о'му воскр\снiю твоему`.

Всеси'льну хр\сто'ву бж~еству` ка'кw не диви'мся; w\т 
страсте'й о_у='бw всjь^мъ вjь^рнымъ, безстра'стiе и= 
нетлjь'нiе точа'щу, w\т ребра' же ст~а'гw и=сто'чникъ 
безсме'ртiя и=ска'пающу, и= живо'тъ и=з\ъ гро'ба 
присносу'щный.

Jа='кw бл~голjь'пенъ жена'мъ а='гг~лъ ны'нjь jа=ви'ся, 
свjь'тлыя нося` w='бразы _е=сте'ственныя невеще'ственныя 
чистоты`, зра'комъ же возвjьща'я свjь'тъ воскр\снiя, 
зовы'й: воскр~се гд\сь.

Бг~оро'диченъ: Пресла^вная возглаго'лашася w= тебjь` 
въ родjь'хъ родw'въ, бг~а сло'ва во чре'вjь вмjь'щшая, 
чиста' же пребы'вши бц\де мр~i'е. тjь'мже тя` вси` 
почита'емъ, су'щее по бз~jь заступле'нiе на'ше.

Канw'нъ кр\стовоскр\снъ.

I=рмо'съ: Во'ду проше'дъ jа='кw су'шу:

Взя'шася врата` болjь'зненная, и= о_у=жасо'шася 
вра'тницы а='довы, зря'ще въ преиспw'днjьйшая соше'дшаго, 
и='же на высотjь` всjь'хъ превы'ше _е=стества`.

О_у=диви'шася чи'ни а='гг~льстiи, зря'ще на пр\сто'лjь 
посажде'но _о='ч~и, w\тпа'дшее _е=стество` 
человjь'ческое, затворе'ное въ преиспо'днихъ земли`.

Чи'ни тя` а='гг~льстiи и= человjь'честiи, 
безневjь'стная мт~и, хва'лятъ непреста'ннw: зижди'теля бо 
си'хъ, jа='кw мл\днца на w=б\ъя'тiихъ твои'хъ носи'ла 
_е=си`.

И='нъ канw'нъ прест~jь'й бц\дjь.

I=рмо'съ: Пои'мъ гд\севи, прове'дшему лю'ди своя^:

Преч\стая бц\де, вопло'щшееся присносу'щное и= 
пребж~е'ственное сло'во, па'че _е=стества` ро'ждши, 
пое'мъ тя`.

Гро'здъ тя` живоно'сенъ, всемi'рнагw и=ска'пающъ 
сла'дость сп~се'нiя, дв~а хр\сте` роди`.

Ро'дъ а=да'мль, ко _е='же па'че о_у=ма` бл~же'нству, 
тобо'ю возведе'нный бц\де, досто'йнw сла'витъ тя`.

Катава'сiа: W\тве'рзу о_у=ста` моя^:

\cslemph{Пjь'снь г~}

I=рмо'съ: О_у=твержде'й въ нача'лjь нб~са` ра'зумомъ, 
и= зе'млю на вода'хъ w=снова'вый, на ка'мени мя`, 
хр\сте`, за'повjьдей твои'хъ о_у=тверди`, jа='кw нjь'сть 
ст~ъ, па'че тебе` _е=ди'не чл~вjьколю'бче.

W=сужде'на бы'вша а=да'ма вкуше'нiемъ грjьха`, пло'ти 
твоея` сп~си'тельною стр\стiю w=правда'лъ _е=си` хр\сте`: 
са'мъ бо непови'ненъ сме'ртнагw и=ску'са бы'лъ _е=си` 
безгрjь'шне.

Воскр\снiя свjь'тъ возсiя` су'щымъ во тьмjь`, и= 
сjь'ни сме'ртнjьй сjьдя'щымъ, бг~ъ мо'й i=и~съ, и= 
свои'мъ бж~ество'мъ крjь'пкаго связа'въ, сегw` сосу'ды 
расхи'тилъ _е='сть.

Бг~оро'диченъ: Херувi'мwвъ и= серафi'мwвъ превы'шши 
jа=ви'лася _е=си` бц\де: ты' бо _е=ди'на прiя'ла _е=си` 
невмjьсти'маго бг~а въ твое'мъ чре'вjь, нескве'рная: 
тjь'мже тя` вjь'рнiи вси` пjь'сньми ч\стая, 
о_у=бл~жа'емъ.

\cslemph{И='нъ} I=рмо'съ: Нб\снагw кру'га верхотво'рче гд\си:

W\тве'ргшагося пре'жде за'повjьди, гд\си, и=зринове'на 
мя` w\т тебе` сотвори'лъ _е=си`, въ него'же воwбрази'вся, 
послуша'нiю же @навы'къ, себjь` па'ки назда'лъ@ 
{научи'въ, къ себjь` всели'лъ} _е=си` распя'тiемъ.

Прему'дростiю вся^ проувjь'дjьвый гд\си, и= ра'зумомъ 
твои'мъ водрузи'вый преиспw'дняя, не не сподо'билъ _е=си` 
снизхожде'нiемъ твои'мъ сло'ве бж~iй, воскр~си'ти, _е='же 
по w='бразу твоему`.

Бг~оро'диченъ: Всели'вся въ дв~у тjьле'снjь гд\си, 
jа=ви'лся _е=си` человjь'кwмъ, jа='коже подоба'ше 
ви'дjьти тебе`: ю='же и= показа'лъ _е=си` jа='кw 
и='стинную бц\ду, и= вjь'рныхъ помо'щницу, _е=ди'не 
чл~вjьколю'бче.

\cslemph{И='нъ} I=рмо'съ: Ты` _е=си` о_у=твержде'нiе:

Да'ждь на'мъ по'мощь твои'ми мл~твами всеч\стая, 
прило'ги w\тража'ющи лю'тыхъ w=бстоя'нiй.

_Е='vjь прама'тери ты` и=справле'нiе была` _е=си`, 
нача'льника жи'зни мi'рови, хр\ста` бц\де ро'ждши.

Препоя'ши мя` си'лою всеч\стая, jа='же вои'стинну бг~а 
ро'ждши пло'тiю, _о='ч~ую v=поста'сную си'лу.

\cslemph{Пjь'снь д~}

I=рмо'съ: Ты` моя` крjь'пость гд\си, ты` моя` и= 
си'ла, ты` мо'й бг~ъ, ты` мое` ра'дованiе, не w=ста'вль 
нjь'дра _о='ч~а, и= на'шу нищету` посjьти'въ. тjь'мъ съ 
пр\оро'комъ а=вваку'момъ зову' ти: си'лjь твое'й сла'ва 
чл~вjьколю'бче.

Ты` врага` су'ща мя` sjьлw` возлюби'лъ _е=си`: ты` 
и=стоща'нiемъ стра'ннымъ соше'лъ _е=си` на зе'млю, 
бл~гоутро'бне сп~се, послjь'днягw моегw` досажде'нiя не 
w\тве'ргся, и= пребы'въ на высотjь` преч\стыя твоея` 
сла'вы, пре'жде безче'ствованнаго просла'вилъ _е=си`.

Кто` зря` вл\дко, ны'нjь не о_у=жаса'ется, стр\стiю 
сме'рть разруша'ему; кр\сто'мъ бjьжа'щее тлjь'нiе, и= 
сме'ртiю а='дъ бога'тства и=стощава'емый, бж~е'ственною 
си'лою тебе` распя'тагw; чу'дно дjь'ло, чл~вjьколю'бче!

Бг~оро'диченъ: Ты` вjь^рнымъ похвала` _е=си` 
безневjь'стная, ты` предста'тельнице, ты` и= прибjь'жище 
хр\стiа'нъ, стjьна` и= приста'нище: къ сн~у бо твоему` 
мольбы^ но'сиши всенепоро'чная, и= сп~са'еши w\т бjь'дъ, 
вjь'рою и= любо'вiю бц\ду ч\стую тебе` зна'ющихъ.

\cslemph{И='нъ} I=рмо'съ: О_у=слы'шахъ гд\си.

На кр\стjь' тя пригвозди'ша законопресту'пныхъ дjь'ти, 
хр\сте` бж~е: и='мже сп~слъ _е=си` jа='кw бл~гоутро'бенъ, 
сла'вящыя твоя^ страда^нiя.

Воскр~съ w\т гро'ба, вся^ совоскр~си'лъ _е=си` су'щыя 
во а='дjь м_е'ртвыя, и= просвjьти'лъ _е=си` jа='кw 
бл~гоутро'бенъ, сла'вящыя твое` воскр\снiе.

Бг~оро'диченъ: Бг~а, _е=го'же родила` _е=си` преч\стая 
мр~i'е, того` моли` дарова'ти рабw'мъ твои^мъ 
согрjьше'нiй проще'нiе.

\cslemph{И='нъ} I=рмо'съ то'йже.

Кла'съ возрасти'вшая животво'рный, неwра'нная ни'во, 
подаю'щаго мi'рови жи'знь, бц\де, сп~са'й пою'щыя тя`.

Бц\ду тя` всеч\стая, просвjь'щшiися вси` 
проповjь'дуемъ: сл~нце бо пра'вды родила` _е=си` 
приснодв~о.

W=чище'нiе да'руй на'шымъ невjь'дjьнi_емъ, jа='кw 
безгрjь'шенъ: и= о_у=мири` мi'ръ тво'й бж~е, мл~твами 
ро'ждшiя тя`.

\cslemph{Пjь'снь _е~}

I=рмо'съ: Вску'ю мя` w\три'нулъ _е=си` w\т лица` 
твоегw` свjь'те незаходи'мый, и= покры'ла мя` _е='сть 
чужда'я тьма` _о=кая'ннаго; но w=брати' мя, и= къ свjь'ту 
за'повjьдей твои'хъ пути^ моя^ напра'ви, молю'ся.

W=дjь'ятися претерпjь'лъ _е=си` въ багряни'цу пре'жде 
стр\сти твоея` сп~се, поруга'емь, первозда'ннагw 
покрыва'я без\ъwбра'зное w=бнаже'нiе: и= на'гъ 
пригвозди'лся _е=си` на кр\стjь` пло'тiю, совлача'я 
хр\сте`, ри'зу о_у=мерщвле'нiя.

W\т пе'рсти сме'ртныя, ты` па'дшее мое` па'ки назда'лъ 
_е=си` существо`, воскр~съ: и= нестарjь'ющееся хр\сте` 
о_у=стро'илъ _е=си`, jа=ви'въ па'ки jа='коже ца'рскiй 
w='бразъ, нетлjь'нiя жи'знь блиста'ющъ.

Бг~оро'диченъ: Мт~рнее дерзнове'нiе, _е='же къ сн~у 
твоему` и=му'щи всеч\стая, сро'днагw промышле'нiя, _е='же 
w= на'съ, не пре'зри, мо'лимся: jа='кw тебе` и= _е=ди'ну 
хр\стiа'не ко вл\дцjь w=чище'нiе мл\стивнw предлага'емъ.

\cslemph{И='нъ} I=рмо'съ: Просвjьти` на'съ:

Наста'ви на'съ си'лою кр\ста` твоегw` хр\сте`: тjь'мъ 
бо тебjь` припа'даемъ, ми'ръ пода'ждь на'мъ 
чл~вjьколю'бче.

W=корми` живо'тъ на'шъ, @jа='кw пребл~гi'й@{бж~е 
на'шъ} пою'щихъ твое` воста'нiе, и= ми'ръ пода'ждь на'мъ 
чл~вjьколю'бче.

Бг~оро'диченъ: О_у=моли` ч\стая, сн~а твоего` и= бг~а 
на'шего, неискусобра'чная мр~i'е преч\стая, _е='же 
низпосла'ти на'мъ вjь^рнымъ ве'лiю мл\сть.

\cslemph{И='нъ} I=рмо'съ: О_у='тренююще, вопiе'мъ ти`:

О_у=толи` нестерпи'мую бу'рю страсте'й мои'хъ, jа='же 
бг~а ро'ждшая w=корми'теля и= гд\са.

Слу'жатъ рж\ству` твоему` преч\стая бц\де, 
а='гг~льстiи чи'нове, и= человjь'кwвъ собра'нiе.

Мр~i'е бц\де безневjь'стная, о_у=пова^нiя врагw'въ 
w=суети`, и= пою'щыя тя` возвесели`.

\cslemph{Пjь'снь s~}

I=рмо'съ: W=чи'сти мя` сп~се, мнw'га бо беззакw'нiя 
моя^, и= и=з\ъ глубины` sw'лъ возведи`, молю'ся: къ 
тебjь' бо возопи'хъ, и= о_у=слы'ши мя`, бж~е сп~се'нiя 
моегw`.

Дре'вомъ крjь'пкw низложи' мя началоsло'бный: ты' же 
хр\сте`, возне'сся на кр\стjь`, @крjьпча'е@ 
{крjь'пльшаго} низложи'лъ _е=си`, посрами'въ сего`, 
па'дшаго же воскр~си'лъ _е=си`.

Ты` о_у=ще'дрилъ _е=си` сiw'на, возсiя'вый w\т гро'ба, 
но'ваго вмjь'стw ве'тхагw соверши'въ, jа='кw 
бл~гоутро'бенъ, бж~е'ственною твое'ю кро'вiю: и= ны'нjь 
ца'рствуеши въ не'мъ во вjь'ки, хр\сте`.

Бг~оро'диченъ: Да и=зба'вимся w\т лю'тыхъ 
прегрjьше'нiй, мольба'ми твои'ми бг~ороди'тельнице 
ч\стая, и= да о_у=лучи'мъ преч\стая бж~е'ственное 
сiя'нiе, и=з\ъ тебе` неизрече'ннw воплоще'ннагw сн~а 
бж~iя.

\cslemph{И='нъ} I=рмо'съ: Мл~тву пролiю` ко гд\су:

Дла^ни на кр\стjь` распросте'рлъ _е=си`, и=сцjьля'яй 
неудержа'ннw просте'ртую во _е=де'мjь ру'ку 
первозда'ннагw: и= твое'ю во'лею же'лчи вкуси'въ, и= 
сп~слъ _е=си` хр\сте` jа='кw си'ленъ, сла'вящыя твоя^ 
страда^нiя.

Сме'рти и=зба'витель вкуси`, дре'внягw w=сужде'нiя, 
jа='кw да и= тлjь'нiя ца'рство разруши'тъ, и= во а='дская 
сше'дъ, воскр~се хр\сто'съ, и= сп~се` jа='кw си'ленъ 
пою'щыя _е=гw` воскр\снiе.

Бг~оро'диченъ: Не преста'й за ны` моля'щи преч\стая 
бц\де дв~о, jа='кw вjь^рнымъ о_у=твержде'нiе ты` _е=си` 
наде'ждою твое'ю крjьпи'мся, и= любо'вiю тя`, и= и=з\ъ 
тебе` воплоще'ннаго неизрече'ннw сла'вимъ.

\cslemph{И='нъ} I=рмо'съ: Ри'зу мнjь` пода'ждь свjь'тлу:

Хра'мъ тя` бж~iй и= ковче'гъ, и= черто'гъ 
w=душевле'нный, и= две'рь нб\сную, бц\де вjь'рнiи 
возвjьща'емъ.

Тре'бищъ разруши'тель jа='кw бг~ъ, бы'вшее рж\ство` 
твое` мр~i'е бг~оневjь'сто, покланя'емо _е='сть со 
_о=ц~е'мъ и= дх~омъ.

Сло'во бж~iе тебе` земны^мъ, бц\де, показа` нб\сную 
лjь'ствицу: тобо'ю бо къ на'мъ сни'де.

Конда'къ, гла'съ и~: Подо'бенъ: Jа='кw нача'тки:

Воскр~съ и=з\ъ гро'ба, о_у=ме'ршыя воздви'глъ _е=си`, 
и= а=да'ма воскр~си'лъ _е=си`, и= _е='vа лику'етъ во 
твое'мъ воскр\снiи, мiрстi'и концы` торжеству'ютъ, _е='же 
и=з\ъ ме'ртвыхъ воста'нiемъ твои'мъ, многомл\стиве.

I='косъ: А='дwва ца'рствiя плjьни'вый, и= м_е'ртвыя 
воскреси'вый, долготерпjьли'ве, ж_ены` мv"ронw'сицы 
срjь'тилъ _е=си`, вмjь'стw печа'ли ра'дость принесы'й: и= 
а=п\слwмъ твои^мъ возвjьсти'лъ _е=си` побjьди'т_ельная 
сп~се мо'й зна'м_енiя живода'телю, и= тва'рь 
просвjьща'еши чл~вjьколю'бче. сегw` ра'ди и= мi'ръ 
сра'дуется, _е='же и=з\ъ ме'ртвыхъ воста'нiю твоему` 
многомл\стиве.

\cslemph{Пjь'снь з~}

I=рмо'съ: Бж~iя снизхожде'нiя _о='гнь о_у=стыдjь'ся въ 
вавv"лw'нjь и=ногда`: сегw` ра'ди _о='троцы въ пещи` 
ра'дованною ного'ю, jа='кw во цвjь'тницjь лику'юще 
поя'ху: бл~гослове'нъ _е=си` бж~е _о=т_е'цъ на'шихъ.

Сла'вное и=стоща'нiе, бж~е'ственное бога'тство твоея` 
нищеты` хр\сте`, о_у=дивля'етъ а='гг~лы, на кр\стjь` 
зря'щыя тя` пригвожда'ема, за _е='же сп~сти` вjь'рою 
зову'щыя: бл~гослове'нъ _е=си` бж~е _о=т_е'цъ на'шихъ.

Бж~е'ственнымъ твои'мъ соше'ствiемъ свjь'та 
и=спо'лнилъ _е=си` преиспw'дняя, и= тьма` прогна'на 
бы'сть пре'жде гоня'щая. w\тню'дуже воскр~со'ша и=`же w\т 
вjь'ка ю='зницы, зову'ще: бл~гослове'нъ бг~ъ _о=т_е'цъ 
на'шихъ.

Тр\оченъ: Всjь^мъ о_у='бw гд\са, @_е=ди'наго же 
_е=ди'ному@{_е=ди'ному же то'чiю} _е=диноро'дному сн~у 
правосла'внw _о=ц~а`, бг~осло'вяще тя` возвjьща'емъ, и= 
_е=ди'наго вjь'дяще w\т тебе` и=сходя'ща дх~а пра'ваго, 
соесте'ственна и= соприсносу'щна.

\cslemph{И='нъ} I=рмо'съ: W\т i=уде'и доше'дше:

Сп~се'нiе содjь'лалъ _е=си` посредjь` вселе'нныя, 
пр\оро'чески бж~е: на дре'во бо вознесе'нъ бы'въ, вся^ 
призва'лъ _е=си` вjь'рою зову'щыя: _о=т_е'цъ на'шихъ бж~е 
бл~гослове'нъ _е=си`.

Воскр~съ w\т гро'ба, jа='коже w\т сна` ще'дре, всjь'хъ 
и=зба'вилъ _е=си` w\т тли`, тва'рь же о_у=вjьря'ется 
а=п\слы проповjь'дающими воста'нiе: _о=тц_е'въ бж~е 
бл~гослове'нъ _е=си`.

Бг~оро'диченъ: Равнодjь'тельное ро'ждшему, 
равноси'льное сло'во и= соприсносу'щное, во о_у=тро'бjь 
дв~ы, _о=ц~а` бл~говоле'нiемъ, и= дх~а, созида'ется: 
_о=тц_е'въ на'шихъ бж~е бл~гослове'нъ _е=си`.

\cslemph{И='нъ} I=рмо'съ то'йже.

W\т дв~ственныхъ ложе'снъ вопло'щься, jа=ви'лся _е=си` 
на сп~се'нiе на'ше. тjь'мже твою` мт~рь вjь'дяще бц\ду 
правосла'внw зове'мъ: _о=тц_е'въ бж~е бл~гослове'нъ 
_е=си`.

Же'злъ прорасти'ла _е=си` дв~о, w\т ко'рене i=ессе'ова 
всебл~же'нная, пло'дъ цвjьтонося'щи сп~си'тельный, 
вjь'рою сн~у твоему` зову'щымъ: _о=тц_е'въ бж~е 
бл~гослове'нъ _е=си`.

Прему'дрости и=спо'лни всjь'хъ и= си'лы бж~е'ственныя, 
v=поста'сная прему'дросте вы'шнягw, бц\дею, вjь'рою 
тебjь` пою'щихъ: _о=тц_е'въ на'шихъ бж~е бл~гослове'нъ 
_е=си`.

\cslemph{Пjь'снь и~}

I=рмо'съ: Седмери'цею пе'щь халде'йскiй мучи'тель 
бг~очести^вымъ неи'стовнw разжже`, си'лою же лу'чшею 
сп~се'ны сiя^ ви'дjьвъ, творцу` и= и=зба'вителю вопiя'ше: 
_о='троцы бл~гослови'те, сщ~е'нницы воспо'йте, лю'дiе 
превозноси'те во вся^ вjь'ки.

I=и~сова бж~ества` пребж~е'ственная си'ла, въ на'съ 
бг~олjь'пнw возсiя'ла _е='сть: пло'тiю во вку'шъ за 
всjь'хъ сме'рть кр\стную, разруши` а='дову крjь'пость. 
_е=го'же непреста'ннw дjь'ти бл~гослови'те, сщ~е'нницы 
воспо'йте, лю'дiе превозноси'те во вся^ вjь'ки.

Распны'йся воста`, великовы'йный паде`, пады'й и= 
сокруше'нный и=спра'вися, тля` w\тве'ржена бы'сть, и= 
нетлjь'нiе процвjьте`: жи'знiю во ме'ртвенное поже'рто 
бы'сть. дjь'ти бл~гослови'те, сщ~е'нницы воспо'йте, 
лю'дiе превозноси'те во вся^ вjь'ки.

Тр\оченъ: Трисвjь'тлое бж~ество`, _е=ди'ну сiя'ющее 
зарю` w\т _е=ди'нагw трiv"поста'снагw _е=стества`, 
роди'теля безнача'льна: _е=диноесте'ственно же сло'во 
_о=ц~у`, и= сца'рствующаго _е=диносу'щнаго дх~а, дjь'ти 
бл~гослови'те, сщ~е'нницы воспо'йте, лю'дiе превозноси'те 
во вся^ вjь'ки.

\cslemph{И='нъ} I=рмо'съ: Побjьди'тели мучи'теля:

На дре'вjь ру'цjь мнjь` просте'ршаго w=бнаже'нному, 
призыва'юща мя`, свое'ю бл~гоwбра'зною согрjь'яти 
нагото'ю: бл~гослови'те вся^ дjьла` гд\сня, и= 
превозноси'те _е=го` во вjь'ки.

И=з\ъ преиспо'днягw а='да возне'сша мя` па'дшаго, и= 
высокопресто'льною сла'вою роди'теля поче'тшаго: 
бл~гослови'те вся^ дjьла` гд\сня гд\са, и= превозноси'те 
_е=го` во вjь'ки.

Бг~оро'диченъ: А=да'ма дв~о, па'дшагw о_у='бw 
jа=ви'лася _е=си` дщи`, бг~а же мт~и, w=бнови'вшагw мое` 
существо`: _е=го'же пое'мъ вся^ дjьла` jа='кw гд\са, и= 
превозно'симъ во вся^ вjь'ки.

\cslemph{И='нъ} I=рмо'съ: Цр~я` нб\снаго:

Сопроти'вныхъ разжж_е'нныя и= пламенови^дныя на на'съ 
о_у=гаси` стрjь'лы: jа='кw да пое'мъ тя` во вся^ вjь'ки.

Преесте'ственнjь содjь'теля и= сп~са, бг~а сло'ва 
родила` _е=си` дв~о: тjь'мже тя` пое'мъ, и= превозно'симъ 
во вся^ вjь'ки.

Просвjьти'тельную тя`, и= златоза'рную, все'льшiйся въ 
тя` свjь'тъ непристу'пный дв~о, показа` свjьщу` во вся^ 
вjь'ки.

Та'же пое'мъ пjь'снь бц\ды: Вели'читъ душа` моя` 
гд\са: Съ припjь'вомъ: Ч\стнjь'йшую херувi^мъ:

\cslemph{Пjь'снь f~}

I=рмо'съ: О_у=жасе'ся w= се'мъ нб~о, и= земли` 
о_у=диви'шася концы`, jа='кw бг~ъ jа=ви'ся человjь'кwмъ 
пло'тски, и= чре'во твое` бы'сть простра'ннjьйшее нб~съ. 
тjь'мъ тя` бц\ду, а='гг~лwвъ и= человjь^къ чинонача^лiя 
велича'ютъ.

Бж~е'ственнымъ и= безнача'льнымъ _е=стество'мъ про'стъ 
сы'й, сложи'лся _е=си` прiя'тiемъ пло'ти, въ тебjь` 
само'мъ сiю` соста'вивъ сло'ве бж~iй, и= пострада'въ 
jа='кw человjь^къ, пребы'лъ _е=си` кромjь` страсте'й 
jа='кw бг~ъ. тjь'мже тя` во двою` сущ_еству` 
нераздjь'льнw и= неслiя'ннw велича'емъ.

_О=ц~а` по существу` бж~е'ственному, jа='коже 
_е=стество'мъ бы'въ человjь'къ, ре'клъ _е=си` бг~а 
вы'шнiй рабw'мъ снизходя`, воскр~съ w\т гро'ба, 
бл~года'тiю _о=ц~а` земнорw'днымъ поло'жъ, и='же по 
_е=стеству` бг~а же и= вл\дку, съ ни'мже тя` вси` 
велича'емъ.

Бг~оро'диченъ: Jа=ви'лася _е=си`, _w дв~о мт~и бж~iя, 
па'че _е=стества` ро'ждши пло'тiю бг~а сло'ва, _е=го'же 
_о=ц~ъ w\тры'гну w\т се'рдца своегw` пре'жде всjь'хъ 
вjь^къ, jа='кw бл~гъ, _е=го'же ны'нjь и= тjьле'съ 
превы'шша разумjь'емъ, а='ще и= въ тjь'ло w=блече'ся.

\cslemph{И='нъ} I=рмо'съ: О_у=страши'ся:

Бж~iя тя` _е=стество'мъ о_у='бw сн~а, зача'таго во 
о_у=тро'бjь свjь'мы бг~ома'тере, и= бы'вшаго на'съ ра'ди 
человjь'ка, и= зря'ще тя` на кр\стjь` _е=стество'мъ 
о_у='бw стра'ждуща человjь'ческимъ, безстр\стна же jа='кw 
бг~а пребыва'юща велича'емъ.

Разруши'ся тьма` дря'хлая, w\т а='да бо возсiя` сл~нце 
пра'вды хр\сто'съ, земли` просвjьща'я вся^ концы`, сiя'я 
бж~ества` свjь'томъ, нб\сный человjь'къ, бг~ъ земны'й: 
_е=го'же во двою` _е=ст_еству` велича'емъ.

Напрязи`, и= о_у=спjьва'й, и= ца'рствуй, сн~е 
бг~ома'тере, i=сма'ильт_ескiя лю'ди покаря'я, борю'щыя 
ны`, jа='кw _о=ру'жiе непобjьди'мое, приходя'щымъ къ 
тебjь` кр\стъ съ копiе'мъ да'руя.

\cslemph{И='нъ} I=рмо'съ: Вои'стинну бц\ду:

Ра'дости и= весе'лiя и=спо'лнь _е='сть па'мять твоя`, 
приступа'ющымъ и=сцjьл_е'нiя точа'щи, и= бл~гоче'стнw 
бц\ду тя` возвjьща'ющымъ.

_Псалмы` тя` воспjьва'емъ бл~года'тная, и= немо'лчнw, 
_е='же ра'дуйся, прино'симъ: ты' бо и=сточи'ла _е=си` 
всjь^мъ ра'дость.

Красе'нъ бг~оро'дице прорасте` пло'дъ тво'й, не тли` 
причаща'ющымся хода'тайственъ, но жи'зни, вjь'рою тя` 
велича'ющымъ.

По катава'сiи _е=ктенiа` ма'лая.

Та'же, Ст~ъ гд\сь бг~ъ на'шъ. Посе'мъ 
_е=_ксапостiла'рiй.

На хвали'техъ стiхи^ры воскр\сны, гла'съ и~:

Стi'хъ: Сотвори'ти въ ни'хъ су'дъ напи'санъ: сла'ва 
сiя` бу'детъ всjь^мъ прп\дбнымъ _е=гw`.

Гд\си, а='ще и= суди'лищу предста'лъ _е=си` w\т 
пiла'та суди'мый, но не w\тступи'лъ _е=си` w\т пр\сто'ла 
со _о=ц~е'мъ сjьдя`: и= воскр~съ и=з\ъ ме'ртвыхъ, мi'ръ 
свободи'лъ _е=си` w\т рабо'ты чужда'гw, jа='кw ще'дръ и= 
чл~вjьколю'бецъ.

Стi'хъ: Хвали'те бг~а во ст~ы'хъ _е=гw`, хвали'те 
_е=го` во о_у=тверже'нiи си'лы _е=гw`.

Гд\си, _о=ру'жiе на дiа'вола кр\стъ тво'й да'лъ _е=си` 
на'мъ: трепе'щетъ бо и= трясе'тся, не терпя` взира'ти на 
си'лу _е=гw`: jа='кw м_е'ртвыя возставля'етъ, и= сме'рть 
о_у=праздни`. сегw` ра'ди покланя'емся погребе'нiю 
твоему` и= воста'нiю.

Стi'хъ: Хвали'те _е=го` на си'лахъ _е=гw`, хвали'те 
_е=го` по мно'жеству вели'чествiя _е=гw`.

Гд\си, а='ще и= jа='кw ме'ртва во гро'бjь i=уд_е'и 
положи'ша: но jа='кw цр~я` спя'ща во'ини тя` стрежа'ху, 
и= jа='кw живота` сокро'вище, печа'тiю печа'таша: но 
воскр\слъ _е=си`, и= по'далъ _е=си` нетлjь'нiе душа'мъ 
на'шымъ.

Стi'хъ: Хвали'те _е=го` во гла'сjь тру'бнjьмъ, 
хвали'те _е=го` во _псалти'ри и= гу'слехъ.

А='гг~лъ тво'й гд\си, воскр\снiе проповjь'давый, 
стра^жи о_у='бw о_у=страши`, жена'мъ же возгласи` 
глаго'ля: что` и='щете жива'гw съ ме'ртвыми; воскр~се 
бг~ъ сы'й, и= вселе'ннjьй жи'знь дарова`.

И='ны стiхи^ры а=натw'лiевы, гла'съ то'йже.

Стi'хъ: Хвали'те _е=го` въ тv"мпа'нjь и= ли'цjь, 
хвали'те _е=го` во стру'нахъ и= _о=рга'нjь.

Пострада'лъ _е=си` кр\сто'мъ, безстр\стный 
бж~ество'мъ, погребе'нiе прiя'лъ _е=си` тридне'вное, да 
на'съ свободи'ши w\т рабо'ты вра'жiя, и= w=безсме'ртивъ 
w=животвори'ши на'съ хр\сте` бж~е, воскр\снiемъ твои'мъ 
чл~вjьколю'бче.

Стi'хъ: Хвали'те _е=го` въ кv"мва'лjьхъ 
доброгла'сныхъ, хвали'те _е=го` въ кv"мва'лjьхъ 
восклица'нiя: вся'кое дыха'нiе да хва'литъ гд\са.

Покланя'юся, и= сла'влю, и= воспjьва'ю хр\сте` твое` 
и=з\ъ гро'ба воскр\снiе, и='мже свободи'лъ _е=си` на'съ 
w\т а='довыхъ нерjьши'мыхъ о_у='зъ: и= дарова'лъ _е=си` 
мi'рови jа='кw бг~ъ жи'знь вjь'чную, и= ве'лiю мл\сть.

Стi'хъ: Воскр\сни` гд\си бж~е мо'й, да вознесе'тся 
рука` твоя`, не забу'ди о_у=бо'гихъ твои'хъ до конца`.

Жизнопрiе'мнагw твоегw` гро'ба стрегу'ще 
законопресту'пнiи, съ кустwдi'ею запеча'таша тогда`: ты' 
же jа='кw безсме'ртенъ бг~ъ и= всеси'ленъ, воскр\слъ 
_е=си` тридне'венъ.

Стi'хъ: И=сповjь'мся тебjь` гд\си, всjь'мъ се'рдцемъ 
мои'мъ, повjь'мъ вся^ чудеса` твоя^.

Доше'дшу ти` во врата` а='дwва, гд\си, и= сiя^ 
сокруши'вшу, плjь'нникъ си'це вопiя'ше: кто` се'й 
_е='сть, jа='кw не w=сужда'ется въ преиспо'днихъ земли`, 
но и= jа='кw сjь'нь разруши` сме'ртное о_у=зи'лище; 
прiя'хъ того` jа='кw ме'ртва, и= трепе'щу jа='кw бг~а. 
всеси'льне поми'луй на'съ.

Сла'ва, стiхи'ра _е=v\гльская. И= ны'нjь, 
бг~оро'диченъ: Пребл~гослове'нна _е=си` бц\де дв~о: 
Славосло'вiе вели'кое.

Та'же, тропа'рь воскр\снъ:

Воскр~съ и=з\ъ гро'ба, и= о_у='зы растерза'лъ _е=си` 
а='да, разруши'лъ _е=си` w=сужде'нiе сме'рти гд\си, вся^ 
w\т сjьте'й врага` и=зба'вивый: jа=ви'вый же себе` 
а=п\слwмъ твои^мъ, посла'лъ _е=си` я=` на про'повjьдь, и= 
тjь'ми ми'ръ тво'й по'далъ _е=си` вселе'ннjьй, _е=ди'не 
многомл\стиве.

Та'же _е=ктенiи`: и= w\тпу'стъ.
