%<%[Въ суббw'ту ве'чера,  на ма'лjьй вече'рни%],%>

%<на Г%>д\си воззва'хъ, %<поста'вимъ стiхw'въ д~: и= 
пое'мъ стiхи^ры воскр\сны _о=смогла'сника г~, повторя'юще 
а~-й стi'хъ, гла'съ а~. Творе'нiе прп\дбнагw _о=тца` 
на'шегw i=wа'нна дамаски'на%>

%<Стi'хъ: W\т%> стра'жи о_у='треннiя до но'щи, w\т 
стра'жи о_у='треннiя, да о_у=пова'етъ i=и~ль на гд\са.

%<В%>еч_е'рнiя на'шя мл~твы, прiими` ст~ы'й гд\си, и= 
пода'ждь на'мъ w=ставле'нiе грjьхw'въ, jа='кw _е=ди'нъ 
_е=си` jа=вле'й въ мi'рjь воскр\снiе.

%<W=%>быди'те лю'дiе сiw'нъ, и= w=б\ъими'те _е=го`, и= 
дади'те сла'ву въ не'мъ воскр\сшему и=з\ъ ме'ртвыхъ: 
jа='кw то'й _е='сть бг~ъ на'шъ, и=збавле'й на'съ w\т 
беззако'нiй на'шихъ.

%<П%>рiиди'те лю'дiе, воспои'мъ и= поклони'мся хр\сту`, 
сла'вяще _е=гw` и=з\ъ ме'ртвыхъ воскр\снiе: jа='кw то'й 
_е='сть бг~ъ на'шъ, w\т пре'лести вра'жiя мi'ръ 
и=збавле'й.

%<Сла'ва, и= ны'нjь: бг~оро'диченъ догма'тiкъ:%>

%<Д%>в\ственное торжество` дне'сь бра'тiе, да взыгра'ется 
тва'рь, да ликовству'етъ человjь'чество: созва' бо на'съ 
(с. 22) ст~а'я бц\да, нескве'рное сокро'вище дв\ства: 
слове'сный втора'гw а=да'ма ра'й: храни'лище соедине'нiя 
дву` _е=ст_еству`: торжество` спаси'тельнагw примире'нiя: 
черто'гъ, въ не'мже сло'во о_у=невjь'стивый пло'ть 
вои'стинну: ле'гкiй _о='блакъ, и='же над\ъ херувi'мы 
су'щаго, съ тjь'ломъ носи'вшiй. тоя` мл~твами хр\сте` 
бж~е сп~си` ду'шы на'шя.

%<Та'же, С%>вjь'те ти'хiй: %<Прокi'менъ: Г%>д\сь 
воцр~и'ся: со стiхи` _е=гw`. %<И= по С%>подо'би гд\си въ 
ве'черъ се'й: %<I=ере'й _е=ктенiи` не глаго'летъ, но 
пое'мъ на стiхо'внjь стiхи'ру воскр\сну:%>

%<С%>тр\стiю твое'ю хр\сте` w\т страсте'й свободи'хомся, 
и= воскр\снiемъ твои'мъ и=з\ъ и=стлjь'нiя и=зба'вихомся, 
гд\си сла'ва тебjь`.

%<И= и='ны стiхи^ры прест~ы'я бц\ды. Подо'бенъ: 
Н%>б\сныхъ чинw'въ:

%<Стi'хъ: П%>омяну` и='мя твое` во вся'комъ ро'дjь и= 
ро'дjь.

%<П%>репросла'влена _е=си` въ ро'дjь родw'въ, дв~о мт~и 
_о=трокови'це бц\де мр~i'е, мi'ра предста'тельство, 
ро'ждши пло'тiю сн~а безнача'льнагw _о=ц~а`, 
соприсносу'щна же дх~у вои'стинну: _е=го'же моли` 
сп~сти'ся на'мъ.

%<Стi'хъ: С%>лы'ши дщи` и= ви'ждь, и= приклони` о_у='хо 
твое`.

%<С%>одержи'мiи скорбьми` ненача'емыми ч\стая, тя` 
предста'тельство _е=ди'но и=му'ще дв~о, вопiе'мъ 
бл~года'рнw: сп~си' ны всест~а'я бг~оневjь'стная: ты' бо 
_е=си` мi'ра прибjь'жище, и= заступле'нiе ро'да на'шегw.

%<Стi'хъ: Л%>ицу` твоему` помо'лятся бога'тiи лю'дстiи.

%<W=%>бнови'ся мi'ръ въ рж\ствjь` твое'мъ 
бг~ороди'тельнице _о=трокови'це, вjь'рныхъ сп~се'нiе, и= 
неусыпа'емая предста'тельнице бл~гоче'стнw моля'щихъ тя` 
преч\стая, не преста'й моля'щи прилjь'жнw w= всjь'хъ 
пою'щихъ тя`.

%<Сла'ва, и= ны'нjь: догма'тiкъ, гла'съ то'йже:%>

%<_О='%>блакъ тя` свjь'та присносу'щнагw дв~о, пр\оро'къ 
и=менова`: и=з\ъ тебе' бо jа='кw до'ждь на руно` сше'дъ 
сло'во _о='ч~ее, и= и=з\ъ тебе` возсiя'вый, мi'ръ 
просвjьти`, пре'лесть о_у=праздни` хр\сто'съ бг~ъ на'шъ. 
того` моля'щи прилjь'жнw (с. 23) прест~а'я, мо'лимся, не 
преста'й w= на'съ, и=`же и='стинную бц\ду 
и=сповjь'дающихъ тя`.

%<Та'же: Н%>ы'нjь w\тпуща'еши: Т%<рист~о'е. по _О='%>ч~е 
на'шъ: %<тропа'рь воскр\сный. Сла'ва, и= ны'нjь, 
бг~оро'диченъ _е=гw`. _Е=ктенiа` ма'лая, и= w\тпу'стъ.%>
