%<%[Въ недjь'лю на о_у='трени%],%>

%<на Б%>г~ъ гд\сь %<тропа'рь: _Е=%>гда` снизше'лъ 
_е=си`: %<Сла'ва, и= ны'нjь: В%>ся^ па'че смы'сла:

%<По а~-мъ стiхосло'вiи, сjьда'льны воскр\сны, гла'съ 
в~:%>

%<Б%>л~гоwбра'зный i=w'сифъ, съ дре'ва сне'мъ 
преч\стое твое` тjь'ло, плащани'цею ч\стою w=бви'въ, и= 
бл~гоуха'ньми во гро'бjь но'вjь закры'въ положи`: но 
тридне'венъ воскр\слъ _е=си` гд\си, подая` мi'рови ве'лiю 
мл\сть.

%<В%>оскр\сни` гд\си бж~е мо'й, да вознесе'тся рука` 
твоя`, не забу'ди о_у=бо'гихъ твои'хъ до конца`.

%<М%>v"роно'сицамъ жена'мъ при гро'бjь предста'въ 
а='гг~лъ вопiя'ше: мv^ра м_е'ртвымъ су'ть прили^чна, 
хр\сто'съ же и=стлjь'нiя (с. 211) jа=ви'ся чу'ждь. но 
возопi'йте: воскр~се гд\сь, подая` мi'рови ве'лiю мл\сть.

%<Сла'ва, и= ны'нjь, бг~оро'диченъ: П%>репросла'вленна 
_е=си` бц\де дв~о, пое'мъ тя`: кр\сто'мъ бо сн~а твоегw` 
низложи'ся а='дъ, и= сме'рть о_у=мертви'ся, 
о_у=мерщвле'ннiи воста'хомъ, и= живота` сподо'бихомся: 
ра'й воспрiя'хомъ дре'внее наслажде'нiе. тjь'мъ 
бл~годаря'ще, славосло'вимъ jа='кw держа'внаго хр\ста` 
бг~а на'шего, и= _е=ди'наго многомл\стиваго.

%<По в~-мъ стiхосло'вiи сjьда'льны, гла'съ в~.%>

%<К%>а'мень гро'бный запеча'тати не возбрани'въ, 
ка'мень вjь'ры воскр~съ по'далъ _е=си` всjь^мъ, гд\си 
сла'ва тебjь`.

%<Стi'хъ: И=%>сповjь'мся тебjь` гд\си всjь'мъ 
се'рдцемъ мои'мъ, повjь'мъ вся^ чудеса` твоя^.

%<О_у=%>ч~нкw'въ твои'хъ ли'къ съ мv"роно'сицами 
жена'ми ра'дуется согла'снw: _о='бщiй бо пра'здникъ съ 
ни'ми @пра'зднуютъ@ {@пра'зднуемъ@}, въ сла'ву и= че'сть 
твоегw` воскр\снiя. и= тjь'ми вопiе'мъ ти`, 
чл~вjьколю'бче гд\си: лю'демъ твои^мъ пода'ждь ве'лiю 
мл\сть.

%<Сла'ва, и= ны'нjь, бг~оро'диченъ: 
П%>ребл~гослове'нна _е=си` бц\де дв~о, вопло'щшимъ бо ся 
и=з\ъ тебе` а='дъ плjьни'ся, а=да'мъ воззва'ся, кля'тва 
потреби'ся, _е='vа свободи'ся, сме'рть о_у=мертви'ся, и= 
мы` w=жи'хомъ. тjь'мъ воспjьва'юще вопiе'мъ: 
бл~гослове'нъ хр\сто'съ бг~ъ бл~говоли'вый та'кw, сла'ва 
тебjь`.

%<По непоро'чнахъ v=пакои`, гла'съ в~.%>

%<П%>о стр\сти ше'дшя во гро'бъ ж_ены`, во _е='же 
пома'зати тjь'ло твое` хр\сте` бж~е, ви'дjьша а='гг~лы во 
гро'бjь, и= о_у=жасо'шася: гла'съ бо слы'шаху w\т ни'хъ, 
jа='кw воскр~се гд\сь, да'руя мi'рови ве'лiю мл\сть.

%<Степ_е'нна, гла'съ в~.%>

%<%[А=нтiфw'нъ%] а~: И='%>хже стiхи` повторя'юще 
пое'мъ:

%<Н%>а нб~о _о='чи пуща'ю моегw` се'рдца, къ тебjь` 
сп~се, сп~си' мя твои'мъ w=сiя'нiемъ. (с. 212)

%<П%>оми'луй на'съ согрjьша'ющихъ тебjь` мно'гw, на 
вся'кiй ча'съ, _w хр\сте` мо'й, и= да'ждь w='бразъ 
пре'жде конца` пока'ятися тебjь`.

%<Сла'ва: С%>т~о'му дх~у, _е='же ца'рствовати 
подоба'етъ, w=свяща'ти, подвиза'ти тва'рь: бг~ъ бо 
_е='сть, _е=диносу'щенъ _о=ц~у` и= сло'ву.

%<И= ны'нjь, то'йже.%>

%<%[А=нтiфw'нъ%] в~:%>

%<А='%>ще не гд\сь бы бы'лъ въ на'съ, кто` дово'ленъ 
цjь'лъ сохране'нъ бы'ти w\т врага`, ку'пнw и= 
человjькоубi'йцы;

%<З%>убw'мъ и='хъ не преда'ждь сп~се, твоегw` раба`: 
и='бо льво'вымъ w='бразомъ на мя` подвиза'ются врази` 
мои`.

%<Сла'ва: С%>т~о'му дх~у живонача'лiе и= че'сть: вся^ 
бо созда^нная, jа='кw бг~ъ сы'й 
@си'лою@{@о_у=крjьпля'етъ@} соблюда'етъ во _о=ц~jь`, 
сн~омъ же.

%<И= ны'нjь, то'йже.%>

%<%[А=нтiфw'нъ%] г~:%>

%<Н%>адjь'ющiися на гд\са, о_у=подо'бишася горjь` 
ст~jь'й: и=`же ника'коже дви'жутся прило'ги вра'жiими.

%<В%>ъ беззако'нiи ру'къ свои'хъ да не про'струтъ 
бж~е'ственнjь живу'щiи: не w=ставля'етъ во хр\сто'съ 
жезла` на жре'бiи свое'мъ.

%<Сла'ва: С%>т~ы'мъ дх~омъ то'чится вся'ка 
прему'дрость: w\тсю'ду бл~года'ть а=п\слwмъ, и= 
страда'льчествы вjьнча'ются мч~нцы, и= пр\оро'цы зря'тъ.

%<И= ны'нjь, то'йже.%>

%<Прокi'менъ, гла'съ в~: В%>оста'ни гд\си бж~е мо'й 
повелjь'нiемъ, и='мже заповjь'далъ _е=си`, и= со'нмъ 
люде'й w=бы'детъ тя`. Стi'хъ: Гд\си бж~е мо'й на тя` 
о_у=пова'хъ, сп~си' мя.

%<В%>ся'кое дыха'нiе: %<_Е=v\глiе воскре'сно, и= 
прw'чая по ря'ду. В%>оскр\снiе хр\сто'во: %<_псало'мъ 
н~.%>

%<Канw'нъ воскр\снъ. Гла'съ в~.%>

%<%[Пjь'снь а~%]%>

%<I=рмо'съ: В%>о глубинjь` постла` и=ногда`, 
фараwни'тское всево'инство (с. 213)преwруже'нная си'ла, 
вопло'щшееся же сло'во всеsло'бный грjь'хъ потреби'ло 
_е='сть, препросла'вленный гд\сь, сла'внw бо просла'вися.

%<Припjь'въ: С%>ла'ва гд\си, ст~о'му воскр\снiю 
твоему`.

%<М%>iрскi'й кня'зь бл~же, _е=му'же написа'хомся, 
за'повjьди твоея` не послу'шавше, кр\сто'мъ твои'мъ 
w=суди'ся: прирази'ся бо ти` jа='кw сме'ртну: w\тпаде' же 
вла'сти твоея` держа'вою, и= немощны'й w=бличи'ся.

%<И=%>зба'витель ро'да человjь'ческагw, и= 
нетлjь'нному животу` нача'льникъ въ мi'ръ прише'лъ 
_е=си`: воскр\снiемъ бо твои'мъ раздра'лъ _е=си` 
см_е'ртныя пел_ены`, _е='же славосло'вимъ вси`: сла'внw 
бо просла'вися.

%<Бг~оро'диченъ: П%>ревы'шши jа=ви'лася _е=си` ч\стая 
приснодв~о, вся'кiя неви'димыя же и= ви'димыя тва'ри: 
зижди'теля бо родила` _е=си`, jа='кw бл~говоли` 
воплоти'тися во о_у=тро'бjь твое'й, _е=го'же со 
дерзнове'нiемъ моли`, сп~сти` ду'шы на'шя.

%<Другi'й канw'нъ крестовоскр\снъ, [_е=гw'же 
краестро'чiе: П%>ою` хвалу` живоно'сному сло'ву.%<] 
Гла'съ в~.%>

%<%[Пjь'снь а~%]%>

%<I=рмо'съ: Н%>етре'ну, не_обы'чну, немо'креннw 
морску'ю ше'ствовавъ стезю`, и=збра'нный вопiя'ше i=и~ль: 
гд\севи пои'мъ, jа='кw просла'вися.

%<С%>и'ла немощны'хъ, воскр\снiе па'дшихъ, и= 
нетлjь'нiе о_у=ме'ршихъ бы'лъ _е=си` хр\сте`, пло'ти 
твоея` стр\стiю: jа='кw просла'вися.

%<О_у=%>ще'дри па'дшiй w='бразъ, и= воскр~си` 
сокруше'нный, содjь'тель бг~ъ и= w=бнови'тель 
о_у=мерщвле'нъ бы'въ всjь'хъ w=живи`: jа='кw просла'вися.

%<И='нъ канw'нъ прест~jь'й бц\дjь, [_е=гw'же 
краестро'чiе: П%>ою` хвалу` живоно'снjьй 
_о=трокови'цjь.%<] Гла'съ в~.%>

%<Пjь'снь а~. I=рмо'съ то'йже.%>

%<Н%>евеще'ственная дре'вле лjь'ствица, и= стра'ннw 
w=леденjь'вшiй пу'ть мо'ря, твое` jа=вля'ше рж\ство` 
ч\стая, _е='же пое'мъ вси`: jа='кw просла'вися. (с. 214)

%<С%>и'ла вы'шнягw, v=поста'сь соверше'нная, бж~iя 
му'дрость, вопло'щшися преч\стая, и=з\ъ тебе`, къ 
человjь'кwмъ @бесjь'дова@{@прибли'жися@} jа='кw 
просла'вися.

%<П%>ро'йде сквозjь` две'рь непрохо'дную, заключе'нныя 
о_у=тро'бы твоея`, пра'вды сл~нце ч\стая, и= мi'рови 
возсiя`: jа='кw просла'вися.

%<Катава'сiа: W\т%>ве'рзу о_у=ста` моя^:

%<%[Пjь'снь г~%]%>

%<I=рмо'съ: П%>роцвjьла` _е='сть пусты'ня, jа='кw 
крi'нъ, гд\си, jа=зы'ческая неплодя'щая цр~ковь, 
прише'ствiемъ твои'мъ, въ не'йже о_у=тверди'ся мое` 
се'рдце.

%<Т%>ва'рь въ стр\сти твое'й и=змjьня'шеся, зря'щи тя` 
въ нище'тнjь w='бразjь w\т беззако'нныхъ поруга'ема, 
w=снова'вшаго вся^ бж~е'ственнымъ манове'нiемъ.

%<W\т%> пе'рсти по w='бразу мя` руко'ю твое'ю созда'лъ 
_е=си`: и= сокруше'нна па'ки въ пе'рсть сме'ртную за 
грjь'хъ, хр\сте` соше'дъ во а='дъ, совоскр~си'лъ _е=си`.

%<Бг~оро'диченъ: Ч%>и'ни о_у=диви'шася а='гг~льстiи 
преч\стая, и= человjь'ч_еская о_у=страши'шася сердца` w= 
рж\ствjь` твое'мъ: тjь'мже тя` бц\ду вjь'рою чти'мъ.

%<%[И='нъ%]%>

%<I=рмо'съ: Л%>у'къ сокруши'ся си'льныхъ держа'вою 
твое'ю хр\сте`, и= си'лою немощству'ющiи препоя'сашася.

%<И='%>же всjь'хъ вы'шши хр\сто'съ, о_у=ма'лися 
ма'лымъ чи'мъ, стр\стiю плотско'ю, а='гг~льскагw 
_е=стества`.

%<М%>е'ртвъ со беззако'нными вмjьни'вся, сiя'я жена'мъ 
вjьнце'мъ сла'вы хр\сте`, jа=ви'лся _е=си` w\т 
воскр\снiя.

%<И='нъ. I=рмо'съ то'йже.%>

%<И='%>же вре'мене превы'шшiй вся'кагw, jа='кw 
вре'мен_емъ творе'цъ, и=з\ъ тебе` дв~о во'лею мл\днецъ 
созда'ся.

%<Ч%>ре'во простра'ннjьйшее нб~съ воспои'мъ, и='мже 
а=да'мъ на нб~сjь'хъ ра'дуяся живе'тъ. (с. 215)

%<%[Пjь'снь д~%]%>

%<I=рмо'съ: П%>рише'лъ _е=си` w\т дв~ы, не хода'тай, 
ни а='гг~лъ, но са'мъ @гд\си@{@гд\сь@} вопло'щься, и= 
сп~слъ _е=си` всего' мя человjь'ка. тjь'мъ зову' ти: 
сла'ва си'лjь твое'й гд\си.

%<П%>редстои'ши суди'щу jа='кw суди'мь, бж~е мо'й, не 
вопiя` вл\дко, су'дъ и=знося` jа=зы'кwмъ: и=`мже стр\стiю 
твое'ю хр\сте` вселе'ннjьй содjь'лалъ _е=си` сп~се'нiе.

%<С%>тр\стiю твое'ю хр\сте`, врагу` w=скудjь'ша 
_о=ру'жiя, проти'внымъ же _е='же во а='дъ схожде'нiемъ 
твои'мъ, гра'ды разруши'шася, и= мучи'телева де'рзость 
низложе'на бы'сть.

%<Бг~оро'диченъ: Т%>я` приста'нище сп~се'нiя, и= 
стjь'ну недви'жиму, бц\де вл\дчце, вси` вjь'рнiи свjь'мы: 
ты' бо мл~твами твои'ми и=збавля'еши w\т бjь'дъ ду'шы 
на'шя.

%<%[И='нъ%]%>

%<I=рмо'съ: О_у=%>слы'шахъ гд\си, сла'вное твое` 
смотре'нiе, и= просла'вихъ чл~вjьколю'бче, непостижи'мую 
твою` си'лу.

%<В%>и'дjьвши на дре'вjь тя` хр\сте` пригвожде'на 
дв~а, jа='же неболjь'зненнw тя` ро'ждшая, мт~рски 
болjь^зни претерпjь`.

%<П%>обjьжде'на бы'сть сме'рть, ме'ртвъ плjьня'етъ 
а='дwва врата`: всея'дцу бо разо'ршуся, jа=`же па'че 
_е=стества` вся^ ми` дарова'шася.

%<И='нъ. I=рмо'съ то'йже.%>

%<С%>е` превознесе'ся бж~е'ственная гора` до'му 
гд\сня, превы'шше си'лъ, бг~омт~рь jа='вственнjьйше.

%<З%>ако'нwвъ _е=сте'ственныхъ кромjь`, _е=ди'на дв~о 
ро'ждши влады'чествующаго тва'рiю, сподо'билася _е=си` 
бж~е'ственнагw зва'нiя.

%<%[Пjь'снь _е~%]%>

%<I=рмо'съ: Х%>ода'тай бг~у и= человjь'кwмъ бы'лъ 
_е=си` хр\сте` бж~е: тобо'ю бо вл\дко, къ 
свjьтонача'льнику _о=ц~у` твоему`, w\т но'щи 
невjь'дjьнiя, приведе'нiе и='мамы. (с. 216)

%<Jа='%>кw ке'дры хр\сте`, jа=зы'кwвъ шата'нiе 
сокруши'лъ _е=си` во'лею вл\дко: jа='кw и=зво'лилъ _е=си` 
на кv"парi'сjь, и= на пе'vкjь, и= ке'дрjь, пло'тiю 
совозвыша'емь.

%<В%>ъ ро'вjь хр\сте` преиспо'днjьмъ положи'ша тя` 
бездыха'нна ме'ртва: но твое'ю jа='звою забве'нныя 
м_е'ртвыя, и=`же во гробjь'хъ спя'щыя, о_у=я'звенъ съ 
собо'ю воскр~си'лъ _е=си`.

%<Бг~оро'диченъ: М%>оли` сн~а твоего` и= гд\са, дв~о 
ч\стая, на тя` о_у=пова'ющымъ ми'ръ дарова'ти, 
плjьн_е'ннымъ и=збавле'нiе, w\т сопроти'вныхъ настоя'нiй.

%<%[И='нъ%]%>

%<I=рмо'съ: О_у='%>гль и=са'iи проявле'йся, сл~нце 
и=з\ъ дjь'вственныя о_у=тро'бы возсiя`, во тьмjь` 
заблу'ждшымъ, бг~оразу'мiя просвjьще'нiе да'руя.

%<П%>ости'ти w\твергi'йся а=да'мъ, вкуша'етъ 
смертоно'снагw дре'ва пе'рвый: но сегw` грjь'хъ 
потребля'етъ, распны'йся вторы'й.

%<_Е=%>стество'мъ человjь'ческимъ стра'стенъ же и= 
сме'ртенъ бы'лъ _е=си`, и='же безстра'стный 
невеще'ственнымъ бж~ество'мъ, w=безтли'вый 
о_у=мерщвл_е'нныя хр\сте`, w\т преиспо'днихъ а='довыхъ 
воскр~си'лъ _е=си`.

%<И='нъ. I=рмо'съ то'йже.%>

%<_О='%>блацы весе'лiя сла'дость кропи'те су'щымъ на 
земли`: jа='кw _о=троча` даде'ся, и='же сы'й пре'жде 
вjь^къ, w\т дв~ы вопло'щься бг~ъ на'шъ.

%<Ж%>итiю` и= пло'ти мое'й свjь'тъ возсiя`, и= 
дря'хлость грjьха` разруши`: напослjь'докъ w\т дв~ы без\ъ 
сjь'мене вопло'щься вы'шнiй.

%<%[Пjь'снь s~%]%>

%<I=рмо'съ: В%>ъ бе'зднjь грjьхо'внjьй валя'яся, 
неизслjь'дную милосе'рдiя твоегw` призыва'ю бе'здну: w\т 
тли` бж~е мя` возведи`.

%<Jа='%>кw sлодjь'й пра'ведникъ w=суди'ся и= со 
беззако'нными на дре'вjь пригвозди'ся, пови^ннымъ 
w=ставле'нiе свое'ю да'руя кро'вiю.

%<_Е=%>ди'нjьмъ о_у='бw чл~вjь'комъ, пе'рвымъ 
а=да'момъ дре'вле въ мi'ръ вни'де сме'рть, и= _е=ди'нjьмъ 
воскр\снiе сн~омъ бж~iимъ jа=ви'ся. (с. 217)

%<Бг~оро'диченъ: Н%>еискусому'жнw дв~о родила` _е=си`, 
и= вjь'чнуеши дв~а, jа=вля'ющи и='стиннагw бж~ества`, 
сн~а и= бг~а твоегw` w='бразы.

%<%[И='нъ%]%>

%<I=рмо'съ: Г%>ла'съ глагw'лъ моле'бныхъ w\т 
болjь'зненныя вл\дко, души` о_у=слы'шавъ, w\т лю'тыхъ мя` 
и=зба'ви: _е=ди'нъ бо _е=си` на'шегw сп~се'нiя вино'венъ.

%<Б%>люсти'тели положи'лъ _е=си` па'дшему, херувi'мы 
дре'ва живо'тнагw, но ви'дjьвше тя`, дв_е'ри 
w\тверзо'шася: jа=ви'лся бо _е=си` путетворя` разбо'йнику 
въ ра'й.

%<П%>у'стъ а='дъ и= и=спрове'рженъ бы'сть сме'ртiю 
_е=ди'нагw: _е='же бо мно'гое бога'тство 
сокро'виществова, _е=ди'нъ w= всjь'хъ на'съ хр\сто'съ 
и=стощи'лъ _е='сть.

%<И='нъ. I=рмо'съ то'йже.%>

%<_Е=%>стество` человjь'ческое рабо'тающее грjьху`, 
вл\дчце ч\стая, тобо'ю свобо'ду о_у=лучи`: тво'й бо сн~ъ, 
jа='кw а='гнецъ, за всjь'хъ закала'ется.

%<В%>опiе'мъ ти` вси` и='стиннjьй бг~омт~ри, 
прогнjь'вавшыя рабы^ и=зба'ви: _е=ди'на бо дерзнове'нiе 
къ сн~у и='маши.

%<Конда'къ, гла'съ в~: В%>оскр\слъ _е=си` w\т гро'ба 
всеси'льне сп~се, и= а='дъ ви'дjьвъ чу'до, о_у=жасе'ся, 
и= ме'ртвiи воста'ша: тва'рь же ви'дящи сра'дуется 
тебjь`, и= а=да'мъ свесели'тся, и= мi'ръ сп~се мо'й 
воспjьва'етъ тя` при'снw.

%<I='косъ: Т%>ы` _е=си` свjь'тъ w=мрач_е'ннымъ, ты` 
_е=си` воскр\снiе всjь'хъ, и= живо'тъ человjь'кwвъ, и= 
всjь'хъ совоскр~си'лъ _е=си`, сме'ртную держа'ву сп~се 
разо'рь, и= а='дwва врата` сокруши'вый сло'ве: и= 
ме'ртвiи ви'дjьвше чу'до чудя'хуся, и= вся'ка тва'рь 
сра'дуется w= воскр~се'нiи твое'мъ чл~вjьколю'бче. 
тjь'мже и= вси` сла'вимъ и= пое'мъ твое` снизхожде'нiе, 
и= мi'ръ сп~се мо'й, воспjьва'етъ тя` при'снw.

%<%[Пjь'снь з~%]%>

%<I=рмо'съ: Б%>г~опроти'вное велjь'нiе 
беззако'ннующагw мучи'теля высо'къ пла'мень вознесло` 
_е='сть: хр\сто'съ же простре` бг~очести^вымъ _о=трокw'мъ 
ро'су дх~о'вную, сы'й бл~гослове'нъ, и= препросла'вленъ. 
(с. 218)

%<Н%>е терпjь'лъ _е=си` вл\дко, за бл~гоутро'бiе 
сме'ртiю человjь'ка зрjь'ти му'чима, но прише'лъ и= 
сп~слъ _е=си` твое'ю кро'вiю, человjь'къ бы'въ: сы'й 
бл~гослове'нъ, и= препросла'вленъ бг~ъ _о=т_е'цъ на'шихъ.

%<В%>и'дjьвше тя` хр\сте`, w=болче'на во _о=де'жду 
w\тмще'нiя, о_у=жасо'шася вра'тницы а='дwвы: безу'мнаго 
бо мучи'теля раба`, вл\дко, прише'лъ _е=си` о_у=би'ти: 
сы'й бл~гослове'нъ и= препросла'венъ бг~ъ _о=те'цъ 
на'шихъ.

%<Бг~оро'диченъ: С%>т~ы'хъ ст~jь'йшую тя` разумjь'емъ, 
jа='кw _е=ди'ну ро'ждшую бг~а непремjь'ннаго, дв~о 
нескве'рная, мт~и безневjь'стная: всjь^мъ бо вjь^рнымъ 
и=сточи'ла _е=си` нетлjь'нiе, бж~е'ственнымъ рж\ство'мъ 
твои'мъ.

%<%[И='нъ%]%>

%<I=рмо'съ: В%>jьтi'и jа=ви'шася _о='троцы, 
любому'дрjьйшiи дре'вле: w\т бг~опрiя'тныя бо души` 
бг~осло'вяще, о_у=стна'ми поя'ху: пребж~е'ственный 
_о=тц_е'въ и= на'шъ бж~е бл~гослове'нъ _е=си`.

%<W=%>суди` пра'_отца дре'вле во _е=де'мjь 
преслуша'нiе: но во'лею суди'мь бы'сть, престу'пльшему 
разрjьша'я прегрjьш_е'нiя, пребж~е'ственный _о=тц_е'въ 
бг~ъ, и= препросла'вленъ.

%<С%>п~слъ _е=си` о_у=я'звеннаго я=зы'комъ, за'вистiю 
чл~вjькоубi'йцы, @во _е=де'мjь во'льнымъ о_у=грызе'нiемъ: 
во'льною бо стр\стiю@{@во _е=де'мjь: во'льное бо 
о_у=грызе'нiе во'льною стр\стiю@} и=сцjьли'лъ _е=си`, 
пребж~е'ственный _о=тц_е'въ бг~ъ, и= препросла'вленъ.

%<Бг~оро'диченъ: Х%>одя'ща мя` въ сjь'ни сме'ртнjьй 
призва'лъ _е=си` къ свjь'ту, темнозра'чный а='дъ, 
блиста'нiемъ w=бло'жъ бж~ества`, пребж~е'ственный 
_о=тц_е'въ бг~ъ, и= препросла'вленъ.

%<И='нъ. I=рмо'съ то'йже.%>

%<З%>ря'ше въ нощи` о_у='бw i=а'кwвъ, jа='кw въ 
гада'нiи бг~а воплоще'нна, и=з\ъ тебе' же свjь'тлостiю 
jа=ви'ся пою'щымъ: пребж~е'ственный _о=тц_е'въ бг~ъ, и= 
препросла'вленъ.

%<З%>на'м_енiя jа=`же въ тебjь` неизрече'ннагw 
проявля'я сня'тiя, со i=а'кwвомъ бо'рется: и='мже во'лею 
соедини'ся человjь'кwмъ ч\стая: пребж~е'ственный 
_о=тц_е'въ бг~ъ, и= препросла'вленъ.(с. 219)

%<М%>е'рзокъ и='же не проповjь'дуетъ тя` дв~ы сн~а, 
_е=ди'наго w\т препjь'тыя тр\оцы несумнjь'нною 
@вjь'рою@{@мы'слiю@}, и= я=зы'комъ вопiя`: 
пребж~е'ственный _о=тц_е'въ бг~ъ, и= препросла'вленъ.

%<%[Пjь'снь и~%]%>

%<I=рмо'съ: П%>е'щь и=ногда` _о='гненная въ 
вавv"лw'нjь дjь^йства раздjьля'ше, бж~iимъ велjь'нiемъ 
халд_е'и w=паля'ющая, вjь^рныя же w=роша'ющая, пою'щыя: 
бл~гослови'те вся^ дjьла` гд\сня гд\са.

%<К%>ро'вiю твое'ю хр\сте`, w=червле'но пло'ти твоея` 
зря'ще w=дjья'нiе тре'петомъ о_у=жаса'хуся мно'гому 
твоему` долготерпjь'нiю, а='гг~льстiи чи'ни, зову'ще: 
бл~гослови'те вся^ дjьла` гд\сня гд\са.

%<Т%>ы` мое` сме'ртное w=дjь'ялъ _е=си` ще'дре, въ 
безсме'ртiе воста'нiемъ твои'мъ. тjь'мже веселя'щеся 
бл~года'рственнw воспjьва'ютъ тя` и=збра'ннiи лю'дiе 
хр\сте`, зову'ще тебjь`: поже'рта бы'сть вои'стинну 
сме'рть побjь'дою.

%<Бг~оро'диченъ: Т%>ы` и='же _о=ц~у` неразлу'чнаго, во 
о_у=тро'бjь бг~ому'жнw пожи'вша безсjь'меннw зачала` 
_е=си`, и= неизрече'ннw родила` _е=си` бг~ороди'тельнице 
преч\стая: тjь'мже тя` сп~се'нiе всjь'хъ на'съ свjь'мы.

%<%[И='нъ%]%>

%<I=рмо'съ: W=%> подо'бiи зла'тjь, небре'гше 
требл~же'ннiи ю='нwши, неизмjь'нный и= живы'й бж~iй 
w='бразъ ви'дjьвше, среди` _о=гня` воспjьва'ху: 
w=существова'нная да пое'тъ гд\са вся` тва'рь, и= 
превозно'ситъ во вся^ вjь'ки.

%<В%>и'дjьнъ бы'лъ _е=си` на кр\стjь` пригвожда'емь, 
и='же бога'тый въ мл\сти, во'лею погре'блся _е=си`: и= 
тридне'внw воскр\слъ _е=си`, и= и=зба'вилъ _е=си` вся^ 
человjь'ки чл~вjьколю'бче, вjь'рою пою'щыя: да пое'тъ 
гд\са вся` тва'рь, и= превозно'ситъ во вся^ вjь'ки.

%<И=%>зба'вити w\т и=стлjь'нiя, соше'дъ въ 
преиспw'дняя сло'ве бж~iй, _е=го'же созда'лъ _е=си` 
хр\сте`, си'лою твое'ю бж~е'ственною, и= без\ъ 
и=стлjь'нiя сотво'рь, сла'вы присносу'щныя твоея` 
прича'стника содjь'лалъ _е=си`, да пое'тъ, зову'щи, вся` 
тва'рь, и= превозно'ситъ хр\ста` во вjь'ки. (с. 220)

%<И='нъ. I=рмо'съ то'йже.%>

%<В%>и'дjьнъ бы'сть на земли` тобо'ю, и= съ человjь'ки 
поживе`, и='же бл~гостiю несравне'нный и= си'лою, 
_е=му'же пою'ще вси` вjь'рнiи зове'мъ: w=существова'нная 
да пое'тъ гд\са вся` тва'рь, и= превозно'ситъ во вся^ 
вjь'ки.

%<В%>ои'стинну тя` ч\стую проповjь'дающе сла'вимъ 
бц\ду: ты' бо _е=ди'на родила` _е=си` w\т тр\оцы 
воплоще'нна, _е=му'же со _о=ц~е'мъ и= дх~омъ вси` пое'мъ: 
да пое'тъ гд\са вся` тва'рь, и= превозно'ситъ во вся^ 
вjь'ки.

%<Та'же, пое'мъ пjь'снь бц\ды: В%>ели'читъ душа` моя` 
гд\са: %<съ припjь'вомъ: Ч\c%>тнjь'йшую херувi^мъ:

%<%[Пjь'снь f~%]%>

%<I=рмо'съ: Б%>езнача'льна роди'теля сн~ъ, бг~ъ и= 
гд\сь, вопло'щься w\т дв~ы на'мъ jа=ви'ся, w=мрач_е'нная 
просвjьти'ти, собра'ти расточ_е'нная: тjь'мъ всепjь'тую 
бц\ду велича'емъ.

%<Jа='%>кw въ раи` насажде'но на ло'бнjьмъ сп~се, 
требога'тое дре'во твоегw` преч\стагw кр\ста`, кро'вiю и= 
водо'ю бж~е'ственною, jа='кw w\т и=сто'чника 
бж~е'ственнагw, ребра` твоегw` хр\сте`, напоя'емо, 
живо'тъ на'мъ прозябло` _е='сть.

%<Н%>изложи'лъ _е=си` си^льныя, распны'йся всеси'льне, 
и= _е='же ни'зу лежа'щее во а='довjь тверды'ни, 
_е=стество` человjь'ческое возне'съ, на _о='ч~емъ 
посади'лъ _е=си` пр\сто'лjь. съ ни'мже тебjь` гряду'щу 
покланя'ющеся, велича'емъ.

%<Тр\оченъ: _Е=%>ди'ницу тричи'сленную, тр\оцу 
_е=диносу'щную правосла'внw пою'ще вjь'рнiи, сла'вимъ: 
непресjько'мо пребж~е'ственное _е=стество`, трисвjь'тлую, 
невече'рнюю зарю`, _е=ди'ну нетлjь'нную на'мъ свjь'тъ 
возсiя'вшую.

%<И='нъ: I=рмо'съ: W\т%> бг~а бг~а сло'ва:

%<П%>осредjь` w=сужде'нныхъ, jа='кw а='гнецъ 
возвы'шенъ бы'лъ _е=си` хр\сте`, на кр\стjь`, на 
ло'бнjьмъ копiе'мъ въ ребро` пробода'емь, живо'тъ 
дарова'лъ _е=си` на'мъ п_е'рстнымъ jа='кw бл~гъ, вjь'рою 
чту'щымъ бж~е'ственное твое` воскр\снiе. (с. 221)

%<И='%>же свое'ю сме'ртiю, сме'рти держа'ву си'лою 
о_у=праздни'вшему бг~у вси` вjь'рнiи поклони'мся, jа='кw 
и='же w\т вjь'ка м_е'ртвыя совоскр~си`, и= всjь^мъ 
подае'тъ живо'тъ и= воскр\снiе.

%<%[И='нъ%]%>

%<I=рмо'съ: В%>е'сь _е=си` жела'нiе, ве'сь сла'дость, 
сло'ве бж~iй, дв~ы сн~е, бж~е богw'въ, гд\си, ст~ы'хъ 
прест~ы'й. тjь'мъ тя` вси` съ ро'ждшею велича'емъ.

%<Ж%>е'злъ крjь'пости даде'ся _е=стеству` тлjь'нному, 
сло'во бж~iе во о_у=тро'бjь твое'й ч\стая: и= сiе` 
воскр~си`, до а='да попо'лзшееся. тjь'мже тя` всеч\стая, 
jа='кw бц\ду велича'емъ.

%<Ю='%>же и=зво'лилъ _е=си` вл\дко, прiими` мл\стивнw 
мл~твенницу, мт~рь твою` w= на'съ, и= твоея` бл~гости 
вся'ч_еская и=спо'лнятся: да тя` вси` jа='кw 
бл~годjь'теля велича'емъ.

%<По катава'сiи, _е=ктенiа` ма'лая. Та'же, С%>т~ъ 
гд\сь бг~ъ на'шъ. %<Посе'мъ _е=_ксапостiла'рiй 
о_у='треннiй.%>

%<На хвали'техъ стiхи^ры воскр\сны, гла'съ в~:%>

%<Стi'хъ: С%>отвори'ти въ ни'хъ су'дъ напи'санъ: 
сла'ва сiя` бу'детъ всjь^мъ прп\дбнымъ _е=гw`.

%<В%>ся'кое дыха'нiе и= вся` тва'рь, тя` сла'витъ 
гд\си, jа='кw кр\сто'мъ сме'рть о_у=праздни'лъ _е=си`, да 
пока'жеши лю'демъ, _е='же и=з\ъ ме'ртвыхъ твое` 
воскр\снiе, jа='кw _е=ди'нъ чл~вjьколю'бецъ.

%<Стi'хъ: Х%>вали'те бг~а во ст~ы'хъ _е=гw`, хвали'те 
_е=го` во о_у=тверже'нiи си'лы _е=гw`.

%<Д%>а реку'тъ i=уде'е, ка'кw во'ини погуби'ша 
стрегу'щiи цр~я`; почто' бо ка'мень не сохрани` ка'мене 
жи'зни; и=ли` погребе'ннаго да дадя'тъ, и=ли` воскр\сшему 
да покло'нятся, глаго'люще съ на'ми: сла'ва мно'жеству 
щедро'тъ твои'хъ, сп~се на'шъ, сла'ва тебjь`. 

%<Стi'хъ: Х%>вали'те _е=го` на си'лахъ _е=гw`, 
хвали'те _е=го` по мно'жеству вели'чествiя _е=гw`.

%<Р%>а'дуйтеся лю'дiе и= весели'теся, а='гг~лъ сjьдя'й 
на ка'мени гро'бнjьмъ, то'й на'мъ бл~говjьсти`, ре'къ: 
хр\сто'съ воскр~се и=з\ъ ме'ртвыхъ, сп~съ мi'ра, и= 
и=спо'лни вся'ч_еская бл~гоуха'нiя: ра'дуйтеся лю'дiе и= 
весели'теся. (с. 222)

%<Стi'хъ: Х%>вали'те _е=го` во гла'сjь тру'бнjьмъ: 
хвали'те _е=го` во _псалти'ри и= гу'слехъ.

%<А='%>гг~лъ о_у='бw _е='же ра'дуйся, пре'жде твоегw` 
зача'тiя гд\си, бл~года'тнjьй принесе`: а='гг~лъ же 
ка'мень сла'внагw твоегw` гро'ба въ твое` воскр\снiе 
w\твали`: _о='въ о_у='бw въ печа'ли мjь'сто, весе'лiя 
w='бразы возвjьща'я: се'й же въ сме'рти мjь'сто, вл\дку 
жизнода'вца проповjь'дуя на'мъ, тjь'мже вопiе'мъ ти`: 
бл~годjь'телю всjь'хъ, гд\си сла'ва тебjь`.

%<И='ны стiхи^ры а=нато'лiевы, гла'съ s~.%>

%<Стi'хъ: Х%>вали'те _е=го` въ тv"мпа'нjь и= ли'цjь, 
хвали'те _е=го` во стру'нахъ и= _о=рга'нjь.

%<В%>озлiя'ша мv^ра со слеза'ми на гро'бъ тво'й 
ж_ены`, и= и=спо'лнишася ра'дости о_у=ста` и='хъ, внегда` 
глаго'лати: воскр~се гд\сь.

%<Стi'хъ: Х%>вали'те _е=го` въ кv"мва'лjьхъ 
доброгла'сныхъ, хвали'те _е=го` въ кv"мва'лjьхъ 
восклица'нiя: вся'кое дыха'нiе да хва'литъ гд\са.

%<Д%>а похва'лятъ jа=зы'цы и= лю'дiе хр\ста` бг~а 
на'шего, во'лею на'съ ра'ди кр\стъ претерпjь'вшаго, и= во 
а='дjь тридне'вновавшаго: и= да покло'нятся _е=гw` и=з\ъ 
ме'ртвыхъ воскр\снiю, и='мже просвjьти'шася всегw` мi'ра 
концы`.

%<Стi'хъ: В%>оскр\сни` гд\си бж~е мо'й, да вознесе'тся 
рука` твоя`, не забу'ди о_у=бо'гихъ твои'хъ до конца`.

%<Р%>а'спятъ и= погребе'нъ бы'лъ _е=си` хр\сте`, 
jа='коже и=зво'лилъ _е=си`, и=спрове'рглъ _е=си` сме'рть, 
и= воскр~слъ _е=си` во сла'вjь, jа='кw бг~ъ и= вл\дка, 
да'руя мi'рови жи'знь вjь'чную и= ве'лiю мл\сть.

%<Стi'хъ: И=%>сповjь'мся тебjь` гд\си, всjь'мъ 
се'рдцемъ мои'мъ, повjь'мъ вся^ чудеса` твоя^.

%<_W%> вои'стинну беззако'ннiи, печа'тавше ка'мень, 
бо'льшихъ на'съ чуде'съ сподо'бисте! и='мутъ ра'зумъ 
стра'жiе, дне'сь про'йде и=з\ъ гро'ба, и= глаго'лаху: 
рцы'те, jа='кw на'мъ спя'щымъ, прiидо'ша о_у=ч~нцы`, и= 
о_у=крадо'ша _е=го`. и= кто` кра'детъ мертвеца`, па'че же 
и= на'га; са'мъ воскр~се (с. 223) самовла'стнw jа='кw 
бг~ъ, w=ста'вль во гро'бjь и= погреба'т_ельная своя^: 
прiиди'те ви'дите i=уде'_е, ка'кw не расто'рже печа'ти, 
сме'рть попра'вый и= ро'ду человjь'ческому безконе'чную 
жи'знь да'руяй, и= ве'лiю мл\сть.

%<Сла'ва, стiхи'ра _е=v\гльская. И= ны'нjь, 
бг~оро'диченъ: П%>реблагослове'нна _е=си` бц\де: 
%<Славосло'вiе вели'кое. По славосло'вiи тропа'рь:%>

%<В%>оскр~съ и=з\ъ гро'ба, и= о_у='зы растерза'лъ 
_е=си` а='да, разруши'лъ _е=си` w=сужде'нiе сме'рти 
гд\си, вся^ w\т сjьте'й врага` и=зба'вивый, jа=ви'вый же 
себе` а=п\слwмъ твои^мъ, посла'лъ _е=си` я=` на 
про'повjьдь, и= тjь'ми ми'ръ тво'й пода'лъ _е=си` 
вселе'ннjьй, _е=ди'не многомл\стиве.
