На о_у='трени, по шесто_пса'лмiи:

Бг~ъ гд\сь, и= jа=ви'ся на'мъ: на гла'съ _е~: и= 
глаго'лемъ тропа'рь воскр\сный: Собезнача'льное сло'во 
_о=ц~у`: два'жды. Сла'ва, и= ны'нjь, бг~оро'диченъ: 
Ра'дуйся две'ре гд\сня непроходи'мая: Та'же, _о=бы'чная 
стiхоло'гiа _псалти'ра.

По а~-мъ стiхосло'вiи сjьда'льны воскр\сны, гла'съ 
_е~: 

Кр\стъ гд\сень похва'лимъ, погребе'нiе ст~о'е 
пjь'сньми почти'мъ, и= воскр\снiе _е=гw` препросла'вимъ: 
jа='кw совозста'ви м_е'ртвыя w\т грw'бъ jа='кw бг~ъ, 
плjьни'въ сме'рти держа'ву, и= крjь'пость дiа'волю, и= 
су'щымъ во а='дjь свjь'тъ возсiя`.

Стi'хъ: Воскр\сни` гд\си бж~е мо'й, да вознесе'тся 
рука` твоя`, не забу'ди о_у=бо'гихъ твои'хъ до конца`.

Гд\си ме'ртвъ наре'клся _е=си`, о_у=мертви'вый 
сме'рть, во гро'бjь положи'лся _е=си`, и=стощи'вый 
гро'бы: горjь` во'ини гро'ба стрежа'ху, до'лjь w\т вjь'ка 
м_е'ртвыя воскр~си'лъ _е=си`. всеси'льне и= непостижи'ме 
гд\си, сла'ва тебjь`.

Сла'ва, и= ны'нjь, бг~оро'диченъ: Ра'дуйся ст~а'я 
горо` и= бг~опрохо'дная, ра'дуйся w=душевле'нная купино` 
и= неwпали'мая. ра'дуйся _е=ди'на къ бг~у мi'рови мо'сте, 
преводя'й м_е'ртвыя къ вjь'чному животу`. ра'дуйся 
нетлjь'нная _о=трокови'це, неискусому'жнw ро'ждшая 
сп~се'нiе ду'шъ на'шихъ.

По в~-мъ стiхосло'вiи сjьда'льны воскр\сны, гла'съ 
_е~: 

Гд\си, по тридне'внjьмъ твое'мъ воскр\снiи, и= 
а=п\слwвъ поклоне'нiи, пе'тръ вопiя'ше ти`: ж_ены` 
дерзнове'нiе прiя'ша, а='зъ же о_у=боя'хся. разбо'йникъ 
бг~осло'вяше, а='зъ же w\тверго'хся. о_у=`бо призове'ши 
ли мя` про'чее о_у=ч~нка` бы'ти; и=ли` па'ки пока'жеши 
мя` ловца` глуби'ннаго; но ка'ющася прiими' мя бж~е, и= 
сп~си' мя. 

Стi'хъ: И=сповjь'мся тебjь` гд\си, всjь'мъ се'рдцемъ 
мои'мъ, повjь'мъ вся^ чудеса` твоя^.

Гд\си, посредjь` w=сужде'нныхъ пригвозди'ша тя` 
беззако'ннiи, и= копiе'мъ ребро` твое` прободо'ша, _w 
мл\стиве! погребе'нiе же прiя'лъ _е=си`, разруши'вый 
а='дwва врата`, и= воскр\слъ _е=си` тридне'внw. 
притеко'ша ж_ены` ви'дjьти тя`, и= возвjьсти'ша а=п\слwмъ 
воста'нiе: превозноси'мый сп~се, _е=го'же пою'тъ 
а='гг~ли, бл~гослове'нный гд\си, сла'ва тебjь`.

Сла'ва, и= ны'нjь, бг~оро'диченъ: Неискусобра'чная 
невjь'сто бг~ороди'тельнице, jа='же _е='vину печа'ль 
радостотвори'вшая, воспjьва'емъ вjь'рнiи и= покланя'емся 
тебjь`, jа='кw возвела` _е=си` на'съ w\т дре'внiя 
кля'твы: и= ны'нjь моли` непреста'ннw, всепjь'тая, 
прест~а'я, во _е='же сп~сти'ся на'мъ.

V=пакои`, гла'съ _е~:

А='гг~льскимъ зра'комъ о_у='мъ смуща'ющя, и= 
бж~е'ственнымъ воста'нiемъ душе'ю просвjьща'емы, 
мv"ронw'сицы а=п\слwмъ бл~говjьствова'ху: возвjьсти'те во 
jа=зы'цjьхъ воскр\снiе, гд\су содjь'йствующу чудесы`, 
подаю'щему на'мъ ве'лiю мл\сть.

Степ_е'нна, гла'съ _е~. %[А=нтiфw'нъ%] а~. 

И='хже стiхи` повторя'юще пое'мъ:

Внегда` скорбjь'ти мнjь`, давi'дски пою` тебjь` сп~се 
мо'й: и=зба'ви ду'шу мою` w\т я=зы'ка льсти'вагw.

Пусты^ннымъ живо'тъ бл~же'нъ _е='сть, бж~е'ственнымъ 
раче'нiемъ воскриля'ющымся. 

Сла'ва: Ст~ы'мъ дх~омъ w=держа'тся вся^, ви^димая же 
съ неви'димыми: самодержа'венъ бо сы'й, тр\оцы _е=ди'нъ 
_е='сть нело'жнw. 

И= ны'нjь, то'йже. %[А=нтiфw'нъ%] в~:

На го'ры душе`, воздви'гнемся, гряди` та'мw, 
w\тню'дуже по'мощь и='детъ.

Десна'я твоя` рука`, и= мене` хр\сте` каса'ющися, w\т 
ле'сти вся'кiя да сохрани'тъ.

Сла'ва: Ст~о'му дх~у бг~осло'вяще рце'мъ: ты` _е=си` 
бг~ъ, живо'тъ, раче'нiе, свjь'тъ, о_у='мъ: ты` 
бл~госты'ня, ты` цр\ствуеши во вjь'ки. 

И= ны'нjь, то'йже. %[А=нтiфw'нъ%] г~:

W= ре'кшихъ мнjь`: во дворы` вни'демъ гд\сня: ра'дости 
мно'гiя и=спо'лненъ бы'въ, мл~твы возсыла'ю.

Въ дому` дв~довjь стра^шная соверша'ются: _о='гнь бо 
та'мw паля` вся'къ сра'мный о_у='мъ.

Сла'ва: Ст~о'му дх~у живонача'льное досто'инство, w\т 
негw'же вся'кое живо'тно w=душевля'ется, jа='кw во 
_о=ц~jь`, ку'пнw же и= сло'вjь.

И= ны'нjь, то'йже.

Прокi'менъ, гла'съ _е~: Воскр\сни` гд\си бж~е мо'й, да 
вознесе'тся рука` твоя`, jа='кw ты` цр\ствуеши во вjь'ки. 
Стi'хъ: И=сповjь'мся тебjь` гд\си, всjь'мъ се'рдцемъ 
мои'мъ. Вся'кое дыха'нiе: Стi'хъ: Хвали'те бг~а во 
ст~ы'хъ _е=гw`:

_Е=v\глiе о_у='треннее рядово'е.

Воскр\снiе хр\сто'во ви'дjьвше, поклони'мся ст~о'му 
гд\су i=и~су, _е=ди'ному безгрjь'шному: кр\сту` твоему` 
покланя'емся хр\сте`, и= ст~о'е воскр\снiе твое` пое'мъ 
и= сла'вимъ: ты' бо _е=си` бг~ъ на'шъ, ра'звjь тебе` 
и=но'гw не зна'емъ, и='мя твое` и=мену'емъ. Прiиди'те 
вси` вjь'рнiи, поклони'мся ст~о'му хр\сто'ву воскр\снiю: 
се' бо прiи'де кр\сто'мъ ра'дость всему` мi'ру. всегда` 
бл~гословя'ще гд\са, пое'мъ воскр\снiе _е=гw`: распя'тiе 
бо претерпjь'въ, сме'ртiю сме'рть разруши`. 

_Псало'мъ н~: Поми'луй мя` бж~е:

Сла'ва: Мл~твами а=п\слwвъ, мл\стиве, w=чи'сти 
мно'ж_ества согрjьше'нiй на'шихъ. 

И= ны'нjь: Мл~твами бц\ды, мл\стиве, w=чи'сти 
мно'ж_ества согрjьше'нiй на'шихъ.

Та'же, гла'съ s~: Поми'луй мя` бж~е, по вели'цjьй 
мл\сти твое'й, и= по мно'жеству щедро'тъ твои'хъ, 
w=чи'сти беззако'нiе мое`.

Посе'мъ стiхи'ра: 

Воскр~съ i=и~съ w\т гро'ба, jа='коже прорече`, даде` 
на'мъ живо'тъ вjь'чный, и= ве'лiю мл\сть. 

Сп~си` бж~е лю'ди твоя^:

И= возгла'съ: Мл\стiю и= щедро'тами и= 
чл~вjьколю'бiемъ: 

Канw'ны: Воскр\сный на д~: Кр\стовоскр\сный на г~: 
Бг~оро'диченъ на г~: Мине'и на д~. А='ще же пра'зднуется 
ст~ы'й, на s~. Кр\стовоскр\сный на в~: и= бц\ды на в~.

Канw'нъ воскр\снъ, гла'съ _е~:

%[Пjь'снь а~%]

I=рмо'съ: Коня` и= вса'дника въ мо'ре чермно'е, 
сокруша'яй бра^ни мы'шцею высо'кою, хр\сто'съ и=стрясе`: 
i=и~ля же сп~се`, побjь'дную пjь'снь пою'ща.

Припjь'въ: Сла'ва гд\си, ст~о'му воскр\снiю твоему`. 

Тебе` терноно'сный _е=вре'йскiй со'нмъ, любве` 
бл~годjь'телю къ тебjь` не сохра'нь мт~рнiя, хр\сте` 
вjьнча`, родонача'льника разрjьша'юща терно'вное 
запреще'нiе.

Воздви'глъ _е=си' мя па'дшаго въ ро'въ, прекло'нься 
жизнода'вче, безгрjь'шне: и= моея` sлосмра'дныя тли` 
хр\сте`, претерпjь'въ неискуше'нно, бж~е'ственнагw 
существа` мv'ромъ мя` w=бл~гоуха'лъ _е=си`.

Бг~оро'диченъ: Разрjьши'ся кля'тва, печа'ль преста`: 
бл~гослове'нная бо и= бл~года'тная, вjь^рнымъ ра'дость 
возсiя`, бл~гослове'нiе всjь^мъ конц_е'мъ цвjьтонося'щи 
хр\ста`.

Другi'й канw'нъ кр\стовоскр\снъ. 

Пjь'снь а~, гла'съ то'йже.

I=рмо'съ: Сп~си'телю бг~у:

И='же во'лею на кр\стjь` пригвожде'нному пло'тiю, и= 
@дре'внягw w\т и=зрече'нiя дре'вомъ@{@дре'внягw 
w=сужде'нiя и='же дре'вомъ@} па'дшаго свобо'ждшему, тому` 
_е=ди'ному воспои'мъ: jа='кw просла'вися.

И='же и=з\ъ гро'ба мертвецу` воскр\сшу хр\сту`, и= 
па'дшаго совозста'вившу, и= сосjьдjь'нiемъ _о=ч~ескимъ 
о_у=краси'вшему, тому` _е=ди'ному воспои'мъ: jа='кw 
просла'вися.

Бг~оро'диченъ: Преч\стая мт~и бж~iя, и=з\ъ тебе` 
вопло'щшемуся, и= w\т нjь'дръ роди'теля не разлу'чшемуся 
бг~у, непреста'ннw моли'ся, w\т вся'кагw w=бстоя'нiя 
сп~сти`, и=`хже созда`.

И='нъ канw'нъ прест~jь'й бц\дjь [_е=го'же 
краестро'чiе: Свjь'тъ ро'ждшая, просвjьти' мя дв~о.].

Пjь'снь а~, гла'съ то'йже.

I=рмо'съ: Коня` и= вса'дника въ мо'ре чермно'е:

Свjь'та все'льшагося въ тя` преч\стая, и= 
просвjь'щшаго мi'ръ луча'ми бж~ества`, хр\ста` моли`, 
просвjьти'ти вся^ пою'щыя тя`, мт~и дв~о.

Jа='кw о_у=краша'ема добро'тою добродjь'телей 
бл~года'тная, добротво'рное бл~голjь'пiе луче'ю дх~а, 
под\ъя'ла _е=си` всеч\стая, вся'ч_еская о_у=добри'вшаго.

Тебе` дре'вле проwбразу'ющи купина` въ сiна'и, не 
w=пали'ся дв~о, _о=гню` присовоку'пльшися: дв~а бо 
родила` _е=си`, и= дв~а пребыла` _е=си`, па'че смы'сла 
мт~и дв~о.

Катава'сiа: W\тве'рзу о_у=ста` моя^:

%[Пjь'снь г~%]

I=рмо'съ: Водрузи'вый на ничесо'мже зе'млю 
повелjь'нiемъ твои'мъ, и= повjь'сивый неwдержи'мw 
тяготjь'ющую, на недви'жимjьмъ, хр\сте`, ка'мени 
за'повjьдей твои'хъ, цр~ковь твою` о_у=тверди`, _е=ди'не 
бл~же и= чл~вjьколю'бче.

Же'лчь о_у='бw и=`же и=з\ъ ка'мене ме'дъ сса'вшiи, въ 
пусты'ни чудодjь'йствовавшему тебjь` принесо'ша хр\сте`: 
_о='цетъ же за ма'нну воз\ъ бл~годjья'нiе ти` возда'ша 
_о='троцы i=и~л_евы небл~года'рнiи.

И=`же дре'вле свjьтови'днымъ _о='блакомъ покрыва'еми, 
живо'тъ во гро'бjь хр\ста` положи'ша: но самовла'стнw 
воскр~съ, всjь^мъ вjь^рнымъ подаде` та'йнw w=сjьня'ющее 
свы'ше дх~а сiя'нiе.

Бг~оро'диченъ: Ты` мт~и бж~iя несочета'ннw родила` 
_е=си`, и='же w= нетлjь'нна _о=ц~а` возсiя'вшаго, кромjь` 
болjь'зней ма'тернихъ: тjь'мже тя` бц\ду, воплоще'нна бо 
родила` _е=си` сло'ва, правосла'внw проповjь'дуемъ. 

%[И='нъ%] I=рмо'съ: Си'лою кр\ста` твоегw` хр\сте`:

Воскр\слъ _е=си` w\т гро'ба хр\сте`, тли` сме'ртныя 
и=зба'вль воспjьва'ющихъ жизнода'вче, во'льное твое` 
распя'тiе.

Пома'зати мv'ромъ тjь'ло твое` мv"ронw'сицы хр\сте` 
тща'хуся, и= не w=брjь'тшя возврати'шася, воспjьва'ющя 
твое` воста'нiе.

Бг~оро'диченъ: Моли` непреста'ннw ч\стая, 
воплоще'ннаго и=з\ъ боку` твое'ю, и=зба'вити w\т ле'сти 
дiа'воли, воспjьва'ющыя тя` дв~у ч\стую. 

%[И='нъ%] I=рмо'съ: Водрузи'вый на ничесо'мже:

Лjь'ствица, _е='юже къ на'мъ сни'де вы'шнiй, 
и=стлjь'вшее _е=стество` и=спра'вити, ты` jа='вственнw 
ч\стая, всjь^мъ ны'нjь ви'дjьна была` _е=си`, тобо'ю бо 
пребл~гi'й мi'рови @бесjь'довати@{@причасти'тися@} 
бл~говоли`.

_Е='же дре'вле предуста'вленное дв~о та'инство, и= 
пре'жде вjь^къ прови'димое вся^ вjь'дущему бг~у, лjь'тwмъ 
ны'нjь напослjь'докъ въ ложесна'хъ твои'хъ 
всенепоро'чная, коне'цъ прiе'мъ jа=ви'ся.

Разрjьши'ся кля'твы дре'внiя w=сужде'нiе твои'мъ 
хода'тайствомъ дв~о преч\стая: и=з\ъ тебе' бо гд\сь 
jа='влься, всjь^мъ бл~гослове'нiе jа='кw пребл~гi'й 
и=сточи`, _е=ди'на человjь'кwмъ о_у=добре'нiе.

%[Пjь'снь д~%]

I=рмо'съ: Бж~е'ственное твое` разумjь'въ и=стоща'нiе, 
прозорли'вw а=вваку'мъ, хр\сте`, со тре'петомъ вопiя'ше 
тебjь`: во сп~се'нiе люде'й твои'хъ сп~сти` пома^занныя 
твоя^ прише'лъ _е=си`.

Jа=`же w\т ме'рры горча'йшыя во'ды, jа='кw во 
w='бразjь проначерта'я преч\стый кр\стъ тво'й бл~же, 
грjьхо'вное о_у=мерщвля'ющъ вкуше'нiе, дре'вомъ 
о_у=слади'лъ _е=си`.

Кр\стъ за дре'во разу'мное, за сла'дкую же пи'щу 
же'лчь, сп~се мо'й, прiя'лъ _е=си`, за тлjь'нiе же 
сме'рти кро'вь твою` бж~е'ственную и=злiя'лъ _е=си`.

Бг~оро'диченъ: Кромjь` о_у='бw сочета'нiя зачала` 
_е=си` нетлjь'ннw во чре'вjь, и= без\ъ болjь'зни родила` 
_е=си`, и= по рж\ствjь` дв~а, бг~а пло'тiю ро'ждши, 
сохрани'лася _е=си`. 

%[И='нъ%] I=рмо'съ: О_у=слы'шахъ слу'хъ си'лы кр\ста`:

Jа='кw водрузи'ся на земли` на ло'бнjьмъ кр\стъ, 
сокруши'шася вер_еи` и= вра^тницы вjь'чнiи, и= возопи'ша: 
сла'ва си'лjь твое'й гд\си.

Jа='кw сни'де сп~съ къ свя^заннымъ jа='кw ме'ртвъ, 
совоскресо'ша съ ни'мъ и=`же w\т вjь'ка о_у=ме'ршiи, и= 
возопи'ша: сла'ва си'лjь твое'й гд\си.

Бг~оро'диченъ: Дв~а роди`, и= ма'терскихъ не позна`: 
но мт~и о_у='бw _е='сть, дв~а же пребы'сть: ю='же 
воспjьва'юще, ра'дуйся бц\де, взыва'емъ.

%[И='нъ%] I=рмо'съ: Бж~е'ственное твое`:

Се'рдцемъ и= о_у=мо'мъ, душе'ю же и= о_у=сты` 
и=сповjь'дую всебл~гоче'стнw тя` бц\ду вои'стинну, 
ч\стая, сп~се'нiя пло'дъ w=б\ъе'мля, и= сп~са'юся дв~о, 
мл~твами твои'ми.

Созда'вый w\т не су'щихъ вся'ч_еская, w\т тебе` ч\стыя 
созда'тися jа='кw бл~годjь'тель бл~говоли`, на сп~се'нiе 
вjь'рою и= любо'вiю тя` пою'щихъ, всенепоро'чная. 

Пою'тъ твое` рж\ство` всенепоро'чная премi'рнiи ли'цы, 
сп~се'нiю ра'дующеся, и='стинную бц\ду му'дрствующихъ 
тя`, дв~о нескве'рная.

Тя` же'злъ и=са'iа и=менова`, w\т негw'же прозябе` 
на'мъ кра'сный цвjь'тъ, хр\сто'съ бг~ъ, на сп~се'нiе 
вjь'рою и= любо'вiю притека'ющихъ къ покро'ву твоему`.

%[Пjь'снь _е~%]

I=рмо'съ: W=дjья'йся свjь'томъ jа='кw ри'зою, къ 
тебjь` о_у='тренюю, и= тебjь` зову`: ду'шу мою` 
просвjьти` w=мраче'нную хр\сте`, jа='кw _е=ди'нъ 
бл~гоутро'бенъ.

И='же сла'вы гд\сь въ несла'внjь зра'цjь, на дре'вjь 
w=безче'щенъ во'лею ви'ситъ, w= бж~е'ственнjьй мнjь` 
сла'вjь несказа'ннw промышля'я.

Ты' мя преwбле'клъ _е=си` въ нетлjь'нiе, хр\сте`, тли` 
сме'ртныя неистлjь'ннw пло'тiю вку'шъ, и= возсiя'въ и=з\ъ 
гро'ба тридне'венъ.

Бг~оро'диченъ: Ты` пра'вду же и= и=збавле'нiе на'мъ 
ро'ждши хр\ста` без\ъ сjь'мене, свобо'дно содjь'яла 
_е=си` w\т кля'твы бц\де, _е=стество` пра'_отца. 

%[И='нъ%] I=рмо'съ: О_у='тренююще вопiе'мъ ти`:

Просте'рлъ _е=си` дла^ни сп~се на'шъ на дре'вjь, вся^ 
призыва'я къ себjь`, jа='кw чл~вjьколю'бецъ.

Плjьни'лъ _е=си` а='дъ сп~се мо'й твои'мъ 
погребе'нiемъ, и= твои'мъ воскр\снiемъ ра'дости вся^ 
и=спо'лнилъ _е=си`.

Воскр~съ w\т гро'ба тридне'внw жизнода'вче, и= всjь^мъ 
и=сточи'лъ _е=си` безсме'ртiе неги'блющее.

Бг~оро'диченъ: Дв~у по рж\ствjь` воспjьва'емъ тя` 
бц\де: ты' бо бг~а сло'ва пло'тiю мi'рови родила` _е=си`. 

%[И='нъ%] I=рмо'съ: W=дjья'йся свjь'томъ:

Вси` пр\оро'цы тя` jа='вjь предвозвjьсти'ша хотя'щую 
бы'ти бж~iю мт~рь, бц\де ч\стая: _е=ди'на бо w=брjьла'ся 
_е=си` ч\стая, соверше'нна непоро'чная.

Свjь'телъ _о='блакъ тя` живо'тныя воды`, на'мъ ту'чу 
нетлjь'нiя хр\ста` w=дожди'вшiй w\тча^яннымъ, ч\стая, 
познава'емъ. 

Jа='кw бли'зъ всю' тя до'бру и= непоро'чну, 
запечатлjь'нну дв~ствомъ чи'стjь возлюби`, въ тя` 
всели'выйся бг~ъ, jа='кw _е=ди'нъ бл~гоутро'бенъ.

%[Пjь'снь s~%]

I=рмо'съ: Неи'стовствующееся бу'рею душетлjь'нною, 
вл\дко хр\сте`, страсте'й мо'ре о_у=кроти`, и= w\т тли` 
возведи' мя jа='кw бл~гоутро'бенъ.

Въ тлjь'нiе попо'лзся родонача'льникъ, вл\дко хр\сте`, 
преслу'шаннаго бра'шна вку'шъ, и= къ животу` возведе'нъ 
бы'сть стр\стiю твое'ю.

Живо'тъ низше'лъ _е=си` ко а='ду, вл\дко хр\сте`, и= 
тлjь'нiе растлjь'вшему бы'въ, тлjь'нiемъ и=сточи'лъ 
_е=си` воскр\снiе.

Бг~оро'диченъ: Дв~а роди`, и= ро'ждши пребы'сть 
ч\ста`, на руку` нося'щаго вся'ч_еская, jа='кw вои'стинну 
дв~а мт~и поне'сшая. 

%[И='нъ%] I=рмо'съ: W=бы'де мя` бе'здна:

Просте'рлъ _е=си` дла^ни твои`, собира'я дале'че 
расточ_е'нная jа=зы^къ твои'хъ собра^нiя, хр\сте` бж~е 
на'шъ, живоно'снымъ кр\сто'мъ твои'мъ, jа='кw 
чл~вjьколю'бецъ.

Плjьни'лъ _е=си` сме'рть, и= врата` а='дwва сокруши'лъ 
_е=си`, а=да'мъ же свя'занный разрjьше'нъ бы'въ, вопiя'ше 
тебjь`: сп~се' мя десни'ца твоя` гд\си.

Бг~оро'диченъ: Купину' тя неwпали'му, и= го'ру и= 
лjь'ствицу w=душевле'ну, и= врата` нб\сная досто'йнw 
сла'вимъ, мр~i'е сла'вная, правосла'вныхъ похвало`. 

%[И='нъ%] I=рмо'съ: Неи'стовствующееся бу'рею:

И='же всему` вино'вный, и= _е='же бы'ти всjь^мъ 
пода'вый, jа='кw вино'вну и=мя'ше воплоща'емь, _е='же по 
на'мъ, тя` бг~ома'ти всенепоро'чная.

И=сцjьле'нiй вл\дчце, душепита'тельный точа'щiй 
и=сто'чникъ вjь'рнw притека'ющымъ къ покро'ву твоему` 
бл~госла'вному, вjь'мы тя` всенепоро'чная. 

Сп~се'нiю вино'вна жизнода'вца родила` _е=си` на'мъ, 
вjь'чное и=збавле'нiе да'рующаго, и='стинную бц\ду тя` 
проповjь'дающымъ.

Конда'къ, гла'съ _е~. Подо'бенъ: Собезнача'льное 
сло'во:

Ко а='ду сп~се мо'й соше'лъ _е=си`, и= врата` 
сокруши'вый jа='кw всеси'ленъ, о_у=ме'ршихъ jа='кw 
созда'тель совоскр~си'лъ _е=си`, и= сме'рти жа'ло 
сокруши'лъ _е=си`, и= а=да'мъ w\т кля'твы и=зба'вленъ 
бы'сть, чл~вjьколю'бче. тjь'мже вси` зове'мъ: сп~си` 
на'съ гд\си.

I='косъ: О_у=слы'шавшя ж_ены` а='гг~лwвы глаго'лы, 
w\тложи'ша рыда'нiе, ра'достны бы'вшя и= тре'петны, 
о_у='жасъ бо ви'дjьша, и= се` хр\сто'съ прибли'жися къ 
ни^мъ, глаго'ля: _е='же, ра'дуйтеся, дерза'йте, а='зъ 
мi'ра побjьди'хъ, и= о_у='зники свободи'хъ, потщи'теся 
о_у='бw ко о_у=ч~нкw'мъ, возвjьща'ющя и=`мъ: jа='кw 
варя'ю вы` во гра'дjь галiле'йстjьмъ, _е='же 
проповjь'дати. тjь'мже вси` тебjь` зове'мъ: сп~си` на'съ 
гд\си.

%[Пjь'снь з~%]

I=рмо'съ: Превозноси'мый _о=тц_е'въ гд\сь пла'мень 
о_у=гаси`, _о='троки w=роси`, согла'снw пою'щыя: бж~е, 
бл~гослове'нъ _е=си`.

Пло'тiю w=бло'жься jа='коже на о_у='дицjь льще'нiя, 
бж~е'ственною твое'ю си'лою sмi'я низвле'клъ _е=си` 
возводя` вопiю'щыя: бж~е, бл~гослове'нъ _е=си`.

Земли` непрохо'дное w=существова'вый составле'нiе, во 
гро'бjь покрыва'ется пло'тiю невмjьсти'мый. _е=му'же вси` 
пое'мъ: бж~е, бл~гослове'нъ _е=си`.

Бг~оро'диченъ: _Е=ди'ну о_у='бw v=поста'сь во двою` 
_е=ст_еству`, всенепоро'чная родила` _е=си` воплоще'ннаго 
бг~а. _е=му'же вси` пое'мъ: бж~е, бл~гослове'нъ _е=си`.

%[И='нъ%] I=рмо'съ: Въ пещи` _о='гненнjьй пjьсносло'вцы: 

И='же дре'вомъ кр\стнымъ i='дwльскую пре'лесть 
разрjьши'вый, бл~гослове'нъ бг~ъ _о=т_е'цъ на'шихъ.

Воскр~сы'й и=з\ъ ме'ртвыхъ, и= су'щыя во а='дjь 
совоздви'гнувый, бл~гослове'нъ бг~ъ _о=т_е'цъ на'шихъ. 

Твое'ю сме'ртiю хр\сте`, сме'ртную разо'рь держа'ву, 
бл~гослове'нъ бг~ъ _о=т_е'цъ на'шихъ.

Бг~оро'диченъ: И='же w\т дв~ы ро'ждься, и= бц\ду сiю` 
показа'вый, бл~гослове'нъ бг~ъ _о=т_е'цъ на'шихъ. 

%[И='нъ%] I=рмо'съ: Превозноси'мый _о=тц_е'въ:

Неwпредjьле'нный пребы'въ непрело'женъ, пло'ти по 
v=поста'си соедини'ся, jа='кw бл~гоутро'бенъ, въ тебjь` 
прест~jь'й: и='же _е=ди'нъ бл~гослове'нъ бг~ъ _о=т_е'цъ 
на'шихъ.

Невjь'сту тя` всенепоро'чную бц\де вл\дчце, согла'снw 
сла'вимъ и= пр\сто'лъ зижди'теля твоегw`. _е=му'же вси` 
пое'мъ: бж~е, бл~гослове'нъ _е=си`.

Мт~и всjь'хъ цр~я`, w=чи'щшися дх~омъ дв~о, была` 
_е=си`, тебе` созда'вшагw. _е=му'же вси` пое'мъ: бж~е, 
бл~гослове'нъ _е=си`.

Сп~се' мя гд\сь, бг~ома'ти преч\стая, _о=де'ждею 
пло'ти и=з\ъ тебе` w=дjь'явся. _е=му'же вси` пое'мъ: 
бж~е, бл~гослове'нъ _е=си`.

%[Пjь'снь и~%]

I=рмо'съ: Тебjь` вседjь'телю, въ пещи` _о='троцы, 
всемi'рный ли'къ спле'тше поя'ху: дjьла` вся^кая гд\са 
по'йте, и= превозноси'те во вся^ вjь'ки.

Ты` w= во'льнjьй сп~си'тельныя стр\сти помоли'лся 
_е=си` ча'ши, jа='коже нево'льнjьй: два` хотjь^нiя, 
двjьма' бо по коему'ждо но'сиши существо'ма хр\сте`, во 
вjь'ки.

Твои'мъ вседjь'тельнымъ схожде'нiемъ, а='дъ хр\сте`, 
пору'ганный и=зблева` вся^, jа=`же дре'вле ле'стiю 
о_у=мерщвл_е'нныя, тебе` превознося'щыя во вся^ вjь'ки.

Бг~оро'диченъ: Тя` jа='же па'че о_у=ма` бг~ому'жнjь 
сло'вомъ ро'ждшую гд\са, и= дв~ствующую, вся^ дjьла` дв~о 
бл~гослови'мъ, и= превозно'симъ во вся^ вjь'ки. 

%[И='нъ%] I=рмо'съ: И=з\ъ _о=ц~а` пре'жде вjь^къ:

И='же на кр\стjь` во'лею дла^ни просте'ршаго, и= 
о_у='зы см_е'ртныя разруши'вшагw хр\ста` бг~а, сщ~е'нницы 
по'йте, лю'дiе превозноси'те во вся^ вjь'ки. 

И='же jа='кw жениха` и=з\ъ гро'ба возсiя'вшаго хр\ста` 
бг~а, и= мv"роно'сицамъ jа='вльшагося, и= ра'дость тjь^мъ 
провjьща'вша, сщ~е'нницы по'йте, лю'дiе превозноси'те во 
вся^ вjь'ки.

Бг~оро'диченъ: Херувi^мъ превы'шши jа=ви'лася _е=си` 
бц\де ч\стая, во чре'вjь твое'мъ, и='же на тjь'хъ 
носи'маго поне'сши: _е=го'же со безпло'тными человjь'цы 
славосло'вимъ во вся^ вjь'ки. 

%[И='нъ%] I=рмо'съ: Тебjь` вседjь'телю въ пещи`:

Преста` ны'нjь jа='же пра'_отчая печа'ль, ра'дость 
прiе'мши ти` бг~ома'терни. тjь'мже непреста'ннw пое'мъ 
тя` дв~о, и= превозно'симъ во вся^ вjь'ки.

Пое'тъ съ на'ми безпло'тныхъ собо'ръ, дв~о рж\ство` 
твое` непостижи'мое, _е=ди'нъ ли'къ соста'вльше любо'вiю, 
и= превознося'ще _е=го` во вjь'ки.

Струя` прозра'чная безсме'ртiя _о=трокови'це и=з\ъ 
тебе` и=зы'де, и='же всjь'хъ гд\сь, скве'рну w=мыва'я 
вjь'рою тя` пою'щихъ, и= превознося'щихъ во вся^ вjь'ки.

Бж~е'ственный вои'стинну, и= свjьтоно'сный пр\сто'лъ, 
и= скрижа^ли бл~года'ти и=сповjь'дуемъ тя`, сло'во дв~о 
_о='ч~ее jа='кw прiе'мшую: _е=го'же превозно'симъ во вся^ 
вjь'ки.

Та'же, пое'мъ пjь'снь бц\ды: Вели'читъ душа` моя` 
гд\са: Съ припjь'вомъ: Ч\стнjь'йшую херувi^мъ:

%[Пjь'снь f~%]

I=рмо'съ: И=са'iе лику'й, дв~а и=мjь` во чре'вjь, и= 
роди` сн~а _е=мману'ила, бг~а же и= человjь'ка, восто'къ 
и='мя _е=му`: _е=го'же велича'юще, дв~у о_у=бл~жа'емъ.

Па'дшаго человjь'ка воспрiя'лъ _е=си` вл\дко хр\сте`, 
и=з\ъ ложе'снъ дjьви'ческихъ всему` совоку'плься, грjьху' 
же ни _е=ди'ному прича'щься: всего` w\т и=стлjь'нiя ты` 
свободи'лъ _е=си` преч\стыми твои'ми стр\стьми`.

Бг~ото'чною кро'вiю и=стоще'нною вл\дко хр\сте`, w\т 
твои'хъ преч\стыхъ ре'бръ и= животворя'щихъ, же'ртва 
о_у='бw преста` i='дwльская, вся' же земля` тебjь` 
хвале'нiя же'ртву прино'ситъ.

Бг~оро'диченъ: Не бг~а безпло'тна, ниже` па'ки 
человjь'ка про'ста произведе` ч\стая _о=трокови'ца, и= 
нескве'рная: но человjь'ка соверше'нна, и= нело'жнw 
соверше'нна бг~а. _е=го'же велича'емъ со _о=ц~е'мъ же и= 
дх~омъ. 

%[И='нъ%] I=рмо'съ: Тя` па'че о_у=ма`:

Тебе` и='же на кр\стjь` стр\сть под\ъе'мшаго, и= 
а='дову си'лу сме'ртiю сокруши'вшаго, вjь'рнiи 
правосла'внw велича'емъ.

Тебе` и=з\ъ гро'ба тридне'внw воскр~сша, и= а='дъ 
плjьни'вшаго, и= мi'ръ просвjьти'вшаго, вjь'рнiи 
_е=диному'дреннw велича'емъ.

Бг~оро'диченъ: Ра'дуйся бц\де, мт~и хр\ста` бг~а, 
_е=го'же родила` _е=си`, моли` прегрjьше'нiй w=ставле'нiе 
дарова'ти, вjь'рою пою'щымъ тя`. 

%[И='нъ%] I=рмо'съ: И=са'iе лику'й:

W\т ч\стыхъ крове'й твои'хъ о_у=сыри'ся пло'ть 
преесте'ственнw всjь'хъ содjь'телю, _е=диноро'дному сн~у 
роди'телеву, не w\т му'жа, без\ъ сjь'мене же о_у='мна и= 
w=душевле'нна, бц\де приснодв~о.

W=бходи'мое о_у=ста'вила _е=си` сме'рти неудержи'мое 
стремле'нiе, ро'ждши пло'тiю вои'стинну па'че о_у=ма` 
жи'знь вjь'чную: _е='йже прило'жься о_у=сты` го'рькими 
а='дъ о_у=праздни'ся, прест~а'я мт~и дв~о.

На пр\сто'лjь сjьдя` сн~ъ тво'й вл\дчнjь, ря'сны тя` 
златы'ми бж~е'ственныхъ добродjь'телей свjь'тъ сiя'ющу, 
w=десну'ю поста'ви себе` ч\стая, дая` че'сть jа='кw мт~ри 
тебjь`, всенепоро'чная.

Па'че о_у=ма` рж\ство` твое` бг~ома'ти: без\ъ му'жа бо 
зача'тiе въ тебjь`, и= дjьви'чески рожде'нiе бы'сть: 
и='бо бг~ъ _е='сть рожде'йся, _е=го'же велича'юще, тя` 
дв~о о_у=бл~жа'емъ.

По катава'сiи _е=ктенiа` ма'лая. Та'же, Ст~ъ гд\сь 
бг~ъ на'шъ. _Е=_ксапостiла'рiй о_у='треннiй. 

На хвали'техъ стiхи^ры воскр\сны, гла'съ _е~:

Стi'хъ: Сотвори'ти въ ни'хъ су'дъ напи'санъ, сла'ва 
сiя` бу'детъ всjь^мъ прп\дбнымъ _е=гw`.

Гд\си, запеча'тану гро'бу w\т беззако'нникwвъ, 
проше'лъ _е=си` и=з\ъ гро'ба, jа='коже роди'лся _е=си` 
w\т бц\ды: не о_у=разумjь'ша, ка'кw воплоти'лся _е=си`, 
безпло'тнiи твои` а='гг~ли: не чу'вствоваша, когда` 
воскр~слъ _е=си`, стрегу'щiи тя` во'ини. _о=боя' бо 
запечатлjь'стася и=спыту'ющымъ, jа=ви'шася же чудеса` 
кла'няющымся вjь'рою та'инству: _е='же воспjьва'ющымъ, 
возда'ждь на'мъ ра'дость и= ве'лiю мл\сть.

Стi'хъ: Хвали'те бг~а во ст~ы'хъ _е=гw`, хвали'те 
_е=го` во о_у=тверже'нiи си'лы _е=гw`.

Гд\си, вер_еи` вjь^чныя сокруши'въ, и= о_у='зы 
растерза'въ, w\т гро'ба воскр~слъ _е=си`, w=ста'вль твоя^ 
погреба^льная, во свидjь'тельство и='стиннагw 
тридне'внагw твоегw` погребе'нiя: и= предвари'лъ _е=си` 
въ галiле'и, въ пеще'рjь стрего'мый. ве'лiя твоя` мл\сть, 
непостижи'ме сп~се, поми'луй и= сп~си` на'съ.

Стi'хъ: Хвали'те _е=го` на си'лахъ _е=гw`, хвали'те 
_е=го` по мно'жеству вели'чествiя _е=гw`.

Гд\си, ж_ены` теко'ша на гро'бъ, ви'дjьти тя` хр\ста` 
на'съ ра'ди пострада'вшаго, и= прише'дшя, w=брjьто'ша 
а='гг~ла на ка'мени сjьдя'ща, стра'хомъ w\тва'льшемся, и= 
къ ни^мъ возопи` глаго'ля: воскр~се гд\сь, рцы'те 
о_у=ч~нкw'мъ, jа='кw воскр~се w\т ме'ртвыхъ, сп~са'яй 
ду'шы на'шя.

Стi'хъ: Хвали'те _е=го` во гла'сjь тру'бнjьмъ, 
хвали'те _е=го` во _псалти'ри и= гу'слехъ.

Гд\си, jа='коже и=зше'лъ _е=си` w\т запеча'таннагw 
гро'ба, та'кw вше'лъ _е=си` и= две'ремъ заключ_е'нымъ ко 
о_у=ч~нкw'мъ твои^мъ, показу'я и=`мъ тjьл_е'сная 
страда^нiя, jа=`же под\ъя'лъ _е=си` сп~се 
долготерпjьли'вый: jа='кw w\т сjь'мене дв~дова jа='звы 
претерпjь'лъ _е=си`: jа='кw сн~ъ же бж~iй, мi'ръ 
свободи'лъ _е=си`. ве'лiя твоя` мл\сть, непостижи'ме 
сп~се, поми'луй и= сп~си` на'съ. 

И='ны стiхи^ры а=натw'лiевы.

Стi'хъ: Хвали'те _е=го` въ тv"мпа'нjь и= ли'цjь, 
хвали'те _е=го` во стру'нахъ и= _о=рга'нjь.

Гд\си цр~ю` вjькw'въ, и= тво'рче всjь'хъ, на'съ ра'ди 
распя'тiе и= погребе'нiе пло'тiю прiи'мый, да на'съ w\т 
а='да свободи'ши всjь'хъ: ты` _е=си` бг~ъ на'шъ, ра'звjь 
тебе` и=но'гw не вjь'мы.

Стi'хъ: Хвали'те _е=го` въ кv"мва'лjьхъ 
доброгла'сныхъ, хвали'те _е=го` въ кv"мва'лjьхъ 
восклица'нiя: вся'кое дыха'нiе да хва'литъ гд\са.

Гд\си, пресiя^ющая твоя^ чудеса` кто` и=сповjь'сть; 
и=ли` кто` возвjьсти'тъ стра^шная твоя^ та^инства; 
вочл~вjь'чивыйся бо на'съ ра'ди, jа='кw са'мъ восхотjь'лъ 
_е=си`, держа'ву jа=ви'лъ _е=си` си'лы твоея`: кр\сто'мъ 
бо твои'мъ разбо'йнику ра'й w\тве'рзлъ _е=си`, и= 
погребе'нiемъ твои'мъ вер_еи` а='дwвы сокруши'лъ _е=си`, 
воскр\снiемъ же твои'мъ вся'ч_еская w=богати'лъ _е=си`: 
бл~гоутро'бне, гд\си, сла'ва тебjь`.

Стi'хъ: Воскр\сни` гд\си бж~е мо'й, да вознесе'тся 
рука` твоя`, не забу'ди о_у=бо'гихъ твои'хъ до конца`.

Мv"ронw'сицы ж_ены` гро'ба твоегw` дости'гшя, sjьлw` 
ра'нw и=ска'ху тебе` мv'ры пома'зати, безсме'ртнаго 
сло'ва и= бг~а: и= а='гг~ла глаго'лы w=гласи'вшяся, 
возвраща'хуся ра'достiю, а=п\слwмъ возвjьсти'ти jа='вjь, 
jа='кw воскр~слъ _е=си`, животе` всjь'хъ, и= по'далъ 
_е=си` мi'рови w=чище'нiе и= ве'лiю мл\сть.

Стi'хъ: И=сповjь'мся тебjь` гд\си всjь'мъ се'рдцемъ 
мои'мъ, повjь'мъ вся^ чудеса` твоя^.

Бг~опрiя'тнагw гро'ба ко i=уде'wмъ стра'жiе 
глаго'лаху: _w ва'шегw суему'дреннагw совjь'та! стрещи` 
неwпи'саннаго покуси'вшеся, всу'е труди'стеся, сокры'ти 
воскр\снiе распя'тагw хотя'ще, jа='снw показа'сте. _w 
ва'шегw суему'дреннагw собо'рища! что` па'ки сокры'ти 
совjь'туете, _е='же некры'ется; па'че же w\т на'съ 
о_у=слы'шите, и= вjь'ровати восхощи'те бы'вшихъ 
и='стинjь: а='гг~лъ молнiено'сецъ съ нб~се` соше'дъ 
ка'мень w\твали`, _е=гw'же стра'хомъ ме'ртвостiю 
содержи'ми бы'хомъ, и= возгласи'въ крjьпкоу'мнымъ 
мv"роно'сицамъ, глаго'лаше жена'мъ: не зрите' ли страже'й 
о_у=мерщвле'нiя, и= печа'тей разрjьше'нiя, а='дова же 
и=стоща'нiя; почто` побjь'ду а='дову о_у=праздни'вшаго, 
и= сме'ртное жа'ло сокруши'вшаго, jа='кw ме'ртва 
взыску'ете; бл~говjьсти'те же ско'рw ше'дшя а=п\слwмъ 
воскр\снiе, без\ъ стра'ха зову'щя: вои'стинну воскр~се 
гд\сь, и=мjь'я ве'лiю мл\сть.

Сла'ва, стiхи'ра о_у='тренняя _е=v\гльская. И= ны'нjь: 
Пребл~гослове'нна _е=си`: Славосло'вiе вели'кое. 

Та'же, тропа'рь воскр~сенъ:

Дне'сь сп~се'нiе мi'ру бы'сть, пое'мъ воскр\сшему 
и=з\ъ гро'ба, и= нача'льнику жи'зни на'шея: разруши'въ бо 
сме'ртiю сме'рть, побjь'ду даде` на'мъ, и= ве'лiю мл\сть.

Та'же, _е=ктенiи`. И= w\тпу'стъ. 
