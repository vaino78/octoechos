%<%[Въ недjь'лю о_у='тра, на полу'нощницjь%],%>

%<канw'нъ тр\оченъ, [_е=гw'же краестро'чiе: Х%>валю` 
тр\оце тя`, _е=ди'но бг~онача'лiе. %<Митрофа'ново.] 
Гла'съ г~%>

%<%[Пjь'снь а~%]%>

%<I=рмо'съ: В%>о'ды дре'вле, ма'нiемъ бж~е'ственнымъ, 
во _е=ди'но со'нмище совокупи'вый, и= раздjьли'вый мо'ре 
i=и~льт_ескимъ лю'демъ, се'й бг~ъ на'шъ препросла'вленъ 
_е='сть: тому` _е=ди'ному пои'мъ, jа='кw просла'вися.

%<Н%>епостижи'мое _е=ди'но гд\сонача'лiе, и= _е=ди'но 
тр\оческое бг~онача'лiе, трисiя'ннагw свjь'та твоегw' мя 
сподо'би: jа='кw да воспою` тя`, пjьва'емое непреста'ннw 
трист~ы'ми пjь'сньми, а='гг~льскими о_у=сты`.

%<С%>щ~е'ннw невеще'ственнiи вси` чи'нове пою'тъ, 
jа='кw твори'тельную вину`, _е=ди'нственную, 
трисвjь'тлую, нача'льнjьйшую jа='вjь: ю='же согла'снw 
пое'мъ, и= человjь'ч_еская мно'ж_ества, и= вjь'рнw 
сла'вимъ бре'нными о_у=сты`.

%<Сла'ва: О_у='%>мъ, и= сло'во и= дх~ъ, бг~осло'вцы 
прили'чнjь, и= знамени'тельнjь тебе` нари'чутъ: 
безстра'стно рождество` @w\т нерожде'ннагw _о=ц~а`, сн~у 
зна'менающу@{@w\т нерожде'ннагw _о=ц~а` сн~а, 
зна'менающе@}, бж~е _е=динонача'льне всjь'хъ, вку'пjь и= 
дх~а бж~е'ственнагw и=схожде'нiе.

%<И= ны'нjь: Jа='%>кw чл~вjьколю'бецъ _е=стество'мъ, 
человjь'ческое существо` прiе'мъ бж~iй сло'ве, тр\очный 
возсiя'лъ _е=си`, _е=динонача'льнjьйшiй свjь'тъ 
_е=ди'нагw бж~ества`, препросла'вленную всjь^мъ 
показа'вый, ро'ждшую тя` дв~у преч\стую. (с. 356)

%<%[Пjь'снь г~%]%>

%<I=рмо'съ: И='%>же w\т не су'щихъ вся^ приведы'й 
сло'вомъ созида'_емая, соверша'_емая дх~омъ, 
вседержи'телю вы'шнiй, въ любви` твое'й о_у=тверди` 
мене`.

%<О_у=%>тро'ити дре'вле во'ду повелjь'въ и=лiа` на 
полjь'нахъ, w=бра'знw прояви` тр\очную v=поста'сь, 
_е=ди'нственнагw бж~iя гд\сонача'лiя.

%<Т%>лjь'нное _е=стество` пое'тъ тя` земноро'дныхъ, 
_е=ди'наго и= трисвjь'тлаго созда'теля неизмjь'ннаго, и= 
вопiе'тъ ти` вл\дко: всеразли'чнагw мя` премjьне'нiя 
и=зба'ви, и= сп~си' мя.

%<Сла'ва: Р%>а'внw вjьща'юще словес_е'мъ 
пр\оро'ч_ескимъ, сла'вныхъ а=п\слъ же и= проповjь^дникъ 
вjь'ры, равнодjь'тельную тя` тр\оцу славосло'вимъ 
вjь'рнiи, бж~е всjь'хъ.

%<И= ны'нjь: С%>ъ пр\сто'ла высо'кагw сни'де 
хр\сто'съ, человjь'ка возвыша'я, jа='кw чл~вjьколю'бецъ, 
тобо'ю преч\стая, и= трисо'лнечный свjь'тъ всjь^мъ 
возсiя`.

%<Г%>д\си поми'луй, %<три'жды.%>

%<По г~-й пjь'сни сjьда'ленъ, гла'съ г~. Подо'бенъ: 
К%>расотjь` дв~ства:

%<П%>ресу'щественный и= _е=ди'не гд\си хр\сте`, и= 
рожде'нiе пребезнача'льнагw _о=ц~а`, и= дш~е 
бж~е'ственнjьйшiй, поми'луй рабы^ твоя^: вси' бо 
согрjьши'хомъ, но w\т тебе` не w\тступи'хомъ. тjь'мже 
мо'лимъ тя` трiv"поста'сне гд\си, jа='кw и=мjь'яй 
вла'сть, созда'нiе твое` сп~си` w\т вся'кагw w=бстоя'нiя.

%<Сла'ва, и= ны'нjь, бг~оро'диченъ: П%>ресу'щественный 
бг~ъ и= гд\сь и=з\ъ тебе` воплоти'ся за бл~гость, _е='же 
по на'мъ w=существова'выйся, и= пребы'въ _е='же бjь`. 
тjь'мже и= бг~очеловjь'ка сего` почита'юще всеч\стая, тя` 
неискусобра'чную бц\ду проповjь'дуемъ, сла'вяще преве'лiе 
чу'до твоегw` безсjь'меннагw рожде'нiя.

%<%[Пjь'снь д~%]%>

%<I=рмо'съ: П%>оложи'лъ _е=си` къ на'мъ тве'рдую 
любо'вь гд\си, _е=диноро'днаго во твоего` сн~а за ны` на 
сме'рть да'лъ _е=си`. тjь'мже ти` зове'мъ бл~годаря'ще: 
сла'ва си'лjь твое'й гд\си. (с. 358)

%<_О='%>трасль сугу'бъ w\т _о=ц~а`, jа='кw w\т ко'рене 
прозябе` сн~ъ, и= дх~ъ пра'вый, соесте'ственна 
прозябе'нiя, и= бг~осажде'нна, и= цвjь'ти 
собезнача'льнiи: jа='кw трiе'мъ бы'ти свjь'тwмъ 
бж~ества`. %<[Два'жды.]%>

%<Сла'ва: М%>но'ж_ества о_у='мныхъ суще'ствъ, 
непреста'ннw пою'тъ тя`, недомы'слимаго бг~а, съ ни'миже 
мы` сла'вимъ, глаго'люще: тр\оце пресу'щная, твоя^ рабы^ 
сп~си`, jа='кw чл~вjьколю'бецъ.

%<И= ны'нjь: Р%>аспали'лъ _е=си` на'съ въ любо'вь 
твою`, многомл\стиве сло'ве бж~iй, и='же на'съ ра'ди 
вопло'щься непрело'жнjь, и= трисвjь'тлое _е=ди'но 
бж~ество` та'йнw научи'въ: тjь'мже тя` сла'вимъ.

%<%[Пjь'снь _е~%]%>

%<I=рмо'съ: Jа='%>кw ви'дjь и=са'iа w=бра'знw на 
пр\сто'лjь превознесе'нна бг~а, w\т а='гг~лъ сла'вы 
дорv"носи'ма, _w _о=кая'нный, вопiя'ше, а='зъ: 
прови'дjьхъ бо воплоща'ема бг~а, свjь'та невече'рня, и= 
ми'ромъ влады'чествующа.

%<_Е=%>ди'наго гд\сонача'льника, w=бра'знw jа='кw 
ви'дjь и=са'iа бг~а, въ трiе'хъ ли'цjьхъ славосло'вима 
преч\стыми гла'сы серафi^мъ, по'сланъ бы'сть 
проповjь'дати а='бiе трисвjь'тлое существо`, и= 
_е=ди'ницу трисо'лнечную %<[Два'жды.]%>

%<Сла'ва: Jа='%>же всjь'хъ неви'димыхъ и= ви'димыхъ 
_е=стество`, w\т не су'щихъ пре'жде соста'вльшая, 
_е=ди'нице трисо'лнечная, и=`же _е=ди'наго тя` бг~а 
вjь'рнw воспjьва'ющихъ w\т вся'кихъ и=скуше'нiй 
и=зба'вльши, твоея` сла'вы сподо'би.

%<И= ны'нjь: Н%>евjь'стникъ свjьтоно'сенъ и= чи'стъ 
дв~о, бы'вшiй бж~iй, воспjьва'емъ тя` любо'вiю, и= 
блажи'мъ: и=з\ъ тебе' бо роди'ся хр\сто'съ, въ 
существа'хъ и= хотjь'нiихъ сугу'быхъ, и='же _е=ди'нъ w\т 
тр\оцы и= сла'вы сы'й гд\сь.

%<%[Пjь'снь s~%]%>

%<I=рмо'съ: Б%>е'здна послjь'дняя грjьхw'въ w=бы'де 
мя`, и= и=счеза'етъ дх~ъ мо'й: но простры'й вл\дко 
высо'кую твою` мы'шцу, jа='кw петра' мя о_у=пра'вителю 
сп~си`. (с. 358)

%<П%>ресе'льникъ сы'й а=враа'мъ, сподо'бися w=бра'знw 
воспрiя'ти _е=ди'нственнаго о_у='бw гд\са въ трiе'хъ 
v=поста'сjьхъ, пресу'щественна му'жескими же зра'ки. 
%<[Два'жды.]%>

%<Сла'ва: Н%>апра'ви сердца` твои'хъ рабw'въ къ 
свjь'ту непристу'пному, _w трисо'лнечне гд\си, и= твоея` 
сла'вы сiя'нiе пода'ждь душа'мъ на'шымъ, во _е='же 
w=блиста'_емымъ бы'ти твое'ю добро'тою неизрече'нною.

%<И= ны'нjь: W\т%>ве'рзи мнjь` дв_е'ри свjь'та 
ро'ждшагwся w\т о_у=тро'бы твоея` преч\стая: да зрю` 
трисвjь'тлую лучу` бж~ества`, и= сла'влю тя` всесвjь'тлую 
вл\дчцу.

%<Г%>д\си поми'луй, %<три'жды.%>

%<По s~-й пjь'сни сjьда'ленъ, гла'съ г~.%>

%<Подо'бенъ: Б%>ж~е'ственныя вjь'ры:

%<Б%>ж~е'ственнагw _е=стества` _е=диносу'щнагw 
трисо'лнечную пое'мъ держа'ву, и= трист~ы'ми гла'сы 
возопiи'мъ: ст~ъ _е=си` _о='ч~е пребезнача'льный, ст~ъ 
_е=си` сн~е собезнача'льный, и= дш~е ст~ы'й, _е=ди'не 
нераздjь'льный бж~е на'шъ, и= всjь'хъ тво'рче 
чл~вjьколю'бче.

%<Сла'ва, и= ны'нjь, бг~оро'диченъ. Подо'бенъ:%>

%<Чу'до преве'лiе: к%>а'кw содержа'ся невмjьсти'мый во 
чре'вjь твое'мъ и= воплоти'ся, и= jа=ви'ся, jа='кw 
человjь'къ, не претерпjь'вый смjьше'нiя и=ли` 
раздjьле'нiя бж~е'ственнагw и= непрело'жнагw бж~ества`, 
_о=трокови'це всеч\стая: тjь'мже бц\ду тя` вjь'рнw 
проповjь'дуемъ при'снw и= сла'вимъ.

%<%[Пjь'снь з~%]%>

%<I=рмо'съ: Jа='%>коже дре'вле бл~гоч_ести'выя три` 
_о='троки w=роси'лъ _е=си` въ пла'мени халде'йстjьмъ, 
свjь'тлымъ бж~ества` _о=гне'мъ и= на'съ w=зари`, 
бл~гослове'нъ _е=си` взыва'ющыя, бж~е _о=т_е'цъ на'шихъ.

%<Х%>ра'мъ мя` покажи` твоегw` бж~ества` вл\дко, 
трисiя'ннагw, ве'сь свjь'телъ: грjьхо'внагw w=мраче'нiя 
лю'тагw и= страсте'й вы'шша, свjьтода'тельными твои'ми 
сiя'нiи, бж~е _о=т_е'цъ на'шихъ бл~гослове'нъ _е=си`. (с. 
359)

%<Б%>ж~ества` зра'къ _е=ди'нъ возвjьща'емъ, въ трiе'хъ 
v=поста'сныхъ и= раздjь'льныхъ сво'йствахъ, _о=ц~а`, и= 
сн~а, и= дх~а: бл~гослове'нъ _е=си` зову'ще, бж~е 
_о=т_е'цъ на'шихъ.

%<Сла'ва: Jа=%>ви'ся а=враа'му бг~ъ трiv"поста'сный 
о_у= ду'ба дре'вле мамврi'йскагw, w= страннолю'бiи мзду` 
i=саа'ка воздая` за мл\сть: _е=го'же и= ны'нjь сла'вимъ, 
jа='кw бг~а _о=т_е'цъ на'шихъ.

%<И= ны'нjь: Jа=%>ви'ся на земли`, бы'въ человjь'къ 
вседjь'тель, бг~олjь'пнw w\т дjь'вственнагw и= преч\стагw 
твоегw` чре'ва, и= на'съ w=божи`, бл~гослове'нная 
всеч\стая, бц\де преч\стая.

%<%[Пjь'снь и~%]%>

%<I=рмо'съ: Н%>естерпи'мому _о=гню` соедини'вшеся, 
бг~оче'стiя предстоя'ще ю='нwши, пла'менемъ же 
неврежде'ни, бж~е'ственную пjь'снь поя'ху: бл~гослови'те 
вся^ дjьла` гд\сня гд\са, и= превозноси'те во вся^ 
вjь'ки.

%<Б%>езнача'льна _о=ц~а` jа='кw w\т ко'рене, сло'во, 
и= дх~ъ собезнача'льнjь су'ща: jа='кw _о='трасли 
пресу'щественнагw бг~онача'лiя показа'ша тр\оцы сла'ву 
_е=ди'ную и= си'лу: ю='же пое'мъ вси` вjь'рнiи во вjь'ки. 
%<[Два'жды.]%>

%<Сла'ва: О_у=%>правля'еши твои'ми свjьтлостьми` чи'ны 
нб\сныя, воспjьва'ти немо'лчнw трист~ы'ми пjь'сньми 
бж~е'ственными, _о='ч~е, сло'ве соwбра'зне, и= дш~е, 
трисвjь'тлую держа'ву и= равномо'щную. тjь'мже тя` пое'мъ 
во вся^ вjь'ки.

%<И= ны'нjь: П%>рореч_е'нiя пр\оро'ч_еская, твое` 
рж\ство` и=здале'ча ви'дjьвше бц\де, восхваля'ху, jа='кw 
без\ъ сjь'мене, и= па'че _е=стества` ро'ждшагося вл\дчце: 
и= согла'снw сего` пое'мъ, jа='кw гд\са, и= превозно'симъ 
во вся^ вjь'ки.

%<%[Пjь'снь f~%]%>

%<I=рмо'съ: Н%>о'вое чу'до и= бг~олjь'пное, 
дjьви'ческую бо две'рь затворе'нную, jа='вjь прохо'дитъ 
гд\сь, на'гъ во вхо'дjь, и= плотоно'сецъ jа=ви'ся во 
и=схо'дjь бг~ъ, и= пребыва'етъ две'рь затворе'на: сiю` 
неизрече'ннw jа='кw бг~ома'терь велича'емъ. (с. 360)

%<В%>и'дjьти сла'ву трисiя'нную, бг~ови'днiи 
безпло'тныхъ чи'нове jа='вjь, восходи'ти крилы` жела'ютъ 
горjь`: но говjь'ютъ sjьлw` непристу'пнагw свjь'та, и= 
пjь^сни непреста'ннw вопiю'тъ: съ ни'ми согла'снw 
_е=ди'нственная тр\оце тя` сла'вимъ. %<[Два'жды.]%>

%<Сла'ва: Н%>есы'тною любо'вiю и= и=`же на земли`, 
о_у='мную ду'шу w\т тебе` прiе'мше, и= слове'сную, тебе` 
пои'мъ вл\дко, бж~е всjь'хъ, _е=ди'нственное _е=стество` 
вои'стинну, и= тр\очное ли'цы, w\т всегw` се'рдца. 
тjь'мже ще'дре jа='кw многомл\стивъ, на'съ о_у=ще'дри.

%<И= ны'нjь: Х%>ра'мъ мя` покажи` _е=динонача'льнагw, 
и= трисвjь'тлагw твоегw` бг~онача'лiя, свjьтови'денъ, 
чи'стjь служи'ти тебjь` зижди'телю всjь'хъ, и= 
неизрече'нную твою` сла'ву о_у='мнw зрjь'ти: мл~твами 
_е=ди'ныя бц\ды, ю='же досто'йнw jа='кw пресла'вную 
велича'емъ.

%<Та'же тр\очны, григо'рiа сiнаи'та. Д%>осто'йно 
_е='сть: %<И= про'чее полу'нощницы, пи'сано въ концjь` 
кни'ги сея`.%>
