%<%[Въ суббw'ту ве'чера,  на ма'лjьй вече'рни%],%>

%<на Г%>д\си воззва'хъ, %<стiхи^ры воскр\сны, 
_о=смогла'сника, г~, повторя'юще пе'рвую. Гла'съ в~.%>

%<Стi'хъ: W\т%> стра'жи о_у='треннiя до но'щи, w\т 
стра'жи о_у='треннiя, да о_у=пова'етъ i=и~ль на гд\са.

%<П%>ре'жде вjь^къ w\т _о=ц~а` ро'ждшемуся бж~iю 
сло'ву, вопло'щшемуся w\т дв~ы мр~i'и, прiиди'те 
поклони'мся: кр\стъ бо претерпjь'въ, погребе'нiю 
предаде'ся, jа='кw са'мъ восхотjь`: и= воскр~съ и=з\ъ 
ме'ртвыхъ, сп~се' мя заблужда'ющаго человjь'ка.

%<Х%>р\сто'съ сп~съ на'шъ, _е='же на ны` рукописа'нiе 
пригвозди'въ на кр\стjь` загла'ди, и= сме'ртную держа'ву 
о_у=праздни`: покланя'емся _е=гw` тридне'вному 
воскр\снiю.

%<С%>о а=рха'гг~лы воспои'мъ хр\сто'во воскр\снiе: 
то'й бо _е='сть и=зба'витель и= сп~съ ду'шъ на'шихъ, и= 
въ сла'вjь стра'шнjьй и= крjь'пцjьй си'лjь, па'ки 
гряде'тъ суди'ти мi'ру, _е=го'же созда`.

%<Сла'ва, и= ны'нjь, бг~оро'диченъ, догма'тiкъ, гла'съ 
в~:%>

%<_W%> прев_е'лiя та^инства! зря` чудеса`, 
проповjь'дую бж~ество`, _е=мману'илъ бо _е=стества` 
о_у='бw врата` w\тве'рзе jа='кw чл~вjьколю'бецъ, 
дjь'вства же ключи` не разруши` jа='кw бг~ъ: но си'це w\т 
о_у=тро'бы про'йде, jа='коже слу'хомъ вни'де: та'кw 
воплоти'ся, jа='коже зача'тся: безстра'стнw вни'де, 
несказа'ннw и=зы'де, по пр\оро'ку глаго'лющему: сiя^ 
врата` заключ_е'на бу'дутъ: никто'же про'йдетъ и='ми, 
то'кмw _е=ди'нъ гд\сь бг~ъ i=и~левъ, и=мjь'яй ве'лiю 
мл\сть. (с. 195)

%<Та'же, С%>вjь'те ти'хiй: П%<осе'мъ, прокi'менъ: 
Г%>д\сь воцр~и'ся, въ лjь'поту w=блече'ся. %<Три'жды. 
Стi'хъ: W=%>блече'ся гд\сь въ си'лу, и= препоя'сася. 
%<Та'же, С%>подо'би гд\си въ ве'черъ се'й: %<I=ере'й же 
_е=ктенiи` не глаго'летъ: но пое'мъ:%>

%<На стiхо'внjь, стiхи'ру воскр\сну, гла'съ в~:%>

%<В%>оскр\снiе твое` хр\сте` сп~се, всю` просвjьти` 
вселе'нную и= призва'лъ _е=си` твое` созда'нiе: 
всеси'льне гд\си сла'ва тебjь`.

%<И='ны стiхи^ры прест~jь'й бц\дjь, гла'съ то'йже.%>

%<Подо'бенъ: _Е=%>гда` w\т дре'ва тя`:

%<Стi'хъ: П%>омяну` и='мя твое` во вся'комъ ро'дjь и= 
ро'дjь.

%<В%>сjь'хъ скорбя'щихъ ра'досте, и= w=би'димыхъ 
предста'тельнице, и= о_у=бо'гихъ пита'тельнице, 
стра'нныхъ же о_у=тjьше'нiе, и= же'зле слjьпы'хъ, 
немощны'хъ посjьще'нiе, тружда'ющихся покро'ве и= 
засту'пнице, и= си'рыхъ помо'щнице, мт~и бг~а вы'шнягw, 
ты` _е=си` преч\стая: потщи'ся, мо'лимся, сп~сти'ся 
рабw'мъ твои^мъ.

%<Стi'хъ: С%>лы'ши дщи` и= ви'ждь, и= приклони` 
о_у='хо твое`.

%<В%>ся'кое беззако'нiе неща'днw, вся'кiй грjь'хъ 
невозде'ржнw _о=кая'нный содjь'лахъ: вся'кагw w=сужде'нiя 
досто'инъ _е='смь! вины` покая'нiя мнjь` пода'ждь дв~о, 
jа='кw да не w=сужде'нъ та'мw jа=влю'ся, тя' бо напису'ю 
моли'твенницу, тя` призыва'ю предста'тельницу, не 
посрами` мене` бг~оневjь'стная.

%<Стi'хъ: Л%>ицу` твоему` помо'лятся бога'тiи 
лю'дстiи.

%<И=%>но'гw прибjь'жища ч\стая, къ творцу` и= вл\дцjь 
мы` не и='мамы, ра'звjь тебе` бг~оневjь'сто: да не 
w\три'неши на'съ те'плымъ твои'мъ предста'тельствомъ, 
ниже` посрами'ши любо'вiю притека'ющихъ под\ъ кро'въ 
тво'й, мт~и бг~а на'шегw: потщи'ся, и= твою` по'мощь 
да'ждь, и= ны'нjьшнягw гнjь'ва на'съ сп~си`.

%<Сла'ва, и= ны'нjь, бг~оро'диченъ, гла'съ в~:%>

%<К%>то' тя по достоя'нiю похва'литъ и= о_у=блажи'тъ, 
_о=трокови'це бг~оневjь'стная, w= _е='же тобо'ю бы'вшемъ 
мi'рови и=збавле'нiи: благодаря'ще о_у='бw зове'мъ ти`, 
(с. 196) глаго'люще: ра'дуйся, jа='же а=да'ма 
w=божи'вшая, и= разстоя^щая совокупи'вшая. ра'дуйся, 
просвjьти'вшая ро'дъ на'шъ свjьтоно'снымъ воскр\снiемъ 
сн~а твоегw` и= бг~а на'шегw: тя' бо хр\стiа'нскiй ро'дъ 
непреста'ннw о_у=блажа'емъ.

%<Та'же, Н%>ы'нjь w\тпуща'еши: %<Трист~о'е, и= по 
_О='%>ч~е на'шъ: %<Тропа'рь воскр\снъ. _Е=ктенiа` ма'лая, 
и= w\тпу'стъ.%>
